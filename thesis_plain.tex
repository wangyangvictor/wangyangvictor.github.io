\documentclass[12pt]{report}
\usepackage[margin=1in]{geometry}
% \renewcommand{\footnotesize}{\small}

% https://tex.stackexchange.com/questions/55080/how-to-hide-the-table-of-contents-heading
\makeatletter
\renewcommand\tableofcontents{%
    \@starttoc{toc}%
}
\makeatother

\usepackage{amsmath,amssymb,amsthm,bm,colonequals,graphicx,mathrsfs,mathtools,microtype,stmaryrd}
\usepackage[shortlabels]{enumitem}
\usepackage{vywang}

\usepackage[hidelinks,pagebackref=true]{hyperref}
\usepackage[alphabetic,initials]{amsrefs}

\begin{document}

{\centering
\Large Families and dichotomies in the circle method (unofficial version)

\medskip
\normalsize Victor Y. Wang

\medskip
(See \url{http://arks.princeton.edu/ark:/88435/dsp01rf55zb86g} for the official version.)
\par}

\medskip
\noindent\textbf{Abstract.}
In the eighties, Hooley applied the Grand Riemann Hypothesis, and what practically amounts to the general Langlands reciprocity (automorphy) conjecture, in a fresh new way, over certain families of cubic threefolds.
This eventually led to conditional near-optimal bounds for the number $N$ of integral solutions to $x_1^3+\dots+x_6^3 = 0$ in expanding boxes.

Elsewhere, building on Hooley's work, we have given new applications of large-sieve hypotheses, the Square-free Sieve Conjecture, and predictions of Random Matrix Theory type, over the same geometric families---for instance, conditional optimal asymptotics for $N$ in a large class of regions, with applications to sums of three cubes.
The underlying harmonic analysis---which in rough form goes back to Kloosterman---picks up equally significant contributions from the classical major and minor arcs in the circle method.

Here, we mainly provide extended summaries, commentary, and other complementary material, leaving complete traditional accounts to papers available elsewhere.
Two central themes of this thesis are families (of arithmetic or analytic objects) and dichotomies (between structure and randomness).
We especially consider (mainly in relation to the aforementioned cubic questions) families of regions, weights, point counts, oscillatory integrals, exponential sums, Hasse--Weil $L$-functions, and quadratic equations; and dichotomies for point counts over finite and infinite fields.

\bigskip
\noindent{\large \textbf{Contents}}
\medskip

\setcounter{tocdepth}{1}
% https://tex.stackexchange.com/questions/544882/how-to-put-the-abstract-on-the-same-page-as-the-table-of-contents
% {\let\clearpage\relax \tableofcontents}
\tableofcontents
\setcounter{tocdepth}{3}

\section*{Acknowledgements}
\addcontentsline{toc}{section}{Acknowledgements}

Many thanks to Professors Amit Ghosh and Peter Sarnak for suggesting the main problem considered here.
In general, I am most grateful to Professor Sarnak for pushing me in the direction of this thesis; for sharing much knowledge and insight; for positively influencing my taste in math; and for being a wonderful adviser, patient and encouraging to an endlessly ignorant student.
I would also like to thank him---and the audience members of his informal Fall 2020 seminar---for the chance to speak about, clarify, and simplify preliminary versions of my work.
In addition, I thank the late Professor Christopher Hooley, who has, through his papers, amply influenced almost all aspects of this thesis; in my mind, he has been much like a second adviser, inspiring and full of wit.

Furthermore, I thank Professors Manjul Bhargava, Tim Browning, Bill Duke, and Bob Vaughan very much for their support.
I also thank Professor Bhargava for serving as a Reader, and for exposing me to many beautiful ideas in arithmetic statistics and related areas.
Similarly, I thank Professors Nick Katz and Shou-Wu Zhang for examining my defense, for sharing many funny and inspiring stories over the years, and for helping me understand algebraic geometry a little bit better; I have especially learned much about ``life over finite fields'' from Professor Katz.

Also, I must thank Jill LeClair and Dr. Jennifer Johnson, as well as Ankit Tak, Will Crow, and others, for their general kindness, and for their excellent help with logistics and other matters throughout my time at Princeton.

Aside from those listed above, many other people have contributed professionally to my academic development in the last few years, in ways large and small.
At the risk of listing both too many and too few names,
I would like to thank
Jonathan Bober,
Andy Booker,
Brian Conrey,
Izzet Coskun,
Samit Dasgupta,
Simona Diaconu,
Alex Gamburd,
Jayce Getz,
Roger Heath-Brown,
Yotam Hendel,
Bob Hough,
Edna Jones,
Ilya Khayutin,
Valeriya Kovaleva,
Chung-Hang Kwan,
Bao Le Hung,
Mark Levi,
Zhiyuan Li,
Yuchen Liu,
Jasmin Matz,
Danny Neftin,
Fabien Pazuki,
Lillian Pierce,
Morten Risager,
Anurag Sahay,
Will Sawin,
Chad Schoen,
Freydoon Shahidi,
Efthymios Sofos,
Christof Sparber,
Ramin Takloo-Bighash,
Yuri Tschinkel,
Akshay Venkatesh,
Hong Wang,
Junho Whang,
Trevor Wooley,
Liyang Yang,
Shing-Tung Yau,
Ruixiang Zhang,
Wei Zhang,
my academic siblings and neighbors,
and others I have failed to mention,
for their general comments, questions, answers, suggestions, interest, support, and help.

On a more personal note, I thank my academic siblings, and many other fellow graduate students and members of the math department, for making math and life at Princeton much more fun and interesting than it would be otherwise.
Thanks also to Calvin Deng and Wayne Zhao for being fantastic roommates.

I am also grateful to many other people, including many old friends, teachers, and mentors, who never cease to inspire me.
And most of all, I thank my family for their patience, love, support, and spirit.

To friends, family, and adventure, I dedicate this thesis.

\section*{Conventions}
\addcontentsline{toc}{section}{Conventions}

We let $\ZZ_{\geq 0}\defeq \set{n\in \ZZ: n\geq 0}$, and define $\ZZ_{\geq 2}, \RR_{<0}, \dots$ similarly.

We let $\bm{1}_E$ denote the \emph{indicator value} of an event $E$;
i.e.~we let $\bm{1}_E\defeq 1$ if $E$ holds, and $\bm{1}_E\defeq 0$ otherwise.
When it would be too cumbersome to ``restrict'' a sum explicitly via indicator notation, we use the shorthand $\smallpsum$ to denote a \emph{restricted sum}, whose variables are restricted according to context.
% This ``restricted sum'' notation may be best avoided, or at least minimized, in formal publications? (Or maybe one should at least use $\sum^{(1)},\dots,\sum^{(17)},\dots$ instead? This can get tedious at small steps, though.)

In number-theoretic contexts, $p$ will denote a prime, and $d$ a positive divisor.
We let $v_p(-)$ denote the usual $p$-adic valuation.
For $n\in \ZZ_{\geq 1}$,
we let $\tau(n)\defeq \sum_{d\mid n} 1$;
$\omega(n)\defeq \sum_{p\mid n} 1$;
$\phi(n)\defeq n\prod_{p\mid n}(1-p^{-1})$;
$\map{rad}(n)\defeq \prod_{p\mid n} p$;
and $\mu(n)\defeq \bm{1}_{n=\map{rad}(n)}\cdot (-1)^{\omega(n)}$.
Given arithmetic functions $a,b$, we let $a\ast b$ denote their \emph{Dirichlet convolution}.

All $L$-functions will be analytically normalized
% https://www.lmfdb.org/knowledge/show/lfunction.normalization
(to have critical line $\Re(s)=1/2$, or an analogous property in the case of general Hasse--Weil $L$-functions).

By default, $\norm{\bm{x}}$ will refer to the $\ell^\infty$-norm $\norm{\bm{x}}_\infty\defeq \max_{i}(\abs{x_i})$ when $\bm{x}$ is a vector.
And in the context of indices,
$[n]$ will denote the set $\set{1,2,\dots,n}$ when $n\in\ZZ_{\geq0}$.

We let $e(t)\defeq e^{2\pi it}$ (if $t$ ``makes sense'' in $\RR/\ZZ$),
and $e_r(t)\defeq e(t/r)$ (for $r\in\RR^\times$).
% https://en.wikipedia.org/wiki/Unit_(ring_theory)#Group_of_units
% In the delta method of \cites{duke1993bounds,heath1996new}, we will write $S_{\bm{c}}(n)$ instead of $S_q(\bm{c})$ (and thus $I_{\bm{c}}(n)$ instead of $I_q(\bm{c})$), in the spirit of standard $L$-function notation.

% In dyadic decomposition (or similar) arguments, capital letters will often demarcate dyadic (or similar) ranges of corresponding lowercase variables.

We will use algebro-geometric notation freely, both for convenience and for rigor;
but most of our varieties and schemes will be explicitly embedded in a projective (or affine) space, and \emph{applied} to concrete questions.
In general,
we let $V_U(f)_{/R}$, or (by a minor abuse of notation) $V_U(f)_R$, denote the closed subscheme of $U_R$ cut out by $f=0$;
we let $V_U(f_1,\dots,f_r)_R$ denote the scheme-theoretic intersection $\bigcap_{i\in [r]} V_U(f_i)_R$.
If the base ring $R$ is clear from context, we may omit it.

% % Given a smooth \emph{or} ``algebro-geometric'' function $f$ in $m$ variables,
% If $f$ is either a nice function,
% or an abstract polynomial,
% in $m$ variables,
Given a polynomial $f(x_1,\dots,x_m)$,
we let $\map{Hess}(f)$ denote the usual $m\times m$ Hessian matrix,
$\det(\map{Hess}{f})$ the Hessian determinant,
$\map{hess}(f)_R\defeq V_{\Aff^m}(\det(\map{Hess}{f}))_R$ the affine Hessian vanishing locus over $R$,
and (when $f$ is homogeneous)
$\map{hess}(V_{\PP^{m-1}}(f)_R)\defeq V_{\PP^{m-1}}(\det(\map{Hess}{f}))_R$ the projective Hessian hypersurface over $R$.

We use the subscript notation $\partial_x\defeq \partial/\partial x$ for derivatives.
We adopt multi-index notation
% https://en.wikipedia.org/wiki/Multi-index_notation
for multivariable calculus,
especially in the context of derivatives;
% see e.g.~``$\partial_{\bm{c}}^{\bm{r}}F^\vee$'' in the proof of Proposition~\ref{PROP:characterize-diagonal-Disc-vanishing-order}.
e.g.~we let $\partial_{\bm{c}}^{\bm{r}}\defeq\partial_{c_1}^{r_1}\cdots\partial_{c_m}^{r_m}$ (for $\bm{r}\in\ZZ_{\geq0}^m$) and $\abs{\bm{b}}\defeq\sum_{i\in S}b_i$ (for $\bm{b}\in\ZZ^S$).
% We also let
% $\partial_{\log{u}}\defeq u\cdot\partial_u$ over $u\in\RR$,
% and
% $\partial_{\log\norm{\bm{x}}}\defeq\sum_i\partial_{\log{x_i}}$ over $\bm{x}\in\RR^m$.
% %the radial derivative
% (Under our definitions,
% $\partial_{\log\norm{\bm{x}}}f(\bm{x})
% = \partial_{\lambda}f(\lambda\bm{x})\vert_{\lambda=1}$.
% So in any reasonable
% radial system of coordinates $(r,\dots)$---with $r\defeq\norm{\bm{x}}_p$ for some $p>0$---we have
% $\partial_{\log\norm{\bm{x}}_p}
% = \partial_{\log\norm{\bm{x}}}$.)

We write $f\ll_S g$, or $g\gg_S f$, to mean $\abs{f}\ls_S g$,
i.e.~$\abs{f}\leq Cg$ for some $C = C(S)>0$.
We write $f\asymp_S g$ if $f\ll_S g\ll_S f$.
We let $O_S(g)$ denote a quantity that is $\ll_S g$;
and similarly, $\Omega_S(f)$ a quantity $\gg_S f$,
and $\Theta_S(g)$ a quantity $\asymp_S g$.
As usual, the implied constants $C$ throughout an argument will depend on one another in a logical fashion.
A few clarifications on our use of inexplicit inequalities may be helpful:
\begin{enumerate}[(1)]
    \item We will often attach ``size adjectives'' to inexplicit inequalities
    when we really mean to describe their implied constants;
    e.g.~``if $f\ll_{S} g$ is small'' would mean
    % ``if $f\ll_{S} g$ holds with a small implied constant''
    % (or more precisely:
    ``if $C$ is small in terms of $S$, and if $\abs{f}\leq Cg$''.
    
    \item Typically ``$\abs{f}\leq Cg$'' will either appear in a statement as
    (i) a \emph{hypothesis},
    in which case we allow $C$ to be arbitrary,
    unless there is some restriction given explicitly (e.g.~``sufficiently small'') or by context (e.g.~when applying a previous bound (X), the constant $C$ would simply be ``copied'' from (X));
    or (ii) a \emph{conclusion},
    in which case $C$ would ``follow'' or ``result'' from the proof.
    
    % (This point might be worth emphasizing, since our implied constants are perhaps better represented using a directed acyclic graph than a totally ordered sequence.
    % But we have still written our work to be linearly readable.)
    
    \item In the context of the previous point, a phrase of the form ``$P$ unless $Q$'' should be read ``if $\lnot P$, then $Q$'' (with ``hypothesis'' $\lnot P$ and ``conclusion'' $Q$);
    but we will try to minimize our use of such ``negative'' phrases, since ``positive'' phrases like ``if $P$, then $Q$'' or ``$Q$, provided $P$'' (with ``hypothesis'' $P$ and ``conclusion'' $Q$) are generally expressive enough for us.
\end{enumerate}

\chapter{Introduction}
\label{CHAP:intro}

\section{General motivation}

Diophantine equations in the tradition of Hardy--Littlewood, and $L$-functions in the tradition of Riemann, are both central objects in number theory.
Some natural problems and questions about them are the following:
\begin{enumerate}[(1)]
    \item Count, produce, or bound solutions to algebraic equations over the integers ($\ZZ$) or related rings (e.g.~$\FF_p[t]$ or $\FF_p$, for various primes $p$).
    
    \item Prove approximations to the Grand Riemann Hypothesis (GRH) for individual $L$-functions, or analyze statistics (especially those of Random Matrix Theory type) over families.
    
    \item To what extent are (1)--(2) related?
\end{enumerate}

\ex{
[BSD]
\label{EX:BSD}
Let $C$ be a soluble cubic curve in $\PP^2_\QQ$, cut out by $F=0$ for some ternary cubic form $F\in \ZZ[x,y,z]$ with nonzero discriminant.
(For example, one could take $F = x^3+y^3+60z^3$, but not $F = 3x^3+4y^3+5z^3$.)
Then Birch and Swinnerton-Dyer conjectured an equality between two integers:
(1) $\rank J(C)(\QQ)$, which measures how many primitive integral solutions $(x,y,z)\in [-X,X]^3$ to $F=0$ there are as $X\to \infty$;
and (2) $\ord_{s=1/2} L(s,C)$, where $L(s,C)$---the \emph{Hasse--Weil $L$-function} associated to $C$---roughly encodes the number of solutions to $F\equiv 0\bmod{p}$ as $p$ varies.
In general, the ``$\geq$'' direction, i.e.~``producing'' points, remains especially mysterious (even over function fields),
though both directions are difficult and interesting.
But the modularity theorem for the elliptic curve $J(C)$ often allows one to produce points, via Heegner points and the Gross--Zagier theorem; contrast with the use of modularity in Wiles' proof of Fermat's last theorem (showing that certain points do \emph{not} exist).
}

\ex{
[Quadratic equations]
\label{EX:quadratic-equations}
The most difficult part of the solution of Hilbert's eleventh problem (up to questions of effectiveness), namely the part regarding integral representations of integers by ternary quadratic forms with integral coefficients (due to Iwaniec, Duke, and Schulze-Pillot over $\QQ$), also makes essential use of automorphic forms, through subconvex $L$-function bounds obtained through the study of $L$-function families.
See e.g.~\cite{sarnak2000kloosterman} for details.

Rational representations are much simpler, with a very clean existence theory (a local-to-global principle with no exceptions) given by Hasse--Minkowski, quantifiable by the sharpest forms of the circle method (see e.g.~\cites{duke1993bounds,heath1996new} over $\QQ$).
}

Since much of what we discuss below will be conditional on standard number-theoretic conjectures, let us first make two remarks,
to give some reasons to view ``standard conjectures'' as valuable working hypotheses.
% on the value of ``standard conjectures'' as working hypotheses.
% What makes a ``standard'' conjecture?

% In order to clarify the goals of the present thesis,
% we must explain why we consider certain conjectures (like GRH) ``standard'',
% but others (like HLH)
% less so.
% Of course,
% what mathematicians consider ``standard'' at any given time
% is an inherently subjective
% social construct subject to change.
% Still,
% The following examples should demonstrate
% the value of automorphic forms and $L$-functions
% as a \emph{rich yet reliable} source of ideas and working hypotheses in Diophantine analysis.

\rmk{
Given an (appropriately normalized) automorphic $L$-function $L(s,\pi)$,
GRH is \emph{equivalent} to the ``naive probabilistic conjecture'' $\sum_{n\leq N}\mu_\pi(n)\ll_{\pi,\eps}N^{1/2+\eps}$ \cite{iwaniec2004analytic}*{Proposition~5.14}.
But what makes GRH so fascinating is that
% But crucially, despite the innocence of such naive formulations,
in addition to strong direct heuristic
and numerical evidence (at least in low dimensions)
% https://mathoverflow.net/questions/264464/numerical-evidence-for-grand-riemann-hypothesis
for its truth,
there is striking indirect evidence---from
% But crucially, GRH has found fairly strong evidence towards its truth---going beyond what the naive statement would seem to merit---both numerically,
% and psychologically
% (in terms of Random Matrix Theory models and the study of $L$-functions in families).
various comparisons with function field analogs and Random Matrix Theory (RMT) models---going
% (note: "Random Matrix Theory models" (for L-functions and their behavior) is indeed perhaps more accurate than "random matrix models", since "theory" encompasses more aspects)
beyond what the ``naive probabilistic conjecture'' above would seem to merit.
% We have much more evidence for GRH than the naive formulation/statement above would seem to merit.
% in its favor, / towards its truth

%Illustrate through Diophantine problems: subtlety of ternary quadratic form representations?
Meanwhile,
given a (positive definite) ternary integral quadratic form such as $F_0=ax^2+by^2+cz^2$,
the ``naive conjecture'' for $r_{F_0}(n)\defeq\#\set{(x,y,z)\in\ZZ^3:F_0=n}$ as $n\to\infty$
is simply false in general:
while simple probabilistic models
can detect \emph{local} obstructions,
they miss further subtleties such as ``exceptional square classes'' arising from \emph{spinor norm} obstructions.
% (nowadays understood to be examples of \emph{integral Brauer--Manin obstructions}).
As it turns out,
$r_{F_0}(n)$ does satisfy a certain natural asymptotic after adjusting for such obstructions,
but the known proofs (see Example~\ref{EX:quadratic-equations})
all use automorphic forms or $L$-functions in an essential way.
}

\rmk{
% How about homogeneous examples?
% Maybe best example: Even knowing GRH over function fields, the circle method is still hard?
Over global function fields,
the analog of GRH is known.
But modulo (Langlands and) GRH,
the analog of ``counting points in natural regions on varieties'' appears so far to be comparable in difficulty to the problem over number fields.
(There are, however, more techniques available for the analogous softer question of ``producing points'' \cite{tian2017hasse}.)
}

% \ex{
% As an example of a (currently) \emph{non-standard} conjecture,
% we discuss,
% % in ?,
% a ternary quadratic forms conjecture
% (part of which has been previously documented,
% in the introduction of \cite{golubeva1998nonhomogeneous})
% %definite vs. indefinite statement
% that would probably improve \cite{hua1938waring}.
% }

\section{Main problems of interest}

Though BSD remains wide open in general,
one can certainly consider many other interesting questions of a similar local-to-global nature, including both qualitative and quantitative questions about certain cubic equations.
In this thesis, we will focus on Examples~\ref{EX:6-cubes-zero-locus} and~\ref{EX:sums-of-cubes} below.

\ex{
[The Fermat cubic fourfold]
\label{EX:6-cubes-zero-locus}
Let $r_3(a)\defeq \#\set{(x,y,z)\in \ZZ_{\geq 0}^3: x^3+y^3+z^3=a}$, for $a\in \ZZ_{\geq 0}$.
For real $X\to\infty$, let $M_2(X)\defeq \sum_{0\leq a\leq X^3} r_3(a)^2$.
By Cauchy--Schwarz, $\#\set{a\in [0,A]: r_3(a)\neq 0}\gg A^2/M_2(A^{1/3})$ for real $A\to \infty$.

Let $F(\bm{x})=F(x_1,\dots,x_6)\defeq x_1^3+\dots+x_6^3$.
Then $M_2(X) = \#\set{\bm{x}\in \ZZ^6\cap XK: F(\bm{x})=0}$ for some fixed compact region $K\belongs \RR^6$.
Beginning with Hardy and Littlewood in \cite{hardy1925some} (roughly), many authors, inspired in part by connections to the statistics of \emph{sums of three cubes} (via $M_2$, for instance), have sought to estimate the number of solutions $\bm{x}\in \ZZ^6$ to $F(\bm{x}) = 0$ in expanding boxes or other regions.

The integral solutions to $F(\bm{x}) = 0$ are expected to exhibit a randomness-structure dichotomy (along the lines of the Manin conjectures), as we now explain.
The purely probabilistic ``Hardy--Littlewood model'' predicts $M_2(X)\sim c_{\textnormal{HL}}\cdot X^{6-3}$, where the constant $c_{\textnormal{HL}}\defeq \sigma_\RR\cdot \prod_{p}\sigma_p\in \RR_{>0}$ is a product of local densities measuring the ``local'' or ``adelic'' (i.e.~real and $p$-adic) bias of the equation $F=0$ (over the regions $K$ and $\ZZ_p^6$).
But $F=0$ also has $\asymp X^{6/2}$ \emph{special} solutions $\bm{x}\in \ZZ^6\cap XK$, i.e.~$\bm{x}$ with ``$x_i+x_j=0$ in pairs'' (e.g.~$\bm{x}$ with $x_1+x_2=x_3+x_4=x_5+x_6=0$).
In fact, Hooley showed that $M_2(X) - \max\left(c_{\textnormal{HL}}\cdot X^3, \#\set{\textnormal{special}\;\bm{x}\in \ZZ^6\cap XK}\right) \gg X^3$ for all sufficiently large $X\gg 1$ \cite{hooley1986some}*{Theorem~1 and the ensuing sentence}, and conjectured that $M_2(X)\sim c_{\textnormal{HL}}\cdot X^3 + \#\set{\textnormal{special}\;\bm{x}\in \ZZ^6\cap XK}$ \cite{hooley1986some}*{Conjecture~2}.
For convenience, call this conjecture ``HLH'' (or ``HLH for $(F,K)$'').

What is known towards HLH?
By Cauchy between ``structure'' and ``randomness'' (in four and eight variables, respectively), Hua showed that $M_2(X)\ll_\eps X^{7/2+\eps}$ \cite{hua1938waring}.
By isolating a new source of randomness (``typical divisors'' of integers $x\ll X$), Vaughan gave a more robust proof of Hua’s bound, ultimately leading to the improvement $M_2(X)\ll_\eps X^{7/2}(\log{X})^{\eps-5/2}$ \cites{vaughan1986waring,vaughan2020some}.
Conditionally under a certain ``Hypothesis HW'' (practically amounting to automorphy and GRH for the Hasse--Weil $L$-functions associated to smooth projective hyperplane sections of the form $F(\bm{x}) = \bm{c}\cdot\bm{x} = 0$, for $\bm{c}\in \ZZ^6$),
Hooley established the near-optimal bound $M_2(X)\ll_\eps X^{3+\eps}$ \cites{hooley1986HasseWeil,hooley_greaves_harman_huxley_1997}, using an ``upper-bound precursor'' to the delta method of \cites{duke1993bounds,heath1996new};
Heath-Brown proved (among other things) the same result using the delta method \cite{heath1998circle}, independently modulo \cite{hooley1986HasseWeil}.

HLH lies beyond the classical Hardy--Littlewood circle method
(according to square-root ``pointwise'' minor arc considerations),
though the ``Hardy--Littlewood part'' arises satisfactorily from the classical major arcs.
But the delta method opens the door to progress on HLH,
by harmonically decomposing the true minor arc contribution in a ``dual'' fashion (via Poisson summation, essentially applied in the form of the Nyquist--Shannon sampling theorem).
This is the impetus for \cites{wang2021_isolating_special_solutions,wang2021_HLH_vs_RMT}, papers to be discussed in Chapters~\ref{CHAP:isolating-special-solutions}--\ref{CHAP:using-mean-value-L-function-predictions} below.
}

\rmk{
The main results of \cites{wang2021_isolating_special_solutions,wang2021_HLH_vs_RMT} apply to diagonal cubic forms in six variables, but to simplify notation, we will mostly focus on the Fermat case.
The Fermat case is arguably the most interesting anyways;
a simple \Holder argument shows that out of all diagonal cubic forms in six variables,
the Fermat cubic has the greatest number, $\int_0^1 d\theta\, \left(\sum_{x\in \ZZ\cap [-X,X]} e(\theta x^3)\right)^6$, of integral zeros $\bm{x}\in [-X,X]^6$.

On the other hand, to extend our work to general cubic forms in six variables (with nonzero discriminant, say), it would take a lot of technical, but significant, work;
see Chapters~\ref{CHAP:isolating-special-solutions}--\ref{CHAP:using-mean-value-L-function-predictions} below for some remarks on what is presently missing.
}

\ex{
[Sums of cubes]
\label{EX:sums-of-cubes}
For integers $s\geq 1$, let $g_s(y_1,\dots,y_s)\defeq y_1^3+\dots+y_s^3$.
In 1825, Ryley proved $\QQ\belongs g_3(\QQ^3)$ by explicitly constructing $x,y,z\in \QQ(t)$ such that $t = x^3+y^3+z^3$.
(See Example~\ref{EX:Ryley's-theorem} below for details.)

How about writing a given integer $a\in\ZZ$ \emph{integrally} in the form $g_3(\ZZ^3)$?
On the one hand,
$g_1(\ZZ)\belongs\set{0,\pm1}\bmod{9}$,
so $g_3(\ZZ^3)\belongs\set{0,\pm1,\pm2,\pm3}\bmod{9}$,
which \emph{prevents} each integer $a\equiv\pm4\bmod{9}$ from decomposing as a sum of three integer cubes.
On the other hand, as it turns out, there do not exist any other such \emph{local obstructions}.
Does every integer $a\not\equiv\pm4\bmod{9}$ in fact lie in $g_3(\ZZ^3)$?
(Cf.~\cite{heath1992density}*{p.~623}.)

This question seems quite subtle:
the space of possible representations seems ``relatively sparse on average'' over $a$,
and there is no known algorithm that can \emph{provably} determine whether an ``arbitrary input'' $a\not\equiv\pm4\bmod{9}$ is represented or not.
% for a fixed value of $a\not\equiv\pm4\bmod{9}$, it is unknown (beforehand) how far one would have to search to determine if a representation exists or not.
Even just to give a complete affirmative answer to the question in the finite range $\abs{a}<100$,
Booker and Sutherland had to search quite far
(and even further to find a \emph{new} representation of $3=g_3(1,1,1)=g_3(4,4,-5)$); see Theorems~\ref{THM:booker-33} and~\ref{THM:booker-sutherland-42-and-3} below.

In fact,
even the weaker question of proving $\ZZ\belongs g_4(\ZZ^4)$
appears to be open
(though it is known that $\set{a\not\equiv\pm4\bmod{9}}\belongs g_4(\ZZ^4)$ \cite{demjanenko1966sums},
and therefore that $\ZZ\belongs g_5(\ZZ^5)$).

The situation is better \emph{on average}:
\cite{davenport1939waring} showed that asymptotically $100\%$ of integers $a>0$ lie in $g_4(\ZZ_{>0}^4)$.
(However, for \emph{general} quaternary cubic forms with nonzero discriminant,
the analog of \cite{davenport1939waring}'s result
% seems much more difficult
is only known conditionally, by \cite{hooley2016representation}.)
One of the main goals of this thesis is to summarize our work \cite{wang2021_HLH_vs_RMT} proving conditional results of a similar flavor for $g_3(\ZZ_{\geq0}^3)$ and $g_3(\ZZ^3)$.

For now, let us note that $g_3(\ZZ_{\geq0}^3)$ contains
$\gg A^{0.91709477}$ integers $a\in [0,A]$, unconditionally, by \cites{wooley1995breaking,wooley2000sums,wooley2015sums};
and $\gg_\eps A^{1-\eps}$ integers $a\in [0,A]$, conditionally on Hypothesis HW from Example~\ref{EX:6-cubes-zero-locus}, by \cites{hooley1986HasseWeil,hooley_greaves_harman_huxley_1997}.
Both Wooley and Hooley use upper bounds on certain second moments, in the spirit of Example~\ref{EX:6-cubes-zero-locus}.
Wooley uses difficult iterative arguments involving smooth numbers.
}

\rmk{
See Observation~\ref{OBS:the-exceptional-set-for-5y_1^3+12y_2^3+9y_3^3-representations-is-nonempty} and Remark~\ref{RMK:obstruction-subtleties} below for a discussion of possible obstructions (or lack thereof) to the Hasse principle for $g_3(\ZZ^3)$ and related problems.
In general, while rational points can already be very subtle (see e.g.~BSD), integral points can be even subtler (see e.g.~\cites{harpaz2017geometry,wilsch2022integral}).
}


\thm{
[\cite{booker2019cracking}]
\label{THM:booker-33}
% At five past nine in the morning on the 27th of February 2019, Booker’s computer found a solution:
% 33 = 8866128975287528^3 + (-8778405442862239)^3 + (-2736111468807040)^3
% When Tim Browning heard the news, he excitedly replaced his homepage with the solution, which was how it initally became publicly known. (This caused some initial confusion: because he didn’t include any more details, some people assumed that Tim Browning himself had made the discovery; but he quickly set the record straight.)
% https://aperiodical.com/2019/09/42-is-the-answer-to-the-question-what-is-80538738812075974%C2%B3-80435758145817515%C2%B3-12602123297335631%C2%B3/
% (See also https://math.mit.edu/~drew/ANTSXIV/BookerSlides.pdf from the talk "33 and all that" available at https://www.youtube.com/watch?v=so6VBxWwWVI - "At 9:05am GMT on February 27th 2019, a computer in Bristol found the solution to $x^3+y^3+z^3=33$ shown on the previous slide.")
Using integers with $16$ digits,
\mathd{
g_3(8866128975287528,
-8778405442862239,
-2736111468807040)
= 33.
}
}
\thm{
[\cite{booker2021question}]
\label{THM:booker-sutherland-42-and-3}
% 42: Probably found on September 5, 2019?
% New 3: Probably around September 18, 2019 (or maybe a few days earlier)
Using integers with $17$ digits,
\mathd{
g_3(-80538738812075974,
80435758145817515,
12602123297335631)
= 42.
}
Also, using two integers with $21$ digits, and a \emph{third with only $18$ digits},
\mathd{
g_3(569936821221962380720,
-569936821113563493509,
-472715493453327032)
= 3,
}
thus affirmatively answering a question of Mordell.
}

\section{Related background}

For Observation~\ref{OBS:the-exceptional-set-for-5y_1^3+12y_2^3+9y_3^3-representations-is-nonempty} and Example~\ref{EX:Ryley's-theorem}, we need the following technical result:
\prop{
[Cf.~\cite{poonen2017rational}*{Remark~9.4.11}]
\label{PROP:sm-proj-cubic-hypersurface-K-point-implies-K-unirational-implies-K-dense}
Fix a smooth projective cubic hypersurface $W$ of dimension $d\geq 2$ over a field $K$.
Then $W(K)\neq\emptyset$ if and only if $W$ is $K$-unirational;
% and if either condition holds,
and in this case,
we must have $\ol{W(K)}=W$ if $K$ is infinite.
}

\pf{
The unirationality criterion is due to \cite{segre1943note} when $(d,K)=(2,\QQ)$,
and to \cite{kollar2002unirationality} in general.
The rest hinges on the fact that
% is due to / owes to / boils down to (hinges on) / stems from / comes from
if $\#K=\infty$, then $\PP^d_K(K)$ is dense in $\PP^d_K$.
% if $U\belongs\PP^d_K$ is dense open and $\#K=\infty$, then $\ol{U(K)}=U$ (with the closure being taken in $U$; i.e.~$U(K)$ is dense in $U$).
% and https://math.stackexchange.com/questions/114462/a-map-is-continuous-if-and-only-if-for-every-set-the-image-of-closure-is-contai (if $f\maps U\to W$ continuous, then $\ol{f(U(K))}\contains f(\ol{U(K)})$; if $f$ is a dominant morphism, then $f(\ol{U(K)})=f(U)$ is dense in $W$, which implies the result)
}

\obs{
[Based on \cite{cassels1966hasse}]
\label{OBS:the-exceptional-set-for-5y_1^3+12y_2^3+9y_3^3-representations-is-nonempty}
Let $h(\bm{y})=h(y_1,y_2,y_3)\defeq 5y_1^3+12y_2^3+9y_3^3$.
Then the Hasse principle for $h(\ZZ^3)$ is \emph{false}.
More precisely,
the ``exceptional set'' $\mcal{E}(h)\defeq\set{a\in\ZZ: h(\bm{y})=a\;\textnormal{fails Hasse}}$ is \emph{nonempty}.
}

\pf{
For an inhomogeneous ternary cubic $P\in \QQ[\bm{y}]=\QQ[y_1,y_2,y_3]$,
let $W_P$ denote the hypersurface in $\PP^3_\QQ$ cut out by $y_4^3P(\bm{y}/y_4) = 0$.
By \cite{cassels1966hasse},
the surface $W=W_{h+10}$ has $\bd{A}_\QQ$-points, but no $\QQ$-points.
This can be explained by $W(\bd{A}_\QQ)^{\Br}$, the ``Brauer--Manin set'' (BMS) for $W$
(see \cite{manin1986cubic}*{\S47.6} and \cite{poonen2017rational}*{\S9.4.9 and Remark~9.4.30}).

Now for convenience,
let \emph{$R$-soluble} mean ``soluble over $R$''.
Then by the previous paragraph,
$h(\bm{y})+10=0$ is $\QQ_v$-soluble for each place $v\leq\infty$
(by Proposition~\ref{PROP:sm-proj-cubic-hypersurface-K-point-implies-K-unirational-implies-K-dense}, applied to $W_{\QQ_v}$),
but \emph{not} $\QQ$-soluble.
In particular,
there must exist a \emph{constant} $q\in\ZZ\setminus\set{0}$ such that
$h(\bm{y})+10t^3=0$ is $\ZZ_p$-soluble for each $t\in q\ZZ$ and prime $p\leq 5$.

On the other hand,
$h(\bm{y})=a$ is $\ZZ_p$-soluble for each $a\in\ZZ$ and prime $p\geq 7$
(as one can prove using Hensel's lemma and
% https://en.wikipedia.org/wiki/Restricted_sumset#Cauchy%E2%80%93Davenport_theorem
either the Cauchy--Davenport theorem or the Weil conjectures).
We conclude that for each $t\in q\ZZ\setminus\set{0}$,
the equation $h(\bm{y})=-10t^3$ is $\bd{A}_\ZZ$-soluble, but \emph{not} $\ZZ$-soluble.
In other words,
$\mcal{E}(h)\contains\set{-10t^3:t\in q\ZZ\setminus\set{0}}$.
}

\rmk{
\label{RMK:obstruction-subtleties}
The (rational!) BMS for $W_{h-a}$ ``explains'' at least part of $\mcal{E}(h)$.
But that for $W_{g_3-a}$ says nothing about the analogous set $\mcal{E}(g_3)$:
even the \emph{integral} (``enti\`{e}re'') $\ZZ$-BMS for $g_3-a=0$ can never obstruct $\ZZ$-solubility
%``p.~1304, les diverses lois de reciprocite englobees dans l'obstruction de brauer-manin entiere ne permettent d'exclure aucun entier a''
\cite{colliot2012groupe}*{p.~1304}.
(The BMS for $W_{g_3-a}$ \emph{can} obstruct weak approximation over $\QQ$
\cite{heath1992density}*{Theorem~1 and ensuing paragraphs}.
This is relevant to \emph{approximation questions} for $g_3-a=0$,
but we have chosen to focus on \emph{Hasse principles} in Example~\ref{EX:sums-of-cubes}.)

Nonetheless, could there be other obstructions?
(For other affine cubic surfaces,
there can indeed be obstructions \emph{not} ``directly'' explained by the $\ZZ$-BMS;
see \cite{colliot2020brauer}*{Theorem~5.14} and \cite{loughran2021integral}*{Theorem~1.5},
both based ``indirectly'' on the $\ZZ$-BMS.
Cf.~\cite{corwin2020brauer}*{Corollary~1.1} and \cite{liu2015very}*{Corollary~3.10}, over $\QQ$.)
}

Let us now recall some background on producing points, related to Examples~\ref{EX:6-cubes-zero-locus} and~\ref{EX:sums-of-cubes}.
This background will also illustrate how weaker goals (e.g.~producing instead of counting points) sometimes allow for greater flexibility and creativity.

\ex{
\label{EX:Ryley's-theorem}
If a cubic equation over a field $K$ has a sufficiently general solution over a quadratic extension $L/K$, then it has a solution over $K$.
(See also \cite{birch2004heegner}*{\S3}'s discussion of \cite{heegner1952diophantische}
% in this case the norm map on E seems to suffice? see Birch's argument about P being in the nontrivial coset of E(R)/2E(R); then consider (the odd number of) conjugates of P.
for a similar fact called ``Heegner's lemma'', for $L/K$ of odd degree and ``typical'' quartic $y^2=f(x)$ over $K$.)

This idea (or alternatively, Proposition~\ref{PROP:sm-proj-cubic-hypersurface-K-point-implies-K-unirational-implies-K-dense}) can be used to geometrically derive Ryley's theorem that every rational number is a sum of three rational cubes, starting from a ``trivial solution at infinity''; see \cite{manin1986cubic}*{Introduction}.
% by a generalization of the chord-and-tangent process.
}

\ex{
Using ternary \emph{quadratic} forms, \cite{linnik1943representation} proved $G(3)\leq 7$, i.e.~every sufficiently large integer is a sum of $7$ positive cubes.
This remains the record for $G(3)$.
% , though conditional on ideas below it might be possible to show that every integer is a sum of $6$ (not necessarily positive) integer cubes.

The expected asymptotic formula in Waring's problem for $7$ cubes remains unproven, but would follow if one knew $M_2(X)\ll X^{3.25-\delta}$ (with $M_2(X)$ defined as in Example~\ref{EX:6-cubes-zero-locus}).
The asymptotic for $8$ cubes is ``barely'' known \cite{vaughan1986waring}, but an easy proof would follow if one knew $M_2(X)\ll X^{3.5-\delta}$.
}

\ex{
Assume the finiteness of the Tate--Shafarevich group $\Sha(E/K)$ for every quadratic extension $K/\QQ$ and elliptic curve $E = E_A:X^3+Y^3=AZ^3$.
Then \cite{swinnerton2001solubility} proved, over $\QQ$, the Hasse principle for diagonal cubic threefolds $a_1x_1^3+\dots+a_5x_5^3=0$ in $\PP^4$, and for ``typical'' diagonal cubic surfaces in $\PP^3$.
The proof uses that diagonal hypersurfaces are DPC (dominated by a product of curves such as $a_ix_i^3+a_jx_j^3+Bx_0^3=0$), and a version of the \emph{fibration method} (i.e.~finding a ``nice'' $B$).
}

\rmk{
Over $\QQ$, the Hasse principle for diagonal cubic hypersurfaces in $\PP^{s-1}$ is known unconditionally for $s\geq 7$ \cite{baker1989diagonal}.
% The situation for (standard) asymptotics is the same as for $G(3)\leq 7$ and $G(3)\leq 8$ above.
For non-diagonal smooth cubic hypersurfaces in $\PP^{s-1}$, it is known unconditionally for $s\geq 9$ \cite{hooley1988nonary}.
% But for the asymptotic, one might need to localize away from the Hessian in \cite{hooley1988nonary} (though localizing doesn't seem to be necessary in \cite{heath1983cubic}).
}

\section{Outline and overview of chapters}

Recall, from Example~\ref{EX:6-cubes-zero-locus}, Hooley's conjecture on $M_2(X)$.

Chapter~\ref{CHAP:approx-variances} shows that a \emph{slightly deformed} version of Hooley's conjecture would imply that asymptotically $100\%$ of integers $a\not\equiv \pm4\bmod{9}$ are sums of three cubes.
The proof follows \cite{diaconu2019admissible}, up to technical (but important) modifications.
There is some flexibility in the choice of deformation; we will show that ``HLH for clean pairs $(x_1^3+\dots+x_6^3,w)$'', in the sense of Definitions~\ref{DEFN:support-smooth-clean} and~\ref{DEFN:HLH-asymptotic} below, suffices.

Before proceeding, we first give two convenient general definitions.

\defn{
\label{DEFN:weighted-zero-count}
Let $s\in \ZZ_{\geq 1}$.
Given a function $w\maps\RR^s\to\RR$ and a polynomial $P\in\ZZ[\bm{x}]=\ZZ[x_1,\dots,x_s]$,
let $N_{P,w}(X)
\defeq\sum_{\bm{x}\in\ZZ^s}
w(\bm{x}/X)
\cdot\bm{1}_{P(\bm{x})=0}$
for real $X>0$.
%We could specify a https://en.wikipedia.org/wiki/Function_space#Functional_analysis but C_c doesn't include sharp cutoffs (maybe B_c?).
If $w$ is unspecified,
we use the \emph{symmetric} box convention $w(\tilde{\bm{x}})\defeq\bm{1}_{\tilde{\bm{x}}\in[-1,1]^s}$.
}

\defn{
\label{DEFN:projectively-smooth-homogeneous-polynomial}
Let $R$ be a ring.
Let $s\in \ZZ_{\geq 1}$.
Given a homogeneous polynomial $P$ that lies in (or maps canonically into) $R[x_1,\dots,x_s]$,
we say $P$ is \emph{$\PP^{s-1}_R$-smooth} if its projective zero locus $V_{\PP^{s-1}}(P)_{/R}$ is smooth; or equivalently, if $P$ has invertible discriminant in $R$.
}

For analytic purposes, the following definition will prove useful:
\defn{
\label{DEFN:support-smooth-clean}
Let $s\in \ZZ_{\geq 1}$.
We will always
% ``supported on'' (or ``away from'') and ``$\map{Supp}(f)$''
interpret the \emph{support} of a function on $\RR^s$ in the \emph{closed} sense.
%https://en.wikipedia.org/wiki/Support_(mathematics)#Closed_support
Let $w\maps\RR^s\to\RR$ be a function; then $\Supp{w}\defeq \ol{\set{\bm{x}\in \RR^s: w(\bm{x})\neq 0}}$.
Let $P(x_1,\dots,x_s)$ be a $\PP^{s-1}_\RR$-smooth homogeneous polynomial.
Call the pair $(P,w)$ \emph{smooth} if $\bm{0}\notin \Supp{w}$, and \emph{clean} if $(\Supp{w})\cap (\map{hess}{P})_\RR(\RR) = \emptyset$;
in other words, call $(P,w)$ \emph{smooth} (resp.~\emph{clean}) if $w$ is supported away from the locus $x_1=\dots=x_s=0$ in $\RR^s$ (resp.~the locus $\det(\map{Hess}{P}) = 0$ in $\RR^s$).
}

\rmk{
A clean pair $(P,w)$ is smooth, since $P$ is homogeneous.
}

\ex{
\label{EX:Fermat-smooth-clean-pairs-explicitly}
Let $s\in \ZZ_{\geq 1}$.
Say $P = x_1^3+\dots+x_s^3$.
Then $P$ is $\PP^{s-1}_\RR$-smooth, and in fact $\PP^{s-1}_k$-smooth for every field $k$ of characteristic not dividing $3$.
Furthermore, $(P,w)$ is clean if and only if $w$ is supported away from the set $\set{\bm{x}\in \RR^s: x_1\cdots x_s = 0}$.
}

% For convenience in what follows, we let $o_{S; x\to a}(g)$ denote\dots
We now define a singular series, some special loci, some real densities, and a weighted version of HLH.
We emphasize that one could attribute the general formulation of HLH (including that in Definition~\ref{DEFN:HLH-asymptotic}) to \cite{hooley1986some}*{Conjecture~2}, \cite{vaughan1995certain}*{Appendix}, Manin--Peyre, et al.
\defn{
\label{DEFN:HLH-asymptotic}
Let $F\in \ZZ[\bm{x}] = \ZZ[x_1,\dots,x_6]$ be a $\PP^5_\QQ$-smooth $6$-variable cubic form.
Let $S_{\bm{0}}(n)\defeq \sum_{a\in (\ZZ/n)^\times} \sum_{\bm{x}\in (\ZZ/n)^6} e_n(aF(\bm{x}))$ for $n\in \ZZ_{\geq 1}$;
let $\mf{S}_F\defeq \sum_{n\geq 1} n^{-6}S_{\bm{0}}(n)$.
Let $C(\map{SSV})$ denote the set of $3$-dimensional vector spaces $L/\QQ$ such that $F\vert_L = 0$.
Given $L\in C(\map{SSV})$, let $\Lambda\defeq L\cap \ZZ^6$ denote
the unique \emph{primitive} sublattice of $\ZZ^6$ with $\Lambda\cdot\QQ=L$.
Then let $L^\perp\defeq \set{\bm{c}\in \QQ^6: \bm{c}\perp L}$ and $\Lambda^\perp\defeq \set{\bm{c}\in \ZZ^6: \bm{c}\perp \Lambda}$ denote
the \emph{orthogonal complements} of $L$ and $\Lambda$, respectively,
with respect to the usual dot product $\bm{c}\cdot \bm{x}\defeq c_1x_1+\dots+c_6x_6$
(so in particular, $\QQ\cdot \Lambda^\perp = L^\perp$).
Now choose bases $\bm{\Lambda},\bm{\Lambda^\perp}$ of $\Lambda,\Lambda^\perp$,
viewed as $6\times 3$ and $3\times 6$ integer matrices, respectively,
so that $\Lambda=\bm{\Lambda}\ZZ^3$ and $\Lambda^\perp=\ZZ^3\bm{\Lambda^\perp}$
(where we view $\Lambda$ as a ``column space'' and $\Lambda^\perp$ as a ``row space'').

Let $w\in C^\infty_c(\RR^6)$ be a smooth compactly supported weight.
Let $\sigma_{\infty,F,w}\defeq \lim_{\eps\to0} \,(2\eps)^{-1}\int_{\abs{F(\bm{x})}\leq\eps}d\bm{x}\,w(\bm{x})$.
For each $L\in C(\map{SSV})$, choose $\bm{\Lambda^\perp}$ as in the previous paragraph,
and let $\sigma_{\infty,L^\perp,w}\defeq \lim_{\eps\to0}\,(2\eps)^{-3}\int_{\norm{\bm{\Lambda^\perp}\bm{x}}_\infty\leq\eps}d\bm{x}\,w(\bm{x})$.
Say $(F,w)$ is \emph{Hardy--Littlewood--Hooley} (HLH) if, as $X\to \infty$, we have the asymptotic
\mathd{
N_{F,w}(X)
= \left(\ub{\sigma_{\infty,F,w}\mf{S}_F}_\textnormal{randomness}
+ o_{F,w;X\to\infty}(1)
+ \sum_{L\in C(\map{SSV})} \ub{\sigma_{\infty,L^\perp,w}}_\textnormal{structure}\right)\cdot X^3.
}
(Here and elsewhere, we let $o_{S; x\to a}(g)$ denote a quantity $f$ such that the statement ``for any real $\eps>0$, we have $\abs{f}\leq \eps g$ for all $x$ in a neighborhood $I = I(a, \eps, S)$ of $a$'' holds.
We will find this little-o notation occasionally convenient.)
}

\rmk{
Here $\mf{S}_F$ is the usual \emph{singular series},
and $\sigma_{\infty,F,w}, \sigma_{\infty,L^\perp,w}$ are \emph{real densities},
all given in technically convenient forms.\footnote{It is known that $\mf{S}_F$ converges absolutely, and that $\sigma_{\infty,F,w}, \sigma_{\infty,L^\perp,w}$ are finite and well-behaved (though a little care is needed for $\sigma_{\infty,F,w}$ ``at the origin'' if $(F,w)$ is not smooth).}
Also, in the setting above, $\Lambda = (\Lambda^\perp)^\perp$,
i.e.~$\Lambda = \set{\bm{x}\in\ZZ^6:\bm{\Lambda^\perp}\bm{x}=\bm{0}}$,
so $\sum_{\bm{x}\in\Lambda}w(\bm{x}/X) = \sigma_{\infty,L^\perp,w}X^3+O_{L,w,A}(X^{-A})$,
by Poisson summation over $\Lambda$
(or, at least morally, by the circle method applied to $\bm{\Lambda^\perp}\bm{x}=\bm{0}$).
%with the circle method, one might have to be a little careful about smoothing if one wants the X^{-A} error (and not just some fixed power saving), but it seems like it should be possible? (certainly by delta method if necessary).
In particular,
$\sigma_{\infty,L^\perp,w}$ does not depend on the choice of $\bm{\Lambda^\perp}$.
}

\rmk{
In this thesis, it would be OK to require a \emph{power-saving} error term for HLH in Definition~\ref{DEFN:HLH-asymptotic}.
Also, one could analyze \emph{unweighted} regions (see Appendix~\ref{CHAP:sharp-cutoffs}).
But at least in Chapter~\ref{CHAP:approx-variances}, a soft and smooth formulation of HLH has some benefits.
% cleaner / more conceptual, and allows for a more general asymptotic variance analysis theorem
}

Recall the discussion of (unconditional and conditional) upper bounds on $M_2(X)$, and the connection to the equation $x_1^3+\dots+x_6^3=0$, from Example~\ref{EX:6-cubes-zero-locus}.
In Chapter~\ref{CHAP:delta-method-review}, we will introduce the delta method, which connects $N_{F,w}(X)$ (for cubic forms $F$ in some generality, at least when $(F,w)$ is smooth)
to the local behavior of the intersections $F(\bm{x}) = \bm{c}\cdot\bm{x} = 0$ over $\FF_p,\ZZ_p,\RR,\dots$, as $\bm{c}\in \ZZ^6$ and $p$ vary;
and especially to certain associated Hasse--Weil $L$-functions.
Chapter~\ref{CHAP:using-hypotheses-on-average} discusses two conditional approaches to upper-bounding $M_2(X)$ (or $N_{x_1^3+\dots+x_6^3,w}(X)$),
one based on the delta method and a large-sieve hypothesis (a la Bombieri--Vinogradov), and the other based on a family of ternary quadratic forms;
see \S\ref{SEC:using-L-function-hypotheses-on-average} for the former, and \S\ref{SEC:using-quadratic-hypotheses-on-average} for the latter.
The former is probably more significant, but we include the latter for amusement.

The following remark may provide a holistic view of the rough landscape so far.
\rmk{
[A cartoon]
\label{RMK:rough-holistic-view-of-HLH-landscape}
Let $F_0(\bm{y})\defeq y_1^3+y_2^3+y_3^3$ first,
and $F(\bm{x})\defeq x_1^3+\dots+x_6^3$ and $\mcal{W}\defeq V_{\Aff^6}(F)_{/\ZZ}$ second.
Then
% when studying certain statistics,
the map $\Aff^1\gets\Aff^3,\;F_0(\bm{y})\mapsfrom\bm{y}$
and the diagram
\mathd{
\ub{
\Aff^3
\xrightarrow{F_0}
\Aff^1
% \gets
\xleftarrow{F_0}
\Aff^3\times_{F_0} \Aff^3
}_\textnormal{Cf.~Examples~\ref{EX:6-cubes-zero-locus} and~\ref{EX:sums-of-cubes}}
\cong \ub{
\mcal{W}
% \gets
\xleftarrow{\bm{x}}
\set{(\bm{x}, \bm{c})\in \mcal{W}\times \Aff^6: \bm{c}\cdot\bm{x}=0}
\xrightarrow{\bm{c}}
\Aff^6}_\textnormal{Cf.~\cite{kloosterman1926representation},
\dots,
\cites{duke1993bounds,heath1996new},
\dots}
}
% followed from left to right
vaguely depict
how when studying certain statistics (in $\ell^1$ and $\ell^2$),
one can ``reduce''
from the miserly \emph{individual} equations $F_0=a$
to the more generous \emph{family} of auxiliary \emph{hyperplane sections} $\mcal{W}_{\bm{c}}\defeq V_{\Aff^6}(F,\bm{c}\cdot\bm{x})_{/\ZZ}$ (over $\bm{c}\in \ZZ^6\setminus \set{\bm{0}}$)---although
the latter will appear ``adelically'' through $\mcal{W}_{\bm{c}}(\bd{A}_\ZZ)$,
rather than ``integrally'' through $\mcal{W}_{\bm{c}}(\ZZ)$.

(The cartoonish ``right-hand half''
is meant to represent
the delta method
for the affine zero locus $\mcal{W}\belongs \Aff^6$ of $F$;
see Chapter~\ref{CHAP:delta-method-review} for details.)
% Note Dec 20, 2020: O_V(1) (what I wrote originally) is wrong, even though it's called the "hyperplane bundle" (perhaps because its global sections are basically linear forms on A^6 restricted to C(V)). Rather note that the tautological line bundle O_V(-1) lies in A^6_V = V x A^6, and consider the orthogonal complement bundle O_V(-1)^\perp in (A^6_V)*, which sits in a s.e.s. O_V(-1)^\perp -> (A^6_V)* -> O_V(1)... which is basically a "bundle of vanishing hyperplanes over X", or a "bundle of hyperplane sections over (A^6)*.
}

\rmk{
% There is a much more basic idea of
One could try counting $\mcal{W}(\ZZ)$ using inclusion-exclusion on
% the hyperplane sections
$\mcal{W}_{\bm{c}}(\ZZ)$ over $\norm{\bm{c}}\ll X^{1/(6-1)}=X^{1/5}$ (in view of Siegel's lemma on linear equations).
As far as I can tell,
the delta method is distinct---even if it still involves
% the family of hyperplane sections
$\mcal{W}_{\bm{c}}(\bd{A}_\ZZ)$.
}

Chapter~\ref{CHAP:biases-over-finite-fields} discusses, among other things, a near dichotomy between randomness and structure for the point counts of projective cubic threefolds over finite fields (with applications to the aformentioned hyperplane sections $F(\bm{x}) = \bm{c}\cdot\bm{x} = 0$), and raises some further questions in this direction.

Chapter~\ref{CHAP:isolating-special-solutions} discusses how to extract the main terms of HLH in a natural way, for smooth pairs $(F,w)$ with $F$ diagonal in $6$ variables.
One of the main inputs is that for $L\in C(\map{SSV})$, the lattices $\Lambda\defeq L\cap \ZZ^6$ ``remain special'' in some sense (cf.~Chapter~\ref{CHAP:biases-over-finite-fields}) for hyperplane sections modulo $n$.

Chapter~\ref{CHAP:discriminating-pointwise-estimates} discusses \cite{wang2021_HLH_vs_RMT}'s new pointwise estimates for exponential sums and oscillatory integrals appearing in the delta method;
these estimates hold for various kinds of pairs $(F,w)$.
Over primes, the proofs involve Chapter~\ref{CHAP:biases-over-finite-fields};
over prime powers, a study of certain arithmetic fourfolds (relative threefolds over $\ZZ_p$);
and over the reals, a critical use of stationary phase beyond that of Hooley and Heath-Brown.
The estimates give, for instance (somewhat in the spirit of conjectures of Sarnak and Xue on ``naive Ramanujan'' failures, but in a different context), a power-saving bound (conditional on limited ranges of the Square-free Sieve Conjecture) on the frequency of certain ``square-root cancellation'' failures.

Chapter~\ref{CHAP:using-mean-value-L-function-predictions} discusses \cite{wang2021_HLH_vs_RMT}'s conditional proof that HLH holds for clean pairs $(F,w)$ with $F$ diagonal in $6$ variables (and thus, by Chapter~\ref{CHAP:approx-variances}, that the ``$100\%$ Hasse principle'' holds for sums of three cubes);
the proof is conditional on some standard number-theoretic conjectures---the main additions (relative to Hooley and Heath-Brown) being conjectures of Random Matrix Theory (RMT) and Square-free Sieve type.
% Paper~III is long because there are many distinct ``sources of $\epsilon$'' in the works of Hooley and Heath-Brown.
The proof builds on Chapters~\ref{CHAP:isolating-special-solutions} and~\ref{CHAP:discriminating-pointwise-estimates}.
By Chapter~\ref{CHAP:isolating-special-solutions}, it suffices to bound the contribution $\Sigma$ from smooth hyperplane sections in the delta method.
We first decompose $\Sigma$ adelically into discriminant-based pieces;
we then conditionally estimate some of these pieces via local calculations and Poisson summation (among other ingredients),
and conditionally bound other pieces via H\"{o}lder's inequality between good and bad moduli factors (among other ingredients).

\rmk{
Recall Observation~\ref{OBS:the-exceptional-set-for-5y_1^3+12y_2^3+9y_3^3-representations-is-nonempty}.
Our methods for $g_3$
would surely extend to conditionally show that $\mcal{E}(h)$ has density $0$.
These methods do not shed much further light on the true sizes of such sets (a much deeper question).
% But it is still nice to know that
But the statement ``$\mcal{E}(-)$ has density $0$'' certainly \emph{cannot} be improved to ``$\mcal{E}(-)=\emptyset$'' in general.
}

Chapter~\ref{CHAP:variations} discusses questions related to or inspired by the work above.

We will often refer to other papers for details, especially for certain proofs, but not at the expense of the overall story.
% On the other hand, in later versions of those papers, we may sometimes refer to the present thesis for supplementary material, when appropriate.
% present thesis (``Paper~T'') for supplementary material, when appropriate; see \url{https://wangyangvictor.github.io/thesis_links} for details.
% \todo[inline]{The list below should perhaps be moved into a living document so we can keep this user-friendly as versions change.}
% \begin{enumerate}
%     \item For [I, \S1, ``Regarding the question of automorphy\dots''], see
%     \item For [I, \S1, ``The philosophy behind Theorem~1.23 should also\dots''], see
%     \item For [I, Remark~1.22]
%     \item For [I, Remark~2.2]
%     \item For [I, proof of Lemma~2.12]:
%     \item For [I, Remark~B.3]:
% \end{enumerate}

\chapter{Approximate variances}
\label{CHAP:approx-variances}

\section{Introduction}
\label{SEC:intro-to-statistical-analysis}

\defn{
Given $a\in\ZZ$,
let $F_a(\bm{y})=F_a(y_1,y_2,y_3)\defeq y_1^3+y_2^3+y_3^3 - a$, and
let $r_3(a)\defeq\#\set{\bm{y}\in\ZZ_{\geq0}^3:F_a(\bm{y})=0}$,
% denote the number of representations of a given integer $a\in\ZZ$ as a sum of three \emph{nonnegative} cubes,
so that $r_3(a)\leq N_{F_a}(a^{1/3})$.
}

Using the convenient structure $F_a = F_0-a$,
we will first estimate $r_3(a)$ ``coarsely'' in $\ell^1,\ell^2$ over $a\leq B$,
as $B\to\infty$---following a classical strategy (cf.~Remark~\ref{RMK:rough-holistic-view-of-HLH-landscape}).
Later, to obtain a more precise result (Theorem~\ref{THM:enough-HLH-implies-100pct-Hasse-principle-for-3Z^odot3}),
we will choose ``nice'' weights $\nu\maps\RR^3\to\RR$ (cf.~\cite{hooley2016representation})
and bound $N_{F_a,\nu}(X)$ in ``approximate variance'' over $a\ll X^3$,
as $X\to\infty$---essentially following \cites{ghosh2017integral,diaconu2019admissible},
up to smoothness.

Recall that \cite{ghosh2017integral} works with the equations $x^2+y^2+z^2-xyz = a$, which are ``critical'' just like the equations $x^3+y^3+z^3 = a$ considered here and in \cite{diaconu2019admissible}.
It is worth noting that there are significant technical differences between \cite{ghosh2017integral} and \cite{diaconu2019admissible} (even though the overall arguments are formally similar), such as the following:
\begin{enumerate}[(1)]
    \item the nature of ``exceptional'' parametric solutions differs between the two;
    \item delicate issues involving binary quadratic forms arise in \cite{ghosh2017integral}, but not in \cite{diaconu2019admissible};
    and
    \item \cite{ghosh2017integral} engineers close-to-classical regions (for a family of quadratic equations), while \cite{diaconu2019admissible} engineers far-from-classical regions (for a single cubic equation).
\end{enumerate}

\subsection{A moment framework}
% Analysis by moments
\label{SEC:a-moment-framework}

Recall the well-known \emph{first} moment
\mathd{
\sum_{a\leq B}r_3(a)
= \#\set{\bm{y}\in\ZZ_{\geq0}^3:F_0(\bm{y})\leq B}
\propto B+o_{B\to\infty}(B)
\qquad\textnormal{as $B\to\infty$},
}
proven by writing $F_a=F_0-a$,
% expanding $r_3(a)$ vertically along fibers of $F_0\maps\ZZ^3\to\ZZ$,
expanding $r_3(a)$ as a ``vertical sum'' along a fiber of $F_0\maps\ZZ^3\to\ZZ$,
and then approximating the ``total space'' by a continuous volume.

In particular, $\EE_{a\leq B}[r_3(a)]\asymp 1$ holds for all $B>0$.
Now recall that \cite{hardy1925some} formulated the $\ell^\infty$ hypothesis $r_3(a)\ll_\eps a^\eps$ for $a>0$ (\emph{Hypothesis $K$});
but \cite{mahler1936note} showed,
via the identity $(9u^4)^3 + (3uv^3-9u^4)^3 + (v^4-9u^3v)^3 = v^{12}$,
that $r_3(a)\gg a^{1/12}$ holds for twelfth powers $a=v^{12}>0$.

\rmk{
Mahler's identity can be viewed as a ``clever specialization'' of a ``$\QQ$-unirational parameterization'' of $x^3+y^3+z^3=w^3$.
% \footnote{All cubic surfaces over $\QQ$ are rational over $\ol{\QQ}$, but the situation over $\QQ$ is subtler.}
In fact,
Mahler's identity can be recovered from \cite{elkiescomplete},
which parameterizes \emph{all} $\QQ$-points on $x^3+y^3+z^3=w^3$.
% Elkies' formulas define *birational inverses* P^3 ---> V and V ---> P^3 *over Q*, so if we have a rational map P^1 ---> V (as in Mahler) then this must be induced by a rational map P^1 ---> P^3
}

By Mahler's construction,
$r_3(a)$ can be very large for some \emph{individual} $a$'s,
but we can still hope for reasonable statistical behavior as long as $r_3(a)$ is
not ``too large too often''---a notion most readily formalized by taking higher moments of $r_3(a)$.

It seems difficult at present to rigorously analyze the third moment or higher.
(The $k$th moment is connected to the equation $x_1^3+y_1^3+z_1^3 = x_2^3+y_2^3+z_2^3 = \dots = x_k^3+y_k^3+z_k^3$, which can be viewed as a system of $k-1$ Diophantine equations in $3k$ variables.
In light of Mahler's ``special'' $a$'s,
even \emph{formulating reasonable conjectures} for arbitrarily high moments seems difficult,
but see \cite{deshouillers2006density} and \cite{diaconu2019admissible}*{\S4} for plausible random models of sets like $\set{x^3+y^3+z^3: (x,y,z)\in \ZZ_{\geq 0}^3}$.)
But the \emph{second} moment of $r_3(a)$ forms,
by double counting,
a relatively simple ``Diophantine sandwich''
\mathds{
\sum_{a\leq B}r_3(a)^2
\leq N_F(B^{1/3})
&= \int_{\RR/\ZZ}d\theta\,\abs{T(\theta)}^6 \\
&\ll \int_{\RR/\ZZ}d\theta\,(\abs{T_{>0}(\theta)}^6+\abs{T_{\leq0}(\theta)}^6)
\ll \sum_{a\leq 3B}r_3(a)^2
}
(unconditionally),
where $F\defeq x_1^3+\dots+x_6^3$ and $T_S(\theta)\defeq\sum_{\abs{x}\leq B^{1/3}}e(\theta x^3)\cdot\bm{1}_\textnormal{$x$ satisfies $S$}$.
And indeed, \cite{hardy1925some} really only applied the $\ell^\infty$ Hypothesis $K$ through
the statement
\mathd{
\sum_{a\leq B}r_3(a)^2
\ll_\eps B^{1+\eps}
\qquad\textnormal{as $B\to\infty$}
}
(known conditionally, as discussed in Example~\ref{EX:6-cubes-zero-locus}),
a weaker $\ell^2$ hypothesis (termed \emph{Hypothesis $K^\ast$} by \cite{hooley_greaves_harman_huxley_1997}) equivalent to ``$N_F(B^{1/3})\ll_\eps B^{1+\eps}$ as $B\to\infty$''.

Now we can combine the $\ell^1$ expectation $\EE_{a\leq B}[r_3(a)]\asymp 1$
with the ``$\ell^2$ data'' $N_F$,
to get the following classical result (essentially already mentioned in Example~\ref{EX:6-cubes-zero-locus}):
\obs{
[Second moment method]
\label{OBS:3Z^odot3-lower-bound-by-Cauchy}
$\#\set{a\leq B: r_3(a)\neq 0}\gg B^2/N_F(B^{1/3})$.
}

\pf{
By Cauchy, $\sum_{a\leq B}
r_3(a)^2\geq \left(\sum_{a\leq B}
r_3(a)\right)^2/\#\set{a\leq B: r_3(a)\neq 0}$.
(If there are \emph{too few} $a\in\ZZ$ with $r_3(a)\neq 0$, then the second moment is \emph{forced to be large}.
This idea comes in many forms, e.g.~the probabilistic Chung--\Erdos inequality.)
% \href{https://en.wikipedia.org/wiki/Chung\%E2\%80\%93Erd\%C5\%91s_inequality}{Chung--\Erdos inequality}.)
}

In particular, if $N_F(B^{1/3})\ll B$ were known,
then Observation~\ref{OBS:3Z^odot3-lower-bound-by-Cauchy}
would imply that $\set{a\in\ZZ: r_3(a)\neq 0}=F_0(\ZZ_{\geq0}^3)$ has \emph{positive lower density}.
Or if more precise estimates for $N_{F,w}(X)$ were known in sufficient generality,
then it would follow that $F_0(\ZZ^3)$ has density $7/9$ in $\ZZ$,
essentially by \cite{diaconu2019admissible}---which we now introduce.

\subsection{The need for increasingly lopsided regions}

To prove,
\emph{under our methods} (based on \cite{diaconu2019admissible}),
that $F_0=a$ ``almost always'' satisfies the Hasse principle,
it will not suffice to statistically analyze
the sequence $a\mapsto N_{F_a,\nu}(X)$ (over a range of the form $a\ll_{\nu}X^3$)
for only a \emph{single fixed} weight $\nu$.
The following remark more or less explains why.
\rmk{
The upper density of $\set{a\in\ZZ_{\geq0}:r_3(a)\neq0}$ is at most $\Gamma(4/3)^3/6\approx 0.12$ \cite{davenport1939waring}.
In fact,
say for each $a\in\ZZ$ we restrict to $\bm{y}\in a^{1/3}\cdot\Omega$,
with $\Omega\belongs\RR^3$ \emph{fixed} and bounded.
Then as $\abs{a}\to\infty$,
the equation $F_a(\bm{y})=0$ \emph{fails Hasse} over a relative density $0.99$ subset of $a\in\ZZ$ lying in \emph{some} arithmetic progression depending on $\Omega$ \cite{diaconu2019admissible}*{\S1}.
}

Thus we will instead let $\nu$ \emph{vary}.
Specifically,
certain \emph{increasingly skewed} regions
(cf.~the cuspidal regions in \cite{diaconu2019admissible}*{p.~26, Remark with pictures})
can help,
as we now explain.
\obs{
Fix an ``inhomogeneity parameter'' $A\geq 1$.
If $X\gg_A 1$ is sufficiently large,
% then for at least $\Omega(X^3)$ triples $\bm{y}\in \ZZ^3$ with $AX\leq \abs{y_1},\abs{y_2},\abs{y_3}\leq 2AX$,
% we have $\abs{F_0(\bm{y})}\leq X^3$.
then the number of triples $\bm{y}\in \ZZ^3$ with $AX\leq \abs{y_1},\abs{y_2},\abs{y_3}\leq 2AX$ and $\abs{F_0(\bm{y})}\leq X^3$
is $\Theta(X^3)$.
}

\pf{
[Proof sketch]
There are $\asymp (AX)^3$ triples $\bm{y}\in \ZZ^3$ with $\abs{y_1},\abs{y_2},\abs{y_3}\in [AX,2AX]$.
The event $\abs{F_0(\bm{y})}\leq X^3$ occurs with probability $\asymp X^3/(AX)^3=1/A^3$ among these triples,
provided that $AX\gg A^3$ is sufficiently large and $0\in F_0((1,2), (1,2), (-2,-1))$.
% Intuitively: AX >> (AX)^3/X^3 means varying $y_1$ alone (keeping $y_2,y_3$ fixed) provides enough points to fill all length-$X^3$ ``value buckets'' in an interval of size $\ll (AX)^3$; this is equivalent to the derivative $3y_1^2\asymp (AX)^2$ being sufficiently smaller than the lengths $X^3$ we're trying to fill.
}

Along these lines,
Heath-Brown conjectured the following:
\cnj{
[\cite{heath1992density}*{p.~623}]
\label{CNJ:heath-brown-conjecture-infinitude-of-representations-as-3Z^odot3}
If $a\not\equiv\pm4\bmod{9}$,
then $\lim_{X\to\infty}N_{F_a}(X)=\infty$.
}

Conjecture~\ref{CNJ:heath-brown-conjecture-infinitude-of-representations-as-3Z^odot3} for \emph{individual} $a\not\equiv\pm4\bmod{9}$ might be very hard (if true),
even conditionally,
so we will content ourselves with a (conditional) \emph{average} study,
% (still hard, but possibly more tractable),
which we now set up.
\subsection{A sketch of an approximate variance framework}
% (Basics of / An overview of / Introducing / A sketch of an) An approximate variance framework
% Using approximate variances
% (Setting up / defining / idea of) Approximate variances, and an application
% Analysis by an approximate variance

Given $\nu,a,X$,
one can define certain densities $\sigma_{p,F_a},\sigma_{\infty,F_a,\nu}(X)$.
(For details,
see \S\ref{SEC:details-of-variance-analysis},
which begins shortly below.)
For $a\in\ZZ$,
informally write $\mathfrak{S}_{F_a}
\defeq\prod_{p<\infty}\sigma_{p,F_a}$,
% (when the product converges),
for expository purposes.
Now consider the naive \emph{Hardy--Littlewood prediction} $\mathfrak{S}_{F_a}\cdot\sigma_{\infty,F_a,\nu}(X)$ for $N_{F_a,\nu}(X)$.
Smaller moduli should have a greater effect in $\mf{S}_{F_a}$;
furthermore,
$\mf{S}_{F_a}$ itself---\emph{as is}---can be subtle (involving the behavior of $L$-functions at $s=1$; cf.~\cite{duke1990representation}*{Theorem~3 and its proof} and \cite{heath1996new}*{Theorems~6--7}).
% the full infinite product $\mf{S}_{F_a}$ itself? as is?
% Note: Behavior at $s=1$ arises rather than $s=1/2$, since we are dealing with surfaces (as opposed to curves with nontrivial cohomology, as in the $L$-function side of BSD).
So in the following definition,
we work with a ``restricted'' version of $\mathfrak{S}_{F_a}\cdot\sigma_{\infty,F_a,\nu}(X)$.
\defn{[Cf.~\cite{diaconu2019admissible}*{\S3}]
\label{DEFN:M-approximate-variance-for-3Z^odot3}

Fix $\nu\maps\RR^3\to\RR$
and let $w(\tilde{\bm{y}},\tilde{\bm{z}})\defeq\nu(\tilde{\bm{y}})\nu(-\tilde{\bm{z}})$.
Given $M\geq 1$, let $K=K(M)\defeq\prod_{p\leq M}p^{\floor{\log_p{M}}}$ and
$s_{F_a}(K)\defeq K^{-2}\cdot\#\set{\bm{y}\in (\ZZ/K)^3: F_a(\bm{y})=0}$.
Then define the \emph{$M$-approximate} (``finite-precision'') \emph{variance}
\mathd{
\map{Var}(X,M)
\defeq \sum_{a\in\ZZ}[N_{F_a,\nu}(X) - s_{F_a}(K)\sigma_{\infty,F_a,\nu}(X)]^2
\eqdef \Sigma_1-2\Sigma_2+\Sigma_3.
}
}

The most interesting sum among $\Sigma_1,\Sigma_2,\Sigma_3$ is
$\Sigma_1\defeq\sum_{a\in\ZZ}N_{F_a,\nu}(X)^2$, which can be rewritten as $N_{x_1^3+\dots+x_6^3,w}(X)$.
In fact, the main ideas of \cite{diaconu2019admissible}*{\S\S2--3} lead to the following result:
\thm{
[Cf.~\cite{diaconu2019admissible}*{Theorem~3.3}]
\label{THM:enough-HLH-implies-100pct-Hasse-principle-for-3Z^odot3}
Suppose $(x_1^3+\dots+x_6^3,w)$ is HLH (in the sense of Definition~\ref{DEFN:HLH-asymptotic})
for \emph{every fixed} choice of $w\in C^\infty_c(\RR^6)$
with $(x_1^3+\dots+x_6^3,w)$ clean.
Then asymptotically $100\%$ of integers $a\not\equiv\pm4\bmod{9}$ are sums of three cubes.
}

For the proof---differing from \cite{diaconu2019admissible} only in technical aspects---see \S\ref{SEC:details-of-variance-analysis}.

\section{Details}
\label{SEC:details-of-variance-analysis}
% Approximate variance analysis of/for sums of 3 cubes (on average, asymptotically, almost always...)
%Application (of main theorem / first/second main results / HLH) to sums of three cubes on average

% \S\ref{SEC:asymptotic-variance-analysis-for-3Z^odot3} explains the (purely technical) modifications of \cite{diaconu2019admissible} needed to prove Theorem~\ref{THM:enough-HLH-implies-100pct-Hasse-principle-for-3Z^odot3}.

% Below,
% we will prove Theorem~\ref{THM:enough-HLH-implies-100pct-Hasse-principle-for-3Z^odot3},
% following the main ideas of \cite{diaconu2019admissible}*{\S\S2--3}.

\subsection{Defining the local densities of individual fibers}

Recall the notation $F_a(\bm{y})\defeq F_0(\bm{y})-a$,
where $F_0(\bm{y})\defeq y_1^3+y_2^3+y_3^3$.
We first define certain non-archimedean densities.
\defn{
For $a\in\ZZ_p$,
let $\sigma_{p,F_a}
\defeq \lim_{l\to\infty}
(p^{-2l}\cdot\#\set{\bm{y}\in (\ZZ/p^l)^3: F_a(\bm{y})=0})$.
}

Next,
we define certain real densities
(analogously to the $p$-adic densities $\sigma_{p,F_a}$),
by ``$\eps$-thickening'' parallel to the $0$-level set
(i.e.~the fiber over $0$)
of the map $F_a\maps\RR^3\to\RR$.
% eps-thickening chosen to be parallel to level sets (Sarnak course: real densities are determined by a *choice of level sets*!)
\defn{
Fix $\nu\in C^\infty_c(\RR^3)$ with $(F_0,\nu)$ smooth (in the sense of Definition~\ref{DEFN:support-smooth-clean}).
Now fix $X\in\RR_{>0}$.
Then for $(\bm{y},a)\in\RR^3\times\RR$,
write $\tilde{\bm{y}}\defeq\bm{y}/X$ and $\tilde{a}\defeq a/X^3$.
Also,
for all $a\in\RR$,
let
\mathd{
\sigma_{\infty,F_a,\nu}(X)
\defeq \lim_{\eps\to0}\,(2\eps)^{-1}
\int_{\abs{F_a(\bm{y})/X^3}\leq\eps}d(\bm{y}/X)\,\nu(\bm{y}/X).
}
}

\obs{
Here $\sigma_{\infty,F_a,\nu}(X)
= \sigma_{\infty,F_{\tilde{a}},\nu}(1)$.
(Indeed, by definition, we have
$\sigma_{\infty,F_a,\nu}(X)
= \lim_{\eps\to0}\,(2\eps)^{-1}
\int_{\abs{F_0(\tilde{\bm{y}})-\tilde{a}}\leq\eps}d\tilde{\bm{y}}\,\nu(\tilde{\bm{y}})
= \sigma_{\infty,F_{\tilde{a}},\nu}(1)$.)
}

\rmk{
In particular, for $a\neq 0$ at least,
$\sigma_{\infty,F_a,\nu}(a^{1/3})
= \sigma_{\infty,F_1,\nu}(1)$
is \emph{constant}.
% % Nov 30, 2020 pdf comment: If we normalize \nu by a^{1/3}, rather than by X (as in cubic seminar notes v2 part 1), then the real densities may be nicer? But in other aspects, the X normalization is simpler to think about.
% % \todo[inline]{In other words,
% % if we scale the region with $a^{1/3}$,
% % rather than $X$...}
% Also,
% $\sigma_{\infty,F_1,\nu}(1)$ is explicitly computable for many common choices of $\nu$.
Essentially for this reason, perhaps,
many references do not seem to treat real densities very thoroughly.
}

% \rmk{
% Let $f(a)\defeq\int_{u_1^3+u_2^3+u_3^3\leq a}du_1\,du_2\,du_3\,\bm{1}_{\bm{u}\geq\bm{0}}=(1/3)^3\Gamma(1/3)^3a$.
% Let $\nu_\star(\bm{u})\defeq\bm{1}_{\bm{u}\geq\bm{0}}\cdot\bm{1}_{F_0(\bm{u})\leq1}$, so $f(a) = \int_{\RR^3}d\bm{u}\,\nu_\star(\bm{u})\bm{1}_{F_0(\bm{u})\leq a} = \int_{b\leq a}db\,\sigma_{\infty,F_b,\nu_\star}(1)$ and $\sigma_{\infty,F_a,\nu_\star}(1) = f'(a) = (1/3)^3\Gamma(1/3)^3$ is constant.
% }

Before proceeding,
we make two conceptual remarks on real densities,
for completeness.

\rmk{
The function $\tilde{a}\mapsto\sigma_{\infty,F_a,\nu}(X)$ is supported on a bounded range $\tilde{a}\ll_{\nu}1$,
on which the $\eps$-limit converges \emph{uniformly} at a rate depending only on $F_0,\nu$.
Generally,
if $\alpha$ is ``nice'' and $\alpha_\eps\defeq\alpha(t/\eps)$
(e.g.~$\alpha_\eps(t) = \bm{1}_{\abs{t}\leq\eps}$ above),
then
\mathd{
\frac{\int_{\RR^3}d\tilde{\bm{y}}\,\nu(\tilde{\bm{y}})\alpha_\eps(F_0(\tilde{\bm{y}})-\tilde{a})}{\int_{\RR}dt\,\alpha_\eps(t)}
= \sigma_{\infty,F_a,\nu}(X)
+ O_{F_0,\nu,\alpha}(\eps).
}
}

\rmk{
Here $d\tilde{a}\;\sigma_{\infty,F_a,\nu}(X)
= (F_0)_\ast(d\tilde{\bm{y}}\;\nu(\tilde{\bm{y}}))$
as a pushforward measure,
% \href{https://en.wikipedia.org/wiki/Pushforward_measure}{pushforward measure}
% (integrating along fibers of $F_0\maps\RR^3\to\RR$).
%https://en.wikipedia.org/wiki/Integration_along_fibers
so that
$\int_{\RR}d\tilde{a}\,\sigma_{\infty,F_a,\nu}(X)g(\tilde{a})
= \int_{\RR^3}d\tilde{\bm{y}}\,\nu(\tilde{\bm{y}})g(F_0(\tilde{\bm{y}}))$
% (``expanding $d\tilde{a}\;\sigma_{\infty,F_a,\nu}(X)$''
holds for all ``nice'' $g\maps\RR\to\RR$.
}

For convenience
(when working with $\sigma_{\infty,F_a,\nu}(X)$),
we now observe the following:
\obs{
\label{OBS:surface-integral-representation-of-real-density}
Define $\pi_1\maps\RR^3\to\RR$ by $\bm{y}\mapsto y_1$.
Fix $\nu\in C^\infty_c(\RR^3)$ such that $0\notin\pi_1(\Supp\nu)$.
Now fix $(a,X)\in\RR\times\RR_{>0}$.
Then a change of variables from $\tilde{y}_1$ to
% $t\defeq F_0(\tilde{\bm{y}})$
$F_0\defeq F_0(\tilde{\bm{y}})$ proves
\mathd{
\sigma_{\infty,F_a,\nu}(X)
= \int_{\RR^2}d\tilde{y}_2\,d\tilde{y}_3\,\nu(\tilde{\bm{y}})
\cdot\ub{\abs{\partial\tilde{y}_1/\partial F_0}_{F_0=\tilde{a}}}_\textnormal{local $\tilde{y}_1$-density}
%resistance to change
%\textnormal{stickiness of surface}
% \textnormal{attraction rate}
% \textnormal{infinitesimal thickness}
= \int_{\RR^2}d\tilde{y}_2\,d\tilde{y}_3\,\nu(\tilde{\bm{y}})
\cdot\abs{\ub{\partial F_0/\partial\tilde{y}_1}_\textnormal{escape rate}}^{-1}_{F_0=\tilde{a}},
}
%\partial t/\partial\tilde{y}_1 = \partial_1F_0\vert_{\tilde{\bm{y}}} = 3\tilde{y}_1^2
where $\partial F_0/\partial\tilde{y}_1 = 3\tilde{y}_1^2\gg_{\nu}1$ over the support of the integrand.
% for all $\tilde{\bm{y}}\in\Supp\nu$ with $F_0(\tilde{\bm{y}})=\tilde{a}$.
}

\pf{
Cf.~\cite{heath1996new}*{proof of Lemma~11}.
}

At least in the absence of better surface coordinates,
the earlier ``$\eps$-thickening'' still provides greater intuition,
while the surface integral allows for effortless rigor.

The following technical bound will come up later,
when integrating by parts.
\prop{
\label{PROP:uniform-bound-on-derivatives-of-densities}
Fix $\nu\in C^\infty_c(\RR^3)$ with $(F_0,\nu)$ smooth.
Then for integers $k\geq 0$,
we have $\partial_{\tilde{a}}^k[\sigma_{\infty,F_a,\nu}(X)]\ll_{k,\nu} 1$, uniformly over $(a, X)\in \RR\times \RR_{>0}$.
}

\pf{
[Proof when $0\notin\pi_1(\Supp\nu)$]
Consider the surface integral representation of $\sigma_{\infty,F_a,\nu}(X)$ from Observation~\ref{OBS:surface-integral-representation-of-real-density}.
The integrand vanishes unless $\tilde{a},\tilde{y}_2,\tilde{y}_3\ll_{\nu}1$.
Fix $\tilde{y}_2,\tilde{y}_3\ll_{\nu}1$,
and let $\tilde{y}_1$ vary with $\tilde{a}$ according to $F_0(\tilde{\bm{y}})=\tilde{a}$.
Then $\partial_{\tilde{a}}[\tilde{y}_1] = (3\tilde{y}_1^2)^{-1}\ll_{\nu}1$.
Now repeatedly apply $\partial_{\tilde{a}}$ to the integrand
(using Leibniz and the chain rule).
}

\pf{[Proof in general]
Use a suitable partition of unity.
}

% \subsection{Details of approximate variance analysis}
\subsection{Interpreting the approximate variance}
% Interpreting the approximate variance roughly as a diagonal

Let $\bm{y},\bm{z}$ denote $3$-vectors,
and define $x_i\defeq y_i$ for $i\in[3]$ and $x_{i+3}\defeq z_i$ for $i\in[3]$.
Let $\tilde{\bm{y}}\defeq \bm{y}/X$,
etc.
Now fix $\nu\in C^\infty_c(\RR^3)$ with $(F_0,\nu)$ smooth.
Recall,
from Definition~\ref{DEFN:M-approximate-variance-for-3Z^odot3},
the definitions of $w,K(M),s_{F_a}(K),
\map{Var}(X,M),\Sigma_1,\Sigma_2,\Sigma_3$.

By double counting (and the ``symmetric'' $\nu$-factorization of $w$),
the $\ell^2$ moment $\Sigma_1\defeq \sum_{a\in\ZZ}N_{F_a,\nu}(X)^2$ equals $N_{F,w}(X)$, where $F\defeq x_1^3+\dots+x_6^3$.

Also, by analogy with standard probability theory,
we expect an $\ell^1$ calculation to show that $\Sigma_2\approx\Sigma_3$.
However,
a rigorous proof
(of such a fact)
takes some nontrivial work,
since $s_{F_a}(K)\sigma_{\infty,F_a,\nu}(X)$ \emph{varies} with $a$.
To begin, write
\mathd{
\Sigma_3
\defeq \sum_{a\in\ZZ}[s_{F_a}(K)\sigma_{\infty,F_a,\nu}(X)]^2
= \sum_{b\bmod{K}}s_{F_b}(K)^2
\sum_{a\equiv b\bmod{K}}\sigma_{\infty,F_a,\nu}(X)^2
}
by collecting along fibers of $\ZZ\to\ZZ/K$,
and write
\mathds{
\Sigma_2
&\defeq \sum_{a\in\ZZ}N_{F_a,\nu}(X)s_{F_a}(K)\sigma_{\infty,F_a,\nu}(X) \\
&= \sum_{\bm{z}\in\ZZ^3}\nu(\bm{z}/X)s_{F_{F_0(\bm{z})}}(K)\sigma_{\infty,F_{F_0(\bm{z})},\nu}(X) \\
% &= \sum_{b\bmod{K}}s_{F_b}(K)
% \sum_{a\equiv b\bmod{K}}N_{F_a,\nu}(X)\sigma_{\infty,F_a,\nu}(X) \\
&= \sum_{\bm{d}\bmod{K}}s_{F_{F_0(\bm{d})}}(K)
\sum_{\bm{z}\equiv\bm{d}\bmod{K}}\nu(\bm{z}/X)\sigma_{\infty,F_{F_0(\bm{z})},\nu}(X)
}
by expanding $N_{F_a,\nu}(X)$ along fibers of $F_0\maps\ZZ^3\to\ZZ$
and collecting along $\ZZ^3\to(\ZZ/K)^3$.
% $\bm{d}\bmod{K}$ with $F_0(\bm{d})\equiv b\bmod{K}$.

To simplify $\Sigma_2,\Sigma_3$ further,
we will use Poisson summation,
as well as the following observation:
\obs{
The ``pure $L^2$ moment''
$\int_{a\in\RR}d\tilde{a}\,\sigma_{\infty,F_a,\nu}(X)^2$
and the ``mixed $L^1$ moment''
$\int_{\tilde{\bm{z}}\in\RR^3}d\tilde{\bm{z}}\,\nu(\tilde{\bm{z}})\sigma_{\infty,F_{F_0(\bm{z})},\nu}(X)$
both simplify to $\sigma_{\infty,F,w}$.
}

\pf{
[Proof when $0\notin\pi_1(\Supp\nu)$]
First,
$F_a\defeq F_0-a$,
so $\int_{a\in\RR}d\tilde{a}\,\sigma_{\infty,F_a,\nu}(X)^2$ expands (via Observation~\ref{OBS:surface-integral-representation-of-real-density}) to
\mathd{
\int_{\RR^4}d\tilde{y}_2\,\cdots\,d\tilde{z}_3
\int_{\tilde{y}_1\in\RR}dF_0(\tilde{\bm{y}})\,
\frac{\nu(\tilde{\bm{y}})\nu(\tilde{\bm{z}})}{\partial_1F_0\vert_{\tilde{\bm{y}}}}
(\partial_1F_0\vert_{\tilde{\bm{z}}})^{-1}\vert_{F_0(\tilde{\bm{y}})=F_0(\tilde{\bm{z}})},
}
which simplifies to $\int_{\RR^5}d\tilde{y}_2\,\cdots\,d\tilde{y}_1\,
w(\tilde{\bm{x}})\cdot(\partial_1F_0\vert_{\tilde{\bm{z}}})^{-1}\vert_{F(\tilde{\bm{x}})=0} = \sigma_{\infty,F,w}$.

Second,
$F_{F_0(\bm{z})} = F_0(\bm{y})-F_0(\bm{z})$,
so by Observation~\ref{OBS:surface-integral-representation-of-real-density},
\mathd{
\int_{\tilde{\bm{z}}\in\RR^3}d\tilde{\bm{z}}\,\nu(\tilde{\bm{z}})\sigma_{\infty,F_{F_0(\bm{z})},\nu}(X)
= \int_{\RR^3\times\RR^2}d\tilde{\bm{z}}\,d\tilde{y}_2\,d\tilde{y}_3\,\nu(\tilde{\bm{y}})\nu(\tilde{\bm{z}})
\cdot(\partial_1F_0\vert_{\tilde{\bm{y}}})^{-1}\vert_{F_0(\tilde{\bm{y}})=F_0(\tilde{\bm{z}})},
}
which again simplifies to $\sigma_{\infty,F,w}$.
}

\pf{[Proof in general]
Argue in terms of $\eps$-thickenings.
Alternatively,
generalize to a ``bilinear'' statement
(involving \emph{two} weights $\nu_1,\nu_2$,
rather than just one);
then reduce the bilinear statement to a surface integral computation (based on a general version of Observation~\ref{OBS:surface-integral-representation-of-real-density}),
after taking suitable partitions of unity.
}

\prop{
[Cf.~\cite{diaconu2019admissible}*{proof of Lemma~3.1}]
Uniformly over $X,K$ and $b,\bm{d}\bmod{K}$ with $X\geq K\geq 1$,
we have
\mathd{
\sum_{a\equiv b\bmod{K}}\sigma_{\infty,F_a,\nu}(X)^2
= K^{-1}X^3\sigma_{\infty,F,w}
+ O_{j,\nu}((X^3/K)^{-j})
}
and
\mathd{
\sum_{\bm{z}\equiv\bm{d}\bmod{K}}\nu(\bm{z}/X)
\sigma_{\infty,F_{F_0(\bm{z})},\nu}(X)
= K^{-3}X^3\sigma_{\infty,F,w}
+ O_{j,\nu}((X/K)^{-j}).
}
}

\pf{
[Proof sketch]
By Poisson summation
% \footnote{Euler--Maclaurin would also work.}
over $K\cdot\ZZ$ and $K\cdot\ZZ^3$,
the sums are approximately
\mathd{
K^{-1}\int_{a\in\RR}da\,\sigma_{\infty,F_a,\nu}(X)^2
= K^{-1}X^3\sigma_{\infty,F,w}
}
and
\mathd{
K^{-3}\int_{\bm{z}\in\RR^3}d\bm{z}\,\nu(\bm{z}/X)\sigma_{\infty,F_{F_0(\bm{z})},\nu}(X)
% = K^{-3}X^3\int_{\tilde{\bm{z}}\in\RR^3}d\tilde{\bm{z}}\,\nu(\tilde{\bm{z}})\sigma_{\infty,F_0(\bm{y})-F_0(\bm{z}),\nu}(X)
= K^{-3}X^3\sigma_{\infty,F,w},
}
respectively
(where we have simplified the integrals using the previous observation),
up to ``errors''
(i.e.~``off-center contributions'')
of $\ll_{\nu,j}K^{-1}X^3(X^3/K)^{-j}$ and $\ll_{\nu,j}K^{-3}X^3(X/K)^{-j}$,
respectively.
% (Cf.~the fact that the \emph{smoothed} Gauss circle problem is trivial.)
(For proof,
bound the first ``off-center contribution'' \emph{absolutely} by
\mathd{
\sum_{c\neq0}K^{-1}\left\lvert{
\int_{a\in\RR}da\,\sigma_{\infty,F_a,\nu}(X)^2e(-c\cdot a/K)
}\right\rvert;
}
then plug in $a = X^3\tilde{a}$,
repeatedly integrate by parts in $\tilde{a}$,
and invoke Proposition~\ref{PROP:uniform-bound-on-derivatives-of-densities}.
The second ``off-center contribution'' is similar.)
}

It follows from the previous proposition
(and the trivial bound $\abs{s_{F_b}(K)}\leq K$)
that
\mathd{
\Sigma_3
= \mcal{S}(K)\cdot\sigma_{\infty,F,w}X^3
% +\cdots,
+ K^3\cdot O_{j,\nu}((X^3/K)^{-j}),
}
where $\mcal{S}(K)\defeq K^{-1}\sum_{b\bmod{K}}s_{F_b}(K)^2 = K^{-5}\cdot\#\set{\bm{x}\in (\ZZ/K)^6: F(\bm{x})=0}$.

Similarly,
\mathd{
K^{-3}\sum_{\bm{d}\bmod{K}}s_{F_{F_0(\bm{d})}}(K) = K^{-3}\sum_{b\bmod{K}}[K^2\cdot s_{F_b}(K)]\cdot s_{F_b}(K)
= \mcal{S}(K)
}
by collecting along fibers of $F_0\maps(\ZZ/K)^3\to\ZZ/K$,
% fibers of $F_0\bmod{K}$ gives
so
\mathd{
\Sigma_2
= \mcal{S}(K)\cdot\sigma_{\infty,F,w}X^3
% +\cdots.
+ K^4\cdot O_{j,\nu}((X/K)^{-j}).
}

\lem{[Cf.~\cite{diaconu2019admissible}*{Lemma~3.1}]
Uniformly over $X,M\geq1$ with $X\geq K(M)\geq 1$,
we have $\mcal{S}(K) = \mf{S}_F+O_\eps(M^{-2/3+\eps})$ and
\mathd{
\map{Var}(X,M)
= N_{F,w}(X)-\mcal{S}(K)\cdot\sigma_{\infty,F,w}X^3
% +\cdots,
+ K^4\cdot O_{j,\nu}((X/K)^{-j}).
}
}

\pf{
For the second part,
use $\map{Var}(X,M)=\Sigma_1-2\Sigma_2+\Sigma_3$.
% As for $\mcal{S}(K)\to\mf{S}_F$: proof omitted.
% For a weaker bound of O(1/log{M}): see Diaconu p. 32 (36 in pdf), or Hooley 1986, p. 116, (19) and the display preceding Section 4
For the first part,
note that $p^{6v}\mcal{S}(p^v)
= \sum_{a\in \ZZ/p^v} \sum_{\bm{x}\in (\ZZ/p^v)^6} e_{p^v}(aF(\bm{x}))
= \sum_{l\in[0,v]} p^{6(v-l)}S_{\bm{0}}(p^l)$
(cf.~\cite{davenport2005analytic}*{Lemma~5.3}),
whence $\mcal{S}(K)=\sum_{n\mid K}n^{-6}S_{\bm{0}}(n)$.
% (by the Chinese remainder theorem).
Yet $\mf{S}_F=\sum_{n\geq1}n^{-6}S_{\bm{0}}(n)$
(by Definition~\ref{DEFN:HLH-asymptotic}),
and $n\mid K$ for all $n\in[1,M]$,
so
\mathd{
\abs{\mcal{S}(K)-\mathfrak{S}_F}
\leq \sum_{n>M}n^{-6}\abs{S_{\bm{0}}(n)}
\ll_\eps M^{(4-6)/3+\eps}
= M^{-2/3+\eps}
}
by standard bounds (e.g.~Lemma~\ref{LEM:sum-S_0(n)-trivially} below),
as desired.
}

Consequently,
\emph{if} $(F,w)$ is HLH
(in the sense of Definition~\ref{DEFN:HLH-asymptotic}),
then
\mathds{
\map{Var}(X,M)
&= [N_{F,w}(X)-\mf{S}_F\cdot\sigma_{\infty,F,w}X^3]
+ O(M^{-20/31}\cdot\sigma_{\infty,F,w}X^3)
% +\cdots\\
+ \frac{O_\nu(K^4)}{(X/K)^{100}} \\
&= \left[o_{\nu;X\to\infty}(1)
+\sum_{L\in C(\map{SSV})}\sigma_{\infty,L^\perp,w}\right]X^3
+ O\left(\frac{\sigma_{\infty,F,w}X^3}{M^{20/31}}\right)
% +\cdots,
+ \frac{O_\nu(K^4)}{(X/K)^{100}},
}
the most interesting term being the diagonal-type contribution $\sum_{L\in C(\map{SSV})}\sigma_{\infty,L^\perp,w}X^3$.

\subsection{Applying increasingly cuspidal weights}

The preceding analysis applies to arbitrary $\nu\in C^\infty_c(\RR^3)$ with $(F_0,\nu)$ smooth.
Now we finally choose specific $\nu$'s.
Fix a \emph{nonnegative,
even} weight $w_0\in C^\infty_c(\RR)$ with $w_0\vert_{[-1,1]}\geq 1$,
and a \emph{nonnegative} weight $D\in C^\infty_c(\RR_{>0})$ with $D\vert_{[1,2]}\geq 1$.

\defn{
Given $A_0\in \RR_{\geq 1}$, set
\mathd{
\nu(\tilde{\bm{y}}) = \nu_{w_0,D,A_0}(\tilde{\bm{y}})
\defeq w_0(F_0(\tilde{\bm{y}}))
\int_{A\in[1,A_0]}d^\times{A}
\prod_{i\in[3]}D(\abs{\tilde{y}_i}/A),
}
where $d^\times{A}\defeq dA/A$.
% (Recall that $F_0(\bm{y})\defeq y_1^3+y_2^3+y_3^3$.)
Then set $w(\tilde{\bm{x}})
\defeq \nu(\tilde{\bm{y}})\nu(-\tilde{\bm{z}})
= \nu(\tilde{\bm{y}})\nu(\tilde{\bm{z}})$.
% \defeq w_0(\tilde{x}_1^3+\tilde{x}_2^3+\tilde{x}_3^3)w_0(\tilde{x}_4^3+\tilde{x}_5^3+\tilde{x}_6^3)
% \prod_{i\in[m]}\int_{A\in[1,A_0]}d^\times{A}\,D(\abs{\tilde{x}_i}/A),
% $w(\tilde{\bm{x}})\defeq D(\norm{\tilde{\bm{x}}}/A)w_0(F(\tilde{\bm{x}}))$.
}

\rmk{
The integral over $A$ \emph{enlarges the search space} for representations of numbers by $F_0$.
Essentially,
we search among $\bm{y}$ with
$X\ll\abs{y_1}\asymp\abs{y_2}\asymp\abs{y_3}\ll A_0X$
and $\abs{F_0(\bm{y})}\ll X^3$.
It is important that the special locus $y_1+y_2=0$
(with $\tilde{y}_3^3=F_0(\tilde{\bm{y}})\in\Supp{w_0}$)
does not accumulate when integrated over $A$.
Cf.~\cite{diaconu2019admissible}*{discussion on p.~24}.
(By ``switching'' $\int,\prod$,
we could instead consider all $\bm{y}$ with
$X\ll\abs{y_1},\abs{y_2},\abs{y_3}\ll A_0X$
and $\abs{F_0(\bm{y})}\ll X^3$,
but then we would also need to ``manually'' restrict to $\abs{y_1+y_2}\gg X$,
etc.
Cf.~\cite{diaconu2019admissible}*{p.~24, definition of $R^\ast_N$}.)
}

\rmk{
In view of Example~\ref{EX:Fermat-smooth-clean-pairs-explicitly},
the use of the ``dyadic'' weight $D$ ensures
that the pairs $(F_0,\nu)$ and $(F,w)$ are not only smooth,
% \footnote{natural from a projective geometry standpoint}
but also clean.
}

\rmk{
Pointwise,
$\abs{\nu(\tilde{\bm{y}})}
\ll \int_{\RR_{>0}}d^\times{A}\,D(\abs{\tilde{y}_1}/A)
= \norm{D\circ\log}_{L^1(\RR)}
\ll 1$,
and in general $\norm{\nu}_{k,\infty}\ll_{k}1$ (i.e.~the derivatives of $\nu$ of order $\leq k$ are uniformly bounded).
% The integral over $A$ only enlarges $\nu$'s \emph{support}.
On the other hand,
$\vol(\Supp\nu)\asymp\log{A_0}$
truly \emph{grows} with $A_0$.
(To prove $\vol(\Supp\nu)\ll\log{A_0}$,
fix $A\in[1,A_0]$ and $\abs{\tilde{y}_1},\abs{\tilde{y}_2}\in[A,2A]$,
and note that $\set{\tilde{y}_3\in\pm[A,2A]:F_0(\tilde{\bm{y}})\in\Supp{w_0}}$ has length $\ll 1/A^2$.
% The proof of $\vol(\Supp\nu)\gg\log{A_0}$ is similar (after restricting $\tilde{y}_1,\tilde{y}_2$ to $[A,1.1A]$, say).
To prove $\vol(\Supp\nu)\gg\log{A_0}$,
use similar but more careful ideas,
based on the fact that $w_0\vert_{[-1,1]}\geq 1$ and $D\vert_{[1,2]}\geq 1$.)
}

Why require $w_0\vert_{[-1,1]}\geq 1$ and $D\vert_{[1,2]}\geq 1$?
% Given $A\in[1,A_0]$ and $X\gg_{A_0} 1$, we observed earlier that $\abs{F_0(\bm{y})}\leq X^3$ for $\gg X^3$ triples $\bm{y}\in\ZZ^3$ with $\abs{y_i}\in [AX,2AX]$.
Recall that
\mathd{
\sigma_{\infty,F_a,\nu}(X)
= \sigma_{\infty,F_{\tilde{a}},\nu}(1)
= \lim_{\eps\to0}\,(2\eps)^{-1}
\int_{\abs{F_0(\tilde{\bm{y}})-\tilde{a}}\leq\eps}d\tilde{\bm{y}}\,\nu(\tilde{\bm{y}}).
}
Now fix $\eps$,
plug in the definition of $\nu$,
switch the order of $\tilde{\bm{y}},A$,
and fix $A\in[1,A_0]$.
If $\abs{\tilde{a}}\leq 1$ and $\eps\leq 0.1$,
then $w_0(F_0(\tilde{\bm{y}}))
\prod_{i\in[3]}D(\abs{\tilde{y}_i}/A)
\geq 1$ certainly holds on the set
\mathd{
S_{\eps,A,\tilde{a}}\defeq
\set{\tilde{\bm{y}}\in\RR^3:
\textnormal{$\abs{\tilde{y}_i}\in[A,2A]$
and $F_0(\tilde{\bm{y}})\in[\tilde{a}-\eps,\tilde{a}+\eps]\cap[-1,1]$}},
}
and furthermore,
% this set has volume
$\vol(S_{\eps,A,\tilde{a}})
\gg A^3\cdot(\eps/A^{\deg{F_0}})
= \eps$
(as one can prove by
restricting attention to $\tilde{y}_1,\tilde{y}_2\in[A,1.1A]$,
for instance).
% \footnote{Indeed, the shrunken set $\set{\tilde{\bm{y}}/A}$ has volume $\gg\eps/A^3$, since $F_0\vert_{(1,2)\times(1,2)\times(-2,-1)}$ is a submersion onto $(-6,15)$ (which contains the compact set $[-1,1]\contains [-1/A^3,1/A^3]$), which locally looks like a projection.
% Alternatively a simple explicit computation, with $\tilde{y}_1,\tilde{y}_2$ restricted to $[A,1.1A]$, suffices (though one is free to use different restrictions to cover different ranges of [-1,1], e.g. one could cover [-1,0] and [0,1] separately).}
Thus for $\abs{a}\leq X^3$,
we have
\mathd{
\sigma_{\infty,F_a,\nu}(X)
\gg \lim_{\eps\to0}\,(2\eps)^{-1}
\int_{A\in[1,A_0]}d^\times{A}\,\eps
\gg \log{A_0}.
}
This proves the first part of the following observation:
\obs{[Cf.~\cite{diaconu2019admissible}*{\S2's analysis}]
Uniformly over $A_0\geq 1$, we have
\begin{enumerate}[(1)]
    \item $\sigma_{\infty,F_a,\nu}(X)\gg\log{A_0}$ uniformly over $\abs{a}\leq X^3$,
    while
    
    \item $\sigma_{\infty,L^\perp,w}\ll\log{A_0}$
    % for all maximal linear spaces $L$,
    % for all $L/\QQ$ from Definition~\ref{DEFN:HLH-constant} is ambiguous (depending on whether we look at the unrestricted or restricted L's); just change to "for all 3-dimensional vector spaces L/Q contained in C(V)"
    for all $L\in C(\map{SSV})$.
\end{enumerate}
}

\rmk{
The first part implies
$\sigma_{\infty,F,w}
= \int_{a\in\RR}d\tilde{a}\,\sigma_{\infty,F_a,\nu}(X)^2
\gg (\log{A_0})^2$.
In fact,
one can show that $\sigma_{\infty,F_a,\nu}(X)\ll\log{A_0}$ holds for \emph{all} $a\in\RR$,
whence $\sigma_{\infty,F,w}\ll(\log{A_0})^2$ as well.
}

\pf{[Proof of second part]
Fix $L$.
By symmetry,
there are really only two cases:
\begin{enumerate}[(1)]
    \item $L$ is cut out by $y_i+z_i=0$,
    i.e.~$x_i+x_{i+3}=0$,
    over $i\in\set{1,2,3}$.
    
    \item $L$ is cut out by
    $y_1+y_2=z_1+z_2=y_3+z_3=0$,
    i.e.~$x_1+x_2=x_4+x_5=x_3+x_6=0$.
\end{enumerate}
In the first case,
\mathd{
\sigma_{\infty,L^\perp,w}
= \lim_{\eps\to0}\,(2\eps)^{-3}
\int_{\abs{\tilde{y}_i+\tilde{z}_i}\leq\eps}d\tilde{\bm{x}}\,w(\tilde{\bm{y}},\tilde{\bm{z}})
= \int_{\RR^3}d\tilde{\bm{y}}\,w(\tilde{\bm{y}},-\tilde{\bm{y}})
\ll \vol(\Supp\nu)
\ll \log{A_0}.
}
% Second case: omitted (technically important, but OK for our choice of $\nu$).
In the second case,
$\sigma_{\infty,L^\perp,w}
= \int_{\RR^3}d\tilde{y}_1\,d\tilde{z}_1\,d\tilde{y}_3\,w(\tilde{\bm{y}},\tilde{\bm{z}})\vert_{L}
\ll 1$,
because every point $(\tilde{y}_1,\tilde{z}_1,\tilde{y}_3)\in\Supp(w\vert_{L})$ must satisfy the following three conditions:
(a) $w_0(\tilde{y}_3^3)\neq0$,
whence $\tilde{y}_3\ll 1$;
(b) $\abs{\tilde{y}_1}\asymp\abs{\tilde{y}_3}$;
and (c) $\abs{\tilde{z}_1}\asymp\abs{\tilde{z}_3}=\abs{\tilde{y}_3}$.
}

Finally,
we prove Theorem~\ref{THM:enough-HLH-implies-100pct-Hasse-principle-for-3Z^odot3}.

\pf{[Proof of Theorem~\ref{THM:enough-HLH-implies-100pct-Hasse-principle-for-3Z^odot3}]
The hypothesis of Theorem~\ref{THM:enough-HLH-implies-100pct-Hasse-principle-for-3Z^odot3} implies,
in particular,
that our pair $(F,w)$
(constructed above,
given $A_0$)
is HLH for \emph{every fixed} choice of $A_0$.
% i.e. \emph{Hooley's conjecture for $F,w$} above holds,

Now,
towards the conclusion of Theorem~\ref{THM:enough-HLH-implies-100pct-Hasse-principle-for-3Z^odot3},
note that we may restrict attention to $a\geq 0$,
by symmetry.
To proceed,
we follow \cite{diaconu2019admissible}*{proof of Theorem~3.3},
who first shows (via local calculations sketched later) that
\mathd{
K^{-1}\psum_{1\leq a\leq K}\frac{1}{s_{F_a}(K)}
\leq \prod_{p=3}O(1)
\prod_{p\equiv2\bmod{3}}(1+O(p^{-3}))
\prod_{p\equiv1\bmod{3}}(1+O(p^{-3/2}))
\ll 1,
}
where we restrict to $a\not\equiv4,5\bmod{9}$
(i.e.~locally \emph{admissible} integers).

Now fix $\delta>0$ arbitrarily small.
Suppose $9\mid K$ and $X\geq K\geq 1$.
The previous display implies that $s_{F_a}(K)\geq\delta$ for all but an $O(\delta)$ \emph{fraction} of admissible residues $a\bmod{K}$.
But by construction,
$\sigma_{\infty,F_a,\nu}(X)\gg\log{A_0}$ uniformly over $\abs{a}\leq X^3$.
Thus $\rho_{a,\nu}(X)
\defeq s_{F_a}(K)\sigma_{\infty,F_a,\nu}(X)
\gg \delta\log{A_0}$
holds for all but an $O(\delta)$ fraction of admissible integers $a\leq X^3$.
So if $A_0\geq 2$, say, then
\mathds{
\star
&\defeq \#\set{\textnormal{admissible $a\leq X^3$}:
N_{F_a}(\infty)=0} \\
&\leq \#\set{\textnormal{admissible $a\leq X^3$}:
\abs{N_{F_a,\nu}(X)-\rho_{a,\nu}(X)}\geq \rho_{a,\nu}(X)/2} \\
&\leq O\left(\delta X^3
+ \frac{\map{Var}(X,M)}{\delta^2(\log{A_0})^2}\right).
}
(The preceding display---a variant of Chebyshev's inequality---may be quite far from the truth.
But without higher moments, we cannot say more.)
By HLH and the bound $\sigma_{\infty,L^\perp,w}\ll \log{A_0}$,
though,
\mathd{
\map{Var}(X,M)
\leq [o_{\nu;X\to\infty}(1) + O(\log{A_0})]X^3
+ O(X^3\sigma_{\infty,F,w}M^{-20/31})
% +\cdots.
+ K^4O_\nu((X/K)^{-100}).
}
To finish, fix $A_0\gg_{\delta}1$ so that $O(\delta^{-2}(\log{A_0})^{-2})\cdot O(\log{A_0})\leq\delta$.
Recall that $A_0$ determines $\nu$,
and fix $M\gg_{\delta,A_0} 1$ so that $O(\delta^{-2}(\log{A_0})^{-2})\cdot O(\sigma_{\infty,F,w}M^{-20/31})\leq\delta$.
Then $\star\leq O(\delta X^3)$ for all sufficiently large $X\gg_{\delta,A_0,K} 1$.
Since $\delta>0$ was arbitrary,
we are done.
}

\rmk{
If we assumed a power saving in HLH,
with ``$O(A_0^{O(1)}X^{3-\Omega(1)})$'' in place of $o_{\nu;X\to\infty}(X^3)$,
then we would likely be able to let $A_0$ grow as a small power of $X$,
and $M,\delta$ as small powers of $(\log{X})^{\pm 1}$.
We would then likely get ``$\star\ll X^3/(\log{X})^{\Omega(1)}$'' as $X\to\infty$.
For proof,
we would need the following ingredients
(for fixed $w_0,D$),
not all proven above:
\begin{enumerate}[(1)]
    \item $K\ll_\eps X^\eps$
    (true for $M=o(\log{X})$,
    since in general,
    $\log{K(M)}\sim M$ as $M\to\infty$);
    
    \item $(\log{A_0})^{-2}\cdot\sigma_{\infty,F,w}\asymp 1$
    (a fact essentially remarked earlier);
    and
    
    \item replacing ``$K^4O_\nu((X/K)^{-100})$'' with ``$O(\diam(\Supp\nu)^{O(1)}K^4(X/K)^{-100})$
    (by analyzing the $\nu$-dependence in our error estimate from Poisson summation, stemming from Proposition~\ref{PROP:uniform-bound-on-derivatives-of-densities}).
    % To be safe we bound in terms of $\diam(\Supp\nu)^{O(1)}\ll A_0^{O(1)}$ (thinking in terms of the surface integral over y,z, and the support and derivatives of \nu) --- which is fine for us --- but a more careful analysis might lead to a better bound.
\end{enumerate}
}

\pf{
[Sketch of local calculations]
See \cite{diaconu2019admissible}*{pp.~10--11 (proof of Theorem~1.4(ii)) and pp.~32--34}
for details,
including the necessary Hensel lifting to moduli $p^{\geq 2}$ at each prime $p$.
The $3$-adic densities,
in particular,
require a bit more lifting work than the densities for $p\neq 3$ do.
%consider number of points mod 3^{l-1} with F_a divisible by 3^l

At $p\equiv1\bmod{3}$,
the local calculations mostly boil down
%to leading order
to a finite linear combination of
% ``purely multiplicative''
cubic characters $\chi\bmod{p}$
(evaluated at certain points depending on $F_0,a$),
since $F_0=a$ is an affine \emph{diagonal} cubic surface over $\FF_p$ when $a\neq0$.
(Unlike with CM elliptic curves,
Hecke characters of infinite order,
e.g.~signed normalized cubic Gauss sums $-\tilde{g}(\chi)=-p^{-1/2}\sum_{x\in\FF_p}\chi(x)e_p(x)$,
do not arise, except in secondary terms related to $F_0=0$.)
Ultimately here,
\mathds{
\sum_{a\in\FF_p}\frac{1}{s_{F_a}(p)}
&= \sum_{a\in\FF_p}\frac{p^2}{\#\set{\bm{y}\in\FF_p^3:F_a(\bm{y})=0}} \\
&= 1+O(p^{-1/2})
+\frac{(p-1)/3}{1+6/p+O(p^{-3/2})}
+\frac{2(p-1)/3}{1-3/p+O(p^{-3/2})} \\
&= 1+O(p^{-1/2})
+ (p-1)[1 - (1/3)(6/p) + (2/3)(3/p)]
= p+O(p^{-1/2}).
}
(The precise ``$O(p^{-1/2})$'' is roughly proportional to $\frac1p(a_p(E)+O(1))$,
where $E\defeq V_{\PP^2}(F_0)_{/\QQ}$.)

At primes $p\equiv2\bmod{3}$,
the local calculations are easier
(as if $F_0$ were linear),
due to bijectivity of the function $x\mapsto x^3$ on $\FF_p$.
}

\chapter{Review of the delta method}
\label{CHAP:delta-method-review}

\section{The basic setup}
\label{SEC:delta-method-setup}

Let $m\in \ZZ_{\geq 3}$.
Let $F\in \ZZ[\bm{x}]=\ZZ[x_1,\dots,x_m]$ be an $m$-variable cubic form with nonzero discriminant.
Let $\mcal{V}\defeq V_{\PP^{m-1}}(F)_{/\ZZ}$ and $V\defeq \mcal{V}_\QQ = V_{\PP^{m-1}}(F)_{/\QQ}$ be cut out by $F=0$.
Then $V$, the generic fiber of $\mcal{V}$, is a smooth projective hypersurface in $\PP^{m-1}_\QQ$.
For $\bm{c}\in \ZZ^m$, we define hyperplane sections $\mcal{V}_{\bm{c}}, V_{\bm{c}}$ via the following convenient general definition:
\defn{
%https://en.wikipedia.org/wiki/Hyperplane_section
Given a ring $R$,
an $m$-tuple $\bm{c}$ that ``makes sense'' in $R^m$ (or more precisely, an $m$-tuple $\bm{c}$ that maps canonically into $R^m$),
and a closed subscheme $W\belongs \PP^{m-1}_R$,
define $W_{\bm{c}}$ to be the scheme-theoretic intersection $W\cap\set{\bm{c}\cdot\bm{x}=0}$.
}

\rmk{
If $\bm{c}\in\QQ^m$,
then $V_{\bm{c}}$ is a hypersurface in $(\PP^{m-1}_\QQ)_{\bm{c}}$ (since $F$ is irreducible),
where $(\PP^{m-1}_\QQ)_{\bm{c}}\cong\PP^{m-2}_\QQ$ if $\bm{c}\neq\bm{0}$.
}

\defn{
\label{DEFN:S_c(n),sqrt-normalized-S_c(n),and-Phi(c,s)}
If $n\in\ZZ_{\geq1}$,
then for each $m$-tuple $\bm{c}$ that ``makes sense'' in $(\ZZ/n)^m$,
let $S_{\bm{c}}(n)\defeq \sum_{a\in (\ZZ/n)^\times} \sum_{\bm{x}\in(\ZZ/n)^m} e_n(aF(\bm{x}) + \bm{c}\cdot\bm{x})$
and $\wt{S}_{\bm{c}}(n)\defeq n^{-(m+1)/2} S_{\bm{c}}(n)$.
Then for each $\bm{c}\in\ZZ^m$, let $\Phi(\bm{c},s)\defeq \sum_{n\geq1} \wt{S}_{\bm{c}}(n) n^{-s}$.
}

Now fix $w\in C^\infty_c(\RR^m)$ with $(F,w)$ smooth (in the sense of Definition~\ref{DEFN:support-smooth-clean}),
and suppose we are interested in the weighted zero count $N_{F,w}(X)$ (from Definition~\ref{DEFN:weighted-zero-count}) for real $X>0$.
We begin by choosing ``standard cutoff parameters'' in the delta method, following \cites{duke1993bounds,heath1996new,heath1998circle}.
\defn{
\label{DEFN:delta-method-cutoff-parameters}
Set $Y\defeq X^{(\deg F)/2} = X^{3/2}$ to be, roughly,
the largest modulus used in the delta method---and given $\eps_0\in (0, 10^{-10}]$,
set $Z = Z_{\eps_0}\defeq
Y/X^{1-\eps_0} = X^{1/2 + \eps_0}$.
(To have correct epsilon management, we have named this epsilon $\eps_0$ to be safe.)
}

Before proceeding, let us recall the standard intuition behind the delta method---intuition one can formalize via eq.~\eqref{EQN:un-normalized-delta-method}, Proposition~\ref{PROP:rigorous-modulus-cutoff-Y}, and Lemma~\ref{LEM:c-aspect-I_c(n)-estimates} below.
\rmk{
[Intuition]
If $\bm{x}\ll X$,
then $F(\bm{x})\ll X^3$.
So it is natural to try using a total of $\asymp X^3$ harmonics to ``detect'' the condition $F(\bm{x})=0$ over $\bm{x}\ll X$.
Since the delta method morally uses $\asymp Y^2$ harmonics
(corresponding to proper reduced fractions with denominator $\ll Y$),
this suggests setting $Y^2\asymp X^3$.

(Smaller choices of $Y$ could in principle also be worth considering;
see the quartic analysis of \cite{marmon2019hasse}.
See \S\ref{SEC:possibly-averaging-deforming-enlarging-or-smoothing-the-delta-method} for some discussion.)

One then encounters certain ``pseudo-exponential sums to modulus $n\ll Y$'' over $\bm{x}\ll X$.
These sums are typically ``incomplete''
(since $X=o(Y)$).
By ``completing'' these sums using Poisson summation,
we end up with ``dual sums''
% morally of length (up to?) $\approx Y/X\asymp X^{1/2}$.
morally of length $\lessapprox Y/X\asymp X^{1/2}$.
}

% Recall $Y,Z$ from Definition~\ref{DEFN:delta-method-cutoff-parameters}.
By \cite{heath1996new}*{Theorem~2, (1.2)} (based on \cite{duke1993bounds})---or rather, \cite{heath1996new}*{(1.2),
up to easy manipulations from \S3 involving the formula (3.3) and the switching of $n,\bm{c}$}---we have (uniformly over $X>0$)
\begin{equation}
\label{EQN:un-normalized-delta-method}
(1+O_A(Y^{-A}))\cdot N_{F,w}(X)
= Y^{-2}\sum_{n\geq1}
\sum_{\bm{c}\in\ZZ^m}
n^{-m}S_{\bm{c}}(n)I_{\bm{c}}(n),
\end{equation}
where $S_{\bm{c}}(n)$ is defined as in Definition~\ref{DEFN:S_c(n),sqrt-normalized-S_c(n),and-Phi(c,s)},
and where (in terms of a certain function $h(-,-)$ typically left in the background; see e.g.~\cite{heath1998circle}*{eq.~(2.3)} for the definition)
\mathd{
I_{\bm{c}}(n) \defeq \int_{\bm{x}\in\RR^m} d\bm{x}\, w(\bm{x}/X) h(n/Y, F(\bm{x})/Y^2) e_n(-\bm{c}\cdot\bm{x}),
}
for $\bm{c}\in \ZZ^m$.
Dimensional analysis suggests the normalization
\mathd{
\wt{I}_{\bm{c}}(n) \defeq X^{-m}I_{\bm{c}}(n)
= \int_{\tilde{\bm{x}}\in\RR^m} d\tilde{\bm{x}}\, w(\tilde{\bm{x}}) h(n/Y, F(\tilde{\bm{x}})) e_{n/X}(-\bm{c}\cdot\tilde{\bm{x}})
}
(using $Y^2 = X^{\deg F}$ to get $F(\bm{x})/Y^2 = F(\tilde{\bm{x}})$ for $\tilde{\bm{x}} \defeq \bm{x}/X$).
Eq.~\eqref{EQN:un-normalized-delta-method} then becomes
\begin{equation}
\label{EQN:normalized-delta-method}
(1+O_A(X^{-A}))\cdot N_{F,w}(X)
= X^{m-3}\sum_{n\geq1}
\sum_{\bm{c}\in\ZZ^m} n^{-(m-1)/2}\wt{S}_{\bm{c}}(n)\wt{I}_{\bm{c}}(n).
\end{equation}

We have referred to the switching of $n,\bm{c}$ as ``easy''
because the compact support of $w$ guarantees that
$I_{\bm{c}}(n)$ is supported on finitely many moduli $n$,
and is rapidly decaying in $\norm{\bm{c}}$ for each $n$.
These \emph{qualitative} facts---important for \cite{heath1996new}*{\S3, proof of Theorem~2}
(which at one point switches $n,\bm{c}$ to get to \cite{heath1996new}*{(1.2)}),
and for us---follow from the following two standard results,
which will soon begin to play an important \emph{quantitative} role as well.

\prop{
% [\cite{heath1996new}*{p.~180, par.~1 of \S7}]
% [Modulus cutoff]
[Vanishing for large $n$]
\label{PROP:rigorous-modulus-cutoff-Y}
% $I_{\bm{c}}(n) = 0$ for $n\gg Y$,
The functions $n\mapsto I_{\bm{c}}(n)$ are supported on $n\ll_{F,w} Y$,
uniformly over $\bm{c}\in\ZZ^m$.
}

\pf{
See e.g.~\cite{heath1996new}*{par.~1 of \S7}.
The vanishing of $I_{\bm{c}}(n)$ for $n\gg_{F,w}Y$ sufficiently large
is a consequence of the choice $Y\asymp X^{(\deg F)/2}$ and the definition of $h(-,-)$.
}

As a sanity check, note that if $X\ll_{F,w} 1$ is sufficiently small, then Proposition~\ref{PROP:rigorous-modulus-cutoff-Y} yields $I_{\bm{c}}(n)=0$ for all $n\geq 1$, whence the right-hand side of eq.~\eqref{EQN:un-normalized-delta-method} vanishes.
This is consistent with the fact that the ``true'' factor of $1+O_A(Y^{-A})$ on the left-hand side of eq.~\eqref{EQN:un-normalized-delta-method} vanishes for $Y<1$ (cf.~\cite{heath1996new}*{left-hand side of (3.3) for $Q<1$}).

The case $X\leq 1$, say, is similarly uninteresting: if $X$ is bounded, then both sides of eq.~\eqref{EQN:un-normalized-delta-method} are trivially bounded as well.
So from now on, we assume $X\geq 1$.

\lem{
% [\cite{heath1998circle}*{Lemma~3.2, (3.9)}]
[Decay for large $\bm{c}$]
\label{LEM:c-aspect-I_c(n)-estimates}
If $\norm{\bm{c}}\geq Z$ and $n\geq1$,
then $I_{\bm{c}}(n)
\ll_{\eps_0,A}\norm{\bm{c}}^{-A}$.
}

\pf{
See e.g.~\cite{heath1998circle}*{Lemma~3.2, (3.9)}.
}

Proposition~\ref{PROP:rigorous-modulus-cutoff-Y} and Lemma~\ref{LEM:c-aspect-I_c(n)-estimates} easily (and comfortably) imply the \emph{absolute} bound
\mathd{
Y^{-2}\sum_{n\geq 1}\sum_{\norm{\bm{c}}\geq Z}n^{-m}\abs{S_{\bm{c}}(n)}\cdot\abs{I_{\bm{c}}(n)}\ll_{F,w,\eps_0,A}X^{-A}
}
(even if we only use the trivial bound $\abs{S_{\bm{c}}(n)}\leq n^{1+m}$).
To analyze $N_{F,w}(X)$, as $X\to\infty$, via eq.~\eqref{EQN:normalized-delta-method},
it thus precisely remains to understand (for arbitrarily small $\eps_0$) the quantity
\begin{equation}
\label{EXPR:main-delta-method-quantity}
X^{m-3}\sum_{n\geq1}
\sum_{\norm{\bm{c}}\leq Z}n^{-(m-1)/2}\wt{S}_{\bm{c}}(n)\wt{I}_{\bm{c}}(n).
\end{equation}
(Here $\wt{I}_{\bm{c}}(n)
= \wt{I}_{\bm{c}}(n)\cdot\bm{1}_{n\ll Y}$
for a suitable factor $\bm{1}_{n\ll Y}$.
But it is more convenient to keep the factor implicit,
to allow for more flexible technique.)

\section{Exponential sums and \texpdf{$L$}{L}-functions}

The sums $S_{\bm{c}}(n)$ have some nice properties
originating from the \emph{homogeneity} of $F$ (crucial in our analysis, and likewise in \cites{hooley1986HasseWeil,hooley_greaves_harman_huxley_1997,heath1998circle}).
\begin{enumerate}[(1)]
    \item The function $n\mapsto S_{\bm{c}}(n)$ is \emph{multiplicative},
    i.e.~$S_{\bm{c}}(n_1n_2) = S_{\bm{c}}(n_1)S_{\bm{c}}(n_2)$ if $(n_1,n_2) = 1$.
    
    \item The function $(F,\bm{c})\mapsto S_{\bm{c}}(n)$ is \emph{scale-invariant},
    e.g.~$S_{\bm{c}}(n)=S_{\lambda\bm{c}}(n)$ if $\lambda\in(\ZZ/n)^\times$.
\end{enumerate}
In particular,
(2) suggests that $S_{\bm{c}}$ might only depend on \emph{homogeneous geometric} invariants of $(F,\bm{c})$.
This is indeed the case.
We now recall some of the main invariants involved.

\defn{
Let $m_\ast\defeq m-3$.
For a prime power $q$,
and an $m$-tuple $\bm{c}$ that ``makes sense'' in $\FF_q^m$,
let $\rho(q),\rho_{\bm{c}}(q)$ be
the respective $\FF_q$-point counts of $\mcal{V}_{\FF_q},(\mcal{V}_{\FF_q})_{\bm{c}}$.
Normalize the ``errors'' $E(q) \defeq \rho(q) - (q^{m-1} - 1)/(q-1)$ and $E_{\bm{c}}(q) \defeq \rho_{\bm{c}}(q) - (q^{m-2} - 1)/(q-1)$
to get $\wt{E}(q) \defeq q^{-(1+m_\ast)/2}E(q)$ and $\wt{E}_{\bm{c}}(q) \defeq q^{-m_\ast/2}E_{\bm{c}}(q)$.
}

\rmk{
Perhaps somewhat confusingly,
$E_{\bm{0}}(q)\neq E(q)$.
}

\propdefn{
[Classical]
\label{PROPDEFN:unique-geometric-discriminant-form,with-proof-for-diagonal-F}
Up to scaling,
there is a \emph{unique} $F^\vee\in\ZZ[\bm{c}]\setminus\set{0}$ of degree $3\cdot 2^{m-2}$
such that
if $\bm{c}\in\CC^m\setminus\set{\bm{0}}$,
then $F^\vee(\bm{c})=0$ if and only if $(V_\CC)_{\bm{c}}$ is singular.

Now fix $F^\vee$.
Then $F^\vee$ is homogeneous,
and irreducible over $\CC$.
Informally,
we call $F^\vee$ a (geometric) \emph{discriminant form}.
Furthermore,
we may choose $F^\vee$ so that for all $\bm{c}\in\ZZ^m$ and primes $p\nmid F^\vee(\bm{c})$,
the special fiber $(\mcal{V}_{\bm{c}})_{\FF_p}$ is smooth of dimension $m_\ast$.
}

\pf{
In general, see \cite{wang2022dichotomous}*{Appendix~A} (for a discriminant-based perspective) or \cite{wang2021_isolating_special_solutions}*{Remark~A.3} (for a perspective based on dual varieties).
When $F$ is diagonal, see \cite{wang2021_large_sieve_diagonal_cubic_forms}*{Proposition-Definition~1.8} for an explicit treatment;
for example, if $F = x_1^3 + \dots + x_m^3$, then we may take $F^\vee = 3\prod (c_1^{3/2}\pm c_2^{3/2}\pm \dots \pm c_m^{3/2})$.
}

For $\bm{c}\in \ZZ^m$,
recall that $S_{\bm{c}}$ is multiplicative.
So $\wt{S}_{\bm{c}}$ is too.
Now we recall some standard formulas at prime powers.

\prop{
% [Cf.~\cite{hooley1986HasseWeil}*{p.~69, (47)}]
% For all tuples $\bm{c}\in\ZZ^m$ and primes $p\nmid\bm{c}$,
Say $p\nmid \bm{c}$.
Then $S_{\bm{c}}(p) = p^2 E_{\bm{c}}(p) - p E(p)$.
}

\pf{
[Proof sketch]
Although \cite{hooley1986HasseWeil}*{p.~69, (47)} assumes $p\nmid F^\vee(\bm{c})$ and $F=x_1^3+\dots+x_6^3$,
the underlying arguments work more generally;
cf.~\cite{hooley2014octonary}*{Lemma~7}.
% See \cite{hooley1986HasseWeil}*{pp.~64--65, proof of Lemma~6}
% (Lemma~6 being an equivalent formulation of the proposition above)
% for details
% when $p\nmid F^\vee(\bm{c})$ and $F=x_1^3+\dots+x_6^3$.
% But the same argument works more generally; see \cite{hooley2014octonary}*{p.~254, Lemma~7}.
% The key is that
% $F$ is homogeneous, so
% $S_{\bm{c}}(p)$ is invariant under scaling of $\bm{c}$.
% This symmetry is the key.
% if $p\nmid \bm{c}$.
% If $p\mid \bm{c}$, then
% $
% S_{\bm{c}}(p)
% = S_{\bm{0}}(p)
% = p [(p-1)\rho(p) + 1] - p^m
% = p(p-1) E(p)
% $.
% A more uniform proof should also be possible.
Because $F$ is homogeneous,
we have $S_{\bm{c}}(p) = S_{\lambda\bm{c}}(p)$ for all $\lambda\in \FF_p^\times$.
So
\mathds{
(p-1)S_{\bm{c}}(p)
&= \sum_{\lambda, a\in \FF_p^\times} \sum_{\bm{x}\in \FF_p^m} e_p(aF(\bm{x})+\lambda\bm{c}\cdot\bm{x}) \\
&= \sum_{\bm{x}\in \FF_p^m} (p\cdot\bm{1}_{p\mid F(\bm{x})} - 1)(p\cdot\bm{1}_{p\mid \bm{c}\cdot\bm{x}} - 1)
= p^2(p-1)E_{\bm{c}}(p)-p(p-1)E(p)
}
if $p\nmid\bm{c}$.
(The factor $p^2(p-1)$ comes from summing over $\lambda,a$ and taking an affine cone over $(\mcal{V}_{\FF_p})_{\bm{c}}$.)
}

\rmk{
If $p\mid\bm{c}$,
% then the same holds if we replace $E_{\bm{c}}(p)=E_{\bm{0}}(p)$ with $E(p)$.
then $S_{\bm{c}}(p) = S_{\bm{0}}(p) = p^2 E(p) - p E(p)$ instead.
}

In particular,
$\wt{S}_{\bm{c}}(p)
= \wt{E}_{\bm{c}}(p)-p^{-1/2}\wt{E}(p)$
at primes $p\nmid F^\vee(\bm{c})$.
Here $\wt{E}(p)\ll 1$, by the Weil conjectures (after ``absorbing'' bad primes for $F$ into the implied constant).

\prop{
% [Cf.~\cite{hooley1986HasseWeil}*{Lemma~7}]
Say $p\nmid F^\vee(\bm{c})$.
Then $S_{\bm{c}}(p^l) = 0$ for all integers $l\geq2$.
}

\pf{
[Proof sketch]
Although \cite{hooley1986HasseWeil}*{Lemma~7} assumes $F=x_1^3+\dots+x_6^3$,
the underlying proof immediately generalizes;
see \cite{hooley2014octonary}*{Lemma~10}.
This time, scalar symmetry in $\bm{c}$ (using homogeneity of $F$) gives
\mathd{
\phi(p^l)S_{\bm{c}}(p^l)
= \sum_{\bm{x}\in(\ZZ/p^l)^m}[-p^{l-1}\cdot\bm{1}_{p^{l-1}\mid \bm{c}\cdot\bm{x}} + p^l\cdot\bm{1}_{p^l\mid \bm{c}\cdot\bm{x}}] [-p^{l-1}\cdot\bm{1}_{p^{l-1}\mid F(\bm{x})} + p^l\cdot\bm{1}_{p^l\mid F(\bm{x})}].
}
So $S_{\bm{c}}(p^l) = 0$ is equivalent to certain statements about point counts.
One can prove these statements by Hensel lifting;
the lifting calculus follows dimension predictions,
because $p\nmid F^\vee(\bm{c})$ implies
(by Proposition-Definition~\ref{PROPDEFN:unique-geometric-discriminant-form,with-proof-for-diagonal-F})
that the $\FF_p$-variety $(\mcal{V}_{\FF_p})_{\bm{c}}$ is smooth of codimension $2$.
% has no singular $\FF_p$-points
% (in fact, no singular $\ol{\FF}_p$-points).
}

Consequently, we know that
\begin{enumerate}[(1)]
    \item $\wt{S}_{\bm{c}}(p) = \wt{E}_{\bm{c}}(p) - p^{-1/2}\wt{E}(p)$ if $p\nmid\bm{c}$
    (e.g.~if $p\nmid F^\vee(\bm{c})$),
    
    \item $\wt{S}_{\bm{c}}(p) = \wt{S}_{\bm{0}}(p) = (p^{1/2}-p^{-1/2})\wt{E}(p)$ if $p\mid\bm{c}$, and
    
    \item $\wt{S}_{\bm{c}}(p^l)=0$ for $l\geq 2$ if $p\nmid F^\vee(\bm{c})$.
\end{enumerate}
But if $\bm{c}\in \ZZ^m$ and $p\nmid F^\vee(\bm{c})$, then $(\mcal{V}_{\bm{c}})_{\FF_p}$ is a smooth complete intersection in $\PP^{m-1}_{\FF_p}$ of dimension $m_\ast$ and multi-degree $(3,1)$.
By the theory of $\ell$-adic cohomology
(including the Grothendieck--Lefschetz fixed-point theorem and the resolution of the Weil conjectures),
we can thus make the following definition:
\defn{
\label{DEFN:good-frienly-Hasse-Weil-L-function-factors}
Fix $\bm{c}\in\ZZ^m$ with $F^\vee(\bm{c})\neq0$,
i.e.~with $V_{\bm{c}}$ smooth of dimension $m_\ast$.
Then for each prime $p\nmid F^\vee(\bm{c})$,
define the \emph{analytically normalized local factor}
\mathds{
L_p(s,V_{\bm{c}})
&\defeq \exp\left((-1)^{m_\ast}\sum_{r\geq1} \wt{E}_{\bm{c}}(p^r)\frac{(p^{-s})^r}{r}\right) \\
&= \prod_{j=1}^{\dim_{m_\ast}} (1-\tilde{\alpha}_{\bm{c},j}(p)p^{-s})^{-1}
\eqdef \sum_{l\geq0}\tilde{\lambda}_{\bm{c}}(p^l)p^{-ls}
}
of degree $\dim_{m_\ast}
\defeq \rank(H^{m_\ast}_\textnormal{sing}(V_{\bm{c}}(\CC),\ZZ)/H^{m_\ast}_\textnormal{sing}(\PP^{m-1}(\CC),\ZZ))
= A_{3,m_\ast+2}$, where
\mathd{
A_{d,s}
\defeq\#\{\bm{a}\in[d-1]^s:a_1+\dots+a_s\equiv0\bmod{d}\}
%Weil uses letter A for the count, though personally might prefer M(ax) or S(up).
=\frac{(d-1)^s+(-1)^s(d-1)}{d}
}
% denote $\dim H^{m_\ast}_{\map{prim}}$ of a complex smooth degree $d$ diagonal hypersurface in $m-1$ variables
for integers $d,s\geq 1$ (following \cite{weil1949numbers}*{p.~506}).
Here the $\tilde{\alpha}_{\bm{c},j}(p)$ denote certain ``normalized'' Frobenius eigenvalues
(known to satisfy $\abs{\tilde{\alpha}_{\bm{c},j}(p)} = 1$).
}

\rmk{
Each $V_{\bm{c}}$ above is a subvariety of $\PP^{m-1}_\QQ$,
so $H^{m_\ast}_\textnormal{sing}({\cdots})/H^{m_\ast}_\textnormal{sing}({\cdots})$ is well-defined.
Also, each $L_p(s,V_{\bm{c}})$ above is well-defined:
for any $\bm{c},\bm{c}'\in \ZZ^m$ with $V_{\bm{c}}=V_{\bm{c}'}$,
one can show (e.g.~by smooth proper base change) that $E_{\bm{c}}(q) = E_{\bm{c}'}(q)$ holds for all prime powers $q$ coprime to $F^\vee(\bm{c})F^\vee(\bm{c}')$.
% Fancy proof of second point: smooth proper base change; elementary proof: compare homogeneous ideals to show that $\bm{c}\in \QQ^\times \bm{c}'$, and then use the coprimality condition to show that $\bm{c}\in \ZZ_p^\times \bm{c}'$.
(To avoid discussing the two points above,
we could write $-\bm{1}_{2\mid m_\ast} + \rank H^{m_\ast}_\textnormal{sing}(V_{\bm{c}}(\CC),\ZZ)$ in place of $\rank({\cdots}/{\cdots})$,
and $L_p(s,\bm{c})$ in place of $L_p(s,V_{\bm{c}})$.
But the current notation is more transparent and suggestive.)
}

In particular,
if $p\nmid F^\vee(\bm{c})$,
then $(-1)^{m_\ast}\wt{E}_{\bm{c}}(p)
= \sum_j \tilde{\alpha}_{\bm{c},j}(p)
= \tilde{\lambda}_{\bm{c}}(p)$,
so $\wt{E}_{\bm{c}}(p)\ll_m 1$.
(Similarly,
we have $\wt{E}(p)\ll_{F}1$ uniformly over \emph{all} primes $p$.)
So for each \emph{fixed} $\bm{c}\in\ZZ^m$ with $F^\vee(\bm{c})\neq0$,
we roughly expect an approximation of the form
\mathd{
\Phi(\bm{c},s)\approx\prod_{p\nmid F^\vee(\bm{c})}L_p(s,V_{\bm{c}})^{(-1)^{m_\ast}}
\qquad\textnormal{(``to leading order'')}.
}
Indeed, the works \cites{hooley1986HasseWeil,hooley_greaves_harman_huxley_1997,heath1998circle,wang2021_large_sieve_diagonal_cubic_forms} are based on the intuitive notion of a ``first-order approximation'' of a Dirichlet series.
One could imagine many different precise definitions---perhaps useful for different purposes.
We now make the following convenient (but not necessarily comprehensive or all-purpose) definition:
\defn{
\label{DEFN:one-sided-first-order-Euler-product-approximation-of-Phi}
Fix a family of Dirichlet series $\Psi_1(\bm{c},s)$ indexed by $\set{\bm{c}\in\ZZ^m:F^\vee(\bm{c})\neq0}$.
For each $\bm{c}$,
let $b_{\bm{c}}(n), a_{\bm{c}}(n),a'_{\bm{c}}(n)$ be the $n$th coefficients of the Dirichlet series $\Psi_1,\Psi_1^{-1},\Phi(\bm{c},s)/\Psi_1$, respectively.
Suppose that
\begin{enumerate}[(1)]
    \item $b_{\bm{c}}$ is multiplicative,
    i.e.~$\Psi_1$ has a (formal) Euler product;
    
    \item $b_{\bm{c}}(n),a'_{\bm{c}}(n)
    \ll_\eps n^\eps\sum_{d\mid n}\abs{\wt{S}_{\bm{c}}(d)}$
    holds uniformly over $\bm{c},n$;
    and
    
    \item $a'_{\bm{c}}(p)
    \cdot \bm{1}_{p\nmid F^\vee(\bm{c})}
    \ll p^{-1/2}$
    holds uniformly over $\bm{c},p$ with $p$ prime.
\end{enumerate}
Then we call $\Psi_1$ a \emph{(one-sided, first-order, Euler-product) approximation of $\Phi$}.
}

\ex{
\label{EX:one-sided-first-order-approximations-of-Phi}
Suppose $m\in \set{4,6}$ and for each $\bm{c}$,
we let $\Psi_1(\bm{c},s)
\defeq L(s,V_{\bm{c}})^{-1}$,
with the standard Hasse--Weil $L$-function $L(s,V_{\bm{c}})$, defined as in
\cite{hooley1986HasseWeil} for $F=x_1^3+\dots+x_6^3$ (though the definition readily generalizes; see e.g.~\cite{heath1998circle});
for primes $p\nmid F^\vee(\bm{c})$ the local factor agrees with the $L_p(s,V_{\bm{c}})$ defined earlier.
Then $\Psi_1$ is an approximation of $\Phi$,
in the sense of Definition~\ref{DEFN:one-sided-first-order-Euler-product-approximation-of-Phi}.
}

\pf{
[Proof sketch]
Say $p\nmid F^\vee(\bm{c})$.
Then (if $\mu_{\bm{c}}(-)$ denote the coefficients of $1/L(s,V_{\bm{c}})$)
\mathd{
a'_{\bm{c}}(p)
\defeq \wt{S}_{\bm{c}}(p) + \mu_{\bm{c}}(p)
= \wt{S}_{\bm{c}}(p) - \wt{E}_{\bm{c}}(p)
= -p^{-1/2}\wt{E}(p) \ll p^{-1/2}.
}
Also, $a'_{\bm{c}}(p^k) = \mu_{\bm{c}}(p^k) + \wt{S}_{\bm{c}}(p)\mu_{\bm{c}}(p^{k-1}) \ll_\eps p^{k\eps}$ for all $k\geq2$, since $\wt{S}_{\bm{c}}(p^l)=0$ for $l\geq2$.
% (in fact $\wt{S}_{\bm{c}}(p^l)\ll1$ suffices).
More care is needed to handle primes $p\mid F^\vee(\bm{c})$; we need the general bound $b_{\bm{c}}(n),a_{\bm{c}}(n)\ll_\eps n^\eps$, which is luckily known for $m\in \set{4,6}$.
}

\rmk{
\label{RMK:first-mention-of-standard-consequences-of-Langlands-and-GRH}
The aforementioned works \cites{hooley1986HasseWeil,hooley_greaves_harman_huxley_1997,heath1998circle} are based on a certain ``Hypothesis HW'' (mentioned in Example~\ref{EX:6-cubes-zero-locus}) for the Hasse--Weil $L$-functions $L(s,V_{\bm{c}})$.
Fix $\bm{c}$,
let $1/L(s,V_{\bm{c}})\eqdef\sum_{n\geq1}\mu_{\bm{c}}(n)n^{-s}$,
and let $q(V_{\bm{c}})$ denote a certain ``conductor'' associated to $V_{\bm{c}}$.
In lieu of precisely recalling
Hypothesis~HW---which amounts to
\emph{certain Selberg-type axioms}
(e.g.~suitable analytic continuation),
\emph{plus GRH}---we simply record
the following conjectures,
each known to imply the next:
\begin{enumerate}[(1)]
    \item Certain Langlands-type conjectures,
    % (for certain Galois representations associated to $V_{\bm{c}}$),
    plus GRH,
    for $L(s,V_{\bm{c}})$.
    
    \item Hypothesis~HW for $L(s,V_{\bm{c}})$.
    
    \item A certain standard elementary ``uniform square-root cancellation'' bound---namely $\sum_{n\leq N}\mu_{\bm{c}}(n)\ll_{m,3,\eps} q(V_{\bm{c}})^\eps N^{1/2+\eps}$ (with an implied constant depending only on $m,3,\eps$).
\end{enumerate}
Here,
chief among the ``Langlands-type conjectures'' in (1)
is \emph{automorphy},
i.e.~(a general form of)
% https://www.ias.edu/idea-tags/langlands-reciprocity-conjecture
% https://mathoverflow.net/questions/343666/langlands-reciprocity-and-fermats-last-theorem ("I think the confusion here lies in what is being called reciprocity (and perhaps the interpretation of "simple"). If by Langlands reciprocity, you mean a correspondence between classical Artin representations and automorphic representations,..." ... "If you are a bit more liberal, and mean a suitable correspondence between")
\emph{Langlands reciprocity}---a statement
generalizing the \emph{modularity of elliptic curves}.
}

\rmk{
\label{RMK:moral-argument-for-assuming-Langlands-directly-rather-than-HW-type-hypotheses}
Later, in Chapter~\ref{CHAP:using-mean-value-L-function-predictions},
we will work \emph{directly} with (1),
rather than with ``avatars'' like (2).
This is natural,
in view of ``analytic-to-automorphic'' converse theorems
% https://en.wikipedia.org/wiki/Converse_theorem
(in other natural families of Dirichlet series)---and
because (1), modulo GRH,
not only lies in a more conceptual framework,
but also offers the only known approach to proving (2), modulo GRH.
}

\section{Main unconditional general pointwise bounds}

For the rest of Chapter~\ref{CHAP:delta-method-review}, assume $F$ is diagonal.
For technical reasons,
we will analyze the $\bm{c}$'s in groups depending on how many coordinates are zero.
It would be interesting to find a similar notion for non-diagonal forms $F$.

\defn{
Given $\mcal{I}\belongs [m]$ of size $r\leq m$,
we call the set $\mcal{R}\belongs [-Z,Z]^m$ of $\bm{c}\in\ZZ^m$ with $c_j\in [-Z,Z]\setminus\set{0}$ for $j\in \mcal{I}$ and $c_j=0$ for $j\notin \mcal{I}$ a (uniform) $r$-dimensional \emph{deleted box}.
}

Let $\mcal{R}\belongs [-Z,Z]^m$ be a deleted box with $\card{\mcal{I}}=r\in[0,m]$.
Then we can bound the sums $\wt{S}_{\bm{c}}(n)\defeq n^{-(m+1)/2}S_{\bm{c}}(n)$ using
the somewhat crude but \emph{general} pointwise bound given by the next result (available since $F$ is diagonal).

\defn{
For an integer $n\geq 1$,
let $\map{sq}(n)\defeq \prod_{p^2\mid n} p^{v_p(n)}$ denote the \emph{square-full part} of $n$,
and $\map{cub}(n)\defeq \prod_{p^3\mid n} p^{v_p(n)}$ the \emph{cube-full part} of $n$.
}

\prop{
[\cites{hooley1986HasseWeil,heath1998circle}]
\label{PROP:pointwise-bound}
For all $\bm{c}\in\ZZ^m$ and integers $n\geq1$,
we have
\mathd{
n^{-1/2}\abs{\wt{S}_{\bm{c}}(n)}
\ll_{F}
O(1)^{\omega(n)}
\prod_{j\in[m]}\gcd(\map{cub}(n)^{1/6},\gcd(\map{cub}(n),\map{sq}(c_j))^{1/4}).
}
Here we interpret $\gcd(-,-)$ formally in terms of exponents that are allowed to be rational.
}

\pf{
In general $\wt{S}_{\bm{c}}(n) = n^{-(m+1)/2}S_{\bm{c}}(n)$ by definition,
and for diagonal $F$ we have
\mathd{
\abs{S_{\bm{c}}(p^l)}
% &\leq p^l\max_{a\in(\ZZ/p^l)^\times}\prod_{i\in[m]}\left\lvert\sum_{x_i\in\ZZ/n} e_n(aF_ix_i^3+c_ix_i)\right\rvert \\
\ll (p^\infty,O_F(1))^{O(1)}
\cdot p^{l(1+m/2)}
\prod_{j\in[m]}\gcd(\map{cub}(p^l)^{1/6},\gcd(\map{cub}(p^l),\map{sq}(c_j))^{1/4})
}
by \cite{heath1998circle}*{p.~682, (5.1)--(5.2)} for $l\geq 2$ and \cite{heath1983cubic}*{Lemma~11} for $l=1$.
}

\rmk{
We have stated Proposition~\ref{PROP:pointwise-bound} uniformly over $\bm{c}\in\ZZ^m$.
But given $\mcal{R}$,
Proposition~\ref{PROP:pointwise-bound} \emph{implies} that
for all $\bm{c}\in\mcal{R}$ and integers $n\geq1$,
we have
\mathd{
n^{-1/2}\abs{\wt{S}_{\bm{c}}(n)}
\ll_{F}
O(1)^{\omega(n)}
\map{cub}(n)^{(m-r)/6}
\prod_{j\in\mcal{I}} \gcd(\map{cub}(n),\map{sq}(c_j))^{1/4}.
}
It is this simpler statement that we typically use.
}

Now we turn to the \emph{integrals} $\wt{I}_{\bm{c}}(n)\defeq X^{-m}I_{\bm{c}}(n)$,
assuming $r\geq1$.
(We cover $\bm{c}=\bm{0}$ in \S\ref{SEC:singular-series-contribution}.)
The statement \cite{heath1998circle}*{p.~678, Lemma~3.2} covers the essential cases $k=0,1$ in the next result
(which, like Proposition~\ref{PROP:pointwise-bound}, is available since $F$ is diagonal).
We state a generalization to all $k\geq0$,
just in case it comes in handy for some (future) smoothing purposes.

\lem{[Main $n$-aspect bounds]
\label{LEM:n-aspect-I_c(n)-estimates}
% $I_{\bm{c}}(n) = 0$ for $n\gg Y$,
% uniformly over $\bm{c}\in\ZZ^m$.
% In general,
Uniformly over $\bm{c}\in\mcal{R}$ and $n\in[1/2,\infty)$,
we have
\mathd{
n^k\abs{\partial_n^k \wt{I}_{\bm{c}}(n)}
\ll_{k,\eps}
% X^\eps
% \left(\frac{X\norm{\bm{c}}}{n}\right)^{1-(m-r)/4}
% \prod_{i\in \mcal{I}}\left(\frac{n}{X\abs{c_i}}\right)^{1/2}
% =
X^\eps
\left(\frac{X\norm{\bm{c}}}{n}\right)^{1-(m+r)/4}
\prod_{i\in \mcal{I}}\left(\frac{\norm{\bm{c}}}{\abs{c_i}}\right)^{1/2}
\quad\textnormal{for}\;
k=0,1,2,\dots.
}
%https://www.wolframalpha.com/input/?i=1-%28m%2Br%29%2F4+%2B+r%2F2+%3D+1-%28m-r%29%2F4
Furthermore,
if $B\in C^\infty_c(\RR_{>0})$ is supported on $[1/2,1]$,
then $n^{k+1}\abs{\partial_n^k[y^{-1}B(n/y)\wt{I}_{\bm{c}}(n)]}$ satisfies the same bound for all $n\in (0,\infty)$,
uniformly as $y\geq 1$ varies.
}

\pf{
By \cite{density_beats_hua_when}*{Remark~4.1 and Lemma~4.9},
we know that (for all $\bm{c}\in \RR^m\setminus \set{\bm{0}}$)
\mathd{
n^k\cdot\partial_n^k I_{\bm{c}}(n)
\ll_{k,\eps}
\left(\frac{X\norm{\bm{c}}}{n}\right) X^{m+\eps} \prod_{i=1}^{m}
\min[
(n/X\abs{c_i})^{1/2},
(n/X\norm{\bm{c}})^{1/4}
],
}
and that $n^{k+1}\cdot \partial_n^k[y^{-1}B(n/y)I_{\bm{c}}(n)]$ satisfies the same bound.
Now ``replace'' $\min[-,-]$
with $(n/X\abs{c_i})^{1/2}$ for each $i\in\mcal{I}$,
and with $(n/X\norm{\bm{c}})^{1/4}$ for each $i\in[m]\setminus\mcal{I}$;
this bounds the right-hand side by
$X^{m+\eps}(X\norm{\bm{c}}/n)^{1-(m-r)/4}\prod_{i\in\mcal{I}}(n/X\abs{c_i})^{1/2}$,
which (after dividing by $X^m$) simplifies to what we want
(since $\wt{I}_{\bm{c}}(n)\defeq X^{-m}I_{\bm{c}}(n)$).
}

\rmk{
We should emphasize that the above bounds on $I_{\bm{c}}(n)$ are likely
only valid (as written)
for the usual (and present) setting of the parameter $Y \asymp X^{(\deg F)/2} = X^{3/2}$.
}

\rmk{
If $(F,w)$ is clean,
then $n^k\abs{\partial_n^k \wt{I}_{\bm{c}}(n)}
\ll_{k,\eps} X^\eps\min(1, (X\norm{\bm{c}}/n)^{1-m/2})
\ll X^\eps(X\norm{\bm{c}}/n)^{1-m/2}$;
cf.~\cite{hooley2014octonary}*{p.~252}.
}

\rmk{
It would be interesting to know the optimal asymptotics,
e.g.~whether or not the bound remains true with a smaller power of $X\norm{\bm{c}}/n$.
At least for generic $\bm{c}$,
one might expect to reduce the exponent $1-m/2$ using
a deeper stationary phase analysis
(implemented to some extent in Chapter~\ref{CHAP:discriminating-pointwise-estimates}).
}

\section{Contribution from the central terms}
\label{SEC:singular-series-contribution}

Here we address $\bm{c}=\bm{0}$ in \eqref{EXPR:main-delta-method-quantity},
using the theory of $I_{\bm{0}}(n)$ developed in \cite{heath1996new}.
This section is standard, and does not require $F$ to be diagonal (see \cite{wang2021_isolating_special_solutions}*{Appendix~B}), but we keep the diagonality assumption for convenience.
We begin with a slight variant of \cite{vaughan1997hardy}*{Lemma~4.9}.
\lem{
\label{LEM:sum-S_0(n)-trivially}
If $N>0$, then $\sum_{n\asymp N} n^{-m}\abs{S_{\bm{0}}(n)}\ll_\eps N^{(4-m)/3+\eps}$.
}

\pf{
$F$ is diagonal, so by Proposition~\ref{PROP:pointwise-bound} (or \cite{vaughan1997hardy}*{Lemma~4.7}) and \Holder, $n^{-m}S_{\bm{0}}(n)\ll O(1)^{\omega(n)}n^{1-m/2}\map{cub}(n)^{m/6}$.
If $n_3\defeq \map{cub}(n)$ (so that $n_3$ is cube-full), then it follows that
\mathd{
\sum_{n\asymp N} n^{-m}\abs{S_{\bm{0}}(n)}
\ll_\eps \psum_{n_3\ll N} (N/n_3)\cdot N^{1-m/2+\eps}n_3^{m/6}
\asymp \psum_{N_3\ll N} N_3^{1/3}\cdot N^{2-m/2+\eps}N_3^{m/6-1},
}
where $N_3$ ranges over $\set{1,2,4,8,\dots}$.
Since $1/3+(m/6-1) = (m-4)/6\geq 0$, the right-hand side is $\ll_\eps N^{(m-4)/6+\eps}\cdot N^{2-m/2+\eps} = N^{(4-m)/3+2\eps}$, as desired.
}

In particular, by Lemma~\ref{LEM:sum-S_0(n)-trivially} (or \cite{vaughan1997hardy}*{proof of Lemma~4.9}), the singular series
\mathd{
\mathfrak{S}
\defeq \sum_{n\geq1} n^{-m}S_{\bm{0}}(n)
}
converges absolutely for $m\geq 5$.
On the other hand, the real density
\mathd{
\sigma_{\infty,w}
\defeq \lim_{\eps\to0} \,(2\eps)^{-1}\int_{\abs{F(\bm{x})}\leq\eps} d\bm{x}\, w(\bm{x})
}
is $O_{F,w}(1)$ (as one can show using \cite{heath1996new}*{Theorem~3}, for instance).

Yet by \cite{heath1996new}*{Lemma~13},
$\wt{I}_{\bm{0}}(n) = \sigma_{\infty,w} + O_A((n/Y)^A)$
for all sufficiently small $n\ll Y$---hence
for all $n\geq1$
(since $\wt{I}_{\bm{0}}(n) \ll 1$ always,
by \cite{heath1996new}*{Lemma~16}).
On the other hand, $\wt{I}_{\bm{0}}(n)=0$ for all sufficiently large $n\gg Y$
(by Proposition~\ref{PROP:rigorous-modulus-cutoff-Y}).
So by Lemma~\ref{LEM:sum-S_0(n)-trivially}, the sum $\sum_{n\geq1} n^{-m}S_{\bm{0}}(n)\wt{I}_{\bm{0}}(n)$ simplifies to
\mathd{
\left(\mathfrak{S} - \psum_{N\gg Y} O_\eps(N^{(4-m)/3+\eps})\right) \sigma_{\infty,w}
+ \psum_{N\ll Y} O_{A,\eps}\left(N^{(4-m)/3 + \eps} (N/Y)^{A}\right),
}
where $N$ ranges over $\set{1,2,4,8,\dots}$.
For $A = (m-3)/3$, both $N$-sums are geometric series peaking at $N\asymp Y$ (since we have assumed $X\geq 1$).
Since $n^{-m}S_{\bm{0}}(n)=n^{-(m-1)/2}\wt{S}_{\bm{0}}(n)$, it follows that if $m\geq 5$, then
\mathd{
X^{m-3}\sum_{n\geq1}
n^{-(m-1)/2}\wt{S}_{\bm{0}}(n)\wt{I}_{\bm{0}}(n)
= X^{m-3}\cdot [\sigma_{\infty,w}\mathfrak{S}
+ O_\eps(X^{(4-m)/2+\eps})].
}
On the other hand, for all $m\geq 4$,
\mathd{
X^{m-3}\sum_{n\geq1}
n^{-(m-1)/2}\wt{S}_{\bm{0}}(n)\wt{I}_{\bm{0}}(n)
\ll X^{m-3}\sum_{n\geq1}
n^{-m}\abs{S_{\bm{0}}(n)}\cdot \bm{1}_{n\ll Y}
\ll_\eps X^{m-3+\eps}
}
by \Holder and \cite{vaughan1997hardy}*{Lemma~4.9}.

\rmk{
\label{RMK:II.B.4}
One may well do better
by analyzing the Dirichlet series $\sum_{n\geq 1} n^{-s}S_{\bm{0}}(n)$,
using the derivative bound $\partial_n^k I_{\bm{0}}(n) \ll_k n^{-k}X^m$
(stated in \cite{heath1996new}*{Lemma~16} for $k=0,1$,
% same bound for general \bm{c} is in H-B 1998.
but valid for $k\geq0$ with little change in proof).
It would be interesting to extend the above analysis to the case $m=4$
(with powers of $\log{X}$ expected to occur for certain $F$'s);
cf.~\cite{browning2009quantitative}*{\S8.3.3's heuristic analysis of the Fermat cubic surface}.
% Also this is a good chance to learn about universal torsors and see if there's a nice heuristic explanation for the Manin--Peyre constant in that setting.
}

\chapter{Using hypotheses on average}
% (Using) Average hypotheses
% (Using) Hypotheses on average
\label{CHAP:using-hypotheses-on-average}

% \subsection{Thoughts on naive attacks/heuristics for the large sieve}

% Consider
% \mathds{
% &\sum_{n,n'\in I} \bm{1}_{d\perp nn'}\mu(n)^{2m_\ast}\mu(n')^{2m_\ast}\cdot Z^m
% \prod_{p\mid nn', p\nmid \gcd(n,n')} \frac{\card{\mcal{V}(\FF_p)}/p - \card{\PP^{m-3}(\FF_p)}}{p^{(m-3)/2}} \\
% &\prod_{p\mid \gcd(n,n')} \frac{[\rho(p)/p+\rho(p)(\rho(p)-1)/p^2] - 2\rho(p)\card{\PP^{m-3}(\FF_p)}/p + \card{\PP^{m-3}(\FF_p)}^2}{p^{m-3}}.
% }

% % Perhaps one should sum over $nn' = k$.
% Or better, expand the product as a sum over divisors $e\mid n,n'$, like
% \mathds{
% &\sum_{n,n'\in I} \bm{1}_{d\perp nn'}\mu(n)^{2m_\ast}\mu(n')^{2m_\ast}\cdot Z^m
% \prod_{p} \left(\frac{\card{\mcal{V}(\FF_p)}/p - \card{\PP^{m-3}(\FF_p)}}{p^{(m-3)/2}}\right)^{v_p(nn')} \\
% &\sum_{e\mid n,n'}\prod_{p\mid e} \frac{\rho(p)/p-\rho(p)/p^2}{p^{m-3}}.
% }
% This becomes
% \mathd{
% Z^m\sum_{e\geq 1} O(1)\cdot \left(\sum_{n\in I} \bm{1}_{d\perp n}\bm{1}_{e\mid n}\mu(n)^{2m_\ast} \prod_{p\mid n}\frac{\card{\mcal{V}(\FF_p)}/p - \card{\PP^{m-3}(\FF_p)}}{p^{(m-3)/2}}\right)^2.
% }



% why is the suspected main term (dihedral Artin $L$-function, maybe) seemingly appearing conditionally here but unconditionally in the original singular series?
% (EDIT: actually even without cancellation over $n$ one expects $(N/e)\cdot 1/N^{1/2}$ for the $n$-sum.)

% One would hope that
% \mathd{
% \sum_{r\leq Y^2}\frac{Z^m}{r}\frac{1}{r^{(m-3)/2}}\left\lvert\sum_{\bm{x}\in\ZZ^m: F(\bm{x})\equiv0\bmod{r}} \int_{\bm{t}\in\RR^m} d\bm{t}\, w(\bm{t}) e_r(-Z\bm{x}\cdot\bm{t})\right\rvert
% \ll Z^m\cdot Y^{1-\delta},
% }
% ideally with $\delta = 1/2$.
% If $w(\bm{t}) = e^{-\norm{\bm{t}}^2}$ is e.g.~a Gaussian the $\bm{t}$-integral should be clean, something like $\pi^{m/2}\exp(-\pi^2\norm{Z\bm{x}/r}^2)$?

% Toy problem: for square-free $r\ll Y^2$, want to improve on the trivial bound
% \mathd{
% r^{-m_\ast/2}\sum_{\bm{c}\in\ZZ^m} w(\bm{c}/Z) \prod_{p\mid r} (\card{V_{\bm{c}}(p)} - \card{\PP^{m_\ast}(p)})
% \ll Z^m.
% }
% The case $r=p$ a prime is already interesting; however it would be worth thinking about if there's a way to use compositeness, say for $r=pq$ with $p,q\ll Y$ distinct.

% It might be better to replace $\PP^{m_\ast}(p)$ with the hyperplane section $\PP^{m-2}_{\bm{c}}(p)$, especially when $\bm{c}=\bm{0}$.
% But this might not be too relevant, since the hard range is $r\gg Z$.

% \todo[inline]{Maybe try shifted Gaussian.
% Or maybe there is another special function for this family.}

% What is
% \mathd{
% \sum_{\bm{c}\in\ZZ^m} w(\bm{c}/Z) \card{V_{\bm{c}}(p)}
% - \sum_{p\mid \bm{c}} w(\bm{c}/Z) \card{V(p)}
% }


% Maybe better: $\sum w(\bm{c}/Z)\wt{S}_{\bm{c}}(p)$ seems to be close to $\frac{\pi^{m/2}Z^m}{p^{(m+1)/2}}(p-1)$ if $Z\gg p$, say.

% On the other hand, $\sum w(\bm{c}/Z)\wt{S}_{\bm{c}}(p)^2$ is
% \mathd{
% \frac{\pi^{m/2}Z^m}{p^{m+1}}\sum_{a,b}\sum_{\bm{x},\bm{y}} e_p(aF(\bm{x})+bF(\bm{y}))\sum_{\bm{d}\in\ZZ^m}\exp(-\pi^2Z^2\norm{\bm{x}+\bm{y}+p\bm{d}}^2/p^2).
% }
% Let $\bm{w}=\bm{x}+\bm{y}+p\bm{d}$; then this becomes
% \mathd{
% \frac{\pi^{m/2}Z^m}{p^{m+1}}\sum_{a,b}\sum_{\bm{x}\bmod{p}}\sum_{\bm{w}\in\ZZ^m}
% e_p(aF(\bm{x})+bF(\bm{w}-\bm{x}))\sum_{\bm{d}\in\ZZ^m}\exp(-\pi^2Z^2\norm{\bm{w}}^2/p^2).
% }
% If $Z\gg p$, this becomes roughly $\frac{\pi^{m/2}Z^m}{p^{m+1}}\sum_{a,b}\sum_{\bm{x}\bmod{p}}e_p(aF(\bm{x})-bF(\bm{x}))$, which simplifies to $\frac{\pi^{m/2}Z^m}{p^{m+1}}\sum_{\bm{x}} (p\bm{1}_{F(\bm{x})=0} - 1)^2 = \frac{\pi^{m/2}Z^m}{p^{m+1}}(p^2[1+(p-1)\rho(p)] - 2p[1+(p-1)\rho(p)] + p^m)$.

\section{Using large-sieve hypotheses}
% Using average \texpdf{$L$}{L}-function hypotheses
\label{SEC:using-L-function-hypotheses-on-average}

Let $m\defeq 6$ and $F\defeq x_1^3+\dots+x_6^3$.
Recall, from Example~\ref{EX:6-cubes-zero-locus},
the conditional near-optimal bound $M_2(X)\ll_\eps X^{3+\eps}$, and the underlying Hypothesis HW (practically amounting to ``automorphy and GRH'') for the standard Hasse--Weil $L$-functions $L(s,V_{\bm{c}})$ associated to \emph{smooth} hyperplane sections $V_{\bm{c}}$ (see Example~\ref{EX:one-sided-first-order-approximations-of-Phi}).

Since automorphy (in full generality)---and especially GRH---might not be proven for some time,
it is therefore natural to ask whether ``Hypothesis HW'' can be weakened.
To quote \cite{hooley1986HasseWeil}*{pp.~51--52}:
``The removal of the dependence of our work on the Riemann hypothesis is an obvious desideratum.
Some weakening of the hypothesis is certainly possible either by substituting some form of zero density requirement or by insisting merely that the zeros of the Hasse-Weil $L$-functions be to the left of some vertical line lying to the right of the critical line $\sigma=2$.
Yet is has not seemed worthwhile to explore such developments here because the principles of the method would be obscured and because we cannot predict the precise form of the first serviceable alternative to the Riemann hypothesis that might subsequently be established.''
(Regarding the question of automorphy, see Appendix~\ref{CHAP:natural-modularity-questions} for more details in the specific setting of $L(s,V_{\bm{c}})$'s above.)

In fact, by \cite{density_beats_hua_when},
it suffices to assume automorphy,
along with a ``density hypothesis for zeros of height $\lessapprox 1$''---an upper-bound statement
on certain exceptional counts $N(\sigma,\mcal{F},T)
\defeq\#\set{\pi\in\mcal{F}:L(s,\pi)\;\textnormal{has a zero in}\;[\sigma,1]\times[-T,T]}$---of the form
(cf.~\cite{density_beats_hua_when}*{\S6's density hypothesis, with $l(\sigma)\defeq 2(1-\sigma)+\eps$})
\begin{equation}
\label{INEQ:zero-density-hypothesis-for-somewhat-low-lying-zeros}
N(\sigma,\mcal{F},T)\ll_\eps T^{O(1)}\card{\mcal{F}}^{2(1-\sigma)+\eps},
\quad\textnormal{uniformly over $T,\card{\mcal{F}}\geq1$ and $\sigma\in[1/2,1]$}
\end{equation}
(optimal, up to $\eps$, among positive linear-in-$\sigma$ exponents in the $\card{\mcal{F}}$-aspect;
but ``arbitrarily polynomially poor'' in the $T$-aspect).
In general, \eqref{INEQ:zero-density-hypothesis-for-somewhat-low-lying-zeros} is morally provable under an ``optimal'' large sieve inequality over $\mcal{F}$,
i.e.~``approximate $\ell^2$ orthogonality'' of the matrix $\RR^{\card{I}}\to\RR^{\card{\mcal{F}}}$ defined by $n\mapsto (\lambda_\pi(n))_{\pi\in\mcal{F}}$---but currently a ``\Lindelof on average'' hypothesis over $\mcal{F}$ also plays a role in rigorous proof \cite{axiomatize_bombieri-huxley}.
For instance, \eqref{INEQ:zero-density-hypothesis-for-somewhat-low-lying-zeros} is ``close'' to known for the Bombieri--Vinogradov family $\set{\chi\bmod{q}: q\leq Q}$.

Yet eventually, I realized that
a large sieve \emph{by itself}---if true---would suffice
for the original goal of ``recovering'' the main results of \cites{hooley1986HasseWeil,hooley_greaves_harman_huxley_1997,heath1998circle}.
See \cite{wang2021_large_sieve_diagonal_cubic_forms} for details.
Let us roughly compare the argument to that of Hooley and Heath-Brown.
\pf{
[Outline of conditional $M_2(X)\ll_\eps X^{3+\eps}$ proofs]
Let $Y\defeq X^{3/2}$ and $Z\defeq X^{1/2+\eps_0}$ (as in Definition~\ref{DEFN:delta-method-cutoff-parameters}).
For a suitable weight $w$, it suffices to bound the expression \eqref{EXPR:main-delta-method-quantity}.
The contribution to \eqref{EXPR:main-delta-method-quantity} from the locus $F^\vee(\bm{c})=0$ can be unconditionally bounded by $O_{\eps_0}(X^{3+O(\eps_0)})$ (and in fact, it captures the main terms of HLH; see Chapter~\ref{CHAP:isolating-special-solutions}), so we focus on the locus $F^\vee(\bm{c})\neq 0$.
Given $\bm{c}\in\ZZ^m$ with $F^\vee(\bm{c})\neq 0$,
recall the Dirichlet series $\Phi(\bm{c},s)$ from Definition~\ref{DEFN:S_c(n),sqrt-normalized-S_c(n),and-Phi(c,s)},
and let $\Psi_1\defeq 1/L(s,V_{\bm{c}})$ and $\Psi_2\defeq \Phi L$,
so that $\Phi = \Psi_1\Psi_2$.
Let $a'_{\bm{c}}(n)\defeq [n^s]\Psi_2$ denote the $n$th coefficient of the ``error'' $\Psi_2$.
Then the following hold (and can be proven using Example~\ref{EX:one-sided-first-order-approximations-of-Phi} and Proposition~\ref{PROP:pointwise-bound}):
\begin{enumerate}[{label=[B\arabic*']}]
    \item For positive $N\leq Z^3$,
    the first moment $\smallpsum_{\bm{c}\ll Z}\sum_{n\asymp N} \abs{a'_{\bm{c}}(n)}$
    is $O_\eps(Z^{m+\eps}N^{1/2})$.
    
    (To first order,
    this follows from
    \cite{hooley1986HasseWeil}*{pp.~78--79, analysis of $Q(\bm{m};k_2)$},
    % \cite{hooley1986HasseWeil}*{pp.~78--79, $\ell^1$ analysis of $\abs{Q(\bm{m};k_2)}$ via Lemmas~9,~11, and~12},
    % \cite{hooley_greaves_harman_huxley_1997}*{\S4, par.~1, derivations of (21)--(22)},
    or alternatively from
    \cite{heath1998circle}*{Lemma~5.2}.)
    
    \item For positive $N\leq Z^3$,
    the second moment $\smallpsum_{\bm{c}\ll Z}(\sum_{n\asymp N} \abs{a'_{\bm{c}}(n)})^2$
    is $O_\eps(Z^{m+\eps}N)$.
    
    (This follows directly from \cite{wang2021_large_sieve_diagonal_cubic_forms}*{\S2.4, Proposition~2.16}.
    This is a surprisingly delicate point; see Remark~\ref{RMK:[B2']-intuition} below.)
\end{enumerate}
The arguments in \cites{hooley1986HasseWeil,hooley_greaves_harman_huxley_1997} and \cite{heath1998circle}
can then loosely be interpreted as
\begin{enumerate}[label=(H\arabic*)]
    \item using partial summation over $n\asymp N\ll Y$ to ``factor out'' $\wt{K}\defeq n^{-(m-1)/2}\wt{I}_{\bm{c}}(n)$ from $\sum_{n} \wt{S}\wt{K}$
    (though in fact,
    partial summation is more like
    a certain ``weighted decoupling'' over $n$),
    and then bounding
    the $\wt{K}$-contribution
    in $\ell^\infty(\set{n\asymp N})$;
    
    \item expanding $\wt{S}=\wt{S}_{\bm{c}} = \mu_{\bm{c}}\ast a'_{\bm{c}}$ using $\Phi = \Psi_1\Psi_2$;
    
    \item using GRH to bound the $\Psi_1$-contribution in $\ell^\infty(\set{\bm{c}\ll Z})$;
    and
    
    \item using [B1'] \emph{afterwards},
    to bound the $\Psi_2$-contribution in $\ell^1(\set{\bm{c}\ll Z})$.
\end{enumerate}
Routine \emph{dyadic bookkeeping} then bounds \eqref{EXPR:main-delta-method-quantity} by $O_{\eps_0}(X^{3+O(\eps_0)})$, under Hypothesis HW.
But in fact, by \cite{wang2021_large_sieve_diagonal_cubic_forms}, one can replace Hypothesis HW with a clean ``elementary GRH on average in $\ell^2$'' statement, which can in turn be reduced to an ``optimal'' large sieve inequality for the $L$-function family $\bm{c}\mapsto L(s,V_{\bm{c}})$.
}

\rmk{
\label{RMK:need-for-new-integral-decay-estimates}
In the aforementioned \emph{dyadic bookkeeping},
each dyadic range of \emph{moduli} $n$
contributes \emph{roughly equally} to the final bound $O_{\eps_0}(X^{3+O(\eps_0)})$.
Thus in Chapter~\ref{CHAP:using-mean-value-L-function-predictions} we will need new integral bounds that decay, as $n\to0$, fairly uniformly over $\bm{c}$.
We will develop these (and other estimates) in Chapter~\ref{CHAP:discriminating-pointwise-estimates}.
}

\rmk{
\label{RMK:using-partial-summation-to-factor-out-integral}
% https://en.wikipedia.org/wiki/Summation_by_parts#Newton_series
Roughly speaking, \cite{wang2021_large_sieve_diagonal_cubic_forms} uses partial summation specifically to deduce, for a suitable probability measure $\nu=\nu_N$ supported on $[N,2N]$, that
\mathd{
\left\lvert{\sum_{n\in [N,2N)} \wt{S}(n)\wt{K}(n)}\right\rvert
\ll \left(\sup_{n\in[N,2N]}\left(\abs{\wt{K}(n)}, N\cdot \abs{\partial_n\wt{K}(n)}\right)\right)
\int_{x\in [N,2N]} d\nu(x)\, \abs{B([N,x))},
}
where $B(J)\defeq \sum_{n\in J} \wt{S}(n)$ for intervals $J$.
The actual argument is subtler, but it builds on this idea.
The point is that the large sieve only accepts uniform vectors;
yet our ``initially given'' vectors (in the delta method) are only approximately uniform over $\bm{c}$,
due to variation in the archimedean component $\wt{K}$, \emph{and} in the error factor $a'$.
% It is plausible (e.g.~using ideas from \cite{wang2021_HLH_vs_RMT}) that one could do double summation by parts in $n_0,n_1$, but for a given $\bm{c}$ (with $F^\vee(\bm{c})\neq0$), the dominant terms in the sum $\sum_{n_0\geq1} a'_{\bm{c}}(n_0)$ are sparse enough that it may be difficult to get additional cancellation.
}

\rmk{
\label{RMK:[B2']-intuition}
What is [B2'] really saying?
% Fix $\bm{c}$.
Say $n$ is \emph{good} if $n\perp F^\vee(\bm{c})$, and \emph{purely bad} if $n\mid F^\vee(\bm{c})^\infty$;
then see Table~\ref{TAB:[B2']-contributions} below for a breakdown of [B2'].
(Note that $\EE_{c\ll Z}[\map{sq}(c)^{1/2}]\ll_\eps Z^\eps$; this is surprisingly delicate \cite{wang2021_large_sieve_diagonal_cubic_forms}*{Remark~2.17}.)

\begin{table}[ht]
\centering
%Different $n$-type contributions to\dots
\caption{Types of contributions to $\smallpsum_{\bm{c}\ll Z}(\sum_{n\asymp N} \abs{a'_{\bm{c}}(n)})^2$ in [B2']}
% \rule{0pt}{9ex}
\vspace*{1ex}
\begin{tabular}{c|c|c}
    \rule{0pt}{2.5ex}
    If $n$ is (frequency) & then $a'_{\bm{c}}(n)$ is $O_\eps(n^\eps)\cdot \fbox{?}$ & contributing $\ll$ (up to $\eps$) \\ [0.5ex]
    \hline \rule{0pt}{2.5ex}
    good, sq.-free (common) & $n^{-1/2}$ & $Z^m(N\cdot N^{-1/2})^2 = Z^m N$ \\
    good, sq.-full (rare) & $1$ & $Z^m(N^{1/2}\cdot 1)^2 = Z^mN$ \\
    purely bad (very rare) & $n^{1/2}\prod_{j\in[m]}\map{sq}(c_j)^{1/4}$ & $Z^m(1\cdot N^{1/2})^2 = Z^mN$ \\
    arbitrary (factor mixture) & mixture & mixture \\
\end{tabular}
\label{TAB:[B2']-contributions}
\end{table}

Looking ahead to Chapters~\ref{CHAP:discriminating-pointwise-estimates}--\ref{CHAP:using-mean-value-L-function-predictions}, the first two rows (``sources of $\eps$'') in Table~\ref{TAB:[B2']-contributions} will inspire a better approximation of $\Phi$, while the third source (from bad $n$) will present a tougher challenge (hence the ``B'' in [B1']--[B2']).
}

Let us now state \cite{wang2021_large_sieve_diagonal_cubic_forms}'s hypotheses more precisely.
We first state a hypothesis
that can morally be thought of
\emph{either as} an elementary GRH-on-average statement (in a certain range of parameters),
\emph{or as} an elementary manifestation of a zero-density hypothesis.
Let $\Psi_1$ be an approximation of $\Phi$, in the sense of Definition~\ref{DEFN:one-sided-first-order-Euler-product-approximation-of-Phi}.

\defn{
\label{DEFN:second-moment-hypothesis-for-Psi_1}
% Given $\Psi_1$ as in Definition~\ref{DEFN:one-sided-first-order-Euler-product-approximation-of-Phi},
The \emph{second-moment hypothesis} for $\Psi_1$ refers to the statement that
for all $(Z,\beta)\in\RR_{>0}^2$,
for all (positive reals) $Y\leq Z^3$,
for all (positive reals) $N\leq \beta Y$,
and for all (real) intervals $I\belongs[N/2,2N]$,
we have
\mathd{
\psum_{\bm{c}\in[-Z,Z]^m}
\left\lvert
\sum_{n\in I}b_{\bm{c}}(n)
\right\rvert^2
\ll_{\beta,\eps} Z^\eps
\max(Z^m,Y)\cdot N
\qquad\textnormal{(uniformly over $Z,Y,N,I$)},
}
where we restrict the $\bm{c}$-sum to $\set{\bm{c}\in\ZZ^m:F^\vee(\bm{c})\neq0}$.
(Here $Z^m\gg Y$,
since $m\geq 3$.
But we write $\max(Z^m,Y)$ in connection with the large sieve to be discussed soon.)
}

\rmk{
[Counting, or ``density'', interpretation]
Since $m\geq 4$,
the family $\mcal{F}\approx [-Z,Z]^m$ is in fact \emph{strictly} larger than
the range of moduli, $Y$.
Given $N,I$ in Definition~\ref{DEFN:second-moment-hypothesis-for-Psi_1},
we tolerate roughly $\card{\mcal{F}}/N^{2\sigma-1}$ contributions of size $N^{\sigma-1/2}$.
In particular,
we are OK with roughly $\card{\mcal{F}}/N$ ``extremely exceptional'' $\bm{c}$'s with $\abs{\sum_{n\in I}b_{\bm{c}}(n)}
\gg N$ matching the ``heuristic trivial bound''.\footnote{In the full generality of Definition~\ref{DEFN:one-sided-first-order-Euler-product-approximation-of-Phi},
we cannot always ``actually trivially bound'' $\abs{\sum_{n\in I} b_{\bm{c}}(n)}$ for approximations $\Psi_1$ of $\Phi$.}
Contrast with ``almost covering'' problems (about e.g.~small primes modulo $c$),
% e.g. https://mathoverflow.net/questions/217956/update-for-2015-least-prime-of-form-nq1-with-q-prime (but on average over residues and/or moduli)
where---in analogy with the ``Sarnak--Xue density philosophy''---an
% (as explained to us by Sarnak)
% cf.~\url{https://arxiv.org/abs/1905.11165} (\emph{Cutoff on Graphs and the Sarnak-Xue Density of Eigenvalues}, by Golubev--Kamber)
``expander property'' (e.g.~a sufficiently strong prime number theorem)
is often needed near $\sigma=1$.
%In other words, the degree (here deg(F) = 3) growing actually makes the family larger than the modulus; for larger degrees deg(F) >= 4 the issue is that the large sieve loses any additional internal cancellation over the family.
}

Here $b_{\bm{c}}(n)$ ``morally contains'' a M\"{o}bius factor $\mu(n)$.
But at least in favorable situations,
the second-moment hypothesis for the Dirichlet series $\Psi_1=\sum_{n\geq1}b_{\bm{c}}(n)n^{-s}$
can be reduced to a large-sieve hypothesis involving
the family of ``friendlier'' (``M\"{o}bius-free'') Dirichlet coefficient vectors $(a_{\bm{c}}(n))_{n\geq1}$.

\defn{
Call $\Psi_1$ \emph{standard} if $b_{\bm{c}}(n),a_{\bm{c}}(n)
\ll_\eps n^\eps$
holds uniformly over $\bm{c},n$.
}

\ex{
Say $\Psi_1$ is defined as in Example~\ref{EX:one-sided-first-order-approximations-of-Phi}.
Then $\Psi_1$ is standard.
}

We now come to our ``main'' hypothesis:
a large sieve in certain ranges.
\defn{
\label{DEFN:large-sieve-hypothesis-for-Psi_1-or-1/Psi_1}
If $\Psi_1$ is standard,
then let $\gamma\defeq\bm{1}_{n=\map{rad}(n)}\cdot a$,
or let $\gamma\defeq b$;
if $\Psi_1$ is non-standard,
then let $\gamma\defeq b$.
We define the \emph{large-sieve hypothesis} for $\gamma$ to be the statement that
for all $(Z,\beta)\in\RR_{>0}^2$,
and for all (positive reals) $Y\leq Z^3$,
we have
\mathd{
\psum_{\bm{c}\in [-Z,Z]^m}
\left\lvert
\sum_{n\leq 2\beta Y}v_n
\cdot \gamma_{\bm{c}}(n)
\right\rvert^2
\ll_{\beta,\eps} Z^\eps \max(Z^m,Y)\cdot \sum_{n\leq 2\beta Y} \abs{v_n}^2
\quad\textnormal{for all $\bm{v}\in\RR^{\floor{2\beta Y}}$},
}
uniformly over $Z,Y$.
(Here $Z^m\gg Y$,
since $m\geq 3$.
But we write $\max(Z^m,Y)$ to avoid potential confusion with large sieves in other ranges of parameters.)
}

\rmk{
The large-sieve hypothesis for $\gamma\defeq a$
(though excluded from Definition~\ref{DEFN:large-sieve-hypothesis-for-Psi_1-or-1/Psi_1}
for expository convenience)
would directly imply that for $\gamma\defeq\bm{1}_{n=\map{rad}(n)}\cdot a$.
The factor $\bm{1}_{n=\map{rad}(n)}$ is simply ``restriction to square-free moduli $n$''.
}

\rmk{
For concreteness,
say $\Psi_1$ is defined as in Example~\ref{EX:one-sided-first-order-approximations-of-Phi}.
Say $\gamma\defeq \bm{1}_{n=\map{rad}(n)}\cdot a$.
Then $\bm{c}\mapsto (\gamma_{\bm{c}}(n))_{n\geq1}$ is genuinely a family of Hasse--Weil coefficients,
up to $\bm{1}_{n=\map{rad}(n)}$.
For $\Psi_1,\gamma$,
it remains open to prove the associated large sieve
(or satisfactory partial results towards it, if true).
The difficulty (of the problem) seems unclear.
One approach might be to dualize and try to adapt \cite{louvel2011distribution},
at least over $\ZZ[\zeta_3]$.
Alternatively,
one could try elementary geometric and analytic arguments.

For a discussion of large sieves in general
(and what one can or cannot expect to be true),
we refer the reader to \cite{iwaniec2004analytic}*{\S7}, \cite{djankovic2012larger}*{\S1}, and \cite{dunn2021bias}*{\S1.4}.
From the point of view of \emph{geometric} families,
a nice example of a large sieve is \cite{heath1995mean}'s quadratic-character large sieve
(``optimal'' if one restricts to square-free moduli,
% or alternatively replaces the $\ell^2$-norm for ``input vectors $\bm{v}$'' with a suitable $\ell^\infty$-norm
as is OK for most applications),
applied in \cite{perelli1997averages} to quadratic twist families of elliptic curves.
(Interestingly,
the conductors in the latter setting grow significantly faster than those in the former,
but \cite{heath1995mean} applies equally well to the two settings,
as far as the large sieve is concerned.)

Optimistically,
one might hope that
% suitably ``large'' geometric families
% at least geometric families with ``big monodromy'' (in some sense)
a ``natural'' geometric family---especially
% how about Gamma_1(q) counterexample (Iwaniec--Li? see Djankovic) --- even though this is in some sense geometric (associated to some modular curve), maybe the fact that we don't just get elliptic curves (but rather abelian varieties of varying dimensions, and perhaps also more complicated things since we're on X_1(q)/J_1(q) rather than X_0(q)/J_0(q) --- EDIT: actually Shimura associated all Gamma_1 forms to abelian varieties) plays some role.
one with ``big monodromy'' (in some sense)---would
% how about quadratic twist family? presumably vertical monodromy (if we fix a prime p) here is small
have a good chance of satisfying an ``optimal'' large sieve.
For a discussion of the (closely related) ``expected random matrix symmetry type'' of our present $V_{\bm{c}}$ families (with $2\mid m$),
we refer the reader to Chapter~\ref{CHAP:using-mean-value-L-function-predictions}.
}

\rmk{
The philosophy behind \cite{wang2021_large_sieve_diagonal_cubic_forms} should also apply to other problems;
see \S\ref{SEC:other-open-problems-of-a-similar-nature} for some potential examples in the spirit of the problems discussed so far.
In geometric settings in particular,
% like the current one,
one might also wonder about
the structure of the conjecturally relevant automorphic objects;
% or what one can say from an automorphic point of view;
% (perhaps with the aim of eventually applying automorphic techniques);
Appendix~\ref{CHAP:natural-modularity-questions} discusses a basic preliminary question
% along these lines
% on this matter
in this direction.
}

\section{Using average quadratic hypotheses}
\label{SEC:using-quadratic-hypotheses-on-average}

Throughout \S\ref{SEC:using-quadratic-hypotheses-on-average},
let $n\defeq3$,
and let $Q_{\bm{h}}\defeq\sum_{i\in[n]}h_iy_i^2\in\ZZ[\bm{y}]$ for each $\bm{h}\in\ZZ^n$.
Also let $F_0(\bm{y})\defeq y_1^3+\dots+y_3^3$ and $F(\bm{x})\defeq x_1^3+\dots+x_6^3$.

\subsection{A uniform conjecture on ternary quadratic equations}
% A uniform conjecture on integer solutions to ternary quadratic equations
% integer points on affine ternary quadrics

\cnj{
\label{CNJ:uniform-ternary-quadratic-form-representation-conjecture}
\emph{Uniformly} over reals $X,H\geq1$ and pairs $(\bm{h},k)\in\ZZ^3\times\ZZ$ with $\norm{\bm{h}}\in[H,2H]$ and $-h_1h_2h_3k\notin\set{0}\cup(\QQ^\times)^2$,
we have
\mathd{
\#\set{\bm{y}\in\ZZ^3\cap[-X,X]^3
: Q_{\bm{h}}(\bm{y})=k}
\ll_\eps (XH)^\eps
\cdot \left(\frac{X}{\abs{h_1h_2h_3}^{1/3}}
+ \left(X^2H\right)^{1/4}\right).
}
}

\rmk{
In the positive-definite case $\bm{h}\in \ZZ_{>0}^3$,
\cite{golubeva1998nonhomogeneous}*{before the statements of the main results} mentions a similar ``quite plausible'' conjecture asserting the $O(k^{1/4})$-boundedness of the $k$th coefficient of the cusp form of weight $3/2$ corresponding to $Q_{\bm{h}}$,
uniformly over $\bm{h}$.
}

\rmk{
Using \cites{duke1993bounds,heath1996new}'s delta method along the lines of \cite{browning2017twisted},
one can likely give a \emph{heuristic} argument for Conjecture~\ref{CNJ:uniform-ternary-quadratic-form-representation-conjecture},
assuming ``generic square-root cancellation'' over the modulus $q$ (for ``generic'' $\bm{c}$'s).
Some details,
including the analysis of the oscillatory integral for small $q$'s,
would likely require some care to flesh out (cf.~\cite{browning2017twisted}),
because $k\neq0$.

(The ``homogeneous case'' $k=0$ would be easier to analyze---at least under standard hypotheses---but
we have excluded it from Conjecture~\ref{CNJ:uniform-ternary-quadratic-form-representation-conjecture} for simplicity.)
}

\rmk{
The $X/\abs{h_1h_2h_3}^{1/3}$ comes from a loose real-density bound:
note that
\mathds{
&(2\eps)^{-1}\vol\left\{\bm{y}\in\prod_{i\in[3]}[X_i,2X_i]
: Q_{\bm{h}}(\bm{y})\in[k-\eps,k+\eps]\right\} \\
&\ll \frac{X_1X_2X_3}{\max_{i\in[3]}(\abs{h_i}X_i^2)}
\leq \frac{X_1X_2X_3}{\prod_{i\in[3]}(\abs{h_i}X_i^2)^{1/3}}
= \frac{(X_1X_2X_3)^{1/3}}{\abs{h_1h_2h_3}^{1/3}}
\ll \frac{X}{\abs{h_1h_2h_3}^{1/3}}
}
holds uniformly over $(\bm{X},k,\eps)\in(0,X]^3\times\RR\times(0,1]$, say.
(For ``lopsided'' $\bm{h}$,
the bound $X/\abs{h_1h_2h_3}^{1/3}$ can likely be improved at the cost of cleanliness.)
}

\rmk{
Fix $(\bm{h},k)\in\ZZ^3\times\ZZ$ with $h_1+h_2+h_3=0$ and $k=-3F_0(\bm{h})$.
Then $k=-9h_1h_2h_3$,
so $-h_1h_2h_3k=(3h_1h_2h_3)^2$ is a square.
And in fact,
here $Q_{\bm{h}}(\bm{y})=k$ does have many ``trivial'' solutions $\bm{y}\in\ZZ^3$,
% for instance: $y_1-3h_1 = y_2+3h_2$, $y_2-3h_2=y_3+3h_3$, $y_3-3h_3=y_1+3h_1$
including $(t+3h_2,t-3h_1,t)$ for each $t\in\ZZ$.
% h_1(t+3h_2)^2 + h_2(t-3h_1)^2 + (-h_1-h_2)t^2 = 9h_1h_2(h_1+h_2) holds by https://www.wolframalpha.com/input/?i=h_1%28t%2B3h_2%29%5E2+%2B+h_2%28t-3h_1%29%5E2+%2B+%28-h_1-h_2%29t%5E2+%3D+9h_1h_2%28h_1%2Bh_2%29
So in the range
$X^\delta\leq\norm{\bm{h}}\leq\delta X$,
for instance,
$Q_{\bm{h}}(\bm{y})=k$ would have $\gg X$ solutions $\bm{y}\in[-X,X]^3$ as $X\to\infty$.
% the conclusion of Conjecture~\ref{CNJ:uniform-ternary-quadratic-form-representation-conjecture} fails for said $(\bm{h},k)$'s as $X\to\infty$.

Thus in Conjecture~\ref{CNJ:uniform-ternary-quadratic-form-representation-conjecture},
we need \emph{some} nontrivial requirement on $(\bm{h},k)$,
even if $h_1h_2h_3\neq0$.
Conjecture~\ref{CNJ:uniform-ternary-quadratic-form-representation-conjecture}'s assumption ``$-h_1h_2h_3k\notin\set{0}\cup(\QQ^\times)^2$'' is thus quite natural.
}

\subsection{A cubic application via differencing}

To bound $N_F(X)$,
we will first ``dilute and fiber $N_F(X)$ into ternary affine quadrics''
using multidimensional van der Corput differencing
(inspired by \cite{marmon2019hasse}*{Lemma~3.3},
which similarly reduced a quartic problem---in some ranges---to
a problem about mixed cubic-quartic exponential sums)
and then apply Conjecture~\ref{CNJ:uniform-ternary-quadratic-form-representation-conjecture} to bound most of the resulting fibers.

\rmk{
The basic idea of studying ``quadratic fibers'' of cubic Diophantine problems
also appears in
\cites{golubeva1998nonhomogeneous,wooley2013waring},
for instance,
and dates back at least to \cite{linnik1943representation}'s proof that $G(3)\leq7$.
(For more details on $G(3)$ and its history,
see \cite{vaughan2002waring}.)
}


Since Conjecture~\ref{CNJ:uniform-ternary-quadratic-form-representation-conjecture} does not apply to all $(\bm{h},k)\in\ZZ^3\times\ZZ$,
we will also need the following conjecture:
\cnj{
\label{CNJ:bounding-the-locus-of-square-values-of-3h_1h_2h_3F_0(h)}
Uniformly over reals $H>0$,
we have
\mathd{
\#\set{(\bm{h},z)\in(\ZZ^3\setminus\set{\bm{0}})\times\ZZ
: \norm{\bm{h}}\leq H
\;\textnormal{and}\;
3h_1h_2h_3(h_1^3+h_2^3+h_3^3)
= z^2}
\ll_\eps H^{2+\eps}.
}
}

\rmk{
% special locus with $3h_1h_2h_3(h_1^3+h_2^3+h_3^3) = z^2$ a square (probability $H^3/H^6$ generically, but only need to show probability $\ll 1/H$) might be interesting, as an extension of Vaughan--Wooley on $h_1+h_2+h_3=0$ (though they got an asymptotic for the total number of solutions, with almost an asymptotic for the *nontrivial* solutions (see de la Breteche for even more precise results), and we only need an upper bound up to $\eps$).
% For $h_1,h_2$ fixed, the LHS is a quartic in $h_3$, which should be susceptible to elliptic curve height bounds (probably effective?); note that $(h_3,z) = (0,0), (-h_1-h_2,\pm 3h_1h_2h_3)$ are solutions.
% %https://mathoverflow.net/questions/230712/uniform-bounds-on-the-number-of-integer-points-on-a-family-of-elliptic-curves
% %https://mathoverflow.net/questions/239746/birationally-transforming-a-quartic-elliptic-curve

% unlikely intersections

For fixed $h_1,h_2$ with $h_1h_2(h_1^3+h_2^3)\neq0$,
the equation
$z^2=3h_1h_2h_3F_0(\bm{h})$
in $h_3,z$
cuts out a genus $1$ affine plane curve,
which contains the points
$(0,0)$ and $(-h_1-h_2,\pm 3h_1h_2h_3)$---and thus
defines an elliptic (in fact, a Mordell) curve,
namely
% defines an elliptic curve with ``twisted Weierstrass model''
\mathd{
y^2
= 3h_1h_2\left((h_1^3+h_2^3)x^3
+ 1\right),
\quad\textnormal{where}\;
(x,y)\defeq(1/h_3,z/h_3^2).
}

In particular,
we may view
$\set{z^2=3h_1h_2h_3F_0(\bm{h})}$
as an ``elliptic fibration'' over $h_1,h_2$.
Conjecture~\ref{CNJ:bounding-the-locus-of-square-values-of-3h_1h_2h_3F_0(h)} would thus follow from
a certain standard hypothesis on the rank growth of elliptic curves over $\QQ$
(which would itself follow from BSD and GRH,
by \cite{iwaniec2004analytic}*{Proposition~5.21});
cf.~\cite{bonolis2021uniform}*{discussion after Theorem~1.1}.

What can be said unconditionally towards Conjecture~\ref{CNJ:bounding-the-locus-of-square-values-of-3h_1h_2h_3F_0(h)}?
(The fact that we are interested only in certain integral points,
rather than all rational points,
may help.)
}

\prop{
\label{PROP:beat-Hua-by-diluting-and-difference-fibering-N_F(X)-into-ternary-quadratics}
Assume Conjectures~\ref{CNJ:uniform-ternary-quadratic-form-representation-conjecture} and~\ref{CNJ:bounding-the-locus-of-square-values-of-3h_1h_2h_3F_0(h)}.
Then $N_F(X)\ll_\eps X^{45/13+\eps}$ as $X\to\infty$.
(Here $45/13=3.4615\ldots<7/2$.)
}

\pf{
Fix a small absolute constant $c>0$ to be specified later.
Then fix a nonzero nonnegative weight $\nu\in C^\infty_c(\RR^n)$ supported on $[1,1+c]^n$.
(Recall that $n\defeq3$.)

Now let $w(\bm{y},\bm{z})\defeq\nu(\bm{y})\nu(-\bm{z})$.
Then by \Holder and dyadic decomposition,
it certainly suffices to show that $N_{F,w}(X)\ll_{\eps,\nu}X^{45/13+\eps}$.
But
\mathd{
N_{F,w}(X)
= \sum_{a\in\ZZ}N_{F_a,\nu}(X)^2.
}

Now fix a van der Corput differencing set $\mcal{H}\belongs\ZZ^n\cap[0,X]^n$,
% let $t_{\bm{h}}(\bm{y})\defeq\bm{y}+\bm{h}$ denote translation by $\bm{h}$,
let $\nu_X\defeq\nu\circ X^{-1}$ and $w_X\defeq w\circ X^{-1}$ for convenience,
and ``dilute'' $N_{F_a,\nu}(X)$ by $\mcal{H}$ to get
% averaging/diluting/tessellating
\mathd{
N_{F_a,\nu}(X)
\defeq \sum_{\bm{x}_0\in\ZZ^n}
\nu_X(\bm{x}_0)\bm{1}_{F_0(\bm{x}_0)=a}
= \sum_{\bm{x}_0\in\ZZ^{n}}
\card{\mcal{H}}^{-1}
\sum_{\bm{h}\in\mcal{H}}
\nu_X(\bm{x})\bm{1}_{F_0(\bm{x})=a},
}
where $\bm{x}\defeq\bm{x}_0+\bm{h}$.
Then by Cauchy over $\bm{x}_0\ll X$,
it follows uniformly over $a\in\ZZ$ that
\mathds{
N_{F_a,\nu}(X)^2
&\ll X^n\sum_{\bm{x}_0\in\ZZ^n}
\card{\mcal{H}}^{-2}
\sum_{\bm{h}_1,\bm{h}_2\in\mcal{H}}
\nu_X(\bm{x}_2)\bm{1}_{F_0(\bm{x}_2)=a}
\cdot \nu_X(\bm{x}_1)\bm{1}_{F_0(\bm{x}_1)=a} \\
&= X^n\card{\mcal{H}}^{-2}
\sum_{\bm{x}_1,\bm{x}_2\in\ZZ^n}
D(\bm{x}_2-\bm{x}_1) \cdot \nu_X(\bm{x}_2)\bm{1}_{F_0(\bm{x}_2)=a}
\cdot \nu_X(\bm{x}_1)\bm{1}_{F_0(\bm{x}_1)=a} \\
&= X^n\card{\mcal{H}}^{-2}
\sum_{\bm{h}\in\ZZ^n} D(\bm{h})
\sum_{\bm{x}_1\in\ZZ^n}
\nu_{X,\bm{h}}(\bm{x}_1)
\bm{1}_{F_0(\bm{x}_1+\bm{h})=F_0(\bm{x}_1)=a},
}
where $D(\bm{h})\defeq
\#\set{(\bm{h}_1,\bm{h}_2)\in\mcal{H}^2
: \bm{h}=\bm{h}_2-\bm{h}_1}
\leq \#\mcal{H}$
and $\nu_{X,\bm{h}}(\bm{y})\defeq
\nu_X(\bm{y}+\bm{h})\nu_X(\bm{y})$.
Summing over $a\in\ZZ$,
and letting $\bm{y}\defeq\bm{x}_1$,
leads to the bound
\mathds{
N_{F,w}(X)
&\ll X^n\card{\mcal{H}}^{-2}
\sum_{\bm{h}\in\ZZ^n} D(\bm{h})
\sum_{\bm{y}\in\ZZ^n}
\nu_{X,\bm{h}}(\bm{y})
\bm{1}_{F_0(\bm{y}+\bm{h})=F_0(\bm{y})} \\
&\ll X^n\card{\mcal{H}}^{-1}
\sum_{\bm{h}\in\mcal{H}-\mcal{H}}
\sum_{\bm{y}\in\ZZ^n}
\bm{1}_{\bm{y}+\bm{h},\bm{y}\in[X,(1+c)X]^n}
\bm{1}_{F_{\bm{h}}(\bm{y})=0},
}
where $F_{\bm{h}}(\bm{y})\defeq
F_0(\bm{y}+\bm{h})-F_0(\bm{y})$.

Inspired by \cite{marmon2019hasse},
we now let $K\defeq cX^\theta$ for an exponent $\theta\in(0,1]$ to be specified,
and let $\mcal{H}\defeq\set{\bm{d}\in[0,cX]\times[0,cK]^{n-1}:6\mid\bm{d}}$.
The key point is that \emph{if}
$(\bm{h},\bm{y})\in(\mcal{H}-\mcal{H})\times[X,(1+c)X]^n$
and $\abs{h_1}\geq K$,
then $\abs{h_2},\abs{h_3}\leq cK\leq c\abs{h_1}$,
so the mean value theorem for $y\mapsto y^3$ implies
\mathd{
\abs{F_{\bm{h}}(\bm{y})}
\geq \abs{h_1}\cdot(3-o_{c\to0}(1))X^2
- (\abs{h_2}+\abs{h_3})\cdot(3+o_{c\to0}(1))X^2
\geq \abs{h_1}X^2
}
(provided we chose $c$ to be sufficiently small),
whence $F_{\bm{h}}(\bm{y})\neq0$.
Thus
\mathd{
N_{F,w}(X)
\ll X^n\card{\mcal{H}}^{-1}
\sum_{\bm{h}\in\ZZ^n}
\bm{1}_{\bm{h}\in[-K,K]^n}
\bm{1}_{6\mid\bm{h}}
\sum_{\bm{y}\in\ZZ^n}
\bm{1}_{\bm{y}\in[X,(1+c)X]^n}
\bm{1}_{F_{\bm{h}}(\bm{y})=0}.
}

For each $\bm{h},\bm{y}$ contributing to the sum above,
let $\bm{h}'\defeq\bm{h}/6\in\ZZ^n\cap[-K/6,K/6]^n$
and $\bm{y}'\defeq\bm{y}+3\bm{h}'\in\ZZ^n\cap[(1-c)X,(1+2c)X]^n$;
then
\mathd{
F_{\bm{h}}(\bm{y})
= \sum_{i\in[n]}[(y_i+h_i)^3-y_i^3]
= \sum_{i\in[n]}[3h_i(y_i+h_i/2)^2 + h_i^3/4]
= 18Q_{\bm{h}'}(\bm{y}')+54F_0(\bm{h}').
}
Let $k_{\bm{h}'}\defeq -3F_0(\bm{h}')$;
then by Conjectures~\ref{CNJ:bounding-the-locus-of-square-values-of-3h_1h_2h_3F_0(h)} and~\ref{CNJ:uniform-ternary-quadratic-form-representation-conjecture},
and the ``trivial bound''
\mathds{
&\#\set{\bm{y}'\in\ZZ^n\cap[-2X,2X]^n
: Q_{\bm{h}'}(\bm{y}')=k_{\bm{h}'}} \\
&\ll X^n\bm{1}_{\bm{h}'=\bm{0}}
+ X^{n-1}\left(
\sum_{\pi\in S_3}
\bm{1}_{h'_{\pi(1)}=h'_{\pi(2)}=0}
+ \sum_{\pi\in S_3}
\bm{1}_{h'_{\pi(1)}=h'_{\pi(2)}+h'_{\pi(3)}=0}\right)
+ O_\eps(X^{n-2+\eps})
}
(proven later, soon below),
we conclude that $N_{F,w}(X)$ is
\mathds{
&\ll_{c,\eps}
\frac{X^n(XK)^\eps}{1+XK^{n-1}}
\left(
X^n + KX^{n-1} + K^2X^{n-2}
+ \psum_{\bm{h}'\ll K}
\left(\frac{X}{\abs{h'_1h'_2h'_3}^{1/3}}
+ \left(X^2K\right)^{1/4}\right)
\right) \\
&\ll_{c,\eps} \frac{X^{n+2\eps}}{1+XK^{n-1}}
\left(
X^n + KX^{n-1} + K^2X^{n-2}
+ K^{2n/3}X
+ K^n\left(X^2K\right)^{1/4}
\right),
}
where the sum over $\bm{h}'$ is restricted to $\set{-h'_1h'_2h'_3k_{\bm{h}'}\notin\set{0}\cup(\QQ^\times)^2}$.
Finally,
recall that $n\defeq3$,
and set $\theta\defeq(4/13)\cdot(5/2)=10/13$ to obtain the desired bound $N_{F,w}(X)\ll_\eps X^{45/13+\eps}$.
}

\pf{
[Loose ends]
To derive the ``trivial bound'' for
$\#\set{\bm{y}'\ll X
: Q_{\bm{h}'}(\bm{y}')=k_{\bm{h}'}}$,
note that
\begin{enumerate}[(1)]
    % \item uniformly over
    % $(a,t)\in(\ZZ\setminus\set{0})\times\ZZ$,
    % we have $\#\set{u\in[-2X,2X]
    % : au^2=t}\ll 1$;
    
    \item uniformly over
    $(a,b,t)\in(\ZZ\setminus\set{0})^2\times\ZZ$,
    we have
    \mathd{
    \#\set{(u,v)\in[-2X,2X]^2
    : au^2+bv^2=t}
    \ll_{\eps} X\cdot\bm{1}_{t=0}
    + (\abs{ab}X)^\eps,
    }
    by using
    $\tau(\abs{t})\cdot\bm{1}_{t\neq0}\ll_\eps\abs{t}^\eps$
    if $-ab$ is a square,
    and by working in $\QQ(\sqrt{-ab})$ otherwise
    % for the (\abs{ab}X)^\eps fact, note that class group can only help, while the unit group can hurt in the indefinite case but not significantly (the units grow exponentially)
    (noting that $p^e$ has at most $(e+1)^2=\tau(p^e)^2$ ideal divisors in $\QQ(\sqrt{-ab})$,
    and that if $-ab>0$ then the fundamental unit $\varepsilon$ is $\geq\min(1+\sqrt2,\frac{1+\sqrt5}{2})$);
    
    \item if $h'_1h'_2h'_3\neq0$,
    then $k_{\bm{h}'}-h'_3(y'_3)^2$ is \emph{nonzero} for all but at most $O(1)$ integers $y'_3$;
    
    \item if $\pi\in S_3$
    and $h'_{\pi(1)}=0$,
    but $h'_{\pi(2)}h'_{\pi(3)}(h'_{\pi(2)}+h'_{\pi(3)})\neq0$,
    then $h'_{\pi(2)}h'_{\pi(3)}k_{\bm{h}'}\neq0$;
    and
    
    \item if $\pi\in S_3$
    and $h'_{\pi(1)}=h'_{\pi(2)}(h'_{\pi(2)}+h'_{\pi(3)})=0$,
    but $\bm{h}'\neq\bm{0}$,
    then $h'_{\pi(3)}\neq0$,
    whence for each $y'_{\pi(1)},y'_{\pi(2)}$ there are $\leq O(1)$ integers $y'_{\pi(3)}$ with $Q_{\bm{h}'}(\bm{y}')=k_{\bm{h}'}$.
\end{enumerate}
The desired ``trivial bound'' follows
upon combining (1)--(4).
% in a fairly simple manner.
}

\subsection{Commentary on the proof of the proposition}

\rmk{
One can certainly relax the assumptions of Proposition~\ref{PROP:beat-Hua-by-diluting-and-difference-fibering-N_F(X)-into-ternary-quadratics}.
For example,
an ``$\ell^1$-average'' version of Conjecture~\ref{CNJ:uniform-ternary-quadratic-form-representation-conjecture}
(over $\bm{h}\ll K$)
would suffice in place of Conjecture~\ref{CNJ:uniform-ternary-quadratic-form-representation-conjecture}.
Also,
we could relax
the exponent in Conjecture~\ref{CNJ:bounding-the-locus-of-square-values-of-3h_1h_2h_3F_0(h)}
from $2+\eps$ to $2.6+\eps$.

In a more qualitative direction,
note that $y\mapsto y^3$ is increasing.
So if $\bm{h}\neq\bm{0}$,
and $\bm{h}\geq0$
(or $\bm{h}\leq0$),
then $F_{\bm{h}}(\bm{y})>0$
(resp.~$F_{\bm{h}}(\bm{y})<0$)
holds for all $\bm{y}\in\RR^n$.
% \subsection{Coefficient sizes and signature}
Thus we only
need Conjecture~\ref{CNJ:uniform-ternary-quadratic-form-representation-conjecture}
in the \emph{indefinite} case
(i.e.~when $h_1,h_2,h_3$ do not all have the same sign)---and
in fact,
we may further assume that
$\max(\bm{h})\asymp\abs{\min(\bm{h})}$.
}

\rmk{
At the beginning,
we localized to $x\in[X,(1+c)X]$.
In fact,
we could have localized to
$\abs{x-X}\leq X/\log{X}$ or $\abs{x-X}\leq X^{1-\eps_0}$, say,
but it is unclear if this would help.
}

\rmk{
For approximate lattices $\mcal{H}$,
the estimate $D(\bm{h})\ll\#\mcal{H}$ is close to the truth.
% It may be interesting (but at first glance difficult) to find a useful non-lattice choice of $\mcal{H}$.
}

\rmk{
The ``lopsided'' choice of $\mcal{H}$ in Proposition~\ref{PROP:beat-Hua-by-diluting-and-difference-fibering-N_F(X)-into-ternary-quadratics} saves a factor of roughly $K/X$ over what we would have gotten from a more ``uniform'' choice of $\mcal{H}$.
This is essentially the observation behind
the ``averaged van der Corput differencing'' in \cite{marmon2019hasse}*{\S3.2}---which
perhaps originated in work of Heath-Brown \cite{heath2007cubic}*{\S4}.
The point is,
in order for something like $h_1x_1^2+h_2x_2^2+h_3x_3^2$ to vanish
(or be small at all),
two of the $h_ix_i^2$ must be comparable.
Above we only used a real (archimedean) version of this observation;
is there a useful way to also introduce divisibility (non-archimedean) conditions on $\mcal{H}$?

And is there a more conceptual justification for our choice of $\mcal{H}$ above?
% Why is our choice of $\mcal{H}$ above reasonable / is there a heuristic reason to believe it is optimal or at least a natural way to motivate it?
% \rmk{
% Heuristically, it makes sense that there exist sets $\mcal{H}$ for which only $O(H/X)$ of the differences $\bm{h}\in\mcal{H}-\mcal{H}$ really contribute.
% Indeed, if we choose $\bm{h}$ at random from $[-cX,cX]^3$, then $h_1x_1^2+h_2x_2^2+h_3x_3^2$ has an $H/X$ chance of being small, i.e. $O(HX^2)$ (at least if we fix $\bm{x}$, but what does this say about varying $\bm{x}$\dots maybe I've confused myself, or it may be something about the tangent space giving an $O(HX^2)$ contribution in all but one direction, in which it gives an $O(HX^2)$ contribution with probability $H/X$)?
% % Indeed, if $F_{\bm{h}}(\bm{x})=0$ then $F_{\bm{h}}(\bm{x})\ll $
% }
Choosing $\mcal{H}$ still seems to be an art.
What is the ``best'' choice of $\mcal{H}$?
It may be worth trying $\mcal{H}\approx[cX]^2\times [cK]$,
for instance---or
more generally,
$\mcal{H}\approx[cX]\times [cK_2]\times [cK_3]$ with $X\gg K_2\gg K_3$.
}




\rmk{
Instead of applying van der Corput differencing to $N_{F_a,\nu}(X)$
(once for each $a\in\ZZ$),
we could have applied van der Corput differencing to $T(\theta)^n=T(\theta)^3$
(once for each $\theta\in\RR/\ZZ$),
where $T(\theta)$ denotes a weighted cubic Weyl sum.
(Recall that $N_F(X)$ is the sixth moment of a certain cubic Weyl sum.)
However,
at least with the approach above,
the end results do not seem to differ.
}

\rmk{
If one wanted to ``amplify'' the Hua problem (for $n=3$) to some $n\geq4$---and
perhaps replace Proposition~\ref{PROP:beat-Hua-by-diluting-and-difference-fibering-N_F(X)-into-ternary-quadratics} with a question about $n$-variable quadratic forms
(instead of ternary ones)---one would need to ``restrict to minor arcs''
(though the precise definition,
including the initial choice of Dirichlet covering of $\RR/\ZZ$,
might require a little care).
% ---it would be worthwhile to work this out at some point).
}

\rmk{
Recall that Hua proved
$N_F(X)\ll_\eps X^{7/2+\eps}$
by interpolating between (unconditional) results in $L^4$ and $L^p$ for some $p\in[8,\infty]$.
The simplest such ``ingredients'' might be Hua's lemma in $L^4$ and $L^8$---which can be proven by applying Weyl differencing once and twice, respectively.
There are more elaborate arguments for the fourth and eight moments
(roughly due to geometry and smooth numbers,
respectively)
that remove Hua's $\eps$'s,
or even give log-power savings for $N_F(X)$
(see \cite{vaughan2020some}*{around Theorem~1.2} for details),
but here we focus on classical differencing.

Thus in some sense,
Hua used Weyl differencing ``strictly between $1$ and $2$ times'' for the sixth moment.
And Proposition~\ref{PROP:beat-Hua-by-diluting-and-difference-fibering-N_F(X)-into-ternary-quadratics} can be viewed as an attempt to ``perturb Hua'' using a single (\emph{more general}) van der Corput differencing \emph{directly applied} to the sixth moment.
}

\subsection{Final questions and remarks}

\rmk{
\cite{mahler1936note} used the identity $(9u^4)^3 + (3uv^3-9u^4)^3 + (v^4-9u^3v)^3 = v^{12}$ to show that $r_3(N)\gg N^{1/12}$ for $N=v^{12}>0$.
Although ``localizing'' to $x\approx X$ would rule out such solutions
(one would need $9u^4\approx 3uv^3-9u^4$, i.e.~$v^3\approx 6u^3$,
but then $v^4-9u^3v\approx (6u^3-9u^3)v < 0$---a contradiction),
this example suggests that some bases or fibrations are more biased than others.
How biased is the ``differenced'' basis used in Proposition~\ref{PROP:beat-Hua-by-diluting-and-difference-fibering-N_F(X)-into-ternary-quadratics}?
}

\rmk{
Suppose one applied the delta method directly on $r_3(N)$ to see what one gets assuming the most optimistic square-root cancellation over the modulus $q$
(but no further cancellation over the dual variable $\bm{c}$).
If $X\asymp N^{1/3}$,
then one should expect roughly $r_3(N)\ll_\eps X^{3s/4 - 3/2+\eps} = X^{3/4+\eps}$,
by extrapolating \cite{heath1998circle}'s work for $s=4,6$ to $s=3$.

The ``total heuristic bound''
$\sum_{\bm{y}\ll X}r_3(F_0(\bm{y}))
\ll_\eps X^{3.75+\eps}$
seems to arise naturally in other \emph{exact}\footnote{i.e.~``dilution-free'' (no nontrivial van der Corput differencing), etc.}
``complete geometric exponential sum'' approaches to bounding $N_F(X)$ as well:
\begin{enumerate}[(1)]
    \item If one applies Deligne rather than GRH in \cites{hooley_greaves_harman_huxley_1997,heath1998circle},
    one gets
    $N_F(X)\ll_\eps X^{3.75+\eps}$
    (rigorously,
    in fact---but unfortunately,
    this is worse than the Hua bound).
    
    \item The delta method heuristic for Conjecture~\ref{CNJ:uniform-ternary-quadratic-form-representation-conjecture} has an error term of $(X^2H)^{1/4+\eps}$,\footnote{the heuristic being uniform in $k$,
    though $k=-3F_0(\bm{h})$ is what is relevant to us}
    which for $H\asymp X$ gives $X^{3/4+\eps}$ for each $\bm{h}$.
    
    % \item In the van der Corput approach with parameter $H_{max} \approx X^{3/4}$, one also ultimately gets $X^{3.75}$ when applying the trivial Diophantine bound $X^{1+\eps}$ individually to each $\bm{h}$.
    % (But at least at first glance, this may not be in the same vein as the ``complete exponential sum'' approaches.
    % Also, by decreasing $H_{max}$ slightly one might be able to get an exponent between $3.5$ and $3.75$, like it seems $X^{11/3=3.666\dots}$.)
    % % X^5/K^2 + (X^2/K^2)K^3*X... X^{11/3=3.666...}
\end{enumerate}

Is the $X^{3/4+\eps}$ commonality a red herring?
It might be interesting to see if there is a deeper connection between
the cubic hyperplane section Hasse--Weil $L$-functions of \cites{hooley1986HasseWeil,heath1998circle}
on the one hand,
and the various
Kloosterman--Sali\'{e} sums and modular forms
associated to quadratic problems arising from Weyl or van der Corput differencing
on the other.
}

\rmk{
It may be interesting to try attacking
other problems---including some of those listed in \S\ref{SEC:other-open-problems-of-a-similar-nature}---by
differencing or fibering.
% \cite{wang2021_large_sieve_diagonal_cubic_forms}*{Appendix~B}
}

\chapter{Biases in finite-field point counts}
% Bias in point counts over finite fields
\label{CHAP:biases-over-finite-fields}

\section{Introduction}

The Weil conjectures imply in particular that point counts of smooth projective complete intersections over finite fields satisfy a certain randomness heuristic of ``square-root cancellation'' type.
As \cite{hooley1991number}*{second paragraph after Theorem~2} notes,
the same heuristic fails for \emph{some} singular complete intersections, and it would be nice to have a ``satisfactory criterion'' to determine when, but ``on the scanty evidence at present available, all we can say as yet is that it seems as if there were only a minority of singular varieties for which the result of the theorem cannot be improved.''
Yet \cite{hooley1991number}*{Theorem~2} itself is silent on this issue (see Remark~\ref{RMK:Hoo91-silence} below), and the full truth is far from known in general.
Let us now summarize positive results of \cite{wang2022dichotomous} in this direction.

Given a base field $k$, an integer $m\geq 3$, and a homogeneous polynomial $F\in k[\bm{x}]=k[x_1,\dots,x_m]$, let $V$ denote $V_{\PP^{m-1}}(F)$, and let $V_{\bm{c}}$ denote the intersection $V_{\PP^{m-1}}(F,\bm{c}\cdot\bm{x})$ for $\bm{c}\in k^m$ (a hyperplane section of $V$ if $\bm{c}\neq \bm{0}$).
(We will repeatedly use this setup with $k,m,F,V,V_{\bm{c}}$; call it the \emph{Main Setup} for convenience.)
Our main general result, Theorem~\ref{THM:codimension-2-criteria-for-boundedness-of-E_c} below, shows that if $k$ is finite, $V$ is smooth, and $\deg{F}\geq 3$, then under mild conditions, the $V_{\bm{c}}$'s satisfy ``square-root cancellation'' for all $\bm{c}$ away from an explicit locus of codimension two.
This result does not extend to $\deg{F}=2$ in general (see Proposition~\ref{PROP:dichotomy-for-quadrics}),
but its truth for $\deg{F}=3$ is significant for Chapters~\ref{CHAP:discriminating-pointwise-estimates}--\ref{CHAP:using-mean-value-L-function-predictions}.

Theorem~\ref{THM:codimension-2-criteria-for-boundedness-of-E_c} directly leads to general progress on Problem~\ref{PROB:codimension-of-failures-of-sqrt-cancellation} below.
In special cases, one can do better.
The quadratic case is more or less understood (in odd characteristic, at the very least), so we focus on cubics.
Corollary~\ref{COR:optimal-criteria-for-boundedness-of-E_c} shows that if $k$ is finite, $V$ is smooth, $\deg{F}=3$, and $m\in \set{4,6}$, then $V_{\bm{c}}$ fails ``square-root cancellation'' if and only if $(V_{\bm{c}})_{\ol{k}}$ contains a certain kind of subvariety of dimension $(m-2)/2$.
This is at least morally significant for Chapter~\ref{CHAP:isolating-special-solutions} on special subvarieties in Manin-type conjectures.
This also leads to the following (vague) question:
\ques{
\label{QUES:correlate-special-subvarieties-Manin-vs-F_q}
To what extent do special subvarieties in Manin's conjectures correlate with special subvarieties in the sense of the present chapter?
For example, it would be interesting to determine whether the special quadratic locus $x_1+x_2+x_3=x_4+x_5+x_6=(x_1^2+x_2^2+x_3^2)-(x_4^2+x_5^2+x_6^2)=0$ on the $6$-variable quartic $x_1^4+x_2^4+x_3^4 = x_4^4+x_5^4+x_6^4$ (which I learned from a talk of Wooley; see \cite{wooley2019talk}) remains special for hyperplane sections of the quartic over finite fields.
}

\prob{
\label{PROB:codimension-of-failures-of-sqrt-cancellation}
Fix $G\in \ZZ[\bm{x}]$, homogeneous of degree $d\geq 2$ in $m\geq 3$ variables, with nonzero discriminant.
Let $M_{d,m}\defeq 72(3+2d)^m$.
Given a prime $p$ and integer $r\geq 1$, let $q\defeq p^r$.
Given $\bm{c}\in \FF_q^m$, let $E_{\bm{c}}(q)\defeq \#V_{\PP^{m-1}}(G,\bm{c}\cdot\bm{x})(\FF_q)-\#\PP^{m-3}(\FF_q)$.
Let
\mathd{
N_{G,d,m}(q)\defeq \#\set{\bm{c}\in \FF_q^m\setminus \set{\bm{0}}: \abs{E_{\bm{c}}(q)} > M_{d,m}\cdot q^{(m-3)/2}}.
}
As $G,d,m$ vary, estimate the \emph{exponent} $\sigma_{G,d,m}\defeq \limsup_{p\to\infty}\sup_{r\to\infty} \log_q(N_{G,d,m}(q))$.
}

\rmk{
Problem~\ref{PROB:codimension-of-failures-of-sqrt-cancellation} is close in spirit to \cite{lindner2020hypersurfaces}*{Theorem~1.3 and the line after}, who studies a different aspect (with $q,m$ fixed and $d\to \infty$), works with a different family (a universal family of hypersurfaces), and works with a related cohomological problem instead of concrete point counts.
}

\rmk{
Morally, $\sigma_{G,d,m}$ is the ``numerical dimension'' of
% the locus of \emph{failures of square-root cancellation}. --- is there a better / clearer wording? e.g. failures -> violators? or "locus of c's violating square-root cancellation" or "locus of c's for which square-root cancellation fails".
the \emph{locus of $\bm{c}$'s for which square-root cancellation fails}.
Most other natural measures of failure (e.g.~those suggested by Proposition~\ref{PROP:amplification-for-projective-varieties}) are harsher, i.e.~$\geq \sigma_{G,d,m}$.
In any case, the Weil conjectures (plus Lang--Weil) give the bound $\sigma_{G,d,m}\leq m-1$ (as does \cite{hooley1991number}*{Theorem~2}), which we improve to $\sigma_{G,d,m}\leq m-2$ for $d\geq 3$ in general (see Remark~\ref{RMK:codimension-two-and-SFSC-significance}), and to $\sigma_{G,d,m}\leq m-3$ in many cases with $(m,d) = (6,3)$ (see e.g.~Example~\ref{EX:diagonal-cubic-example}, where $m\in \set{4,6}$ and $\sigma_{F,3,m} = m/2$).
Based on this, it would be reasonable to conjecture that $\sigma_{G,d,m}\leq m-3$, or perhaps even more, holds when $d\geq 3$ and $m\geq 6$ (if one believes that ``randomness'' should increase with $d,m$).
}

We now introduce some relevant conventions and notions.

\defn{
Let $k$ be a base field.
A \emph{$k$-scheme} is a scheme equipped with a morphism to $\Spec{k}$.
A \emph{variety over $k$} (or \emph{$k$-variety} for short) is a separated $k$-scheme of finite type (not necessarily reduced or irreducible).
In the context of projective varieties, and especially their singular loci, let $\dim(\emptyset)\defeq -1$ and $\PP^{-1}_k\defeq \emptyset$.
If $X$ is a projective $k$-variety and $k$ is finite, let $E(X)\defeq \#X(k)-\#\PP^{\dim{X}}(k)$.
(The base field $k$ of $X$ is essential to this definition.)

% https://stacks.math.columbia.edu/tag/0C3H (Section 31.10: The singular locus of a morphism); see also https://stacks.math.columbia.edu/tag/0C3K (have good intrinsic scheme structure on singular locus for pure varieties over a field).
For a $k$-variety $X$ of pure dimension $d\geq 0$,
let $X_{\map{sing}}$ denote the \emph{singular subscheme} of $X$,
i.e.~the closed subscheme of $X$ cut out by the $d$th Fitting ideal of the cotangent sheaf $\Omega_{X/k}$ (or informally, ``by the Jacobian criterion''), following \cite{stacks-project}*{\href{https://stacks.math.columbia.edu/tag/0C3H}{Tag 0C3H}}.
% (Note: cotangent sheaf https://en.wikipedia.org/wiki/Cotangent_sheaf is sometimes called sheaf of differentials; see https://stacks.math.columbia.edu/tag/01UM for instance.)
% (The casual reader can, for the most part, ignore the scheme structure on $X_{\map{sing}}$.)
(Most important to us is $X_{\map{sing}}(\ol{k})$, the set of singular $\ol{k}$-points of $X$.)
For a scheme $Y$, we let $\abs{Y}$ denote the underlying topological space;
thus, for instance, $\abs{X_{\map{sing}}}$ denotes a topological space,
while $\card{X_{\map{sing}}(\ol{k})} = \#X_{\map{sing}}(\ol{k})$ denotes an integer.
}

Our main concern is the notion of \emph{error-goodness} (or \emph{$\abs{E}$-goodness} for short), which we now define alongside some related notions (related by Proposition~\ref{PROP:amplification-for-projective-varieties}).
\defn{
\label{DEFN:error-goodness-for-projective-X/F_q}
Let $k$ denote an arbitrary finite field, and $X$ a projective $k$-variety.
\begin{enumerate}[(1)]
    \item Given $f\in \set{\abs{E}, +E, -E}$, say $X$ is \emph{$f$-good} (with constant $C$) if there exists $C\in \RR_{>0}$ such that for all finite extensions $k'/k$, we have $f(X_{k'})\leq C\card{k'}^{(\dim{X})/2}$.
    % Note: Originally terminology involved $(+1)$-side and $(-1)$-side error-goodness, but this is unnecessarily cumbersome.
    Say $X$ is \emph{$f$-bad} if it is not $f$-good.
    
    \item Given a property \emph{blah} in (1), say $X$ is \emph{potentially} (resp.~\emph{stably}) \emph{blah} if $X_{k'}$ is blah for some (resp.~for every) finite extension $k'/k$.
\end{enumerate}
}

\rmk{
\label{RMK:Hoo91-silence}
Fix a projective complete intersection $Y/k$.
Then \cite{hooley1991number}*{Theorem~2} gives $\abs{E(Y_{k'})}\leq O(\card{k'}^{(1+\dim(Y_{\map{sing}})+\dim{Y})/2})$ as $[k':k]\to \infty$.
Like the Weil conjectures, this only proves that $Y$ is $\abs{E}$-good if $\dim(Y_{\map{sing}})=-1$, i.e.~$Y$ is smooth.
% Like the Weil conjectures, this (together with Proposition~\ref{PROP:amplification-for-projective-varieties}) gives $\sigma_{G,d,m}\leq m-1$ in Problem~\ref{PROB:codimension-of-failures-of-sqrt-cancellation}.
}

For reference, we recall a useful principle of Zak:
% % (see e.g.~\cite{hooley1991nonary}*{p.~110, Lemma~39}
% % or \cite{katz1999estimates}*{p.~896, before Theorem~23}),
% (see e.g.~\cite{hooley1991number}*{Katz's Appendix, Theorem~2}):
% % for diagonal $F$, this can be easily proven by hand.
\thm{
[See e.g.~\cite{hooley1991number}*{Katz's Appendix, Theorem~2}]
\label{THM:Zak's-principle}
Let $k,m,F,V,V_{\bm{c}}$ be as in the Main Setup.
If $V$ is smooth, then $\dim((V_{\bm{c}})_{\map{sing}})\leq 0$ for all $\bm{c}\in k^m\setminus\set{\bm{0}}$.
}

We now state a useful amplification-type result.
The proof (see \cite{wang2022dichotomous}*{\S2}) uses general foundational results due to Katz and others (see especially Theorem~\ref{THM:general-perversity-result}).
\prop{
\label{PROP:amplification-for-projective-varieties}
Let $k$ be a finite field.
Let $X$ be a projective $k$-variety of the form $V(F_1,\dots,F_r)\belongs \PP^n_k$, with $n,r\geq 1$ and $\max_{i\in [r]}\deg{F_i}\leq d$.
Consider the following four conditions:
\begin{enumerate}[(1)]
    \item $X$ is $\abs{E}$-good with constant $18(3+rd)^{n+1}2^r$;
    \item $X$ is $\abs{E}$-good;
    \item $X$ is potentially $\abs{E}$-good;
    \item $X$ is potentially $(-1)^{1+\dim{X}}E$-good;
\end{enumerate}
In general, (1)--(2) are equivalent, (2) implies (3), and (3) implies (4).
If $\codim{X} = r$ and $\dim(X_{\map{sing}})\leq 0$, then (1)--(4) are equivalent.
}

\rmk{
In particular, a projective complete intersection $Y/k$ with $\dim(Y_{\map{sing}})\leq 0$ is $\abs{E}$-good if and only if it is potentially $\abs{E}$-good.
% It would be nice to know if this equivalence holds under weaker hypotheses.
It would be nice to have this more generally.
}

To state our main general result, Theorem~\ref{THM:codimension-2-criteria-for-boundedness-of-E_c}, we need two definitions.

\defn{
For a variety of pure dimension $n-1$ over $K=\ol{K}$, an \emph{$A_1$ singularity} is
% a hypersurface singularity of analytic type $z_1^2+\dots+z_n^2$; or equivalently,
a point at which the completed local ring is $\cong K[[z_1,\dots,z_n]]/(z_1^2+\dots+z_n^2)$.
}

\rmk{
\label{RMK:A_1-vs-ordinary-vs-nondegenerate-double-points}
In characteristic $p\neq 2$, the following notions coincide:
$A_1$ singularity, non-degenerate double point, and ordinary double point.
(See e.g.~\cite{poonen2020valuation}*{\S4}.)
}

\defn{
\label{DEFN:informal-disc(P,a)-definition}
Given integers $d\geq 2$ and $m\geq 3$,
let $\disc(P, \bm{a})$ be a discriminant polynomial associated to the ``universal'' intersection $V(P, \bm{a}\cdot\bm{x})\belongs \PP^{m-1}$ defined by the ``universal'' homogeneous polynomials $P(\bm{x}), \bm{a}\cdot\bm{x}$ of respective degrees $d, 1$ in $\bm{x}=(x_1,\dots,x_m)$.
(See e.g.~\cite{terakado2018determinant}*{\S1.1}, or \cite{wang2022dichotomous}*{Appendix~A}, for details.)
}

\thm{
\label{THM:codimension-2-criteria-for-boundedness-of-E_c}
In the Main Setup with $k,m,F,V,V_{\bm{c}}$,
suppose $V$ is smooth, $d\defeq \deg{F}\geq 3$, and $k$ is finite of characteristic $p\neq 2$.
Let $\bm{c}\in k^m\setminus\set{\bm{0}}$, and suppose $V_{\bm{c}}$ is $\abs{E}$-bad (see Definition~\ref{DEFN:error-goodness-for-projective-X/F_q} and Proposition~\ref{PROP:amplification-for-projective-varieties}).
Then the following hold:
\begin{enumerate}[(1)]
    \item Either $V_{\bm{c}}\times\ol{k}$ has $\geq 2$ singularities, or it has a non-$A_1$ singularity.
    
    \item If $p\nmid d(d-1)$, then $\bm{c}$ is a singular zero of the polynomial $\disc(F,-)$.
\end{enumerate}
}

\rmk{
\label{RMK:codimension-two-and-SFSC-significance}
The significance of (2) comes from its codimension-two nature; when combined with Proposition~\ref{PROP:amplification-for-projective-varieties} and Lang--Weil, it gives $\sigma_{G,d,m}\leq m-2$ in Problem~\ref{PROB:codimension-of-failures-of-sqrt-cancellation} for $d\geq 3$.
Though (2) is easier to use in certain applications,
(1) can also be useful in view of \cite{poonen2020valuation}*{Theorem~1.1}.
}

\rmk{
[Sawin]
One could perhaps prove a less explicit version of Theorem~\ref{THM:codimension-2-criteria-for-boundedness-of-E_c}(2) using the perversity-based strategy of \cite{grimmelt2021representation}*{proof of Lemma~3.1}.
}

\rmk{
Theorem~\ref{THM:codimension-2-criteria-for-boundedness-of-E_c}(2) may have an analog for the universal family of hypersurfaces of a given degree $\geq 3$ and dimension $\geq 0$.
We have not checked.
}

The proof of Theorem~\ref{THM:codimension-2-criteria-for-boundedness-of-E_c} (see \cite{wang2022dichotomous}*{\S3}) mainly combines results on discriminants, duality, and $\ell$-adic cohomology (especially \cite{lindner2020hypersurfaces}*{Theorem~1.2}).
When $(m,d)=(6,3)$, one can replace the ingredient \cite{lindner2020hypersurfaces}*{Theorem~1.2} with the following proposition, proven later (after Lemma~\ref{LEM:standard-real-amplification-lemma}) using some geometry (going back to Clemens--Griffiths) specific to cubic threefolds.
\prop{
\label{PROP:cubic-threefold-replace-Lindner}
Let $X$ be a projective cubic threefold over a finite field $k$.
Assume that $X_{\map{sing}}(\ol{k}) = \set{x}$, with $x\in X(\ol{k})$ either nodal or mildly cuspidal (i.e.~either of analytic type $z_1z_2+z_3z_4=0$ or $z_1z_2+z_3^2+z_4^3=0$; see e.g.~\cite{casalaina2009moduli}*{Corollary~3.2(i)--(ii)} and \cite{van2010note}*{paragraph before Proposition~2.1} for the equivalence).
Then $x$ is defined over $k$, and $X$ is $\abs{E}$-good.
}

Theorem~\ref{THM:codimension-2-criteria-for-boundedness-of-E_c} has the following corollary (proven in \cite{wang2022dichotomous}*{\S4} by point counting), which when $2\mid m$ reproves a consequence of the Deligne--Katz equidistribution theorem.
(The case $2\nmid m$ may or may not involve ``exceptional monodromy'', which could complicate an attempt to use Deligne--Katz.)
% https://web.math.princeton.edu/~nmk/moments-67.pdf
It suggests that there may be a deeper connection between monodromy, moments, and Problem~\ref{PROB:codimension-of-failures-of-sqrt-cancellation}.

\cor{
\label{COR:smooth-locus-moment-calculations}
Let $G,d,m,E_{\bm{c}}(q)$ be as in Problem~\ref{PROB:codimension-of-failures-of-sqrt-cancellation}.
If $d,m\geq 3$, then as $p\to \infty$, the following hold (with implied constants depending only on $G$):
\begin{enumerate}[(1)]
    \item $\EE_{\bm{c}\in \FF_p^m}[E_{\bm{c}}(p)\cdot \bm{1}_{p\nmid \disc(G,\bm{c})}]
    = p^{-1}\cdot E(V_{\PP^{m-1}}(G)_{/\FF_p}) + p^{-1}\cdot O(p^{(m-3)/2})$.
    
    \item $\EE_{\bm{c}\in \FF_p^m}[E_{\bm{c}}(p^2)\cdot \bm{1}_{p\nmid \disc(G,\bm{c})}]
    = (1+O(p^{-1/2}))\cdot p^{m-3}$.
    
    \item $\EE_{\bm{c}\in \FF_p^m}[E_{\bm{c}}(p)^2\cdot \bm{1}_{p\nmid \disc(G,\bm{c})}]
    = (1+O(p^{-1/2}))\cdot p^{m-3}$.
\end{enumerate}
}

The next result shows that the assumption $d\geq 3$ in Theorem~\ref{THM:codimension-2-criteria-for-boundedness-of-E_c} is essential when $2\mid m$; take $\dim(X) = m-3\equiv 1\bmod{2}$ and $d = \dim(X)-1$ to see why.

\defn{
Let $k$ be a base field.
A \emph{cone} is a projective cone over a projective $k$-variety, with vertex a $k$-point.
(Informally, an embedded projective $k$-variety is a cone if and only if it is ``missing a variable'' after some $k$-linear change of coordinates.
For algebraic convenience, we will consider a hypersurface $X\belongs \PP^1_k$ with $\#X(k) = \#X(\ol{k}) = 1$ to be a cone.)
An \emph{iterated cone} is obtained by taking cones one or more times.
}

\prop{
[Quadric dichotomy]
\label{PROP:dichotomy-for-quadrics}
Let $X$ be a projective quadric of dimension $\geq 0$ over a finite field $k$.
Then $X$ is stably $\abs{E}$-bad if and only if $X_{\ol{k}}$ is an iterated cone over a smooth projective quadric of dimension $d\in [0, \dim(X)-1]$ with $2\mid d$.
}

\rmk{
If $2\nmid \card{k}$, then the adverb ``stably'' can be removed, and the cone condition is equivalent to ``$X$ is of the form $V(Q)$ with $2\mid \rank(Q)\in [2, \dim(X)+1]$''.
}

Proposition~\ref{PROP:dichotomy-for-quadrics} is essentially classical.
We will sketch a proof using the following
general fact:
\prop{
[Routine]
\label{PROP:cone-construction-multiplies-E-by-q}
Let $k$ be a finite field.
If $C(Y)$ is a cone over a projective $k$-variety $Y$, then $E(C(Y)) = \card{k}\cdot E(Y)$.
}

\pf{
Proof by calculation (one could probably alternatively use cohomology somehow): $\card{C(Y)(k)} = 1 + \card{k}\cdot \card{Y(k)}$ (``vertex + lines through vertex'') and likewise $\card{\PP^{\dim C(Y)}} = \card{C(\PP^{\dim Y})} = 1 + \card{k}\cdot \card{\PP^{\dim Y}}$; now subtract.
}

\pf{
[Proof sketch for Proposition~\ref{PROP:dichotomy-for-quadrics}]
The key ingredients are the following:
\begin{enumerate}[(1)]
    \item If $X$ is smooth of dimension $d\geq 0$, then $E(X) = 0$ if $2\nmid d$, and $E(X) = \pm \card{k}^{d/2}$ if $2\mid d$.
    (One can prove this using the Weil conjectures.
    If $2\nmid \card{k}$, one can alternatively diagonalize and then follow e.g.~\cite{weil1949numbers}.)
    % If $2\nmid q$, then every nonzero rank-$r$ quadratic form $Q/\FF_q$ in $s\geq 3$ variables has $\#V(Q)(\FF_q) = \#\PP^{s-2}(\FF_q)\pm q^{(s-r)/2}\bm{1}_{2\mid r}\cdot q^{(s-2)/2}$.
    
    \item If $X$ is non-reduced, then $E(X)=0$.
    On the other hand, a reduced projective $\ol{k}$-quadric (of dimension $\geq 0$) is singular if and only if it is an iterated cone over a smooth projective quadric (of dimension $\geq 0$).
\end{enumerate}
By (1)--(2) and Proposition~\ref{PROP:cone-construction-multiplies-E-by-q}, $X$ is potentially $\abs{E}$-good if and only if $X_{\ol{k}}$ is either smooth or non-reduced, or an iterated cone over a smooth projective quadric of dimension $d\in [0, \dim(X)-1]$ with $2\nmid d$.
The result follows from (2).
}

The following results show that Theorem~\ref{THM:codimension-2-criteria-for-boundedness-of-E_c} is far from the full truth in general.

\defn{
A \emph{$\bm{d}$-CI} is a (projective) complete intersection of multi-degree $\bm{d}$.
}

\defn{
\label{DEFN:cubic-scroll-of-dimension-d}
Let $d\geq 1$ be an integer.
Over $K=\ol{K}$, a \emph{cubic $d$-scroll} is an embedded projective $d$-fold $\Sigma\belongs \PP^n$, integral of degree $3$, with $\dim(\Span(\Sigma)) = d+2$.
(See Proposition~\ref{PROP:basic-cubic-scroll-facts} below, and its proof, for some background on cubic scrolls.)
}

\thm{
\label{THM:near-dichotomy-for-cubic-curves-and-threefolds}
Let $n\in \set{2,4}$.
Let $X$ be a $(3)$-CI in $\PP^n$ over a finite field $k$.
Suppose $X_{\ol{k}}$ is not a cone over a cone.
If $n=2$, then $X$ is $\abs{E}$-bad if and only if $X_{\ol{k}}$ contains a line.
Now suppose $n=4$, and consider the following four conditions:
\begin{enumerate}[(1)]
    \item $X$ is stably $E$-bad (see Definition~\ref{DEFN:error-goodness-for-projective-X/F_q} and Proposition~\ref{PROP:amplification-for-projective-varieties});
    \item $X$ is stably $\abs{E}$-bad;
    \item $X_{\ol{k}}$ contains a plane or a singular cubic $2$-scroll in $\PP^4_{\ol{k}}$;
    and
    \item there exists a $(2,2)$-CI of the form $V(Q_1,Q_2)\belongs \PP^4_{\ol{k}}$, with $V(Q_1)_{\map{sing}}\cap V(Q_2)_{\map{sing}}\neq \emptyset$, such that $X_{\ol{k}}$ contains a nonempty open subscheme of $V(Q_1,Q_2)$.
\end{enumerate}
In general, (1)--(2) are equivalent, (2) implies (3), and (3)--(4) are equivalent.
If $\dim(X_{\map{sing}})\leq 0$, then (1)--(4) are equivalent.
}

\rmk{
The case when $X_{\ol{k}}$ is a cone over a cone can still be fully analyzed (using Proposition~\ref{PROP:cone-construction-multiplies-E-by-q}), but it is less interesting than the opposite case.
}

\rmk{
The equivalence of (1)--(2) suggests that (stable) $\abs{E}$-badness might be explained by ``excess'' points from ``special'' subvarieties, which we have tried to pinpoint in (3)--(4).
Theorem~\ref{THM:near-dichotomy-for-cubic-curves-and-threefolds} is close to a complete dichotomy.
To ``complete'' it (for $n=4$), one would need to analyze the case $\dim(X_{\map{sing}})\geq 1$, which might be tricky (see e.g.~\cite{wang2022dichotomous}*{proof of Lemma~5.5}).
}

\rmk{
Our methods can also be used to show that a projective cubic surface $X/k$ is stably $\abs{E}$-bad if and only if $X_{\ol{k}}$ is either reducible, or a cone over a smooth cubic curve.
We omit this from Theorem~\ref{THM:near-dichotomy-for-cubic-curves-and-threefolds} because it has a different flavor.
% different flavor, and because a better formulation may be possible.
}

The proof of Theorem~\ref{THM:near-dichotomy-for-cubic-curves-and-threefolds} (see \cite{wang2022dichotomous}*{\S5}) uses (for $n=4$) base change and some situation-specific geometry, including some classification results over $\ol{k}$ (see e.g.~Proposition~\ref{PROP:using-k-rational-singular-point-to-count-points} and Lemma~\ref{LEM:description-of-low-degree-components-of-a-CI} below).
It would be interesting to find different proofs of Theorem~\ref{THM:near-dichotomy-for-cubic-curves-and-threefolds} that work directly over $k$, or that generalize naturally.
Condition~(4) in Theorem~\ref{THM:near-dichotomy-for-cubic-curves-and-threefolds} seems especially suggestive as to what one might try more generally.
One might also try using mixed Hodge theory, in the spirit of e.g.~\cites{dimca1990betti,kloosterman2022maximal}.

\rmk{
Originally we sought to prove Theorem~\ref{THM:near-dichotomy-for-cubic-curves-and-threefolds} (for $n=4$) using (an extension of) \cite{bombieri1967local}'s ``conic bundle'' method; see \S\ref{SEC:conic-bundle-analysis-of-cubic-threefolds-over-F_q} below (and specifically, Remark~\ref{RMK:original-conic-bundle-approach-to-dichotomy}).
This approach inspired condition~(4).
But our present approach is overall more efficient in the singular case.
}

\cor{
\label{COR:optimal-criteria-for-boundedness-of-E_c}
In the Main Setup,
suppose $F$ is a cubic form in $m\in \set{4,6}$ variables over a finite field $k$,
and $V$ is smooth.
Let $\bm{c}\in k^m\setminus \set{\bm{0}}$.
Then $V_{\bm{c}}$ is $\abs{E}$-bad if and only if $(V_{\bm{c}})_{\ol{k}}$ contains an $(m-2)/2$-plane or a singular cubic $2$-scroll in $\PP^{m-1}_{\ol{k}}$.
}

\pf{
Combine Theorem~\ref{THM:Zak's-principle}, Proposition~\ref{PROP:amplification-for-projective-varieties}, and Theorem~\ref{THM:near-dichotomy-for-cubic-curves-and-threefolds}.
}

\ex{
\label{EX:diagonal-cubic-example}
Let $m\in \set{4,6}$ and $\bm{F}=(F_1,\dots,F_m)\in (\ZZ\setminus \set{0})^m$.
Suppose $F = F_1x_1^3+\dots+F_mx_m^3$, and assume the characteristic of $k$ is sufficiently large in terms of $\bm{F}$.
Then by \cite{wang2022dichotomous}*{Proposition~B.3} (proven by a calculation---a singularity analysis---involving, among other things, $3\times 3$ Vandermonde determinants arising from diagonality), the phrase ``or a singular cubic $2$-scroll'' in Corollary~\ref{COR:optimal-criteria-for-boundedness-of-E_c} is unnecessary for $F$.
Furthermore, the $(m-2)/2$-planes on $V_{\ol{k}}$ are known to be cut out by systems of equations of the form ``$F_ix_i^3+F_jx_j^3=0$ in pairs''.
Thus a given $V_{\bm{c}}$ is $\abs{E}$-bad if and only if ``$c_i^3/F_i=c_j^3/F_j$ in pairs''.
}

Our proof of \cite{wang2022dichotomous}*{Proposition~B.3} makes use of the following technical fact (though it would be nice to know how much the hypotheses can be weakened; cf.~\cite{MO211153intersect_components}):
\prop{
\label{PROP:lower-bound-on-singular-scheme-of-a-reducible-variety}
Let $X_1,X_2$ be subvarieties of $\PP^n$ of pure dimension $d$ over $K=\ol{K}$, where $d,n\geq 1$.
Let $X\defeq X_1\cup X_2$.
Assume the following hypotheses:
\begin{enumerate}[(1)]
    \item $\dim(X_1\cap X_2) = 0$;
    and
    \item for each $x\in X_1\cap X_2$, there exist a subvariety $Y$ of $\PP^n$ of pure dimension $2d$, and an open neighborhood $U$ of $x$ in $\PP^n$,
    such that $X_1\cap U$ and $X_2\cap U$ are Cohen--Macaulay, $Y\cap U$ is smooth, and $Y\cap U\contains X\cap U$.
\end{enumerate}
Then $X_{\map{sing}}\contains X_1\cap X_2$.
}

\pf{
% The statement is local, so we may assume that $X_1\cap X_2\belongs \Aff^n$, and that there exist ideals $I_1,I_2,J\belongs K[x_1,\dots,x_n]$, and an affine open neighborhood $U = \Spec{R}$ of $X_1\cap X_2$ in $\Aff^n$, such that $X_i=V_U(I_i)$ and $X_i$ is Cohen--Macaulay (for all $i\in [2]$), the variety $Y\defeq V_U(J)$ is smooth and irreducible of dimension $2d$, and $J\belongs I_1\cap I_2$.
The statement is local, so we may assume $X_1\cap X_2$ is supported on a singleton $\set{x}$.
Now let $Y,U$ be as in hypothesis~(2); by shrinking $U$ if necessary, we may assume that $U$ is affine and that $X_1,X_2,Y$ are closed subschemes of $U$.
Say $U = \Spec{R}$, and let $I_1,I_2,J\belongs R$ be the ideals defining $X_1,X_2,Y$, respectively.
Then $X_1,X_2$ are Cohen--Macaulay, $Y$ is regular, and $J\belongs I_1\cap I_2$.
By \cite{SpeyerMO49261product_ideal} and \cite{serre2000local}*{Proposition~11 in \S{IV.B.1}, and Corollary to Theorem~4 in \S{V.B.6}},
it follows that $(I_1/J)\cap (I_2/J) = (I_1/J)\cdot (I_2/J)$ in $R/J$ (cf.~\cite{DaoMO49299product_ideal}).
So if $f\in I_1\cap I_2$, then $f\equiv h\bmod{I_1I_2}$ for some $h\in J$, whence $Df\equiv Dh\bmod{(I_1,I_2)}$ for all derivations $D\maps R\to R$.
But $\Omega_{Y/K}$ is locally free of rank $2d\geq d+1$, so the $d$th Fitting ideal of $\Omega_{Y/K}$ is $0$.
% See https://stacks.math.columbia.edu/tag/02G1 and https://stacks.math.columbia.edu/tag/05P8 (cf.~proof of https://stacks.math.columbia.edu/tag/0C3K)
% But $Y$ is smooth, so any $(n-d)\times (n-d)$ matrix of the form $(D_j h_i)_{i,j}$ has determinant $\equiv 0\bmod{J}$.
% https://mathoverflow.net/questions/236054/canonical-scheme-structure-on-the-singular-locus-of-a-variety
Thus $V_U(I_1\cap I_2)_{\map{sing}}\contains V_U(I_1,I_2)$, i.e.~$X_{\map{sing}}\contains X_1\cap X_2$.
}

Problem~\ref{PROB:codimension-of-failures-of-sqrt-cancellation} and Corollary~\ref{COR:optimal-criteria-for-boundedness-of-E_c} motivate the following question:
\ques{
Given a smooth cubic hypersurface $X\belongs \PP^5_\CC$, let $S\belongs (\PP^5_\CC)^\vee$ parameterize hyperplane sections of $X$ containing a singular cubic $2$-scroll in $\PP^5_\CC$.
What is the best possible upper bound on the dimension of the Zariski closure of $S$?
}

\rmk{
It is known that $\card{S} = 1$ for sufficiently general $X$ containing a (not necessarily singular) cubic $2$-scroll; cf.~especially \cite{hassett1996special}*{proof of Lemma~2.11, and the subsequent dimension counting} and \cite{hassett2010flops}*{Propositions~3.3 and~6.1}.
And $S$ may well be finite for $X = V_{\PP^5}(x_1^3+\dots+x_6^3)_{/\CC}$, though our analysis in \cite{wang2022dichotomous}*{Appendix~B} falls short of a proof (due to interference from the planes on $X$).
}

\section{Miscellaneous writeups}

The proof of Theorem~\ref{THM:near-dichotomy-for-cubic-curves-and-threefolds} (see \cite{wang2022dichotomous}*{\S5}) begins with a classical ``rationality''-type idea (cf.~\cite{dolgachev2016corrado}*{\S1}).
\prop{
\label{PROP:using-k-rational-singular-point-to-count-points}
Let $X\belongs \PP^n$ be a $(3)$-CI over a finite field $k$, where $n\geq 2$.
Assume $[0:\dots:0:1]\in X_{\map{sing}}$.
Then $X = V(f_2x_{n+1}+f_3)$ for some $f_i\in k[x_1,\dots,x_n]$, homogeneous of degree $i$, with $(f_2,f_3)\neq (0,0)$.
Furthermore, $E(X) = \card{k}\cdot E(V_{\PP^{n-1}}(f_2,f_3)) - E(V_{\PP^{n-1}}(f_2)) + \card{k}^{n-1}\cdot \bm{1}_{\dim V(f_2,f_3) = \dim V(f_2)}$.
}

\pf{
The first part is clear.
So there are two kinds of points $[\bm{x}]\in X(k)$: (i) those with $f_2\neq 0$ and $x_{n+1} = -f_3/f_2$, and (ii) those with $f_2=0$ and $f_3=0$.
Therefore $\card{X(k)} = \card{(\PP^{n-1}\setminus V(f_2))(k)} + \card{C(V_{\PP^{n-1}}(f_2,f_3))(k)}$.
Proposition~\ref{PROP:cone-construction-multiplies-E-by-q}, and casework on $\dim V(f_2)-\dim V(f_2,f_3)\in \set{0,1}$, then lead to the desired equality.
}

To study $E(X)$ using Proposition~\ref{PROP:using-k-rational-singular-point-to-count-points}, one needs to analyze low-degree complete intersections in some detail.
\cite{wang2022dichotomous}*{\S5} repeatedly uses the following lemma describing the low-degree components of non-integral $(2,2)$-CI's and $(2,3)$-CI's:
\lem{
\label{LEM:description-of-low-degree-components-of-a-CI}
Let $n\geq 2$ and $K=\ol{K}$.
Let $Y$ be a $(d,e)$-CI of the form $V(A,B)\belongs \PP^n_K$, with $d=2$ and $e\in \set{2,3}$.
Then $Y$ is non-integral if and only if it has an irreducible component of degree $\leq 3$.
Let $Z$ be any such component, equipped with the reduced induced scheme structure.
Then the following dichotomy holds:
\begin{enumerate}[(1)]
    \item If $\deg{Z}\leq 2$, or $e=3$ and $A$ is reducible, then $\dim(\Span(Z))\leq n-1$.
    
    \item If $\deg{Z}=3$, and $e=2$ or $A$ is irreducible, then $\dim(\Span(Z)) = n$.
\end{enumerate}
Furthermore, $\dim(\Span(Z))\leq n-1$ if and only if $Z$ is a $(1,\deg{Z})$-CI.
}

\pf{
The first part is clear, since $\deg{Y} = de\leq 6$.
Now fix $Z$.
Since $\dim{Z} = n-2$, and $Z$ is integral, the final sentence is clear: both conditions are equivalent to ``$Z$ lies in an $(n-1)$-plane (scheme-theoretically)''.
So it remains to prove (1)--(2).

If $\deg{Z}\leq 2$, then $\dim(\Span(Z))\leq \dim(Z)+1 = n-1$ by \cite{eisenbud1987varieties}*{Proposition~0}.
This proves (1)--(2) when $\deg{Z}\leq 2$.

Now suppose $\deg{Z}=3$.
If $A$ is reducible, then since $Z$ is integral, there must exist a  nontrivial (and thus linear) factor $L\mid A$ such that $V(L,B)\contains Z$; and thus $e=3$ and $Z = V(L,B)\belongs V(L)$.
Conversely, suppose $Z = V(L,C)$ for some nonzero linear form $L$ and cubic form $C$ (with $L\nmid C$).
Then $V(A,B)\contains V(L,C)$, i.e.~$A,B$ lie in the saturated homogeneous ideal $(L,C)$.
So for degree reasons, $L\mid A$ and $e=3$ (or else $L\mid A,B$, which is impossible).

Thus we have shown that if $\deg{Z}=3$, then $Z$ is a $(1,3)$-CI if and only if $A$ is reducible, in which case $e=3$ must hold.
This proves (1)--(2) when $\deg{Z}=3$.
}

Let us now record some results and proofs of a folklore nature.

\rmk{
Technical points (some implicit) in \cite{wang2022dichotomous} include
taking care to work with saturated homogeneous ideals when necessary (as in the proof of Lemma~\ref{LEM:description-of-low-degree-components-of-a-CI});
also recall that for homogeneous ideals, prime implies radical implies saturated.
}

\prop{
\label{PROP:basic-cubic-scroll-facts}
For a cubic scroll $\Sigma$ over $K=\ol{K}$, the following hold:
\begin{enumerate}[(1)]
    \item $\Sigma$ contains a nonempty open subscheme of a $(2,2)$-CI in $\Span(\Sigma)$.
    
    \item $\Sigma$ is singular if and only if it is a cone, in which case it is a cone over a cubic scroll of dimension $\dim(\Sigma)-1$.
\end{enumerate}
}

\pf{
By Definition~\ref{DEFN:cubic-scroll-of-dimension-d}, $\Sigma$ is, in the sense of \cite{eisenbud1987varieties}, a ``variety of minimal degree'' in $\Span(\Sigma)\cong \PP^{d+2}$, where $d\defeq \dim(\Sigma)$.
Now inspect the statement \cite{eisenbud1987varieties}*{Theorem~1} and elaboration \cite{eisenbud1987varieties}*{pp.~4--6 of \S1}.
We find that $\Sigma$ is, in the language of \cite{eisenbud1987varieties}, a ``rational normal scroll'' of the form $S(a_1,\dots,a_d)$, where $a_1,\dots,a_d$ are integers with $0\leq a_1\leq \cdots\leq a_d$ and $a_1+\dots+a_d = 3$;
that $S(a_1,\dots,a_d)$ is smooth if $a_1\geq 1$, and a cone over $S(a_2,\dots,a_d)$ if $a_1 = 0$;
and that the schemes $S(1,1,1)\belongs \PP^5$, $S(1,2)\belongs \PP^4$, and $S(3)\belongs \PP^3$ are (up to linear change of coordinates) defined by the homogeneous ideals $(x_3x_6-x_4x_5, x_1x_6-x_2x_5, x_1x_4-x_2x_3)$, $(x_3x_5-x_4^2, x_1x_5-x_2x_4, x_1x_4-x_2x_3)$, and $(x_2x_4-x_3^2, x_1x_4-x_2x_3, x_1x_3-x_2^2)$, respectively.
% $x_1/x_2 = x_3/x_4 = x_5/x_6$ and $x_1/x_2 = x_3/x_4 = x_4/x_5$ and $x_1/x_2 = x_2/x_3 = x_3/x_4$, respectively.

Claim~(1) follows, since $V_{\PP^5}(x_1x_6-x_2x_5, x_1x_4-x_2x_3)$, $V_{\PP^4}(x_1x_5-x_2x_4, x_1x_4-x_2x_3)$, and $V_{\PP^3}(x_1x_4-x_2x_3, x_1x_3-x_2^2)$ are $(2,2)$-CI's that coincide with $S(1,1,1)$, $S(1,2)$, and $S(3)$, respectively, away from $V(x_1)$.
It also follows that if $\Sigma$ is singular, then it is a cone.
On the other hand, if $\Sigma$ is a cone, say over $Y$,
% (so in particular, $Y$ is a hyperplane section of $\Sigma$)
then $Y\belongs \Sigma$ is an integral $(d-1)$-fold with $\deg{Y} = \deg{\Sigma} = 3$ and $\dim(\Span(Y)) = \dim(\Span(\Sigma)) - 1 = d+1$, i.e.~$Y$ is a cubic $(d-1)$-scroll;
so by \cite{eisenbud1987varieties}*{Theorem~1} and \cite{eisenbud1987varieties}*{pp.~4--6 of \S1}, we have $d-1\geq 1$, and $Y\cong S(b_1,\dots,b_{d-1})$ and $\Sigma\cong S(0,b_1,\dots,b_{d-1})$ for some integers $b_1,\dots,b_{d-1}$;
and thus $\Sigma$ is singular.
}

\prop{
\label{PROP:every-cubic-surface-contains-a-line}
Let $S\belongs \PP^3$ be a $(3)$-CI over $K=\ol{K}$.
Then $S$ contains a line.
}

\pf{
% http://www-personal.umich.edu/~mmustata/lectures_rationality.html
% http://www-personal.umich.edu/~mmustata/lecture2_rationality.pdf
See \cite{mustata2017lecture2}*{Theorem~1.1}.
}

For the rest of this section, let $k$ denote a finite field.

\defn{
\label{DEFN:nontrivial-or-error-relevant-Frob-eigenvalues-for-projective-X/F_q}
Fix a prime $\ell\nmid \card{k}$.
Let $Y,X$ denote arbitrary projective $k$-varieties.
\begin{enumerate}[(1)]
    \item Define $H^\bullet(Y)\defeq H^\bullet(Y\times\ol{k},\QQ_\ell)$ using $\ell$-adic cohomology with $\QQ_\ell$-coefficients.
    
    \item For each $i\geq 0$, let $\mcal{E}^i(Y)$ denote the multiset of (geometric) Frobenius eigenvalues on $H^i(Y)$.
    
    \item Let $\mcal{E}^i_{\triangle}(X,\PP)\defeq (\mcal{E}^i(X)\cup \mcal{E}^i(\PP^{\dim{X}}_k))\setminus (\mcal{E}^i(X)\cap \mcal{E}^i(\PP^{\dim{X}}_k))$ for $i\geq 0$.
    In other words, if $\alpha\in \ol{\QQ}_\ell$ has multiplicities $j_1,j_2\geq 0$ in $\mcal{E}^i(X),\mcal{E}^i(\PP^{\dim{X}}_k)$, let it have multiplicity $\abs{j_1-j_2}$ in $\mcal{E}^i_{\triangle}(X,\PP)$.
    
    \item Define $\mcal{E}_{\triangle}(X,\PP)\defeq \sum_{i\geq 0}\mcal{E}^i_{\triangle}(X,\PP)$ by ``summing multiplicities'' over $i$.
\end{enumerate}
}

\rmk{
Deligne's purity theorem implies that each $\mcal{E}^i(Y)$ above consists of $\card{k}$-Weil numbers $\alpha\in \ol{\QQ}_\ell$ of weight $w(\alpha)\leq i$.
% https://wiki.epfl.ch/quantumchaos2013/documents/PhMichel_SeminarTamas.pdf
(See e.g.~\cite{kiehl2001weil}*{Remark~I.7.2 and Theorem~I.9.3(2)} for a precise textbook reference in English.)
}

The following statement combines \cite{hooley1991number}*{Katz's Appendix, assertion~(2) in the proof of Theorem~1} and \cite{ghorpade2008etale}*{first sentence of Remark 3.5}.
Both are important for us, but the latter seems to appear without proof in \cite{ghorpade2008etale}, so we sketch some.
\thm{
[Katz, Skorobogatov, et al.]
\label{THM:general-perversity-result}
Fix $\ell\nmid \card{k}$.
Fix integers $n,N\geq 0$,
and a complete intersection $X\belongs \PP^n_k$ with $\dim{X} = N$ and $\codim{X}\geq 1$.
Let $D\defeq \dim(X_{\map{sing}})$.
If $i\in \ZZ$, then
(1) if $i\geq N+D+2$, then $\mcal{E}^i_{\triangle}(X,\PP) = \emptyset$;
and (2) if $i = N+D+1$, then $\mcal{E}^i(\PP^{N}_k)\belongs \mcal{E}^i(X)$.
}

\rmk{
By \cite{ghorpade2008etale}*{Proposition~3.2}, one could add the following to Theorem~\ref{THM:general-perversity-result}:
(3) if $i = N$, then $\mcal{E}^i(\PP^{N}_k)\belongs \mcal{E}^i(X)$;
and (4) if $0\leq i\leq N-1$, then $\mcal{E}^i_{\triangle}(X,\PP) = \emptyset$.
But we only need (1)--(2).
}

\pf{
[Proof sketch for Theorem~\ref{THM:general-perversity-result}]
When $X$ is a hypersurface in a smooth projective complete intersection $Y/k$ of dimension $\geq 2$, Theorem~\ref{THM:general-perversity-result} follows from \cite{skorobogatov1992exponential}*{Corollary~2.2, up to Veronese embedding}, Theorem~\ref{THM:Zak's-principle}, ``Betti comparison'' for $Y$, and the geometric irreducibility of $Y$.
In general, one can prove Theorem~\ref{THM:general-perversity-result} either inductively or directly.

\emph{Inductive proof.}
% [Inductive proof]
If $D\geq N-1$, use \cite{poonen2017rational}*{Corollary~7.5.21}.
Now suppose $D\leq N-2$ (which implies $N\geq 1$, since $D\geq -1$).
Using $N-D\geq 2$,
induct on $\codim{X}$ using \cite{skorobogatov1992exponential}*{Corollary~2.2}, \cite{ghorpade2008etale}*{Lemma~1.1(ii)}, and \cite{ghorpade2008etale}*{proofs of Theorem~2.4 and Proposition~2.5, up to Veronese embeddings}.
% (using the fact that $\dim\Sing$ decreases by at most $1$ for a hyperplane section (so that $\codim\Sing$ is weakly decreasing under taking hyperplane sections; note that \cite{ghorpade2008etale}*{Lemma~1.1(ii)} gives this), and the fact that projective complete intersections with $\codim\Sing \geq 2$ are always irreducible (cf.~\cite{ghorpade2008etale}*{proof of Proposition~2.5}); and in the case where $\codim\Sing = 1$, we can directly use \cite{poonen2017rational}*{Corollary~7.5.21} instead of inducting.)
% If $X$ is a hypersurface in a geometrically irreducible projective complete intersection $Y/k$ with $\dim{Y}\geq 2$ and $\dim(X_{\map{sing}})\leq \dim(Y)-2 = \dim(X)-1$ (e.g.~OK if $\dim(Y_{\map{sing}})\leq \dim(Y)-3$), and $Y$ satisfies Theorem~\ref{THM:general-perversity-result}, then so does $X$ (by \cite{skorobogatov1992exponential}*{Corollary~2.2, up to Veronese embedding}).

\emph{Direct proof.}
% [Direct proof]
Suppose $N\geq 1$ and $D\leq N-1$.
Claim~(1) follows from Katz \cite{hooley1991number}*{loc. cit.}.
And if $D = -1$, then (2) follows from weak Lefschetz.
Now assume $D\geq 0$, and let $i\defeq N+D+1$.
If $2\nmid i$, then $\mcal{E}^i(\PP^{N}_k) = \emptyset$, so (2) holds trivially.
Now suppose $2\mid i$.
Let $q\defeq \card{k}$.
Then $\mcal{E}^i(\PP^{N}_k) = \set{q^{i/2}}$, since $i\leq 2N$.
It remains to show that $q^{i/2}\in \mcal{E}^i(X)$.
Let $T\defeq \Aff^1_k$.
Following Katz, we can reduce to the case in which there exists
a closed subscheme $Z\belongs \PP^n_T$, flat over $T$, such that (i) $Z_0 = X$ and (ii) $Y\defeq Z_1$ is a smooth complete intersection in $\PP^n_k$ with $\dim{Y} = N$.
% Note: Clearly $Z/T$ is proper (since $Z$ is closed in $\PP^n_T$).
In this case, \cite{grothendieck1972groupes}*{Deligne's Expos\'{e}~I, Corollaire~4.3} implies that the specialization map $H^i(Z_t)\to H^i(Z\times_{T} \ol{k(T)}, \QQ_\ell)$ is an isomorphism at $t=1$ (since $D\geq 0$), and a surjection at $t=0$.
So by $G_k$-equivariance, $\mcal{E}^i(Y)\belongs \mcal{E}^i(X)$.
But $i\geq N+1$ (since $D\geq 0$), so $\mcal{E}^i(Y) = \mcal{E}^i(\PP^{N}_k)$ (by (1) for $Y$).
Thus $\set{q^{i/2}} = \mcal{E}^i(\PP^{N}_k)\belongs \mcal{E}^i(X)$.
}

The following standard result is also essential in the proof of Proposition~\ref{PROP:amplification-for-projective-varieties}:
\lem{
[Real amplification]
\label{LEM:standard-real-amplification-lemma}
Fix an integer $l\geq 0$ and a tuple $\bm{\beta}\in \set{z\in \CC: \abs{z}\geq 1}^l$.
Then $\limsup_{n\to \infty}\Re(\beta_1^{dn}+\dots+\beta_l^{dn})\geq l$ holds for every integer $d\geq 1$.
}

\pf{
We may assume $l\geq 1$ and $d=1$.
Then, Dirichlet's approximation theorem implies that $\limsup_{n\to \infty}\Re({\cdots})$ is $+\infty$ if $\norm{\bm{\beta}}_\infty > 1$, and $l$ otherwise.
}

\pf{
[Proof of Proposition~\ref{PROP:cubic-threefold-replace-Lindner}]
The first part is clear:
$X_{\map{sing}}(\ol{k})$ is $G_k$-invariant, so the geometric point $x\maps \Spec\ol{k}\to X$ must factor through $\Spec k$.
It remains to prove the second part.
Suppose $k=\FF_q$, and let $k_r\defeq \FF_{q^r}$ and $N_r(Y)\defeq \#Y(k_r)$ for any given integer $r\geq 1$ and $k$-scheme $Y$.
Clearly $X$ is reduced, so by \cite{debarre2021lines}*{\S2.3, eq.~(8)},
the Galkin--Shinder formulas imply (for all $r\geq1$) the equality
\mathd{
N_r(F(X))
= \frac{N_r(X)^2
- 2(1+\card{k_r}^{\dim{X}})N_r(X)
+ N_{2r}(X)}{2\card{k_r}^2}
+ \card{k_r}^{-2+\dim{X}}N_r(X_{\map{sing}}).
}

Now recall our assumption on $X_{\map{sing}}(\ol{k})$.
By an inspection of the three cases of \cite{debarre2021lines}*{\S4.4, Proposition~4.8},
it follows that there exists a smooth genus $4$ curve $C/\FF_q$ such that $N_r(F(X)) = \frac12\left(N_r(C)^2 + O(N_r(C)) + N_{2r}(C)\right)$.
Thus $N_r(F(X)) = \card{k_r}^2 + O(\card{k_r}^{3/2})$.
Since $\dim{X}=3$ and $N_r(X_{\map{sing}})=O(1)$,
it follows that
\mathd{
N_r(X)^2
- 2(1+\card{k_r}^3)N_r(X)
+ N_{2r}(X)
= 2\card{k_r}^4
+ O(\card{k_r}^{7/2}).
}
For all $d\geq1$,
let $E_d(X)\defeq N_d(X)-\card{\PP^3(k_d)}$.
Then $N_r(X)-\card{k_r}^3=E_r(X)+\card{k_r}^2+O(\card{k_r})$ and $N_{2r}(X)=E_{2r}(X)+\card{k_r}^6+\card{k_r}^4+O(\card{k_r}^2)$,
so we easily obtain
\mathd{
(E_r(X)+\card{k_r}^2)^2
+ O(E_r(X)\cdot\card{k_r})
+ O(N_r(X))
+ E_{2r}(X)
= \card{k_r}^4+O(\card{k_r}^{7/2}).
}

Now fix $\ell\neq p$,
fix an embedding $\iota\maps\QQ_\ell\inject\CC$,
and let $\alpha_1,\dots,\alpha_b\in\CC$ denote the weight-$4$ eigenvalues on $\CC\otimes_{\QQ_\ell}H^4(X)$.
For each $i\in[b]$,
let $\tilde{\alpha}_i\defeq\card{k}^{-4/2}\alpha_i\in S^1$.
Then $E_d(X) = \card{k_d}^{4/2}(\tilde{\alpha}_1^d+\dots+\tilde{\alpha}_b^d-1) + O_X(\card{k_d}^{3/2})$ uniformly over $d\geq1$, by Theorem~\ref{THM:general-perversity-result}(1).
In particular, $E_d(X)\ll_{X}\card{k_d}^2$ and $N_d(X)\ll_{X}\card{k_d}^3$, so ultimately,
\mathd{
\left((\tilde{\alpha}_1^r+\dots+\tilde{\alpha}_b^r-1)+1\right)^2
+ \left(\tilde{\alpha}_1^{2r}+\dots+\tilde{\alpha}_b^{2r}-1\right)
= 1+O_X(\card{k_r}^{-1/2}).
}
Dirichlet's approximation theorem (applied to the \emph{phases} $\tilde{\alpha}_1,\dots,\tilde{\alpha}_b$) then yields $b^2 + (b-1) = 1$ (cf.~Lemma~\ref{LEM:standard-real-amplification-lemma}),
whence $b=1$ (since $b\geq0$).
Theorem~\ref{THM:general-perversity-result}(2) then gives $\tilde{\alpha}_1 = 1$,
which implies (by the trace formula) that $X$ is $\abs{E}$-good.
}

\section{Analyzing cubic threefolds as conic bundles}
\label{SEC:conic-bundle-analysis-of-cubic-threefolds-over-F_q}

Fix a finite field $k\defeq\FF_q$.
Let $X\defeq V(C)$ denote a projective cubic hypersurface in $\PP^4_k$.
\emph{Assume} $\dim(X_{\map{sing}})\leq 0$.
(In particular,
$C$ must be absolutely irreducible,
by Theorem~\ref{THM:general-perversity-result}(1) and \cite{poonen2017rational}*{Corollary~7.5.21}.)

\subsection{Counting using a conic bundle structure}

In addition to assuming $\dim(X_{\map{sing}})\leq 0$,
we now make the following assumptions.
\begin{enumerate}[(1)]
    \item We assume $p\geq 3$ whenever necessary or convenient.
    
    \item We \emph{assume} that $X_{\ol{k}}$ does \emph{not} contain any $2$-plane $P$.
\end{enumerate}
% assumed to be plane-free with isolated singularities.
Under the above assumptions,
we will proceed to count points roughly along the lines of \cite{debarre2021lines}*{\S4.3} and \cite{github-alaface-CubLin-2016}
(but note that we have not assumed smoothness).

First, it is known that every projective cubic threefold over $\ol{k}$ contains a line
(since every projective cubic surface does).
By enlarging $k$
(and applying a $\GL_5(k)$-transformation)
if necessary,
we may assume that $X$ contains $L\defeq V(x_1,x_2,x_3)$.

Next, given a $5$-vector $\bm{x}$,
let $\pi(\bm{x})$ denote the $3$-vector $(x_1,x_2,x_3)$.
Then we may write
\mathd{
C(\bm{x})
= f
+ 2q_1x_4+2q_2x_5
+ l_1x_4^2+2l_2x_4x_5+l_3x_5^2,
}
for some forms $f,q_i,l_j$ in $\pi(\bm{x})$,
with $f$ cubic, $q_i$ quadratic, and $l_j$ linear.

% \prop{
% $X$ contains a given plane $P$ if and only if (?) $X_{\map{sing}},P$ intersect with multiplicity $\geq 4$ (after base change to $\ol{k}$, say).
% }

% \prop{
% In the above setting,
% $X$ does \emph{not} contain any planes $P\contains L$.
% (EDIT: THIS MAY OR MAY NOT BE TRUE)
% }

% \pf{
% Suppose for contradiction
% that $X\contains P\contains L$ for some plane $P$.
% By applying a $\GL_3(k)$-transformation if necessary,
% we may assume that $P = V(x_1,x_2)$,
% i.e.~that $P$ is the plane passing through $[e_3],[e_4],[e_5]$.
% Then every monomial of $C$ must be divisible by $x_1$ or $x_2$.
% Therefore $\partial_{x_3}C,\partial_{x_4}C,\partial_{x_5}C$ must vanish identically on the cone of $P$
% (i.e.~at points of the form $\bm{x}=(0,0,x_3,x_4,x_5)$).

% Furthermore, if we write $C(\bm{x}) = O(x_1^2,x_1x_2,x_2^2) + \sum_{i,j}l_{ij}x_ix_j$ for $i,j\in\set{3,4,5}$,
% then $\partial_{x_1}C,\partial_{x_2}C$ simplify to $Q_1\defeq\sum_{i,j}(\partial_{x_1}l_{ij})x_ix_j$ and $Q_2\defeq\sum_{i,j}(\partial_{x_2}l_{ij})x_ix_j$ (ternary quadratic forms in $x_3,x_4,x_5$) on the cone of $P$.

% Without further conditions on the original singular points of $X$,
% this could be problematic if the resulting conics intersect at $\leq 2$ distinct points.
% % https://math.stackexchange.com/questions/631815/two-conics-have-exactly-one-intersection-point
% }

By assumption, if $\bm{x}'=(x_1,x_2,x_3)\in k^3\setminus\set{\bm{0}}$,
the plane $P_{[\bm{x}']}$ through $L,[\bm{x}']$ (i.e. through $[e_4],[e_5],[\bm{x}']$)
is \emph{not} contained in $X$.
In other words, $(f,q_i,l_j)(\bm{x}')\neq\bm{0}$.
So let $Q_{\bm{x}'}$ denote the projective conic
\mathd{
V(fy_1^2
+ 2q_1y_1y_2+2q_2y_1y_3
+ l_1y_2^2+2l_2y_2y_3+l_3y_3^2)
\belongs\set{[\bm{y}]\in\PP^2},
}
canonically embedded into $X$ via $[\bm{y}]\mapsto [y_1\bm{x}'+y_2e_4+y_3e_5]$,
with image $Q_{[\bm{x}']}\belongs X$, say.
% Then up to $k$-isomorphism, $Q_{\bm{x}'}$ only depends on $[\bm{x}']\in\PP^2(k)$.

Note that the (restricted) projection $[\pi]\maps X\setminus L\to\PP^2$ fibers $X\setminus L$ into affine conics $Q_{[\bm{x}']}\setminus L$;
equivalently, the plane $P_{[\bm{x}']}$ intersects $X$ in the product of $L$ and $Q_{[\bm{x}']}$.

\rmk{
We are essentially writing $\pi(\bm{x}) = y_1\bm{x}'$,
and ``factoring out'' $y_1$ to get $Q_{\bm{x}'}$.
}

\defn{
Let $M\defeq\inmat{f & q_1 & q_2 \\ q_1 & l_1 & l_2 \\ q_2 & l_2 & l_3}$
and $(\delta_1,\delta_2,\delta_3)\defeq (l_1l_3-l_2^2,fl_3-q_2^2,fl_1-q_1^2)$.
Let $\Gamma\defeq V(\det{M})\belongs\set{[\bm{x}']\in\PP^2}$,
where
\mathd{
\det{M}
= f\delta_1
- q_1(q_1l_3-q_2l_2)
+ q_2(q_1l_2-q_2l_1)
= f\delta_1
- q_1^2l_3 - q_2^2l_1
+ 2q_1q_2l_2
\in k[\bm{x}'].
}
}

\prop{
Here
\mathds{
\#X(k)-\#\PP^3(k)
= &- q^2\#\set{[\bm{x}]\in L(k): l_1x_4^2+2l_2x_4x_5+l_3x_5^2=0} \\
&+ q\sum_{[\bm{x}']\in\Gamma(k)}(-1+\#\set{\textnormal{distinct $k$-lines on $Q_{[\bm{x}']}$}}).
}
}

\rmk{
As $[\bm{x}']$ varies over $\Gamma$,
the lines on $Q_{[\bm{x}']}$ presumably cut out a closed subscheme $\Gamma'$ of the Fano scheme of lines $F(X)$,
which in turn sits in the Grassmannian $\Gr(2,5)\belongs\PP(\bigwedge^2{k^5})$.
%https://encyclopediaofmath.org/wiki/Fano_scheme
However, it is unclear
(both theoretically and computationally)
how suitable or well-developed the theory of $\Gamma'$ is
for our purposes.
}

\pf{
[Proof sketch]
Here each $Q_{[\bm{x}']}$ is in fact a conic (due to the plane-free assumption on $X$).
So we may follow the casework of \cite{debarre2021lines}*{pp.~8--9, proof of Proposition~4.6}.
Specifically, write $\#X(k)-\#L(k)$ as
\mathd{
\sum_{[\bm{x}']\in\PP^2(k)}\#Q_{[\bm{x}']}(k)
- \sum_{[\bm{x}']\in\PP^2(k)}\#\set{[\bm{x}]\in L(k):l_1x_4^2+2l_2x_4x_5+l_3x_5^2=0}.
}
The first sum $\Sigma_1$ simplifies (upon considering smooth and singular $Q$'s separately) to
\mathd{
(q+1)(q^2+q+1)
+ q\sum_{[\bm{x}']\in\Gamma(k)}(-1+\#\set{\textnormal{distinct $k$-lines on $Q_{[\bm{x}']}$}}),
}
while the second sum $\Sigma_2$ simplifies (upon ``switching the order of $\bm{x}',\bm{x}$'') to
\mathd{
(q+1)^2
+ q^2\#\set{[\bm{x}]\in L(k): l_1x_4^2+2l_2x_4x_5+l_3x_5^2=0\;\forall\bm{x}'}.
}
The result follows upon noting $\#L(k) + (q+1)(q^2+q+1) - (q+1)^2 = \#\PP^3(k)$.
%https://www.wolframalpha.com/input/?i=%28q%2B1%29%2B%28q%2B1%29%28q%5E2%2Bq%2B1%29-%28q%2B1%29%5E2
}

\rmk{
The proof above works for an arbitrary projective cubic threefold $X/k$ containing $L$
such that there exists no $2$-plane $P/\ol{k}$ with $L_{\ol{k}}\belongs P\belongs X_{\ol{k}}$.
(No hypothesis on isolated singularities is needed.)
However, the formula could easily (but somewhat tediously) be adjusted to allow for such $2$-planes.
}

\prop{[Cf.~\cite{debarre2021lines}*{p.~9, Proposition~4.7}]
\label{PROP:splitting-type-of-singular-plane-conic}
% See also notes in pp. 4--5 of \verb|2019-08-18_Computing_L_functions_and_verifying_automorphy|.
Fix $[\bm{x}']\in\Gamma(k)$.
Then $Q_{[\bm{x}']}$ is
\begin{enumerate}[(1)]
    \item \emph{reducible over $k$},
    i.e.~the product of two \emph{distinct} $k$-lines,
    if and only if $(\delta_1,\delta_2,\delta_3)\neq\bm{0}$ and $-\delta_i\in (k^\times)^2\cup\set{0}$ for all $i\in[3]$;
    
    \item \emph{non-reduced} over $k$,
    i.e.~a \emph{double} $k$-line,
    if and only if $(\delta_1,\delta_2,\delta_3)=\bm{0}$;
    and
    
    \item \emph{integral over $k$},
    i.e.~the product of two \emph{distinct conjugate} lines defined over $\ol{k}$ but not over $k$,
    otherwise.
\end{enumerate}
}

\pf{
By assumption, $Q_{[\bm{x}']}$ is in fact a conic.
Here $Q_{[\bm{x}']}\cong_{k} Q_{\bm{x}'}\defeq V(\bm{y}^TM\bm{y})$,
where
\mathd{
\bm{y}^TM\bm{y}
= fy_1^2
+ 2q_1y_1y_2+2q_2y_1y_3
+ l_1y_2^2+2l_2y_2y_3+l_3y_3^2.
}
By assumption, $\det{M}=0$,
from which a routine computation
yields $\delta_2\delta_3 = (q_1q_2-fl_2)^2\in k^2$.
% See notes on p.~5 of \verb|2019-08-18_Computing_L_functions_and_verifying_automorphy|.
Similarly, $\delta_1\delta_2\in k^2$ and $\delta_1\delta_3\in k^2$.
In particular, if $-\delta_i\in(k^\times)^2$ for \emph{some} $i\in[3]$,
then $-\delta_j\in k^2$ for \emph{all} $j\in[3]$.

Now fix $i\in[3]$.
Essentially by \cite{debarre2021lines}*{first paragraph of proof of Proposition~4.7}
(i.e.~casework on $\card{(Q_{\bm{x}'}\cap V(y_i))(k)}$),
we know that \emph{if $\delta_i\neq0$},
then $V(y_i)\not\belongs Q_{\bm{x}'}$, and
\begin{itemize}
    \item $Q_{\bm{x}'}$ is $k$-reducible if $-\delta_i\in(k^\times)^2$,
    while
    
    \item $Q_{\bm{x}'}$ is $k$-reduced \emph{and} $k$-irreducible
    (i.e.~$k$-integral)
    otherwise.
\end{itemize}

Consequently, \emph{if $\bm{\delta}\neq\bm{0}$},
then we have established the desired dichotomy between (1) and (3).
Finally, \emph{suppose $\bm{\delta}=\bm{0}$}.
Then it remains precisely to show that $Q_{\bm{x}'}$ is a double $k$-line,
or equivalently (by Galois theory) that $(Q_{\bm{x}'})_{\ol{k}}$ is a double $\ol{k}$-line.
To this end, write $(Q_{\bm{x}'})_{\ol{k}}$ as the product of two $\ol{k}$-lines $L_1,L_2$,
and note that for each $i\in[3]$,
we either have $V(y_i)_{\ol{k}}\in\set{L_1,L_2}$,
or else $V(y_i)_{\ol{k}}\cap L_1 = \set{p_i} = V(y_i)_{\ol{k}}\cap L_2$ for some point $p_i\in\PP^2(k)$.
In any case, we can find points $q_i\in\PP^2(k)$ with $q_i\in V(y_i)_{\ol{k}}\cap L_1\cap L_2$.
Here we must have $\set{q_1}\cap\set{q_2}\cap\set{q_3}=\emptyset$,
since $V(y_1,y_2,y_3)_{\ol{k}}=\emptyset$.
But also, $\set{q_1,q_2,q_3}\belongs L_1\cap L_2$.
Thus $L_1=L_2$, as desired.
}

% \pf{[Proof when $(f,l_1,l_3)\neq\bm{0}$]
% By assumption, $Q_{[\bm{x}']}$ is in fact a conic.
% Here $Q_{[\bm{x}']}\cong_{k} Q_{\bm{x}'}\defeq V(\bm{y}^TM\bm{y})$,
% where
% \mathd{
% \bm{y}^TM\bm{y}
% = fy_1^2
% + 2q_1y_1y_2+2q_2y_1y_3
% + l_1y_2^2+2l_2y_2y_3+l_3y_3^2.
% }
% In the present proof,
% we will treat $(f,q_i,l_j)$ merely as an \emph{arbitrary element} of $k^6\setminus\set{\bm{0}}$
% with $(f,l_1,l_3)\neq\bm{0}$ and $\det{M}=0$.
% Thus, by symmetry, we may \emph{assume} $f\neq\bm{0}$.

% Now substitute $(y_2,y_3) = (sz_2,sz_3)$,
% so that $\bm{y}^TM\bm{y}$ becomes a binary quadratic form $\mcal{Q}\in(k[\bm{z}])[y_1,s]$.
% Then a $k[\bm{y}]$-factorization of $\bm{y}^TM\bm{y}$
% directly induces a $(k[\bm{z}])[y_1,s]$-factorization of $\mcal{Q}$.
% Likewise, a $(k[\bm{z}])[y_1,s]$-factorization of $\mcal{Q}$
% induces a $k[\bm{y},s,s^{-1}]$-factorization of $\bm{y}^TM\bm{y}$,
% which in turn (upon ``cancelling'' powers of $s$)
% must directly simplify to a $k[\bm{y}]$-factorization of $\bm{y}^TM\bm{y}$.

% In the ``correspondence of factors'' above,
% the ``$y_1$-degrees of the factors'' always match.
% Therefore, \emph{if $\deg_{y_1}\mcal{Q}=2$} (as we have \emph{assumed}), then to determine the factorization type of $\bm{y}^TM\bm{y}$ in $k[\bm{y}]$,
% it suffices to determine the factorization type of $\mcal{Q}$ in $(k[\bm{z}])[y_1,s]$,
% or equivalently (by Gauss' lemma) in $k(\bm{z})[y_1,s]$.

% But the factorization type of $\mcal{Q}$ in $k(\bm{z})[y_1,s]$
% is completely determined by
% the ``square class'' of the discriminant $-4\det\mcal{Q}$ in $k(\bm{z})$.
% Here
% \mathds{
% -\det\mcal{Q}
% &= (q_1z_2+q_2z_3)^2
% - f\cdot(l_1z_2^2+2l_2z_2z_3+l_3z_3^2) \\
% &= -\delta_3z_2^2
% +2(q_1q_2-fl_2)z_2z_3
% -\delta_2z_3^2
% \in k[\bm{z}].
% }
% A routine computation, using $\det{M}=0$,
% shows that $\delta_2\delta_3 = (q_1q_2-fl_2)^2\in k^2$;
% % See notes on p.~5 of \verb|2019-08-18_Computing_L_functions_and_verifying_automorphy|.
% similarly, $\delta_1\delta_2\in k^2$ and $\delta_1\delta_3\in k^2$.
% In particular, if $-\det\mcal{Q}\neq0$,
% then $(-\delta_2,-\delta_3)\neq\bm{0}$,
% and furthermore the desired dichotomy between (1) and (3) holds.
% Finally, suppose $\det\mcal{Q}=0$.
% Then $(\delta_2,\delta_3)=\bm{0}$,
% and it remains precisely to show that $\delta_1=0$.
% The formal identity
% \mathd{
% f^2\delta_1
% = (fl_1)(fl_3)-(fl_2)^2
% \equiv q_1^2q_2^2-(fl_2)^2
% \bmod{(\delta_2,\delta_3)}
% }
% completes the proof (since $q_1q_2-fl_2=0$ and $f\neq0$).
% }

% \pf{[Proof when $(f,l_1,l_3)=\bm{0}$]
% Here $\det{M}=0$ implies $2q_1q_2l_2=0$, so $\bm{y}^TM\bm{y}$ consists of at most two monomials among $2q_1y_1y_2, 2q_2y_1y_3, 2l_2y_2y_3$, and the only possibility (since $(f,q_i,l_j)\neq\bm{0}$) ends up being (1).
% }



To go further, we must analyze $X$'s singularities (or lack thereof).

\obs{
$C$ is singular at a given (geometric) point $\bm{x}=[x_4e_4+x_5e_5]\in L(\ol{k})$ if and only if $l_1x_4^2+2l_2x_4x_5+l_3x_5^2$ vanishes identically as a linear form in $\bm{x}'$.
}

\pf{
By assumption, ``$C\vert_L$'' vanishes identically, so $C,\partial_{x_4}C,\partial_{x_5}C$ vanish on the (affine) cone of $L$.
Next, if we write $C(\bm{x}) = O((\bm{x}')^2) + l_1x_4^2+2l_2x_4x_5+l_3x_5^2$,
then $\partial_{x_i}C$, for $i\in [3]$, simplifies to $(\partial_{x_i}l_1)x_4^2+2(\partial_{x_i}l_2)x_4x_5+(\partial_{x_i}l_3)x_5^2$ (a binary quadratic form in $x_4,x_5$) on the cone of $L$.
But $l_1,l_2,l_3$ are linear,
so the desired result immediately follows.
}

\rmk{
In particular, $\#(X_{\map{sing}}\cap L)(\ol{k})\leq 2$,
since $X$ has isolated singularities.
There is a conceptual proof of this fact:
``$\grad{x_i}C\vert_L$'' is a quadratic polynomial on $L$
for each $i$, so $L\not\belongs X_{\map{sing}}$ implies that there exists $i$ such that ``$\grad{x_i}C\vert_L$'' has at most $2$ distinct roots.
}

% As a sanity check,
% we can numerically verify the propositions (and observation) in an
% \ex{
% No $2$-planes through $L$: check that $f,q_1,q_2,l_1,l_2,l_3$ have no common zeros $[\bm{x}']\in\PP^2(\ol{k})$.
% }
\ex{
The data in \cite{github-singular-cubic-threefold-2021}*{{\tt Data for...~X (in general).xlsx}} seems consistent with the propositions (and observation) above.
}


\subsection{A convenient choice of a line}

By enlarging $k$
(and applying a $\GL_5(k)$-transformation)
if necessary,
we may assume that
\mathd{
\card{X_{\map{sing}}(\ol{k})\cap \set{[e_4],[e_5]}}
= \min(\card{X_{\map{sing}}(\ol{k})}, 2),
}
and that $X$ contains the line $L\defeq V(x_1,x_2,x_3)$ through $[e_4],[e_5]$.
To justify the existence of such a ``convenient line'' $L$ requires some proof,
which we now give (by casework).

\pf{
[Proof when $\card{X_{\map{sing}}(\ol{k})}\geq 2$]
Say $X_{\map{sing}}(\ol{k})\contains \set{[e_4],[e_5]}$.
Then the binary cubic form ``$C\vert_L$''
vanishes identically,
because $C$ itself is singular at $[e_4],[e_5]\in L(\ol{k})$
(forcing ``$C\vert_L$'' to vanish to order $\geq 4$).
% ``$C\vert_L$'' must vanish to order $\geq 2$ at the singularities $[e_4],[e_5]\in (X\cap L)(\ol{k})$.
% (More explicitly, we could evaluate the derivatives $\partial_{x_4}C,\partial_{x_5}C$ at $e_4,e_5$.)
}

\pf{
[Proof when $\card{X_{\map{sing}}(\ol{k})} = 1$]
Say $X_{\map{sing}}(\ol{k}) = \set{[e_4]}$.
We must show that $[e_4]$ is contained in a line on $X_{\ol{k}}$.
To this end,
consider an arbitrary (nontrivial) hyperplane section $S$ of $X_{\ol{k}}$.
% Is it true, for instance, that every singular point on a cubic surface must be contained in a line on that surface?
Then $S$ is a projective cubic surface,
so it must contain a line $l$.
%http://www.math.ucsd.edu/~jmckerna/Teaching/13-14/Winter/203B/l_15.pdf
If $[e_4]\in l$, then we are done,
so suppose not.
Now let $P$ denote the (unique) plane through $[e_4],l$.
If $P\belongs S$, then we may simply choose any line on $P$ through $[e_4]$.
So suppose $P\not\belongs S$.
Then $P\cap S$ is a plane cubic containing $[e_4],l$.
But, crucially,
the ternary cubic form ``$C\vert_{P\cap S}$'' must be singular at $[e_4]$
(since $C$ itself is singular at $[e_4]$).
Since $[e_4]\notin l$, we conclude that $[e_4]$ is a singular point of a \emph{conic} $Q$,
which can only occur if $Q$ is a product of two lines passing through $[e_4]$.
Either line then suffices.
}

\pf{
[Proof when $X_{\map{sing}}(\ol{k}) = \emptyset$]
We must show that $X_{\ol{k}}$ contains a line.
It suffices to choose any (nontrivial) hyperplane section $S\belongs X_{\ol{k}}$,
and then any line $l\belongs S$.
(By Bertini's theorem, we can choose $S$ to be smooth if desired, but this is unnecessary.)
% In fact (citation needed?), every point $x\in X(\ol{k})$ is contained in a line on $X_{\ol{k}}$ (provided $p\gg 1$?).
}

Henceforth, we fix a ``convenient line'' $L$ as above,
though in principle other lines could also be used.

\subsection{The smooth case}

See \cite{debarre2021lines}*{par.~4 of \S4.3}.
(Here $\Gamma'\to\Gamma$ is a fairly ``nice'' double cover,
already analyzed by \cite{bombieri1967local}.)

\subsection{The case of exactly one singular point contained in \texpdf{$L$}{L}}

By assumption, either $l_1=0$ or $l_3=0$.
Assume the former ($l_1=0$).
Then either $l_2=0$ and $l_3\neq 0$,
or else $l_2,l_3$ are linearly independent.
(Otherwise, $X_{\map{sing}}$ would contain a point in $L\setminus\set{[e_4]}$.)
In particular,
\mathd{
\#\set{[\bm{x}]\in L(k): l_1x_4^2+2l_2x_4x_5+l_3x_5^2=0}
= \#\set{[\bm{x}]\in L(k): x_5=0}
= 1.
}

Here $(\delta_1,\delta_2,\delta_3) = (-l_2^2,fl_3-q_2^2,-q_1^2)$ and
\mathd{
\det{M}
= f\delta_1-q_1^2l_3+2q_1q_2l_2
= -fl_2^2 + 2q_2l_2q_1-l_3q_1^2.
% = (-fl_2+2q_1q_2)l_2 - q_1^2l_3$.
}
Here $C = f+2q_1x_4+2q_2x_5+2l_2x_4x_5+l_3x_5^2$.

\prop{
Here $\det{M}$ is a \emph{nonzero} ternary quintic form in $k[\bm{x}']$.
}

\pf{[Proof when $l_2\neq0$]
Noting the linearity of $C$ in $x_4$, we write
\mathd{
C
= f+2q_2x_5+l_3x_5^2
+ 2(q_1+l_2x_5)x_4.
}
By assumption, $C$ is (absolutely) irreducible,
so $\gcd(f+2q_2x_5+l_3x_5^2, q_1+l_2x_5) = 1$ in $\ol{k}[\bm{x}',x_5]$.
Consequently, Gauss' lemma implies that $\gcd(f+2q_2x_5+l_3x_5^2, q_1+l_2x_5) = 1$ in $\ol{k}(\bm{x}')[x_5]$.
But in $\ol{k}(\bm{x}')[x_5]$,
the remainder of $f+2q_2x_5+l_3x_5^2$ modulo $l_2x_5+q_1$
is precisely $-l_2^{-2}\det{M}$,
so $\det{M}\neq 0$, as desired.
}

\pf{[Proof when $l_2=0$]
Here $l_3\neq 0$, and $\det{M} = -l_3q_1^2$.
Also, $C = f+2q_1x_4+2q_2x_5+l_3x_5^2$.
Suppose for contradiction that $q_1=0$.
% Then $C$ is a cubic form in $4$ variables (namely $\bm{x}',x_5$),
Then $X$ is the (projective) cone over a projective cubic surface $S$.
Now fix a line $l$ on $S_{\ol{k}}=X_{\ol{k}}\cap\set{x_4=0}$.
Then $l$ extends to a plane on $X_{\ol{k}}$,
contradicting our plane-free assumption on $X$.
Thus in fact $q_1\neq0$, so $\det{M}\neq0$.
}

\cor{
Here
\mathd{
\#X(k)-\#\PP^3(k)
\leq -q^2
+ (q^2+O(q^{3/2}))
\#\set{\textnormal{irreducible components of $\Gamma_{\ol{k}}$}}.
}
}

\pf{[Proof sketch]
Similar to the analogous corollary when $\card{X_{\map{sing}}(\ol{k})}\geq 2$ (treated below).
}

\prop{
If $[\bm{x}']\in\Gamma(k)$,
then the number of distinct $k$-lines on $Q_{[\bm{x}']}$ is
\begin{enumerate}[(1)]
    % \item exactly $1$,
    % if $(l_2,l_3,q_1,q_2)=\bm{0}$;
    
    \item exactly $1+(\frac{-\delta_2(\bm{x}')}{k})$,
    defined using the Legendre symbol over $k$,
    if $l_2(\bm{x}')=q_1(\bm{x}')=0$;
    and
    
    \item exactly $2$,
    otherwise.
\end{enumerate}
}

\pf{
Apply Proposition~\ref{PROP:splitting-type-of-singular-plane-conic}.
}

% \pf{
% Fix $\bm{x}'\in k^3\setminus\set{\bm{0}}$ with $\det{M}=0$.
% Then $fl_2^2-2q_2l_2q_1+l_3q_1^2=0$.

% Here $Q_{[\bm{x}']}\cong_{k} Q_{\bm{x}'}\defeq V(\bm{y}^TM\bm{y})$,
% where
% \mathd{
% \bm{y}^TM\bm{y}
% = fy_1^2+2q_2y_1y_3+l_3y_3^2
% +2(q_1y_1+2l_2y_3)y_2.
% }
% By assumption, $(f,q_1,q_2,l_2,l_3)\neq\bm{0}$ at $\bm{x}'$.
% If $l_2=0$, then $l_3q_1^2=0$.
% Observe:
% \begin{itemize}
    
%     \item If $l_2=l_3=0$,
%     then $\bm{y}^TM\bm{y}=y_1(fy_1+2q_1y_2+2q_2y_3)$,
%     and $(f,q_1,q_2)\neq\bm{0}$.
    
%     \item If $l_2=q_1=0$,
%     then $\bm{y}^TM\bm{y}=fy_1^2+2q_2y_1y_3+l_3y_3^2$,
%     and $(f,q_2,l_3)\neq\bm{0}$.
    
%     \item If $l_2\neq0$, then $f = [2q_2l_2-l_3q_1]q_1/l_2^2$,
%     % $f = 2q_2(q_1/l_2) - l_3(q_1/l_2)^2$,
%     so
%     \mathd{
%     \bm{y}^TM\bm{y}
%     = l_2^{-2}(q_1y_1+l_2y_3)([2q_2l_2-l_3q_1]y_1 + 2l_2^2y_2+l_2l_3y_3).
%     }
%     %cf. https://www.wolframalpha.com/input/?i=q_1%282q_2l_2-q_1l_3%29+%2B+%282q_1x_4%2B2q_2x_5%2B2l_2x_4x_5%2Bl_3x_5%5E2%29l_2%5E2
% \end{itemize}
% In the first and third situations,
% $\bm{y}^TM\bm{y}$ is a product of $2$ linear forms in $k[\bm{y}]$,
% which are proportional if and only if $l_2=0$ and $q_1=q_2=0$.
% (In the third situation, note that $l_2,2l_2^2\neq0$.)
% The second situation can be analyzed separately.
% }



\subsection{The case of two or more isolated singularities in \texpdf{$L$}{L}}

By assumption, $l_1=l_3=0$.
Hence $l_2\neq 0$ (or else $X_{\map{sing}}$ would contain $L$ entirely).
In particular,
\mathd{
\#\set{[\bm{x}]\in L(k): l_1x_4^2+2l_2x_4x_5+l_3x_5^2=0}
= \#\set{[\bm{x}]\in L(k): x_4x_5=0}
= 2.
}

Here $(\delta_1,\delta_2,\delta_3) = (-l_2^2,-q_2^2,-q_1^2)$ and $\det{M} = f\delta_1+2q_1q_2l_2 = (-fl_2+2q_1q_2)l_2$.

\prop{
Here $\det{M}$ is a \emph{nonzero} ternary quintic form in $k[\bm{x}']$.
}

\pf{
Here $C = f+2q_1x_4+2q_2x_5+2l_2x_4x_5$.
But by assumption, $C$ is (absolutely) irreducible,
so $f\cdot 2l_2\neq 2q_1\cdot 2q_2$.
(Otherwise, there would exist forms $a,b,c,d$ with $(f,2l_2) = (ab,cd)$ and $(2q_1,2q_2) = (ac,bd)$, and
we would have $C = (a+dx_5)(b+cx_4)$,
%https://www.wolframalpha.com/input/?i=ab+%2B+ac*x_4+%2B+bd*x_5+%2B+cd*x_4x_5
with both factors necessarily nonzero of degree $\geq 1$;
contradiction.)

Earlier, we noted that $l_2\neq 0$.
So in fact, $\det{M} = (-fl_2+2q_1q_2)l_2\neq 0$.
}

\cor{
Here
\mathd{
\#X(k)-\#\PP^3(k)
\leq -2q^2
+ (q^2+O(q^{3/2}))
\#\set{\textnormal{irreducible components of $\Gamma_{\ol{k}}$}}.
}
}

\pf{[Proof sketch]
Apply the trivial bound $-1+\#\set{\textnormal{distinct $k$-lines on $Q_{[\bm{x}']}$}}\leq 1$.
Then note that
\mathd{
\card{\Gamma(k)}
= O(1)
+ (q+O(q^{1/2}))
\#\set{\textnormal{irreducible components of $\Gamma_{\ol{k}}$ defined over $k$}}
}
by the Lang--Weil bound, for instance.
(Note that if $\mcal{C}/\ol{k}$ is an embedded projective curve \emph{not} defined over $k$,
then there exists a conjugate $\mcal{C}'/\ol{k}$ distinct from $\mcal{C}$,
so that ``$\mcal{C}(k)$'' is contained in the \emph{finite} set $(\mcal{C}\cap\mcal{C}')(\ol{k})$ of size $O(1)$.)
}

\prop{
If $[\bm{x}']\in\Gamma(k)$,
then the number of distinct $k$-lines on $Q_{[\bm{x}']}$ is exactly $1$ if $(l_2,q_1,q_2)(\bm{x}')=\bm{0}$,
and exactly $2$ otherwise.
}

\pf{
Apply Proposition~\ref{PROP:splitting-type-of-singular-plane-conic}.
}

% \pf{
% Fix $\bm{x}'\in k^3\setminus\set{\bm{0}}$ with $\det{M}=0$.
% Then either $fl_2=2q_1q_2$ or $l_2=0$.

% Here $Q_{[\bm{x}']}\cong_{k} Q_{\bm{x}'}\defeq V(\bm{y}^TM\bm{y})$,
% where
% \mathd{
% \bm{y}^TM\bm{y}
% = fy_1^2+2q_1y_1y_2+2q_2y_1y_3+2l_2y_2y_3.
% }
% By assumption, $(f,q_1,q_2,l_2)\neq\bm{0}$.
% Observe:
% \begin{itemize}
%     \item If $l_2=0$, then $\bm{y}^TM\bm{y}=y_1(fy_1+2q_1y_2+2q_2y_3)$, and $(f,q_1,q_2)\neq\bm{0}$.
    
%     \item If $l_2\neq0$, then $f = 2q_1q_2/l_2$,
%     so $\bm{y}^TM\bm{y}=2l_2^{-1}(q_1y_1+l_2y_3)(q_2y_1+l_2y_2)$.
%     %https://www.wolframalpha.com/input/?i=%282q_1q_2%2Fl_2%29y_1%5E2+%2B+2q_1y_1y_2+%2B+2q_2y_1y_3+%2B+2l_2y_2y_3
% \end{itemize}
% So in each case,
% $\bm{y}^TM\bm{y}$ is a product of $2$ linear forms in $k[\bm{y}]$,
% which are proportional if and only if $l_2=0$ and $q_1=q_2=0$.
% }

%%%%%%%% ARCHIVE %%%%%%%%%%%%%%%%%
% Approaches to proving error-riskiness:
% \begin{enumerate}
%     \item Can we choose $L$ (containing a singularity if possible) to lie on a hypothetical plane and/or cubic scroll in $X$?
    
%     \item Alternatively, can we prove a lower bound on $E_X$ by blowing up along a hypothetical $P$ or $\Sigma$ (for $P$, cf.~Paper~II)?
%     Concretely this might amount to intersecting $X$ with varying hyperplanes containing $P$, or with varying quadrics containing $\Sigma$.
% \end{enumerate}

% Approaches to finding planes and/or cubic scrolls:
% \prop{
% [1.3.5, $l_2=0$]
% Analyze $-\delta_2$
% }

% \prop{
% [1.3.5, $l_2\neq 0$, $l_2\mid q_1$]
% Analyze $-\delta_2$
% }

% \prop{
% [1.3.5, $l_2\neq 0$, $l_2\nmid q_1$]

% }

% \prop{
% [1.3.6, $(fl_2-2q_1q_2)$ is a prime power away from $(l_2)$]
% }

% \prop{
% [1.3.6, $fl_2-2q_1q_2$ has a linear factor coprime to $l_2$]
% }

% \prop{
% [1.3.6, $fl_2-2q_1q_2$ has an irreducible quadratic factor]
% }
%%%%%%%% END ARCHIVE %%%%%%%%%%%%%%%%%

% %%%
% ARCHIVING PAST WORK TOWARDS BOUNDING THE NUMBER OF IRREDUCIBLE COMPONENTS:
% %%%
% \obs{
% Here $l_2\nmid q_1q_2$.
% (EDIT: This may or may not be true without additional hypotheses.
% However, it might not be that important either, since the only enemy in the amplification \emph{proof} strategy is if the $H^4$ contribution is too \emph{positive}.)
% }

% \pf{
% [Progress]
% Suppose for contradiction that $l_2\mid q_1$, say.
% Now choose $\bm{x}'\in\ol{k}^3\setminus\set{\bm{0}}$
% % with $f = (q_1/l_2)\cdot 2q_2$.
% with $l_2=q_2=0$; then automatically $q_1=0$ as well.
% But then 
% }

% \obs{
% Here, if $\card{X_{\map{sing}}(\ol{k})}=2$,
% then $fl_2-2q_1q_2$ is (nonzero and) absolutely irreducible.
% (EDIT: This may or may not be true without additional hypotheses.)

% NOTE 2022-01-17: What really matters is whether $fl_2-2q_1q_2$ has $\geq 2$ prime factors (without multiplicity) distinct from $(l_2)$; this can only occur if $L\mid fl_2-2q_1q_2$ for some $(L)\neq (l_2)$ (easy case, since we get that $C\bmod{L}$ is reducible, and therefore has a linear factor $L'\bmod{L}$ such that $C\bmod{L,L'}$ vanishes), or else (if that doesn't occur) if $fl_2-2q_1q_2=AB$ with $\deg{A}=\deg{B}=2$ and $A,B$ irreducible (hard case, but at least we can see that $l_2\nmid q_1q_2$ in this case, and $C$ contains---or almost contains?---a one-parameter family of $(2,2)$ complete intersections $A-\lambda(q_1+l_2x_4) = \lambda B+2(q_1+l_2x_5) = 0$?
% Note also that for $\lambda=0$ we get a rank $\leq 4$ quadric pencil, i.e.~a cone, and in fact an argument mod $l_2$ shows that we can arrange for either $A=q_1+l_2x_5=0$ or $A=q_2+l_2x_4=0$, or both, to contain a plane of the form $l_2=?=0$).
% }

\rmk{
[Original approach to one direction of Theorem~\ref{THM:near-dichotomy-for-cubic-curves-and-threefolds}]
\label{RMK:original-conic-bundle-approach-to-dichotomy}
Suppose $\Gamma_{\ol{k}}$ has $\geq 3$ irreducible components.
Then $fl_2-2q_1q_2$ has $\geq 2$ prime factors (counted without multiplicity) distinct from $(l_2)$.

\emph{Case~1: $A\mid fl_2-2q_1q_2$ for some linear form $A\in \ol{k}[x_1,x_2,x_3]$ with $(A)\neq (l_2)$.}
Then the reduction of $C = f+2q_1x_4+2q_2x_5+2l_2x_4x_5$ modulo $A$ is reducible, and therefore has a linear factor $B\bmod{A}$ such that $C\bmod{(A,B)}$ vanishes identically.
So $X_{\ol{k}} = V_{\PP^4}(C)_{\ol{k}}$ contains the $2$-plane $A=B=0$ in in $\PP^4_{\ol{k}}$.
Thus the condition~\ref{THM:near-dichotomy-for-cubic-curves-and-threefolds}(3) holds; so does condition~\ref{THM:near-dichotomy-for-cubic-curves-and-threefolds}(4) (cf.~\cite{wang2022dichotomous}*{Proposition~5.9(3)$\Rightarrow$(1)}).

\emph{Case~2: $fl_2-2q_1q_2=AB$ for some irreducible quadratic forms $A,B\in \ol{k}[x_1,x_2,x_3]$.}
% We could also assume $A,B$ are coprime, but this is not necessary.
Then $C\cdot l_2 = AB + 2(q_1+l_2x_5)(q_2+l_2x_4)$.
Let $U$ denote the open subscheme $\set{l_2\neq 0}\belongs \PP^4_{\ol{k}}$.
Then $X_{\ol{k}}$ contains the scheme-theoretic intersection $V_U(A, q_1+l_2x_5)$, where $A, q_1+l_2x_5$ are quadratic forms ``missing'' the variable $x_4$ (and therefore sharing a singularity in $\PP^4_{\ol{k}}$).
Furthermore, $l_2\nmid A$ (since $A$ is irreducible), so there exists $(x_1,x_2,x_3,x_5)\in \ol{k}^4\setminus \set{\bm{0}}$ with $A=0\neq l_2$ and $x_5 = -q_1/l_2$; thus $V_U(A, q_1+l_2x_5)$ is nonempty.
So the condition~\ref{THM:near-dichotomy-for-cubic-curves-and-threefolds}(4) holds.
}

% \ques{
% Does $x_1^3+\dots+x_6^3=0$ contain infinitely many $(2,2,1)$ complete intersections?
% }

% \pf{
% [Progress]
% Earlier, we showed that $fl_2-2q_1q_2$ is a \emph{nonzero} ternary quartic form in $k[\bm{x}']$.
% Now suppose for contradiction
% that $fl_2-2q_1q_2 = AB$ with $1\leq\deg{A}\leq\deg{B}$.
% Then from $C = f+2q_1x_4+2q_2x_5+2l_2x_4x_5$, we deduce
% \mathd{
% C\cdot l_2
% = (AB+2q_1q_2)
% +(2q_1x_4+2q_2x_5+2l_2x_4x_5)\cdot l_2
% = AB
% + 2(q_1+l_2x_5)(q_2+l_2x_4).
% }
% %https://www.wolframalpha.com/input/?i=factor+2q_1q_2+%2B%282q_1x_4%2B2q_2x_5%2B2l_2x_4x_5%29*l_2


% Now choose $\bm{x}'\in\ol{k}^3\setminus\set{\bm{0}}$ such that $A=B=0$.
% If $l_2(\bm{x}')\neq0$, then choosing $x_5 = -q_1/l_2$ and $x_4 = -q_2/l_2$ makes $[\bm{x}'+x_4e_4+x_5e_5]$ a singular zero of $C\cdot l_2$,
% which (since $l_2\neq0$) contradicts our assumption that $X_{\map{sing}}\belongs L$.

% Thus we may assume $V(\sqrt{(A,B)})\belongs V(l_2)$.
% Applying a $\GL_3(k)$-transformation if necessary,
% we may assume $l_2=x_3$.
% Now, fix $\bm{x}' = (x_1,x_2,0)$ with $A=B=0$.
% \begin{itemize}
%     \item If $q_1=q_2=0$ at $\bm{x}'$,
%     then $fx_3 = 2q_1q_2 + AB$ implies $\partial_{x_3}(fx_3)\vert_{\bm{x}'} = 0$,
%     which
%     % (since $x_3(\bm{x}')=0$ and $\partial_{x_3}(x_3)=1\neq0$)
%     implies $f(\bm{x}')=0$.
%     But then $C(\bm{x}'+x_4e_4+x_5e_5) = f(\bm{x}')+0+0+0 = 0$ for all $x_4,x_5\in\ol{k}$.
%     It follows that $X$ contains the plane through $[\bm{x}'],[e_4],[e_5]$,
%     contradicting our plane-free assumption.
    
%     \item Thus we may assume (by symmetry between $q_1,x_4$ and $q_2,x_5$) that $q_1(\bm{x}')\neq0$.
%     But $fx_3 = 2q_1q_2 + AB$ implies $(q_1q_2)(\bm{x}')=0$,
%     so $q_2(\bm{x}')=0$.
%     If $\partial_{x_3}q_2\vert_{\bm{x}'}=0$, then in fact $\partial_{x_3}(q_1q_2)\vert_{\bm{x}'}=0$,
%     which implies $\partial_{x_3}(fx_3)\vert_{\bm{x}'}=0$,
%     leading to the same contradiction as before.
    
%     \item Thus we may assume that $\partial_{x_3}q_2\vert_{\bm{x}'}\neq0$,
%     so that $q_2(\bm{y}) = r_2(y_1,y_2) + s_2(\bm{y})y_3$ with $r_2(x_1,x_2)=0$ but $s_2(x_1,x_2,0)\neq0$.
%     In fact, $-2q_1q_2\equiv AB\bmod{x_3}$ implies (since $q_1(\bm{x}')\neq0$) that $r_2(y_1,y_2)\propto (x_1y_2-x_2y_1)^2$.
%     (Meanwhile, $A(y_1,y_2,0)$ and $B(y_1,y_2,0)$ must each have only a \emph{simple} root at $(x_1,x_2)$.)
%     Hence $\partial_{x_1}q_2,\partial_{x_2}q_2$ vanish at $\bm{x}'$.
% \end{itemize}

% Note: $\partial_{x_4}C = 2q_1+2l_2x_5$, and $\partial_{x_5}C = 2q_2+2l_2x_4$, so a necessary condition for $C$ to be singular at $\bm{x}$ (so in particular, $C=0$) is that $q_1+l_2x_5=q_2+l_2x_4=0$ and $AB$ be singular at $\bm{x}$ (so in particular, $AB=0$).
% Also, if $l_2=x_3$, then $\partial_{x_3}C = \partial_{x_3}f + 2(\partial_{x_3}q_1)x_4 + 2(\partial_{x_3}q_2)x_5 + 2x_4x_5$,
% while $\partial_{x_i}C = \partial_{x_i}f + 2(\partial_{x_i}q_1)x_4 + 2(\partial_{x_i}q_2)x_5$ for $i\in[2]$.
% }

% % \ex{
% % % A = x_1x_2+x_3^2
% % % B = -2x_1x_2
% % % l_2 = x_3
% % % q_1 = x_1^2
% % % q_2 = x_2^2
% % % f = -2x_1x_2x_3
% % % Replace C with C/2.
% % If $C = -x_1x_2x_3 + x_1^2x_4 + x_2^2x_5 + x_3x_4x_5$,
% % then
% % \mathd{
% % \grad{C}
% % = (-x_2x_3 + 2x_1x_4, -x_1x_3 + 2x_2x_5, -x_1x_2 + x_4x_5, x_1^2 + x_3x_5, x_2^2 + x_3x_4).
% % }
% % Suppose $\grad{C}(\bm{x})=0$.
% % \begin{itemize}
% %     \item If $x_3=0$, then $x_1=0$ and $x_2=0$,
% %     so $x_4x_5=0$,
% %     i.e. $[\bm{x}]\in\set{[e_4],[e_5]}$.
    
% %     \item If $x_3\neq0$, then $x_5 = -x_1^2/x_3$ and $x_4 = -x_2^2/x_3$,
% %     so $x_1x_2 = x_4x_5 = x_1^2x_2^2/x_3^2$.
% %     If $x_1=0$, then $x_5=0$ and $x_2x_3=0$, so $x_2=0$, so $x_3x_4=0$, so $x_4=0$, making $[e_3]$ a new singular point!
% % \end{itemize}
% % Alternatively, note that $C\vert_{x_1=x_5=0}=0$.

% % Thus some things to try are:
% % either finding a singular point with $x_3\neq0$ (i.e. $l_2\neq0$),
% % or creatively finding a plane with e.g. $x_4=0$ or $x_5=0$.
% % }

% \ex{
% % A = x_1x_2+x_3^2
% % B = -2x_1x_2+x_3^2
% % l_2 = x_3
% % q_1 = x_1^2+x_2x_3
% % q_2 = x_2^2+x_1x_3
% % f = 2x_1^3 + x_1x_2x_3 + 2x_2^3 + x_3^3
% %https://www.wolframalpha.com/input/?i=2%28x_1%5E2%2Bx_2x_3%29%28x_2%5E2%2Bx_1x_3%29+%2B+%28x_1x_2%2Bx_3%5E2%29%28-2x_1x_2%2Bx_3%5E2%29
% %C = (2x_1^3 + x_1x_2x_3 + 2x_2^3 + x_3^3) + 2(x_1^2+x_2x_3)x_4 + 2(x_2^2+x_1x_3)x_5 + 2x_3x_4x_5
% %C = 2x_1^3 + 2x_1^2x_4 + x_1x_2x_3 + 2x_1x_3x_5 + 2x_2^3 + x_3^3 + 2x_2x_3x_4 + 2x_2^2x_5 + 2x_3x_4x_5
% % R.<x_1, x_2, x_3, x_4, x_5> = QQ[]
% % P.<x_1, x_2, x_3, x_4, x_5> = ProjectiveSpace(4, QQ)
% % I = R.ideal(2*x_1^3 + 2*x_1^2*x_4 + x_1*x_2*x_3 + 2*x_1*x_3*x_5 + 2*x_2^3 + x_3^3 + 2*x_2*x_3*x_4 + 2*x_2^2*x_5 + 2*x_3*x_4*x_5)
% % X = P.subscheme(I)
% % X.Jacobian().radical()
% % ---
% % Ideal (x_3, x_2, x_1, x_4*x_5) of Multivariate Polynomial Ring in x_1, x_2, x_3, x_4, x_5 over Rational Field
% It seems that for
% \mathd{
% C = (2x_1^3 + x_1x_2x_3 + 2x_2^3 + x_3^3) + 2(x_1^2+x_2x_3)x_4 + 2(x_2^2+x_1x_3)x_5 + 2x_3x_4x_5,
% }
% the only singular points are $[e_4],[e_5]$.
% (Here $fx_3-2q_1q_2 = (x_1x_2+x_3^2)(-2x_1x_2+x_3^2)$.)

% Is there a plane $P$ with $C\vert_P = 0$?
% (Casework: Is $x_5$ free or dependent on $x_1,x_2,x_3,x_4$?
% Then, is $x_4$ free or dependent on $x_1,x_2,x_3$?
% Etc., but in total we are only allowed $2$ constraints.
% Note that $x_4,x_5$ can't both be free, or else we would have $x_3=0$ and then $x_1^2=0$ and $x_2^2=0$.

% TRY TO SHOW THAT if $X$ contains a plane $P$, then it must contain a plane through $[e_4],[e_5]$, i.e. a plane with $x_4,x_5$ ``free''.
% Perhaps first show $P'\contains [e_4]$, then extend this to $[e_5]$.
% MAY BE HARD/FALSE to show say $[e_4]\in P'$, because unlike the statement for lines, using the fact that a singular conic is reducible,
% a singular quadric surface may simply be a cone rather than reducible.)

% NOTE 2022-01-17: StdConicBundleErrorCount(F,GF(q)) for [101,10007,10009] gives [10100, 100130042, 100170072] for $E_X(q)$.
% Also if $X$ contains a plane $P$, it must pass through either $[e_4]$ or $[e_5]$, but not both (it would require there to be linear forms $L_1,L_2$ in $x_1,x_2,x_3$ with $C\bmod{L_1,L_2}$ vanishing, which is impossible since $l_2,q_1,q_2$ have no common zeros).
% }

% \todo[inline]{Do statistical computations for the number of irreducible components of $\Gamma$\dots
% (and for $\#\set{[\bm{x}]\in L(k): l_1x_4^2+2l_2x_4x_5+l_3x_5^2=0}$, as a sanity check\dots).
% Work with appropriate selection of hyperplane sections of Fermat cubic $x_1^3+\dots+x_6^3=0$.}

\chapter{Isolating special integral solutions}
% Isolating special solutions in the delta method
\label{CHAP:isolating-special-solutions}

\section{Introduction}

As in \S\ref{SEC:using-L-function-hypotheses-on-average}, let $m\defeq 6$ and $F\defeq x_1^3+\dots+x_6^3$.
Fix $w\in C^\infty_c(\RR^m)$ with $(F,w)$ smooth in the sense of Definition~\ref{DEFN:support-smooth-clean}.
As we mentioned in \S\ref{SEC:using-L-function-hypotheses-on-average}, the analysis of the delta method for $N_{F,w}(X)$ (see \eqref{EQN:normalized-delta-method}--\eqref{EXPR:main-delta-method-quantity}) naturally breaks up into two parts: one over the locus $F^\vee(\bm{c})\neq 0$ (where ``reciprocal'' Hasse--Weil $L$-functions $1/L(s,V_{\bm{c}})$ naturally arise), and one over the locus $F^\vee(\bm{c})=0$ (where the $\QQ$-varieties $V_{\bm{c}}$ are singular, and thus need to be treated separately).

Since $m\geq 5$, it is known that the $\bm{c}=\bm{0}$ terms isolate the singular series in HLH for $(F,w)$ (see Definition~\ref{DEFN:HLH-asymptotic} and \S\ref{SEC:singular-series-contribution}).
In \cite{wang2021_isolating_special_solutions}, we interpret the sum
\mathd{
Y^{-2}\sum_{n\geq1}
\sum_{\bm{c}\in\ZZ^m}
n^{-m}S_{\bm{c}}(n)I_{\bm{c}}(n)
\cdot \bm{1}_{\bm{c}\neq\bm{0}}
\cdot \bm{1}_{V_{\bm{c}}\;\textnormal{is singular}}
}
in terms of special $\QQ$-subvarieties (specifically, linear spaces) on $V$,
which allows one to cleanly reformulate HLH for $(F,w)$ in a useful way (for Chapter~\ref{CHAP:using-mean-value-L-function-predictions}).
The main goal of the present chapter is to summarize the work done in \cite{wang2021_isolating_special_solutions}.
Throughout Chapter~\ref{CHAP:isolating-special-solutions}, we let $\smallpsum_{\bm{c}\in\ast}\cdots$ denote $\sum_{\bm{c}\in\ast\;\textnormal{with}\;F^\vee(\bm{c})=0}\cdots$,
unless specified otherwise.

\thm{
\label{THM:contribution-from-generically-singular-c's}
For some absolute constant $\delta>0$, we have (uniformly over $X\geq 1$)
\mathd{
Y^{-2}\psum_{\bm{c}\in\ZZ^m}
\bm{1}_{\bm{c}\neq\bm{0}}
\cdot \sum_{n\geq 1} n^{-m}S_{\bm{c}}(n)I_{\bm{c}}(n)
= O_{F,w}(X^{m/2-\delta})
+ \sum_{L\in C(\map{SSV})}
\sum_{\bm{x}\in L\cap\ZZ^m} w(\bm{x}/X).
}
}

\rmk{
The set $C(\map{SSV})$ is known to be finite
for general reasons (recalled in \S\ref{SEC:maximal-linear-under-duality} below).
For diagonal $F$ as in Theorem~\ref{THM:contribution-from-generically-singular-c's},
one can in fact compute the set $C(\map{SSV})$ explicitly:
see Observation~\ref{OBS:characterize-diagonal-L/Q} (essentially classical).
In any case,
on the Diophantine (right-hand) side of Theorem~\ref{THM:contribution-from-generically-singular-c's},
the \emph{sum} over $L$ is roughly equivalent to a \emph{union},
since $\#C(\map{SSV})<\infty$ and the pairwise intersections fit in the error term.
}

\rmk{
% As pointed out to us by Sarnak,
The \emph{$h$-invariant} introduced by \cites{davenport1962exponential,davenport1964non}
provides an equivalent way to think about linear subvarieties of cubic varieties.
% (The equivalence is not hard to prove, but
(See e.g.~\cite{dietmann2017h}*{Lemma~1.1} for a modern record of a more general equivalence.)
In fact,
the proof of the important Lemma~\ref{LEM:stating-bias-for-generic-trivial-c's} below (see \cite{wang2021_isolating_special_solutions}*{\S6}) essentially relies on
(a convenient choice of)
``$h$-decompositions of $F$''
corresponding to the $L$'s in Theorem~\ref{THM:contribution-from-generically-singular-c's}.
% Given L, there are infinitely many possible h-decompositions (i.e. h-decompositions do not uniquely correspond to L's). But at least for diagonal F, there's a sort of natural/convenient choice coming from the "van der corput / weyl differencing" structure.
}

Here is the main result here (in the present chapter) needed for Chapter~\ref{CHAP:using-mean-value-L-function-predictions}:
\cor{
\label{COR:Disc-vanishing-sum-with-explicit-linear-density}
For some absolute $\delta>0$, we have (uniformly over $X\geq 1$)
\mathds{
Y^{-2}\psum_{\bm{c}\in \ZZ^m}
\sum_{n\geq 1} n^{-m}S_{\bm{c}}(n)I_{\bm{c}}(n)
= \sigma_{\infty,F,w}\mathfrak{S}_FX^{m-3}
&+ O_{F,w}(X^{m/2-\delta}) \\
&+ \sum_{L\in C(\map{SSV})}\sigma_{\infty,L^\perp,w}X^{m/2}.
}
}

\pf{
Combine Theorem~\ref{THM:contribution-from-generically-singular-c's} with
the routine $\bm{c}=\bm{0}$ computation in \S\ref{SEC:singular-series-contribution} (isolating the singular series).
In the exponents,
note that $m/2=m-3=3$.
}

\rmk{
\label{RMK:II.1.15}
Theorem~\ref{THM:contribution-from-generically-singular-c's} and Corollary~\ref{COR:Disc-vanishing-sum-with-explicit-linear-density}
are \emph{completely unconditional} results.
These results let us \emph{reformulate}
HLH for $(F,w)$
as a statement purely about \emph{cancellation} in
the sum $\sum_{\bm{c}\in\ZZ^m}\bm{1}_{F^\vee(\bm{c})\neq0}
\cdot \sum_{n\geq1}n^{-m}S_{\bm{c}}(n)I_{\bm{c}}(n)$;
see Chapter~\ref{CHAP:using-mean-value-L-function-predictions}.
With any luck,
a similar reformulation might be possible much more generally.
% (perhaps even for higher-degree $F$'s).
But at least for cubic forms in $4$ variables,
subtleties in the Manin--Peyre constant
would likely demand
% a more sophisticated explanation
a more sophisticated ``delta method recipe''
% that incorporates more than just the locus $F^\vee(\bm{c})=0$.
beyond ``restriction to $F^\vee=0$''.
}

Let us sketch the proof of the theorem.
(In this sketch, we restrict $L$ to $C(\map{SSV})$.)

We \emph{start} generally,
% (without the diagonality assumption on $F$),
observing that $F^\vee\vert_{L^\perp}=0$ for all $L$
(see the first part of Proposition~\ref{PROP:dual-linear-subvariety}).
Conversely, at least for diagonal $F$,
most $\bm{c}$'s on the left-hand side of Theorem~\ref{THM:contribution-from-generically-singular-c's}
are in fact \emph{trivial}, in the sense of the following definition:
\defn{
\label{DEFN:orthogonally-trivial-c's}
Call a solution $\bm{c}\in\ZZ^m$ to $F^\vee(\bm{c})=0$ \emph{trivial}
if $\bm{c}\in L^\perp$ for some $L$.
}

Actually,
we cannot analyze all trivial $\bm{c}$'s uniformly,
but only the ``least degenerate'' ones.
% from the lens of the second part of Proposition~\ref{PROP:dual-linear-subvariety}.
Under (plausibly mild) hypotheses satisfied by diagonal $F$,
the second part of Proposition~\ref{PROP:dual-linear-subvariety}
establishes a vanishing baseline for the jets $j^\bullet{F^\vee}$ over $\bigcup_{L}L^\perp$---which inspires the following definition:
\defn{
\label{DEFN:unsurprising-Disc-zero-locus}
% For ``typical'' or ``generically unbiased'' $F$: $F^\vee=0$ should have few solutions?
% How to make this precise?
% For an (open set of) general $F$ containing all diagonal $F$: away from $m/2$-dimensional vector spaces, $F^\vee=0$ should have few solutions?
% How to make this precise?
Call $F^\vee$ \emph{unsurprising} if uniformly over $C\geq 1$,
the equation $F^\vee(\bm{c})=0$ has at most $O_\eps(C^{m/2-1+\eps})$ integral solutions $\bm{c}\in[-C,C]^m$ that are
either nontrivial, or else trivial with $j^{2^{m/2-1}}{F^\vee}(\bm{c})=\bm{0}$.
% with $\bm{c}\notin\bigcup_{L}L^\perp$ or $\bm{c}\in\map{Sing}(F^\vee)$.
% , where $L$ ranges over all $m/2$-dimensional vector spaces contained in $F=0$.
% (Here the \emph{jet} $j^r$ denotes a Taylor polynomial, as in Proposition~\ref{PROP:dual-linear-subvariety}.)
}

Indeed,
in \cite{wang2021_isolating_special_solutions}*{\S5},
we prove (using the diagonality of $F$) that $F^\vee$ is unsurprising---in which case
``almost all solutions to $F^\vee=0$ are trivial with nonzero $2^{m/2-1}$-jet''
(qualitatively speaking).
\rmk{
In particular (for the Gauss map $\gamma\maps V\to V^\vee$ introduced in \S\ref{SEC:alg-geom-background}),
$\gamma(\QQ)\maps V(\QQ)\to V^\vee(\QQ)$ can be
far from surjective,
even though $\gamma\maps V\to V^\vee$ itself is
birational and roughly log-height--preserving (up to a constant factor).
}

\rmk{
A weaker bound of the form $O(C^{m/2-\delta})$ in Definition~\ref{DEFN:unsurprising-Disc-zero-locus} would likely suffice for our main purposes.
}

For the ``least degenerate'' trivial $\bm{c}$'s,
Lemma~\ref{LEM:stating-bias-for-generic-trivial-c's} isolates an explicit positive bias
\mathd{
\wt{S}_{\bm{c}}(p^l)
= [A_{p^l}(\bm{c}) + O(p^{-l/2})]
\cdot (1-p^{-1})\cdot p^{l/2}
}
for most primes $p$,
with $A_{p^l}(\bm{c})\ll 1$ essentially a combinatorial factor measuring the $p$-adic ``extent of speciality'' of $\bm{c}$;
on average \emph{vertically}, $\EE[A_{p^l}(\bm{c})]\approx 1$.
% ``degeneracy'' (or ``extent of triviality'')
In the dominant ranges
(i.e.~for $n$ large),
the resulting reduction in arithmetic complexity of $S_{\bm{c}}(n)$ lets us
dramatically simplify each sum of the form $\sum_{\bm{c}\in\Lambda^\perp}S_{\bm{c}}(n)I_{\bm{c}}(n)$
by ``undoing'' Poisson summation to $I_{\bm{c}}(n)$ over various individual residue classes $\bm{c}\equiv\bm{b}\bmod{n_0\Lambda^\perp}$ with $n_0\ll X^{1/2}$ dividing $n$.

Ultimately,
this process produces corresponding sums over $\bm{x}\in\Lambda$
(as desired for Theorem~\ref{THM:contribution-from-generically-singular-c's}).
However,
when $m=6$,
we must be careful to separate $\bm{c}=\bm{0}$ from $\bm{c}\neq\bm{0}$;
Lemma~\ref{LEM:sum-S_0(n)-trivially} (decay of the singular series over large moduli)
provides the necessary input,
when contrasted with Lemma~\ref{LEM:n-aspect-I_c(n)-estimates} giving
decay of $I_{\bm{c}}(n)$ over small moduli for $\bm{c}\neq \bm{0}$.

For a full cross-referenced outline of
the proof of Theorem~\ref{THM:contribution-from-generically-singular-c's},
see \cite{wang2021_isolating_special_solutions}*{\S5}.

\rmk{
% % To do so, we first eliminate all but the
% The main contribution over $\bm{c}\neq \bm{0}$ comes from ``orthogonally trivial'' $\bm{c}$'s.
% % whose coordinates match in pairs.
% % (e.g.~$c_{2i-1} = c_{2i}$ for all $i$).
% % (i.e.~$c_{2i-1} = c_{2i}$ for all $i$, or a permutation thereof).
% For most such $\bm{c}$'s, we isolate an explicit positive bias in most of the exponential sums $S_{\bm{c}}(p^l)$.
% % To do so, we first eliminate all but the
% The main contribution over $\bm{c}\neq \bm{0}$ comes from ``dually trivial'' $\bm{c}$'s whose coordinates match in pairs.
% % (e.g.~$c_{2i-1} = c_{2i}$ for all $i$).
% % (i.e.~$c_{2i-1} = c_{2i}$ for all $i$, or a permutation thereof).
% For most such $\bm{c}$'s and primes $p$, we isolate an explicit positive bias in $S_{\bm{c}}(p^l)$.
We do not need \emph{horizontal} cancellation over $n$:
at least morally,
the terms $S_{\bm{c}}(n),I_{\bm{c}}(n)$ are positive for trivial $\bm{c}$'s,
while nontrivial $\bm{c}$'s are relatively sparse.
% However, for trivial $\bm{c}$ and $l\geq 2$, the constant $A_{p^l}(\bm{c})$ will usually depend on $(\frac{c_ic_j}{p})$.
% % certain quadratic symbols modulo $p$, 
% For convenience, we will assume GRH for quadratic Dirichlet $L$-functions to simplify horizontally over $n$ (by partial summation, say) before summing over $\bm{c}$.
This \emph{morally} explains why we can prove Theorem~\ref{THM:contribution-from-generically-singular-c's} unconditionally.
In fact,
the deepest result we use on $L$-functions (when $m=6$) is
the (purely local) Weil bound for hyperelliptic curves of genus $\leq 2$.
% It could be interesting to obtain further cancellation over our error terms, possibly using quadratic Dirichlet $L$-functions, as well as $L$-functions of hyperelliptic curves of genus $\leq 2$ for $m=6$, but we do not need such a deep analysis here.
}

\rmk{
% It may be worth mentioning that
The full proof of Theorem~\ref{THM:contribution-from-generically-singular-c's}
requires an \emph{error analysis}
to ``reduce consideration'' to biases.
Fortunately,
several convenient features make
our error analysis here
(reducing consideration to biases over $F^\vee(\bm{c})=0$)
``half an inch'' easier, or clearer,
than that in Chapter~\ref{CHAP:using-mean-value-L-function-predictions}
(reducing to $L$-functions over $F^\vee(\bm{c})\neq 0$).
Specifically,
small moduli $n$ and bad primes $p$ here cause
% relatively little pain.
very little pain compared to those in Chapter~\ref{CHAP:using-mean-value-L-function-predictions}.
}

See Remark~\ref{RMK:non-diagonal-F} for a discussion of what is missing for non-diagonal $F$.

\section{Algebraic geometry background}
\label{SEC:alg-geom-background}
% \subsection{Algebraic geometry in the delta method}

Let $F,V,\dots$ be as in \S\ref{SEC:delta-method-setup}.
For general $F$, we need some classical results on
the gradient map $\grad{F}$, its image, and its ramification.
For diagonal $F$,
a more explicit analysis is possible
(see \S\ref{SUBSEC:diagonal-dual-example} below).

\subsection{The dual hypersurface and the discriminant form}

Since $V$ is smooth,
the \emph{polar map} $[\grad{F}]\maps\PP^{m-1}\to\PP^{m-1}$ is
\emph{regular}, i.e.~defined everywhere.
(It would be better to write $\PP^{m-1}\to(\PP^{m-1})^\vee$,
but no confusion should arise.)
The map $[\grad{F}]$ is finite of degree $2^{m-1}$ \cite{dolgachev2012classical}*{p.~29}
between irreducible equidimensional projective varieties,
hence surjective.
%https://mathoverflow.net/questions/41390/morphism-between-projective-varieties
%https://math.stackexchange.com/questions/2643347/finite-morphism-fx-to-mathbbp-kn-is-surjective
Since $\PP^{m-1}$ is smooth,
$[\grad{F}]$ must then be flat
(by ``miracle flatness'').

Upon restricting $[\grad{F}]$ to $V$,
% the ambient polar map $[\grad{F}]$ to $V$,
we get the finite surjective \emph{Gauss map} $\gamma\maps V\to V^\vee$,
where $V^\vee\belongs (\PP^{m-1}_\QQ)^\vee$ denotes the \emph{dual variety} of $V$,
i.e.~the closure of the union of
% tangent hyperplanes at \emph{smooth} points of $V$.
hyperplanes tangent to the \emph{smooth locus} $V_{\map{sm}}$ of $V$.
(For us,
$V$ is a hypersurface,
and $V_{\map{sm}}=V$.
So $V^\vee=\ol{[\grad{F}]V_{\map{sm}}}=\ol{[\grad{F}]V}=[\grad{F}]V$.
Hence $\gamma$ is indeed well-defined and surjective;
%quasi-finite map between projective varieties is finite (in general quasi-finite + proper + mild hypotheses should imply finite); see https://math.stackexchange.com/questions/289772/references-for-quasi-finite-and-proper-implies-finite
and $\gamma$ is finite because it is a quasi-finite map between projective varieties.)

Here $V/\QQ$ is irreducible over $\CC$,
so $V^\vee = \gamma(V)$ must be too.
% by \cite{gelfand2008discriminants}*{p.~15, Proposition~1.3} (general fact that doesn't seem necessary here).
Since $\gamma$ is finite, $V^\vee$ must therefore be a geometrically integral hypersurface,
%google "reference for dual hypersurfaces"
% https://mathoverflow.net/questions/158187/dual-of-a-smooth-hypersurface
%https://math.stackexchange.com/questions/1515239/dual-of-a-smooth-hypersurface-is-still-a-hypersurface
namely the zero locus of $F^\vee$ from Proposition-Definition~\ref{PROPDEFN:unique-geometric-discriminant-form,with-proof-for-diagonal-F}.
(At least for diagonal $F$,
one can explicitly compute $F^\vee$;
see \eqref{EXPR:diagonal-discriminant-factorization} in \S\ref{SUBSEC:diagonal-dual-example} below.)
The notation $F^\vee$ is thus convenient for us, but it is likely \emph{not} standard.

% We now recall the symmetry
% (or rather duality)
% between $V,V^\vee$.
% State reflexivity
% First,
It is known that $(V^\vee)^\vee=V$ \cite{dolgachev2012classical}*{p.~30, Reflexivity Theorem}.
% Explain how definitions and reflexivity imply divisibilities
Furthermore,
the definition of $V^\vee$
(together with the fact that $V,V^\vee$ are hypersurfaces with $V$ irreducible)
implies the divisibility $F(\bm{x})\mid F^\vee(\grad{F}(\bm{x}))$,
and reflexivity
(together with the fact that $V^\vee,V$ are hypersurfaces with $V^\vee$ irreducible)
similarly implies the divisibility $F^\vee(\bm{c})\mid F(\grad{F^\vee}(\bm{c}))$.
These (symmetric!) divisibilities capture much of the basic duality theory for $V$.

% The apparent symmetry is deceptively simple, however.
The apparent symmetry between $V,V^\vee$ thus far is deceptive, however.
What complicates matters is that $V^\vee$, or equivalently $F^\vee$,
must be \emph{singular} if $\deg{F}\geq 3$.
(Otherwise,
$(V^\vee)^\vee$ would be a hypersurface of degree larger than $\deg{F}$,
% contradicting $(V^\vee)^\vee = V$.}
contradicting reflexivity.)
Hence the polar map $[\grad{F^\vee}]\maps\PP^{m-1}\dashrightarrow\PP^{m-1}$ is only a \emph{rational} map,
defined away from $\map{Sing}(V^\vee)$, the \emph{set of singular points} of $V^\vee$.
%https://mathoverflow.net/questions/147159/singularities-of-the-dual-variety-of-a-surface
(Here $\map{Sing}(V^\vee)$ is a proper closed subset of $V^\vee$.)

Nonetheless,
given \emph{smooth} points $[\bm{x}]\in V$ and $[\bm{c}]\in V^\vee$,
the biduality theorem says that
$[\grad{F}(\bm{x})] = [\bm{c}]\iff[\grad{F^\vee}(\bm{c})] = [\bm{x}]$.
(See e.g.~\cite{dolgachev2012classical}*{p.~30, sentence after Reflexivity Theorem},
which however does not explicitly refer to the biduality theorem by name.)
%the singular locus is a proper subvariety by a noether normalization argument, presumably?... actually it is easier, as in Milne's notes, just note that F_x = 0 on F implies (if F irreducible) that F_x is divisible by F, so F_x = 0 identically, so F is constant in x, etc.
%https://math.stackexchange.com/questions/1874535/is-the-singular-locus-of-a-variety-as-a-variety-itself-a-smooth-variety
%https://mathoverflow.net/questions/316595/degree-of-the-variety-of-singular-points
For us,
$V_{\map{sm}}=V$,
so biduality implies that $[\grad{F}],[\grad{F^\vee}]$ restrict to
\emph{inverse morphisms} between $V\setminus[\grad{F}]^{-1}(\map{Sing}(V^\vee))$ and $V^\vee\setminus\map{Sing}(V^\vee)$.

\rmk{
The map $\gamma\maps V\to V^\vee$ is finite surjective
% of (generic) degree $1$ (by biduality),
(and birational, i.e.~of degree $1$),
but not necessarily \emph{flat}
% (Indeed, $V^\vee$ need not be smooth.)
% https://stacks.math.columbia.edu/tag/02K9 (29.48 Finite locally free morphisms)
(or equivalently, \emph{locally free}).
In fact,
% by biduality,
the finite $\mcal{O}_{V^\vee}$-algebra $\gamma_\ast\mcal{O}_V$
% https://en.wikipedia.org/wiki/Sheaf_of_algebras
is isomorphic to $\mcal{O}_{V^\vee}$ \emph{generically} over $V^\vee$,
but \emph{not necessarily everywhere}---for instance,
% https://en.wikipedia.org/wiki/Spectrum_of_a_ring
% https://math.stackexchange.com/questions/607934/underlying-set-of-the-scheme-theoretic-fiber
% https://mathoverflow.net/questions/121932/geometric-fibers-of-schemes
the geometric fiber $V\times_{\gamma}\ol{k(p)} = \Spec((\gamma_\ast\mcal{O}_V)_p\otimes_{\mcal{O}_{V^\vee,p}}\ol{k(p)})$,
and in fact the analogous set-theoretic geometric fiber,
may have size $\geq2$ at some point $p\in\map{Sing}(V^\vee)$.
}

\ques{
What is known about the fiber of $\gamma$ over a \emph{singular} point $[\bm{c}]\in V^\vee$?
}
Presumably the singular structure of $V_{\bm{c}}$ should play a role.
Perhaps works of Aluffi and Cukierman (such as \cites{aluffi1993multiplicities,aluffi1995singular})
can help to give a precise statement.

\subsection{Ramification and the Hessian}

$[\grad{F}]$ is a finite surjective morphism of smooth $\QQ$-varieties,
so its ramification theory is well-behaved.
%e.g.~it is flat, and also generically etale by field theory
% Also, by https://stacks.math.columbia.edu/tag/0475 the unramified locus commutes with base change, and by https://stacks.math.columbia.edu/tag/0BW7 the different ideal (equivalently by https://www.math.columbia.edu/~dejong/wordpress/?p=4048 the ramification divisor) in fact commutes.
By \cite{dolgachev2012classical}*{p.~29, Proposition~1.2.1},
% (and faithfully flat descent along $\Spec\CC\to \Spec\QQ$?),
\begin{enumerate}[(1)]
    \item the \emph{ramification divisor} of $[\grad{F}]\maps\PP^{m-1}\to\PP^{m-1}$
    is the Hessian hypersurface $\map{hess}(V)\belongs\PP^{m-1}$;
    and
    
    \item its image, the \emph{branch divisor},
    is $\map{St}(V)^\vee$, the dual of the Steinerian hypersurface $\map{St}(V)$
    (with $\map{St}(V)$ defined scheme-theoretically as in \cite{dolgachev2012classical}*{p.~19, \S1.1.6}).
\end{enumerate}
In particular,
both $\map{hess}(V),\map{St}(V)^\vee$ are
(possibly reducible or non-reduced)
projective hypersurfaces in $\PP^{m-1}$.
% $\map{hess}(V)$ is a hypersurface in $\PP^{m-1}$, so its image $[\grad{F}](\map{hess}(V)) = \map{St}(V)^\vee$ (under the finite morphism $[\grad{F}]$) is also a hypersurface.
%https://math.stackexchange.com/questions/1758315/do-finite-morphisms-preserve-dimensions
(It is worth noting that the precise geometry of $\map{hess}(V),\map{St}(V)^\vee$ is not so important in \cite{wang2021_isolating_special_solutions}, but it may become more important in the future if one wants to study general $V$.
% The scheme structure, working over $\QQ$ (as opposed to $\ol{\QQ}$ or $\CC$), it also not so important; but to give a full version of Dolgachev's proofs of (1)--(2) above, one should probably work with the universal family of polar maps (a proper, quasi-finite, and surjective --- and thus finite flat --- morphism between smooth quasi-projective varieties, I believe) and reduce to studying general cubic hypersurfaces (where hessian is surely reduced, etc.) via https://stacks.math.columbia.edu/tag/0BW7 (and the invertibility of the universal relative dualizing module given by https://www.math.columbia.edu/~dejong/wordpress/?p=4048 should help).
Most important for us now is that the branch divisor exists, and that it can be computed explicitly when $F$ is diagonal.)

When studying the Hasse principle (or even weak approximation),
one can certainly localize before counting points.
For instance, the following proposition shows that
on an \emph{arbitrary} smooth cubic $V$,
as in \cite{hooley1988nonary} or \cite{hooley2014octonary} for instance,
it is always \emph{non-vacuous} to count points on $V$ with nonvanishing Hessian determinant.
\prop{[\cite{hooley1988nonary}*{Lemma~1 and its proof}]
\label{PROP:localize-away-from-hessian}
If $V$ is a smooth cubic, then $V\not\belongs \map{hess}(V)$.
Furthermore, $V(\RR)$ is Zariski dense in $V$,\footnote{One can prove density using the general Proposition~\ref{PROP:sm-proj-cubic-hypersurface-K-point-implies-K-unirational-implies-K-dense}, or the real \cite{LawtonMO331205real_Zariski_density}.} so $V(\RR)\not\belongs (\map{hess}{V})(\RR)$.
}

\rmk{
[Cf.~\cite{hooley1988nonary}*{remarks in the paragraph before Lemma~1}]
% As Hooley remarks,
% Hooley remarks the following:
The intersection $V\cap\map{hess}(V)$ consists of \emph{inflection points} if $m=3$,
and of \emph{parabolic points} if $m\geq4$
\cite{dolgachev2012classical}*{p.~17, Theorem~1.1.20}.
But it does not seem easy to find a general reference proving
the existence of non-inflection or non-parabolic points on $V$
(according as $m=3$ or $m\geq4$).
}

The following stronger technical question comes up in Proposition~\ref{PROP:dual-linear-subvariety},
although we happen to be able to sidestep it there.

\ques{
\label{QUES:dual-variety-vs-branch-locus}
% In the proof of Proposition~\ref{PROP:dual-linear-subvariety} we took for granted that certain sets are nonempty:
Is it always true (for smooth cubic $V$)
that $V^\vee\not\belongs\map{St}(V)^\vee$?
% $V^\vee\not\belongs\map{Sing}(V^\vee)\cup\map{St}(V)^\vee$
}

\rmk{
If $V\belongs\map{hess}(V)$,
then applying $[\grad{F}]$ would imply $V^\vee\belongs\map{St}(V)^\vee$.
Thus an affirmative answer to the question
would give another proof of Proposition~\ref{PROP:localize-away-from-hessian}.
}


\section{Maximal linear subvarieties under duality}
\label{SEC:maximal-linear-under-duality}
%Linear subvarieties and their duals
% \subsection{Linear subvarieties of maximal possible dimension}

Let $F,V,\dots$ be as in \S\ref{SEC:delta-method-setup}.
If $2\mid m$ but $m\neq 6$,
extend the definition of $C(\map{SSV})$ from Definition~\ref{DEFN:HLH-asymptotic} in the obvious way (involving $m/2$-dimensional vector spaces $L/\QQ$).
The reader only interested in diagonal $F$ can skim forwards to \S\ref{SUBSEC:diagonal-dual-example}, which explicitly analyzes $C(\map{SSV})$ through the lens of $F^\vee$ (in the diagonal case).

\subsection{A preliminary general analysis}

In general,
classical duality theory comes into play,
leading to Proposition~\ref{PROP:dual-linear-subvariety} below.
For the definitions and basic properties of the \emph{polar} and \emph{Gauss} maps $[\grad{F}],\gamma$ associated to $V$,
see the first few paragraphs of \S\ref{SEC:alg-geom-background}.

Consider the affine cone $C(V)$ (i.e.~$F(\bm{x})=0$ sitting in $\Aff^m_\QQ$).
Suppose $m\geq3$ and
$L\belongs C(V)$ is a vector space over $\QQ$
(of arbitrary dimension).
Then differentiating along $L$ implies $L\perp\grad{F}(\bm{x})$
% , i.e.~$\grad{F}(\bm{x})\in L^\perp$,
for all $\bm{x}\in L$.
In other words,
the restriction $\gamma\vert_{\PP L}$ maps into $\PP L^\perp$.
% the restriction $\grad{F}\vert_{L}$ maps into $L^\perp$.
But $[\grad{F}]\maps\PP^{m-1}\to\PP^{m-1}$ is regular and finite,
so $\gamma\vert_{\PP L}$
% the restricted projective map $\gamma\vert_{\PP L}\maps\PP L\to\PP L^\perp$
is itself regular and finite (since $\PP L$ is closed in $\PP^{m-1}$).
Thus $\dim(L)\leq\dim(L^\perp)$,
i.e.~$\dim(L)\leq \floor{m/2}$.
% \footnote{This argument recovers the fact that $\dim(L)\leq\dim(L^\perp)$ \emph{in general} (for $\PP^{m-1}_\QQ$-smooth $F$).}

\rmk{
Although $\PP L^\perp\belongs(\PP^{m-1})^\vee$ is the dual variety of $\PP L\belongs\PP^{m-1}$,
we prefer to write $L^\perp$ instead of $L^\vee$, to avoid confusion with the dual vector space $\Hom(L,\QQ)$.
}

Since $\deg{F}\geq 3$,
% Starr proves this dimension bound and more:
% shows that the affine cone $C(V)$ (i.e.~$F(\bm{x})=0$ sitting in $\Aff^m$) can only contain vector spaces of dimension $\leq \floor{m/2}$---and furthermore,
\cite{debarre2003lines}*{Lemma~3}
(or Starr \cite{starr2005fact_in_browning2006density}*{Appendix}, of which I learned from \cite{DaoMO15111linear_spaces})
proves more:
if $m$ is \emph{even},
then there are at most \emph{finitely many} $L$'s
of dimension $m/2$.
% of dimension $m/2$ (i.e.~the ``largest possible'' dimension).
We would like to understand these ``maximal'' linear spaces $L$ in terms of the delta method.
Proposition~\ref{PROP:dual-linear-subvariety} suggests one plausible \emph{starting} route:
duality (i.e.~detecting $L^\perp$ through $F^\vee$).

\prop{
\label{PROP:dual-linear-subvariety}
Suppose $2\mid m\geq 4$,
and fix $L\in C(\map{SSV})$.
% \emph{Suppose} $\PP L\belongs V$.
Then $\gamma\vert_{\PP L}$ is a finite flat surjective morphism $\PP L\to \PP L^\perp$ of degree $2^{m/2-1}\geq 2$,
% so $L^\perp\belongs C(V^\vee)$
% Conversely, if $L^\perp\belongs C(V^\vee)$ \emph{and} $L^\perp\not\belongs\map{Sing}(F^\vee)$, then $L\belongs C(V)$.
and $\PP L^\perp\belongs\map{Sing}(V^\vee)$.
Furthermore,
\emph{if} $\PP L^\perp\not\belongs[\grad{F}](\map{hess}(V))$ set-theoretically,
% \footnote{with containment defined in the sense of closed subvarieties of $\PP^{m-1}$}
then the affine jet $j^{2^{m/2-1}-1}{F^\vee}$ \emph{vanishes} over $L^\perp$.
% , i.e.~the \emph{evaluation} $(j^{2^{m/2-1}-1}{F^\vee})(\bm{c})$ vanishes for all $\bm{c}\in L^\perp$.
% , or equivalently $(j^{2^{m/2-1}-1}{F^\vee})\vert_{L^\perp}\defeq i^\ast(j^{2^{m/2-1}-1}{F^\vee})=\bm{0}$ where $i\maps L^\perp\inject\Aff^m$ denotes (reduced closed subscheme) inclusion.
%https://math.stackexchange.com/questions/2218018/global-sections-of-a-sheaf-restricted-to-a-closed-subscheme
% array of order $k$ derivatives $D^k{F^\vee}\defeq (\partial^{\bm{l}}F^\vee)_{\abs{\bm{l}}=k}$ vanishes on $L^\perp$ for each nonnegative integer $k\leq 2^{m/2-1}-1$.
}

\rmk{
For diagonal $F$,
we provide an explicit proof of Proposition~\ref{PROP:dual-linear-subvariety} in \S\ref{SUBSEC:diagonal-dual-example}.
% In general we must work harder:
For general $\PP^{m-1}_\QQ$-smooth $F$,
we instead rely on some classical algebraic geometry---duality theory and the ramification behavior of $[\grad{F}]$---detailed in \S\ref{SEC:alg-geom-background}.
}

We now begin the proof of Proposition~\ref{PROP:dual-linear-subvariety}.
We know (from earlier) that $\gamma\vert_{\PP L}$ is a finite map from $\PP L$ to $\PP L^\perp$.
Yet $\dim(L) = \dim(L^\perp)$.
So $\gamma\vert_{\PP L}$ is \emph{surjective},
since $\PP L^\perp$ is irreducible.
But $\gamma$ maps $V$ into $V^\vee$ by definition,
% so $\PP L^\perp\belongs V^\vee$ follows.
% so $\PP L^\perp\belongs\gamma(\PP L)\belongs\gamma(V)\belongs V^\vee$.
so $\PP L^\perp\belongs\im\gamma\vert_{\PP L}\belongs\im\gamma\belongs V^\vee$.

\pf{[Proof of first part]
% Conversely, suppose $L^\perp\belongs C(V^\vee)$.
% Again, the restricted gradient $\grad{F^\vee}\vert_{L^\perp}$ maps into the dual space $(L^\perp)^\perp = L$.
% If the open subvariety $L^\perp\setminus\map{Sing}(F^\vee)$ of $L^\perp$ is \emph{nonempty}, then it has dimension $\dim(L^\perp)$, and dimension theory (say for integral schemes of finite type over a field) shows that the \emph{closure} of $\im(\grad{F^\vee}\vert_{L^\perp})$ in $\Aff^m$ must coincide with $L$ (since $\dim(L)=\dim(L^\perp)$ by assumption).
% But $\im(\grad{F^\vee}\vert_{L^\perp})\belongs C(V^\vee)$ by definition (consider $\map{Sing}(F^\vee)$ and $L^\perp\setminus\map{Sing}(F^\vee)$ separately), and $C(V^\vee)$ is closed, so indeed $L\belongs C(V^\vee)$.

Since $\gamma\vert_{\PP L}$ is finite surjective and $\PP L,\PP L^\perp$ are smooth,
``miracle flatness'' implies flatness of $\gamma\vert_{\PP L}$.
Also,
$\gamma\vert_{\PP L}$ has degree $2^{m/2-1}$ (cf.~\cite{dolgachev2012classical}*{top of p.~29}),
since it is a morphism given by quadratic polynomials, between projective spaces of dimension $m/2-1$.
In particular,
$\gamma\vert_{\PP L}\maps\PP L\to\PP L^\perp$ is \emph{nowhere birational},
% is \emph{not} birational at any point of $\PP L^\perp$
so the biduality theorem implies $\PP L^\perp\belongs\map{Sing}(V^\vee)$.
}

The second part of Proposition~\ref{PROP:dual-linear-subvariety} is inspired by
the factorization of $F^\vee$ over $\ol{\QQ}[\bm{c}^{1/2}]$ when $F$ is diagonal (see \eqref{EXPR:diagonal-discriminant-factorization} in \S\ref{SUBSEC:diagonal-dual-example} below).
However,
giving a rigorous ``factorization'' of $F^\vee$ seems to require a bit of setup,
since the map $[\grad{F}]$ presumably need not be Galois in general.
%find an explicit example of non-Galois [\grad{F}]?

First,
\emph{assume} $\PP L^\perp \not\belongs [\grad{F}](\map{hess}(V))$.
Now consider the hypersurface complements $S\defeq \PP^{m-1}\setminus [\grad{F}](\map{hess}(V))$ and $X\defeq [\grad{F}]^{-1}S\belongs \PP^{m-1}$.
% \footnote{Both $S,X$ are nonempty open subsets in $\PP^{m-1}$, hence irreducible (and connected).}
Then $S\cap\PP L^\perp$ is a \emph{nonempty} open subset of $\PP L^\perp$,
yet $[\grad{F}]\vert_X\maps X\to S$ is finite \emph{\etale} of degree $2^{m-1}$.
Write $\phi\defeq [\grad{F}]\vert_X$.
By Grothendieck's Galois theory,
there exists a finite \etale Galois cover $\pi\maps X'\to X$ with $X'$ connected and $\phi\circ\pi\maps X'\to S$ (finite \etale) Galois.
Let $G\defeq\Aut_S(X')$ and $H\defeq\Aut_X(X')$.
%https://stacks.math.columbia.edu/tag/03SF or https://people.math.ethz.ch/~pink/Theses/2018-Bachelor-Noah-Held.pdf (Szamuely and Lenstra are good standard references)

\subsubsection{Constructing a product ``divisible'' by \texpdf{$F^\vee$}{Fvee}}
% Note that
Over every geometric point $[\bm{c}]\in (S\cap V^\vee)(\ol{\QQ})$,
there exists a (geometric) point $[\bm{x}]\in (X\cap V)_{[\bm{c}]}$,
i.e.~$[\bm{x}]\in X_{[\bm{c}]}$ with $F(\bm{x})=0$.
% \footnote{Furthermore, if $[\bm{c}]\notin\map{Sing}(V^\vee)$, then $[\bm{x}]$ is \emph{unique} by biduality.}
Although $G$ may not act on $X$ itself,
it acts transitively on $X'$---so
% So for any geometric point $p\in X'$, the fiber $X_{\phi\pi(p)}$ over $\phi\pi(p)\in S$ consists of $\pi(gp)$ for $g\in H\backslash G$.
after fixing a (geometric) point $p\in X'_{[\bm{x}]}$,
we can characterize the fiber $X_{[\bm{c}]}$ as the set $\set{\pi(gp):g\in H\backslash G}\belongs X(\ol{\QQ})$.

Now view $F(\bm{x})$ as a section of $\mcal{O}_X(3)$,
and pull it back to $\pi^\ast{F}\in\Gamma(X',\pi^\ast\mcal{O}_X(3))$.
Then the product $\alpha\defeq\prod_{g\in H\backslash G}g^\ast(\pi^\ast{F})$ defines a \emph{$G$-invariant section}\footnote{A \emph{$G$-invariant section} $\alpha\in\Gamma(X',\mcal{L})$ is equivalent in data to a \emph{$G$-equivariant morphism} $\alpha\maps\mcal{O}_{X'}\to\mcal{L}$.}
of the \emph{$G$-equivariant line bundle} $\mcal{L}\defeq\bigotimes_{g\in H\backslash G}g^\ast(\pi^\ast\mcal{O}_X(3))$ on $X'$,
with
\mathd{
\alpha\vert_{(\phi\pi)^{-1}(S\cap V^\vee)} = 0
\quad\textnormal{(since $X\cap V\belongs\phi^{-1}(S\cap V^\vee)$, and $F\vert_{X\cap V} = 0$)}.
}
By faithfully flat (Galois) descent,\footnote{in the form of an equivalence of categories}
there exist line bundles $\mcal{F},\mcal{D}$ on $X,S$ with $\mcal{L}\cong\pi^\ast\mcal{F}$ and $\mcal{F}\cong\phi^\ast\mcal{D}$,\footnote{It suffices to find $\mcal{D}$ with $\mcal{L}\cong(\phi\pi)^\ast\mcal{D}$,
and then define $\mcal{F}\defeq\phi^\ast\mcal{D}$.}
and sections $\beta,\delta$ on $X,S$ vanishing along $\phi^{-1}(S\cap V^\vee),S\cap V^\vee$, respectively,
with $\alpha=\pi^\ast\beta$ and $\beta=\phi^\ast\delta$.

But $S,X$ are hypersurface complements in $\PP^{m-1}$,
so $\Pic(\PP^{m-1})\to\Pic(S)$ and $\Pic(\PP^{m-1})\to\Pic(X)$ are surjective
and we may identify $\mcal{F},\mcal{D}$ with suitable powers of $\mcal{O}_X(1),\mcal{O}_S(1)$, respectively.
Then up to a choice of nonzero constant factors,
we may view $\beta,\delta$ as homogeneous rational functions
(i.e.~ratios of homogeneous $m$-variable polynomials)
with $F^\vee(\bm{c})\mid\delta$ and $F^\vee(\grad{F}(\bm{x}))=\phi^\ast{F^\vee}\mid\beta$.
% , where we interpret divisibility of sections to mean their ratio is a \emph{global section} of the obvious line bundles $\mcal{D}\otimes\mcal{O}_S(-\deg{F^\vee})$ and $\mcal{F}\otimes\phi^\ast\mcal{O}_X(-\deg{F^\vee})$, respectively.
Here we interpret divisibility of two sections on a scheme to mean their ratio is a \emph{global section} of the obvious ``tensor-quotient'' line bundle.

\subsubsection{``Factoring'' \texpdf{$F^\vee$}{Fvee}}

By duality theory,
$F(\bm{x})\mid F^\vee(\grad{F}(\bm{x}))$.
However,
$F(\bm{x})^2\nmid\beta$ on $X$,
since $(\pi^\ast{F})^2\nmid\alpha$ on $X'$.
Indeed,
given a geometric point $[\bm{c}]$ of $(S\cap V^\vee)\setminus\map{Sing}(V^\vee)$,
biduality furnishes a \emph{unique} point $[\bm{x}]\in X_{[\bm{c}]}$ with $F(\bm{x})=0$.
So if $p\in X'_{[\bm{x}]}$ and $g\in G$,
then the section $g^\ast\pi^\ast{F}$ evaluates to $0$ at $p$ \emph{if and only if} $g\in H$.
Thus $\pi^\ast{F}\nmid\prod_{[1]\neq g\in H\backslash G}(g^\ast\pi^\ast{F})$ on $X'$.
It follows that
$(\pi^\ast{F})^2\nmid\alpha$,
whence $F(\bm{x})^2\nmid\beta$;
whence $F^\vee(\bm{c})^2\nmid\delta$.

However,
$F\mid\phi^\ast{F^\vee}$,
and $\phi\pi g = \phi\pi$ for all $g\in G$,
so by inspection,
$\alpha\mid(\pi^\ast\phi^\ast{F^\vee})^{\deg\phi}$,
i.e.~$\delta\mid (F^\vee)^{\deg\phi}$.
% \footnote{Here $\card{H\backslash G} = \deg\phi$.}
By absolute irreducibility of $F^\vee$,
we conclude that
% after scaling, we may assume $\delta = F^\vee(\bm{c})$
% and $\beta = F^\vee(\grad{F}(\bm{x}))$.
in fact $\delta\mid F^\vee$ and $\alpha\mid(\phi\pi)^\ast{F^\vee}$.
(Divisibility also holds in the other direction,
but we will not need this.)

\rmk{
Our proof of $\delta\mid F^\vee$ requires
% $S,X$ are nonempty sets/schemes
$(S\cap V^\vee)\setminus\map{Sing}(V^\vee)\neq\emptyset$,
% For $S$, recall that $[\grad{F}](\map{hess}(V))$ is a hypersurface in $\PP^{m-1}$.
% For $X = [\grad{F}]^{-1}S$, then use surjectivity of $[\grad{F}]$.
% For $(S\cap V^\vee)\setminus\map{Sing}(V^\vee)$,
i.e.~that $S\cap V^\vee$ and $V^\vee\setminus\map{Sing}(V^\vee)$ are nonempty (open) subsets of $V^\vee$;
since $\PP L^\perp\belongs V^\vee$,
the former nonemptiness follows conveniently from our assumption $S\cap\PP L^\perp\neq\emptyset$
(but see Question~\ref{QUES:dual-variety-vs-branch-locus}),
while the latter follows (unconditionally) from ``generic smoothness'' in characteristic $0$ (since $V^\vee$ is reduced).
}

\subsubsection{Differentiating the product}

Using $S\cap\PP L^\perp\neq\emptyset$ one last time
(more seriously than before),
we will now complete the proof of the second part of Proposition~\ref{PROP:dual-linear-subvariety}.
% as follows / as we now explain
\pf{
[Completion of proof]
% Finally, recall that
By assumption,
$S\cap\PP L^\perp\neq\emptyset$.
Yet $\phi\vert_{X\cap\PP L} = \gamma\vert_{X\cap\PP L}\maps X\cap\PP L\to S\cap\PP L^\perp$ is finite \etale of degree $2^{m/2-1}$.
Say $[\bm{c}]\in(S\cap\PP L^\perp)(\ol{\QQ})$ is a geometric point,
and fix $p\in X'_{[\bm{c}]}$.
Then there exist at least $2^{m/2-1}$ cosets $g\in H\backslash G$ with $\pi gp\in (X\cap\PP L)(\ol{\QQ})\belongs V(\ol{\QQ})$.
In a $G$-invariant affine open neighborhood of (the image of) $p$ in $X'$,\footnote{Such a neighborhood exists by \cite{mustatua2011zeta}*{paragraph after Corollary~A.3}, since $X'$ is certainly quasi-projective.}
%googling g-invariant affine chart galois leads to http://www.math.lsa.umich.edu/~mmustata/appendix.pdf from http://www-personal.umich.edu/~mmustata/zeta_book.pdf
the Leibniz rule---applied after locally trivializing $g^\ast\pi^\ast\mcal{O}_X(3)$ for all $g\in G$---thus implies
that $j_p^r{\alpha}(p)=\bm{0}$ for $r\defeq 2^{m/2-1}-1$,
where $j^r\maps\mcal{L}\to J^r\mcal{L}$ denotes the $r$th-order jet map
``along'' $\mcal{L}$ (from $\mcal{L}$ to its $r$th jet bundle $J^r\mcal{L}$).

Since $\alpha\mid(\phi\pi)^\ast{F^\vee}$,
Leibniz then implies $j_p^r{(\phi\pi)^\ast{F^\vee}}(p)=\bm{0}$ ``along'' the pullback line bundle $(\phi\pi)^\ast\mcal{O}_S(\deg{F^\vee})$.
But $\phi\pi\maps X'\to S$ is \etale at $p\in X'$,
so $j_{[\bm{c}]}^r{F^\vee}([\bm{c}])=\bm{0}$ ``along'' $\mcal{O}_S(\deg{F^\vee})$ itself,
over all points $[\bm{c}]\in(S\cap\PP L^\perp)(\ol{\QQ})$.
Finally,
% $\ol{S\cap\PP L^\perp} = \PP L^\perp$ (closure defined with respect to $S$ or $\PP L^\perp$\dots)
$S\cap\PP L^\perp$ is dense in $\PP L^\perp$, so the vanishing of the $r$th-order jet section $j^r F^\vee$ extends to all points $[\bm{c}]\in\PP L^\perp$, as desired.
}

\rmk{
\label{RMK:friendly-etale-local-calculus}
In the friendly setting above,
our \etale morphisms (such as $\phi\pi\maps X'\to S$),
after base change to an algebraically closed field,
always induce isomorphisms on completed local rings.
%https://math.stackexchange.com/questions/3314133/etale-iff-completions-are-isomorphic
%Assume that {\displaystyle Y}Y is locally noetherian and f is locally of finite type. For {\displaystyle x}x in {\displaystyle X}X, let {\displaystyle y=f(x)}y=f(x) and let {\displaystyle {\hat {\mathcal {O}}}_{Y,y}\to {\hat {\mathcal {O}}}_{X,x}}{\hat  {{\mathcal  O}}}_{{Y,y}}\to {\hat  {{\mathcal  O}}}_{{X,x}} be the induced map on completed local rings.
%If in addition all the maps on residue fields {\displaystyle \kappa (y)\to \kappa (x)}\kappa (y)\to \kappa (x) are isomorphisms, or if {\displaystyle \kappa (y)}\kappa (y) is separably closed, then {\displaystyle f}f is étale if and only if for every {\displaystyle x}x in {\displaystyle X}X, the induced map on completed local rings is an isomorphism.[7: EGA IV4, Proposition~17.6.3]
So e.g.~at regular points we can do calculus purely in terms of formal power series
(in general by the Cohen structure theorem, but for us $S$ is already given as a piece of $\PP^{m-1}$).
%https://stacks.math.columbia.edu/tag/0C0S
}

\ques{
For smooth $V$, is $\PP L^\perp\belongs[\grad{F}](\map{hess}(V))$ possible
(for $L\in C(\map{SSV})$)?
If so,
then in such situations,
does the conclusion of Proposition~\ref{PROP:dual-linear-subvariety} still hold?
}

% \ques{
% What is the maximum possible dimension of a vector space $L^\perp\belongs \map{Sing}(F^\vee)$?
% Assume only that $F$ is $\PP^{m-1}_\QQ$-smooth and $m\in\set{4,6}$ (if it helps).
% }

% Proving dimension $\leq m-3$ for $m=4$, or $\leq m-4$ for $m=6$, would be nice.

% \rmk{
% If $\grad{F^\vee}\vert_{L^\perp} = \bm{0}$, then differentiating would imply $\map{Hess}{F^\vee}\vert_{L^\perp}\cdot L^\perp = \bm{0}$ and $(L^\perp)^t\cdot[\map{Hess}{\grad{F^\vee}}]\vert_{L^\perp}\cdot L^\perp = \bm{0}$ (among other things), so that $\rank\map{Hess}{F^\vee}\vert_{L^\perp}\leq m-\dim(L^\perp)$.
% % Fix $L^\perp$ and suppose $\grad{F^\vee}\vert_{L^\perp} = \bm{0}$.
% % For convenience, assume $L^\perp$ is cut out by $c_{d+1}=\dots=c_m=0$, where $d=\dim(L^\perp)$.
% % Then for all $i_1,\dots,i_d\geq 0$, differentiating along $L^\perp$ would imply $J(\partial_{c_1}^{i_1}\dots\partial_{c_d}^{i_d}\grad{F^\vee})\vert_{L^\perp}\cdot L^\perp = \bm{0}$, so that $\rank J(\partial_{c_1}^{i_1}\dots\partial_{c_d}^{i_d}\grad{F^\vee})\vert_{L^\perp}\leq m-d$.

% % At least for \emph{general} $F$, does this give the desired bounds?
% % (Check the Hessian rank when $F$ is diagonal, say?)
% It is plausible that for \emph{general} $F$ (perhaps in some open neighborhood of the set of diagonal forms $F$), the abstract conditions $L^\perp\belongs\map{Sing}(F^\vee)$ and $\rank J(\partial_{c_1}^{i_1}\dots\partial_{c_d}^{i_d}\grad{F^\vee})\vert_{L^\perp}\leq m-d$ alone (forgetting the linear structure of $L^\perp$) might already be enough to prove $\dim(L^\perp)\leq m/2-1$ for $m=4$ or for $m=6$.
% }

\subsection{The diagonal case}
\label{SUBSEC:diagonal-dual-example}

Say $F$ is diagonal,
and write $F = F_1x_1^3 + \dots + F_mx_m^3$.
Then we can explicitly verify all the theory above.
Here $[\grad{F}]\maps [\bm{x}]\mapsto [3F_1x_1^2,\dots,3F_mx_m^2]$ is Galois with abelian Galois group $\mu_2^m/\mu_2\cong (\ZZ/2)^{m-1}$,
and $\map{hess}(V)$ is cut out by $(6F_1x_1)\cdots(6F_mx_m)=0$.

\subsubsection{Describing \texpdf{$C(\map{SSV})$}{C(SS)}}

% Then $L,L^\perp$ are \dots
Since $F$ is diagonal,
the number of $m/2$-dimensional vector spaces $L\belongs C(V)_\CC$ over $\CC$ is $(\deg{F})^{m/2}=3^{m/2}$ times
$C_{m/2-1}\defeq (m-1)!!$, the number of \emph{pairings} of $[m]$.
(See e.g.~\cite{starr2005fact_in_browning2006density}*{Starr's Appendix, top of p.~302} for a more general statement on Fermat hypersurfaces of degree $d\geq3$ in $s\in\set{4,6,8,\dots}$ variables.)
So we make a combinatorial definition:
\defn{
\label{DEFN:diagonal-permissible-pairing}
Let $\mcal{J}=(\mcal{J}(k))_{k\in \mcal{K}}$ denote an \emph{ordered set partition} of $[m]$:
a list of pairwise disjoint nonempty sets $\mcal{J}(k)\belongs [m]$ covering $[m]$, indexed by a set $\mcal{K}\in \set{[1], [2], [3], \dots}$.
Call $\mcal{J},\mcal{J}'$ \emph{equivalent} if they define the same unordered partition of $[m]$ (i.e.~if $\mcal{K}=\mcal{K}'$ and $\set{\mcal{J}(k): k\in \mcal{K}}=\set{\mcal{J}'(k): k\in \mcal{K}'}$).

Call $\mcal{J}$ a \emph{pairing} if
% $\set{\card{\mcal{J}(k)}:k}=\set{2}$.
$\card{\mcal{J}(k)} = 2$ for all $k\in \mcal{K}$.
Call $\mcal{J}$ \emph{permissible} if
for all $k\in \mcal{K}$ and $i,j \in\mcal{J}(k)$, we have $F_j/F_i\in (\QQ^\times)^3$.
For a permissible $\mcal{J}$,
let $\mcal{R}_{\mcal{J}}
\defeq \set{\bm{c}\in\ZZ^m
: \textnormal{if $k\in \mcal{K}$ and $i,j \in\mcal{J}(k)$, then $c_i/F_i^{1/3}=c_j/F_j^{1/3}$}}$---and
given $\bm{c}\in \mcal{R}_{\mcal{J}}$,
define $c\maps \mcal{K}\to \RR$ so that
% $c(k)=c_i/F_i^{1/3}$ holds for all $k,i$ with $i\in\mcal{J}(k)$.
% $\forall{k},\;\set{c_i/F_i^{1/3}:i\in\mcal{J}(k)}=\set{c(k)}$.
for all $k\in \mcal{K}$ and $i \in\mcal{J}(k)$, we have $c_i/F_i^{1/3}=c(k)$.
}

% \defn{
% Let $\mcal{J}=(\mcal{J}(k))_{k\geq0}$ denote an ordered partition of $[m]$.
% We identify $\mcal{J},\mcal{J}'$ as the same if $\set{\mcal{J}(k):k\geq 1}=\set{\mcal{J}'(k):k\geq 1}$ as sets.
% % with parts $(\mcal{J}(0),\set{\mcal{J}(k)}_{k\geq 1})$. 
% % indexed by $k\geq 0$.
% Call $\mcal{J}$ a \emph{pairing} if
% $\mcal{J}(0)=\emptyset$ and
% $\set{\card{\mcal{J}(k)}:k\geq1}\belongs\set{0,2}$.
% Call $\mcal{J}$ \emph{permissible} if
% for each $k\geq1$, the class $F_i\bmod{(\QQ^\times)^3}$ is constant over $i\in\mcal{J}(k)$.
% For permissible $\mcal{J}$,
% let $\mcal{R}_{\mcal{J}}
% \defeq \set{\bm{c}\in\ZZ^m
% : \textnormal{there exists}\;c\maps\ZZ_{\geq0}\to\RR,
% \;\textnormal{with}\;c(0)=0,
% \;\textnormal{such that}\;
% % \set{c_i/F_i^{1/3}:i\in\mcal{J}(k)}=\set{c(k)}\;\forall\;k\geq0,
% c_i/F_i^{1/3}=c(k)\;\textnormal{for all $k\geq0$ and $i\in\mcal{J}(k)$}}$.

% \emph{In practice},
% when working with any given $\mcal{J}$,
% we \emph{assume} $\card{\mcal{J}(1)}\geq\card{\mcal{J}(2)}\geq\cdots$,
% and \emph{restrict attention} to indices $k\geq0$ with $\mcal{J}(k)\neq\emptyset$.
% }

% \ex{[Diagonal case]
% \label{EX:diagonal-dual-example}
% }

Knowing the number of $L\belongs C(V)_{\CC}$,
we can then give an exhaustive construction:
each equivalence class of pairings $\mcal{J}$ yields $3^{m/2}$ distinct $L/\CC$,
obtained by setting $F_ix_i^3+F_jx_j^3=0$ for each part $\mcal{J}(k)=\set{i,j}$.
Over $\QQ$,
we must set $x_i+(F_j/F_i)^{1/3}x_j=0$---which is
valid when $F_i\equiv F_j\bmod{(\QQ^\times)^3}$.
It follows that
the only $m/2$-dimensional $L$'s on $C(V)_\CC,C(V)$ (over $\CC,\QQ$, respectively) are the ``obvious'' ones.

\rmk{
Since we do not know of a reference \emph{proving} the aforementioned statement of Starr,
we should mention that given $d,s$,
% Use https://math.stackexchange.com/questions/1281981/27-lines-on-fermat-surface setup to reduce to automorphisms of Fermat cubic forms, and then cite Shioda/Kontogeorgis (who work projectively, i.e. up to scaling, but that is not substantial over C).
the statement easily follows from
Gaussian elimination,
symmetry,
and the fact that over $\CC$,
the only \emph{linear} automorphisms
of the ``halved'' (i.e.~$s/2$-variable degree-$d$) Fermat hypersurface
are the ``obvious'' ones
(see e.g.~\cite{shioda1988arithmetic} or \cite{kontogeorgis2002automorphisms}*{proofs of Proposition~3.1 and Example~1}).
}

Therefore we obtain the following (essentially classical) result:
\obs{
\label{OBS:characterize-diagonal-L/Q}
There is a canonical bijection,
between $C(\map{SSV})$ and the set of equivalence classes of \emph{permissible} pairings $\mcal{J}$,
characterized by $L\cap\ZZ^m = \mcal{R}_{\mcal{J}}^\perp$ (an equality of sublattices of $\ZZ^m$).
}

% \pf{
% Take rationality into account.
% }

\subsubsection{Analyzing the discriminant}

For convenience,
fix square roots $F_i^{1/2}\in\ol{\QQ}^\times$.
Up to scaling
(which matters in the non-archimedean analysis underlying Lemma~\ref{LEM:stating-bias-for-generic-trivial-c's},
% and that of Papers~I and~III,
but not here),
the discriminant form $F^\vee(\bm{c})$ factors in $\ol{\QQ}[\bm{c}^{1/2}]$ as
\mathn{
\label{EXPR:diagonal-discriminant-factorization}
\prod_{\bm{\eps}}(\eps_1F_1^{-1/2}c_1^{3/2}+\eps_2F_2^{-1/2}c_2^{3/2}+\dots+\eps_mF_m^{-1/2}c_m^{3/2})\in \QQ[\bm{c}],
}
with the product taken over $\bm{\eps} = (\eps_1,\dots,\eps_m)$ with $\eps_1=1$ and $\eps_2,\dots,\eps_m=\pm1$.
(This formula is classical;
see~\cite{wang2021_large_sieve_diagonal_cubic_forms}*{\S1.2, proof of Proposition-Definition~1.8 for diagonal $F$} or \cite{heath1998circle}*{eq.~(4.2)}.)
% Then $\grad{F^\vee}(\bm{c})$, computed over $\ol{\QQ}[\bm{c}^{1/2}]$ (say), is
% \mathd{
% \left(\sum_{\bm{\eps}}\frac{(3/2)\eps_iF_i^{-1/2}c_i^{1/2}F^\vee(\bm{c})}{\eps_1F_1^{-1/2}c_1^{3/2}+\dots+\eps_mF_m^{-1/2}c_m^{3/2}}\right)_i,
% }
% The $(i,j)$ entry of $\map{Hess}{F^\vee}(\bm{c})$ is then
% \mathd{
% \sum_{\bm{\eps}=\bm{\eps}'}\frac{(3/4)\eps_ic_i^{-1/2}\bm{1}_{i=j}F^\vee(\bm{c})}{\eps_1c_1^{3/2}+\dots+\eps_mc_m^{3/2}}
% +\sum_{\bm{\eps}\neq\bm{\eps}'}\frac{(9/4)\eps_i\eps_j(c_ic_j)^{1/2}F^\vee(\bm{c})}{(\eps_1c_1^{3/2}+\dots+\eps_mc_m^{3/2})(\eps'_1c_1^{3/2}+\dots+\eps'_mc_m^{3/2})}.
% }
% Now fix such $\bm{c}$.
% A pairing argument (pairing $\bm{\eps}$ with its ``conjugate with respect to $c_i^{1/2}$'' to get rid of the inverse $c_i^{-1/2}$), with a short casework depending on whether $c_i=0$ or $c_i\neq0$, shows that the sum over $\bm{\eps}=\bm{\eps}'$ must vanish.

% If $\bm{\eps}_1,\bm{\eps}_2$ are unique, then\dots

% If $\bm{\eps}_1,\bm{\eps}_2,\bm{\eps}_3$ exist, then in fact there must exist a fourth distinct $\bm{\eps}_4\in\mu_2^m$ in the additive $\QQ$-span of $\bm{\eps}_1,\bm{\eps}_2,\bm{\eps}_3$.
% (This might be true, but it might not quite give a codimension $4$ condition on $\bm{c}$.)

Now fix a tuple $\bm{c}\neq\bm{0}$ with $F^\vee(\bm{c})=0$,
and fix square roots $c_i^{1/2}\in\ol{\QQ}$.
% Inspection of the factored form of $F^\vee$, with extra care if $c_1\cdots c_m=0$, shows that $\grad{F^\vee}(\bm{c})=\bm{0}$ if and only if there exist at least two distinct $\bm{\eps}$ with $(\eps_1F_1^{-1/2}c_1^{3/2}+\dots+\eps_mF_m^{-1/2}c_m^{3/2})=0$.
% More generally:
Then using formal power series calculus over $c_i\neq 0$
(by Remark~\ref{RMK:friendly-etale-local-calculus}, adapted to $\Aff^1_{\ol{\QQ}}\to\Aff^1_{\ol{\QQ}},\;t\mapsto t^2$ away from the origin),
we will prove the following result,
which precisely characterizes the order of vanishing of $F^\vee$ at $\bm{c}$:
\prop{
\label{PROP:characterize-diagonal-Disc-vanishing-order}
Fix $r\geq 0$.
Then the affine jet $j^r{F^\vee}$ vanishes at a given point $\bm{c}\neq\bm{0}$
if and only if there exist at least $r+1$ distinct $\bm{\eps}$ with $({\cdots})=0$.
}

\rmk{
% \todo[inline]{Interpret number of $\bm{\eps}$'s as a sum over $[\bm{x}]\in(V_{\bm{c}})_{\map{sing}}$, with multiplicity some power of $2$ depending only how many $x_i=0$.}
A short computation yields the equality
\mathd{
\#\set{\textnormal{such distinct $\bm{\eps}$'s}}
= \sum_{[\bm{x}]\in\gamma(\overline{\QQ})^{-1}([\bm{c}])}
2^{\#\set{i\in[m]:x_i=0}},
}
where $\gamma(\overline{\QQ})^{-1}([\bm{c}])\defeq
\set{[\bm{x}]\in V(\ol{\QQ}): [\grad{F}(\bm{x})]=[\bm{c}]}
= \set{\textnormal{singular $\ol{\QQ}$-points of $V_{\bm{c}}$}}$.
(Here $x_i$ corresponds to $\eps_iF_i^{-1/2}c_i^{1/2}$,
with some ambiguity or ``multiplicity'' in $\eps_i$ when $x_i=0$.)
The previous display provides a geometric interpretation of the number of $\bm{\eps}$'s in Proposition~\ref{PROP:characterize-diagonal-Disc-vanishing-order};
thus we can formulate the proposition more geometrically,
without reference to $\bm{\eps}$'s.
Does this \emph{geometric formulation} generalize somehow to arbitrary $\PP^{m-1}_\QQ$-smooth $F$?
}

\pf{
Induct on $r\geq 0$.
% , differentiating repeatedly using the product rule.
The base case $r=0$ follows directly from the factorization of $F^\vee$.
Now suppose $r\geq 1$,
and assume the result for $r-1$.

First,
we prove the forwards implication for $r$.
Here it suffices to work with ``pure'' derivatives $\partial_{c_i}^{\leq r}$,
for just a single index $i$ with $c_i\neq0$.
For example,
if $c_1\neq0$, and there exist \emph{exactly} $r$ distinct $\bm{\eps}_1,\dots,\bm{\eps}_r$ with $({\cdots})=0$,
then $r\leq 2^{m-1}$, and the product rule implies
\mathd{
\partial_{c_1}^r{F^\vee}(\bm{c})\propto_{3,r,F} (c_1^{1/2})^r\prod_{\bm{\eps}\neq\bm{\eps}_1,\dots,\bm{\eps}_r}({\cdots})\neq 0.
}
But if $j^r{F^\vee}(\bm{c})=0$, then by the inductive hypothesis,
there must exist \emph{at least} $r$ distinct $\bm{\eps}$'s with $({\cdots})=0$,
and thus at least $r+1$.
This proves the the forwards implication for $r$.

It remains to prove the backwards implication,
i.e.~that if there exist \emph{at least} $r+1$ distinct $\bm{\eps}$'s with $({\cdots})=0$, then $j^r{F^\vee}(\bm{c})=\bm{0}$.
% By the inductive hypothesis, $j^{r-1}{F^\vee}(\bm{c})=\bm{0}$, so we need only prove $\partial_{\bm{c}}^{\bm{r}}{F^\vee}(\bm{c})=0$ for $\abs{\bm{r}}=r$.
We must take extra care if $c_1\cdots c_m=0$.
Say $c_i=0$ for $i\in I$.
% For each $i\notin I$ we assume $c_i\neq 0$ lies in some fixed simply connected region of $\CC^\times$, and fix a corresponding branch of the square root $c_i^{1/2}$.
Then the following ``formal analytic functions''---indexed by certain triples $(\bm{a},\bm{b},E)$---span a $\ol{\QQ}$--vector space \emph{closed under $\bm{c}$-differentiation}:
% the algebraic expressions
\mathd{
\prod_{i\in I}c_i^{a_i}
\prod_{i\notin I}c_i^{b_i/2}
\prod_{\bm{\eps}\in E}(\eps_1F_1^{-1/2}c_1^{3/2}+\dots+\eps_mF_m^{-1/2}c_m^{3/2})
\in\ol{\QQ}[c_i]_{i\in I}[c_i^{1/2},c_i^{-1/2}]_{i\notin I},
}
% with $a_i\in\ZZ_{\geq 0}$ and $b_i\in\ZZ$, and with $E\bmod{\pm{1}}$ invariant under flipping $\eps_i$ for $i\in I$.
% (of degree $\abs{\bm{a}}+\abs{\bm{b}}/2+3\card{E}/2$),
for $(\bm{a},\bm{b})\in\ZZ_{\geq0}^I\times\ZZ^{[m]\setminus I}$ and $E\belongs\set{\bm{\eps}\in\set{\pm1}^m:\eps_1=1}$ with $E\bmod{\pm{1}}$ invariant under flipping $\eps_i$ for $i\in I$.

Specifically,
differentiating in $c_i$ leads to
terms with $a_i\mapsto a_i-1$ or $(a_i,\card{E})\mapsto (a_i+2,\card{E}-2)$ if $i\in I$,
and to terms with $b_i\mapsto b_i-2$ or $(b_i,\card{E})\mapsto (b_i+1,\card{E}-1)$ if $i\notin I$.
In each case,
applying $\partial_{c_i}$ decreases $\min_{\bm{a},\bm{b},E}(\abs{\bm{a}}+\card{E})$ by at most $1$.

Now fix $\bm{r}\in\ZZ_{\geq0}^m$ with $\abs{\bm{r}}\leq r$.
Then $\partial_{\bm{c}}^{\bm{r}}{F^\vee}$ is a $\ol{\QQ}$-linear combination of functions indexed by
triples $(\bm{a},\bm{b},E)$ with $\abs{\bm{a}}+\card{E}\geq 2^{m-1}-r$ (and thus $\card{E}\geq 2^{m-1}-r$ or $\abs{\bm{a}}\geq 1$).
Each such function must vanish at our original given point $\bm{c}$,
so $\partial_{\bm{c}}^{\bm{r}}{F^\vee} = 0$,
as desired.
}

\subsubsection{Evaluating \texpdf{$F^\vee$}{Fvee} on \texpdf{$L^\perp$}{Lperp} for special \texpdf{$L$'s}{L's}}

Fix $L\in C(\map{SSV})$.
Proposition~\ref{PROP:characterize-diagonal-Disc-vanishing-order} has the following corollary:
\cor{
% Take $r=2^{m/2-1}-1$.
\label{COR:baseline-diagonal-Disc-vanishing}
For $L$ as above,
we have $(j^{2^{m/2-1}-1}{F^\vee})\vert_{L^\perp}=\bm{0}$.
}

\pf{
By Observation~\ref{OBS:characterize-diagonal-L/Q}, $L$ corresponds to some permissible pairing $\mcal{J}$.
For each part $\mcal{J}(k)=\set{i,j}$,
there are exactly two choices of signs $(\eps_i,\eps_j)\in\set{\pm1}^2$---or
only one choice if $1\in\mcal{J}(k)$---such that
$\eps_iF_i^{-1/2}c_i^{3/2}+\eps_jF_j^{-1/2}c_j^{3/2}$ vanishes over \emph{all} $\bm{c}\in L^\perp\cap\ZZ^m = \mcal{R}_{\mcal{J}}$ lying in a given orthant of $\RR^m$.
So given $\bm{c}\in L^\perp\setminus\set{\bm{0}}$,
we can apply Proposition~\ref{PROP:characterize-diagonal-Disc-vanishing-order} ``backwards'' with $r\defeq 2^{m/2-1}-1$.
% there are at least $2^{m/2-1}$ distinct sign choices $\bm{\eps}$ with 
(Of course,
$L^\perp\setminus\set{\bm{0}}$ is dense in $L^\perp$,
so the vanishing then extends to all of $L^\perp$.)
}

Thus we have explicitly verified the conclusion of Proposition~\ref{PROP:dual-linear-subvariety}.
The next result shows that in fact,
$F^\vee$ generally does not vanish to higher order along $L^\perp$.
\obs{
\label{OBS:higher-order-diagonal-Disc-vanishing-critera}
Given $L, \mcal{J}$ as above,
fix $\bm{c}\in L^\perp\cap\ZZ^m = \mcal{R}_{\mcal{J}}$.
For each $k\in \mcal{K}$,
choose a square root $c(k)^{3/2}\defeq F_i^{-1/2}c_i^{3/2}$
% (taken inside the well-defined Galois compositum $\QQ[F_i^{1/2},c_i^{1/2}]_{i\in[m]}$, say)
%$\QQ[(F_ic_i)^{1/2}]_{i\in[m]}$ is also large enough of an extension.
in $\ol{\QQ}$ (say).
Then $j^{2^{m/2-1}}{F^\vee}(\bm{c})=\bm{0}$
if and only if
there exist $l\geq 1$ distinct indices $k_1,\dots,k_l\in \mcal{K}$ such that $c(k_1)^{3/2}\pm\cdots\pm c(k_l)^{3/2}=0$ for some choice of signs.
Consequently,
if $j^{2^{m/2-1}}{F^\vee}(\bm{c})=\bm{0}$,
% then $\set{c(k)^{3/2}}_{k\geq 1}$ must be linearly dependent over $\QQ$.
then $c(k_1)^3c(k_2)^3\in(\QQ^\times)^2\cup\set{0}$ for some distinct $k_1,k_2\in \mcal{K}$.
}

\pf{
For the equivalence,
apply Proposition~\ref{PROP:characterize-diagonal-Disc-vanishing-order} ``forwards'' with $r\defeq 2^{m/2-1}$
(and then ``simplify'' the resulting conclusion using the fact that $\mcal{J}$ is a \emph{pairing}).
To obtain the final conclusion,
note that the condition $c(k_1)^{3/2}\pm\cdots\pm c(k_l)^{3/2}=0$ \emph{implies} the following,
provided $l$ is \emph{minimal} among all possible $l$'s
(as we may certainly assume):
\begin{enumerate}[(1)]
    \item If $l=1$,
    then $c(k)^3=0$ for some $k\in \mcal{K}$.
    
    \item If $l=2$ is \emph{minimal},
    then $c(k_1)^3=c(k_2)^3\in\QQ^\times$ for some distinct $k_1,k_2\in \mcal{K}$.
    % in which case there exists a permissible $\mcal{J}'$ with $\bm{c}\in\mcal{R}_{\mcal{J}'}\subsetneq\mcal{R}_{\mcal{J}}$, or else
    
    \item If $l\geq 3$ is \emph{minimal},
    then $c(k_t)^3\in\QQ^\times$ for all $t\in[l]$,
    and by multi-quadratic field theory in characteristic $0$,
    the square classes $c(k_1)^3,\dots,c(k_l)^3\bmod{(\QQ^\times)^2}$ must all coincide.
    (More precisely,
    given indices $i_t\in\mcal{J}(k_t)$ for $t\in[l]$,
    we must have
    % $c_{i_t}/F_{i_t}=x_{i_t}^2d$, i.e.
    $c(k_t)^3 = F_{i_t}^2x_{i_t}^6d^3\in d\cdot(\QQ^\times)^2$ for some $d,x_{i_t}\in\QQ^\times$ such that
    $F_{i_1}x_{i_1}^3+\dots+F_{i_l}x_{i_l}^3=0$.
    If $\mcal{J}(k_1),\dots,\mcal{J}(k_l)$ cover $[m]$,
    % (e.g.~if $m=6$ then this must be the case, and we must have $l=3$),
    as must be the case if $m=6$,
    then this would imply
    % $[\bm{c}] = [\grad{F}(\bm{x})]$ for some choice of $\bm{x}$, so
    that $[\bm{c}]\in\PP L^\perp$ actually lies in the image of $\gamma(\QQ)$---unlike most points of $\PP L^\perp$.)
\end{enumerate}
% but unless there is some tuple with $l\leq 2$, then there \emph{need not} exist a permissible $\mcal{J}'$ with $\bm{c}\in\mcal{R}_{\mcal{J}'}\subsetneq\mcal{R}_{\mcal{J}}$.
In each case,
the claimed multiplicative relationship exists for some distinct $k_1,k_2\in \mcal{K}$.
This completes the proof.
}

\rmk{
\label{RMK:char-p-diagonal-Disc-analysis}
\S\ref{SUBSEC:diagonal-dual-example} is written in characteristic $0$,
but since $F^\vee\in\ZZ[\bm{c}]$,
the results, with their \emph{algebraic} proofs,
carry over to arbitrary fields of characteristic $p\nmid (6^m)!F_1\cdots F_m$.
% $p\gg_{3,r,F_1,\dots,F_m}1$.
Over $\FF_p$,
such extensions of
Proposition~\ref{PROP:characterize-diagonal-Disc-vanishing-order} (in its geometric formulation),
Corollary~\ref{COR:baseline-diagonal-Disc-vanishing},
and (the equivalence part of) Observation~\ref{OBS:higher-order-diagonal-Disc-vanishing-critera}
prove useful in the proof of the important Lemma~\ref{LEM:stating-bias-for-generic-trivial-c's} stated below.

(Though unimportant for us over $\FF_p$,
the other results of \S\ref{SUBSEC:diagonal-dual-example} also carry over.
For instance,
regarding the field theory behind the last part of Observation~\ref{OBS:higher-order-diagonal-Disc-vanishing-critera}:
if $K$ is a field of characteristic $p\nmid 2$,
and $d_1,\dots,d_l\in K^\times$ are pairwise incongruent modulo $(K^\times)^2$,
then $\sqrt{d_1},\dots,\sqrt{d_l}\in K(\sqrt{d_1},\dots,\sqrt{d_l})$ are linearly independent over $K$.)
}

\section{Some key ingredients}

The following lemma is close in spirit to one direction of Corollary~\ref{COR:optimal-criteria-for-boundedness-of-E_c} (see also Question~\ref{QUES:correlate-special-subvarieties-Manin-vs-F_q}).
The proof in \cite{wang2021_isolating_special_solutions}*{\S6} begins with a change of coordinates somewhat related to van der Corput or Weyl differencing, or (probably) to certain blow-ups along $m/2$-planes;
see \cite{wang2021_isolating_special_solutions}*{Remark~6.5}.
\lem{
[\cite{wang2021_isolating_special_solutions}*{Lemma~5.5}]
\label{LEM:stating-bias-for-generic-trivial-c's}
Assume $F$ is diagonal,
with $m\in\set{4,6}$.
Fix $L\in C(\map{SSV})$.
Suppose $\bm{c}\in \Lambda^\perp$ is trivial,
and $p\nmid (j^{2^{m/2-1}}{F^\vee})(\bm{c})$ is a prime.
Then
\mathd{
\wt{S}_{\bm{c}}(p)
= \phi(p)p^{-1/2}
+ O(1).
}
%S_{\bm{c}}(p) = \phi(p)p^{m/2} + O(p^{(1+m)/2}).
Also, if we take $\mcal{J}$ corresponding to $L$ via Observation~\ref{OBS:characterize-diagonal-L/Q},
then in the notation of Definition~\ref{DEFN:diagonal-permissible-pairing},
$c(k)^3\in\QQ$
% lies in $\ZZ_{(p)}^\times$
is a well-defined $p$-adic unit for all $k$,
and
\mathd{
\wt{S}_{\bm{c}}(p^l)
= \phi(p^l)p^{-l/2}
\cdot \prod_{k\in[m/2-1]}\left[1+\chi\left(c(k)^3c(k+1)^3\right)\right]
\ll \phi(p^l)p^{-l/2}
}
for all integers $l\geq 2$,
if we let $\chi(r)\defeq(\frac{r}{p})$.
Both implied constants depend only on $m$.
}

To state the next ingredient, let $\mcal{S}\belongs\set{\bm{c}\in\ZZ^m:F^\vee(\bm{c})=0}$ be a \emph{homogeneous}
(i.e.~invariant under scaling, so $\bm{c}\in\mcal{S}$ implies $\ZZ\cdot\bm{c}\belongs\mcal{S}$)
subset of $\bm{c}$'s with $F^\vee(\bm{c})=0$.
At several technical points
% of the proof of Theorem~\ref{THM:contribution-from-generically-singular-c's},
in the proof of Theorem~\ref{THM:contribution-from-generically-singular-c's} (see \cite{wang2021_isolating_special_solutions}*{\S5}),
the following lemma lets us cleanly discard various contributions from
\emph{sparse} homogeneous sets $\mcal{S}$---when restricted to $\bm{c}\neq\bm{0}$, at least.
The proof (based on \cite{heath1998circle}*{pp.~688--689}) is rather awkward, due to a lack of a deeper algebro-geometric theory for $S_{\bm{c}}$ when $V_{\bm{c}}$ is singular.
\lem{
[\cite{wang2021_isolating_special_solutions}*{Lemma~4.1}]
\label{LEM:sparse-bound}
Assume $F$ is diagonal.
If $\mcal{S}\cap[-C,C]^m$ has size $O(C^{m/2-\delta+\eps})$ for all $C\gg 1$,
then
\mathd{
Y^{-2}\sum_{\bm{c}\in\mcal{S}\setminus\set{\bm{0}}}
\sum_{n\geq 1}
\abs{I_{\bm{c}}(n)}
\cdot n^{1-m/2}
\cdot \max_{n_\star\mid n} n_\star^{-1/2}\abs{\wt{S}_{\bm{c}}(n_\star)}
\ll_\eps X^{(m-\delta)/2+\eps}
}
provided $4\leq m\leq 6$ and $\delta\leq\min((m+2)/4,(m-1)/2)$.
}

\rmk{
\label{RMK:non-diagonal-F}
% Very conjecturally, we can also generalize the proof.
We can ``axiomatize'' our proof of Theorem~\ref{THM:contribution-from-generically-singular-c's},
in the hope of extending Theorem~\ref{THM:contribution-from-generically-singular-c's}
to general non-diagonal $F$
(though a full diagnosis of the relevant issues
would seem to require algebraic geometry beyond the author's current expertise).
Assume
\begin{enumerate}[(1)]
    \item $m\geq 4$ is even,
    $F$ has nonzero discriminant,
    and $F^\vee$ is
    % unsurprising (Definition~\ref{DEFN:unsurprising-Disc-zero-locus});
    unsurprising (in the sense of Definition~\ref{DEFN:unsurprising-Disc-zero-locus});
    
    \item $(F,w)$ is clean in the sense of Definition~\ref{DEFN:support-smooth-clean};
    
    \item a version of Lemma~\ref{LEM:sparse-bound},
    with the same hypotheses but with a modified conclusion
    \mathd{
    \frac{X^m}{Y^2}\sum_{\bm{c}\in \mcal{S}\setminus\set{\bm{0}}}
    \sum_{n\geq 1}
    \left(1 + \frac{X\norm{\bm{c}}}{n}\right)^{1-m/2} \nu_1\left(\frac{\norm{\bm{c}}}{X^{1/2}}\right)
    \cdot n^{(1-m)/2} \abs{\wt{S}_{\bm{c}}(n)}
    \ll_\eps X^{(m-\delta)/2+\eps},
    }
    remains true,
    where $\nu_1(\star)$ is a fixed function decaying as $O_A(\max(1,\star)^{-A})$;
    and
    
    \item in Lemma~\ref{LEM:stating-bias-for-generic-trivial-c's},
    the formula for $S_{\bm{c}}(p)$ and upper bound for $S_{\bm{c}}(p^{\geq 2})$ remain true,
    provided $p\gg_{3,m,F}1$ exceeds some threshold (only allowed to depend on $3,m,F$).
\end{enumerate}
Then the conclusion of Theorem~\ref{THM:contribution-from-generically-singular-c's}
still holds for $F$,
as does that of Corollary~\ref{COR:Disc-vanishing-sum-with-explicit-linear-density}
if $m\geq 6$.
% (There is some flexibility regarding the precise definition of ``unsurprising'' in (1), and the precise range of applicability of (3).)
(Depending on the precise ``Definition~\ref{DEFN:unsurprising-Disc-zero-locus}'' in (1),
a weaker version of (3) may suffice.)
}

\rmk{
In Remark~\ref{RMK:non-diagonal-F},
we expect that (4) should be relatively routine to prove (if true),
but (1) and (3) may well require substantial new ideas.
}

\section{Discussion of Manin-type conjectures}

Let $F,V,\dots$ be as in \S\ref{SEC:delta-method-setup}.
The Hardy--Littlewood ``randomness'' (singular series) prediction for $F=0$ may fail,
even when $m=6$ (see Example~\ref{EX:6-cubes-zero-locus}).

\rmk{
% (Such failures are expected to only be possible if $m\leq6$.)
The ``randomness'' prediction is expected to hold for $m\geq 7$,
since $V$ is \emph{smooth}.
% , but for non-diagonal $F$ (with $(F,w)$ clean), only $m\geq 8$ has been rigorously analyzed (unconditionally for $m\geq 9$ and conditionally for $m=8$) \cites{hooley2014octonary,hooley2015octonary}.
(At least when $m\geq 8$ and $(F,w)$ is clean,
\cite{hooley2015octonary} provides a conditional affirmative proof, unconditional for $m\geq 9$.)
For \emph{singular} cubics,
however,
failure can occur even when $m\geq7$;
see \cite{brudern2019instance} for an interesting example when $m=8$.
}

% But whenever such a failure occurs, there should be a good \emph{reason} for it!
But every ``randomness failure'' should have a good \emph{excuse}!

A plausible version of Manin's conjecture
(in a smoothed form) for $C(\mcal{V})\belongs\Aff^m_\ZZ$ says that
away from a certain \emph{special structured locus}---namely
the empty set
if $m\geq7$,
and the union of all (\emph{linear}) vector spaces $L\belongs C(V)$ over $\QQ$ of dimension $\floor{m/2}$
if $m\leq 6$---one should have
\mathd{
N_{F,w}(X)
-\sum_{\bm{x}\in\ZZ^m}w(\bm{x}/X)\cdot\bm{1}_{\bm{x}\in\bigcup_{L}L}
= (c+o_{F,w;X\to\infty}(1))\cdot X^{m-3}(\log{X})^{r-1+\bm{1}_{m=4}}
}
% for any fixed weight $w\in C^\infty_c(\RR^m)$,\footnote{assuming smoothness for simplicity of discussion below}
for a certain precise \emph{predicted}
constant $c=c_{C(\textnormal{MP}),w}\in\RR$ and integer rank $r\geq 1$.

\rmk{
If $m\geq 5$, then
$r=1$ always---so the $\log{X}$ disappears---while
$c_{C(\textnormal{MP}),w}$, the ``coned'' (or ``unsieved'') Manin--Peyre constant,
always equals the Hardy--Littlewood constant $c_{\textnormal{HL},w}$.
If $m=4$,
then \emph{typically}, but not always,
% For typically $r=1$ claim when m=4$, see the Hilbert irreducibility argument in http://math.mit.edu/~poonen/papers/random.pdf (for "Galois action as large as possible")
% For typically $c_MP = c_HL$ claim for $m=4$, see \S2 of \verb!Delta_method_heuristics (6).pdf! from October 21, 2019.
we have $(r,c_{C(\textnormal{MP}),w})=(1,c_{\textnormal{HL},w})$---if
one interprets $c_{\textnormal{HL},w}$ generously as in \cite{jahnel2014brauer}*{Chapter~II, Remarks~7.5--7.7}.

In these cases,
``Manin'' differs from ``Hardy--Littlewood''
only in the \emph{special} part.
(But \emph{sometimes} when $m=4$,
Brauer--Manin obstructions or other phenomena
lead to \emph{further} differences;
e.g.~loosely speaking,
it is possible for ``$c_{\textnormal{HL},w}\neq0=c_{C(\textnormal{MP}),w}$'' or related behavior to occur.)
}

% \rmk{
% % Since $V/\QQ$ is \emph{smooth},
% % special subvarieties such as $L\belongs C(V)$
% % should not play a (leading) role when $m\geq 7$.
% % Yet if we allowed \emph{singular} $V/\QQ$, ... could arise even when $m=8$.
% For \emph{singular} cubics,
% special subvarieties
% can arise even when $m=8$ \cite{brudern2019instance}.
% }

For a \emph{general} overview of ``Manin'',
see \cite{bombieri2009problems}*{\S2 and references within} or \cites{browning2009quantitative,jahnel2014brauer}.
In general,
``Manin'' is imprecise regarding the \emph{special} part (though nowadays there are precise, but sometimes complicated, proposals available; see e.g.~\cite{lehmann2018geometric}).
However,
at least when $m=6$ and $F=x_1^3+\dots+x_6^3$,
% (an interesting case of particular interest)
the \emph{specific} conjecture recorded above was---in essence---stated
first, and precisely, by Hooley \cite{hooley1986some}*{Conjecture~2}.
(See also \cite{vaughan1995certain}*{Appendix}.)

Finally, we raise some further questions (about ``more complicated'' varieties)
% that also might merit study (even if done in the ``$\bm{c}$-restricted'' sense of Theorem~\ref{THM:contribution-from-generically-singular-c's}).
that one also might (at first) study in the ``$\bm{c}$-restricted'' sense of Theorem~\ref{THM:contribution-from-generically-singular-c's}.

\ex{
\label{EX:II.1.7}
It would be interesting to extend our analysis to $m=5$,
even just for diagonal $F$.
See \cite{bombieri2009problems}*{\S3} for a discussion of
the potentially infinite family of lines on a cubic threefold $V\belongs\PP^4$.
Can one see these lines via $V^\vee$
(cf.~Proposition~\ref{PROP:dual-linear-subvariety})?
}

Now consider the following example (which I learned from a talk of Wooley)
of a situation in which (one expects that) \emph{nonlinear} special subvarieties arise.
% One could also change "Example" to "Remark" and say "Consider the following remark illustrating that nonlinear special subvarieties arise.", or perhaps say "Consider the following remark." (and begin the remark with "Nonlinear special subvarieties can arise. For example:")
\ex{
[\cite{wooley2019talk}]
\label{EX:Wooley-special-quadratic-subvarieties}
Over boxes $[-X,X]^6$ as $X\to\infty$,
one expects the $6$-variable quartic $x_1^4+x_2^4+x_3^4 = x_4^4+x_5^4+x_6^4$
to have not only a ``purely probabilistic'' source of $\asymp X^2$ points (as $X\to\infty$),
%local density / singular series / singular integral contribution (adelic singular series contribution?)
but also at least two relevant ``special'' sources of points:
% over $\QQ$:
\begin{enumerate}[(1)]
    \item the ``trivial'' or ``diagonal-type'' linear locus
    \mathd{
    x_1\pm x_4 = x_2\pm x_5 = x_3\pm x_6 = 0
    \quad\textnormal{(and the obvious permutations)}
    }
    contributing $\asymp X^3$ points (as $X\to\infty$),
    as well as
    
    \item a secondary \emph{quadratic} locus
    \mathd{
    x_1\pm x_2\pm x_3
    = x_4\pm x_5\pm x_6
    = (x_1^2+x_2^2+x_3^2)
    \pm (x_4^2+x_5^2+x_6^2)
    = 0
    }
    contributing $\asymp X^2\log{X}$ points (as $X\to\infty$).
\end{enumerate}
(The underlying identity behind (2) is $a^4+b^4+(a+b)^4 = 2(a^2+ab+b^2)^2$.)
%https://www.wolframalpha.com/input/?i=factor+x%5E4%2By%5E4%2B%28x%2By%29%5E4
}

\ques{
\label{QUES:II.1.26}
% It would be interesting to try
Can one detect (2) naturally via the delta method; or if not, why?
Where might the quadratic aspect naturally arise?
}

\rmk{
Question~\ref{QUES:correlate-special-subvarieties-Manin-vs-F_q} may or may not be relevant to this question.
}

% \rmk{
% Less formally but more generally,
% it is natural to ask if the delta method can serve as
% a ``predictive telescope'' into the asymptotic behavior of points on Fano \emph{hypersurfaces},
% even when a fully rigorous analysis lies out of reach.
% In particular,
% can one always detect ``thin but special'' sets
% % (in the sense of the Hooley--Manin--Peyre philosophy)
% naturally via the delta method?

% (For \emph{general} Fano varieties,
% the delta method could still be relevant at least sometimes,
% but for simplicity we have restricted attention to Fano hypersurfaces.)
% % (The analogous questions about \emph{general} Fano varieties
% % may need to wait for a suitable analog of the delta method in various ``torsor'' settings.
% % But torsor methods
% % % are interesting in their own right---and
% % % https://www.ldoceonline.com/dictionary/a-life-of-its-own (if something has a life of its own, it exists and develops without depending on other things)
% % have a life of their own---and
% % % have succeeded in many instances of the Manin--Peyre conjecture---and
% % their development may or may not meet the delta method anytime soon.)
% }

\chapter{Discriminating pointwise estimates}
% Discriminating pointwise estimates in the delta method
\label{CHAP:discriminating-pointwise-estimates}

In Chapter~\ref{CHAP:using-mean-value-L-function-predictions}, we would like to ``remove the $\eps$ from \S\ref{SEC:using-L-function-hypotheses-on-average}'' (and go beyond it), at least conditionally.
In the present chapter, we will summarize the estimates from \cite{wang2021_HLH_vs_RMT} designed in response to Remarks~\ref{RMK:need-for-new-integral-decay-estimates} and~\ref{RMK:[B2']-intuition} towards Chapter~\ref{CHAP:using-mean-value-L-function-predictions}.
These estimates may well be related to
Igusa local zeta functions, model theory, o-minimality, real algebraic geometry, or resolution of singularities (in mixed characteristic).
But a consideration of such perspectives will have to wait for now.

Let $F,V,w,\dots$ be as in \S\ref{SEC:delta-method-setup}.
At least for some pairs $(F,w)$,
we will present new unconditional pointwise bounds on the oscillatory integrals $I_{\bm{c}}(n)$ and exponential sums $S_{\bm{c}}(n)$ (see Lemmas~\ref{LEM:uniform-[J1]--[J4]} and~\ref{LEM:bad-sum-vanishing-and-boundedness-criteria} below, respectively),
and new conditional average bounds on $S_{\bm{c}}(n)$ (see Conjecture~\ref{CNJ:bad-sum-average}~(B3) below, and the surrounding discussion).

\section{Pointwise integral estimates}

\defn{
\label{DEFN:decay-weight-nu}
Call $\nu_d\maps\RR^d\to\RR$ a \emph{decay weight}
if $\nu_d\in\mcal{S}(\RR^d)$
(i.e.~if $\nu_d$ is \emph{Schwartz}).
% INCLUDED in Schwartz: and the bound $\nu_d(\bm{a})\ll_b (1+\norm{\bm{a}})^{-b}$ holds for every real $b>0$.
%more general \nu\maps\RR\to\RR_{\geq0} (e.g.~continuous, or just integrable) should also be OK as long as integrability/summability is not an issue.
For such $\nu_d$,
write $\nu_d\geq0$ to mean $\im\nu_d\belongs\RR_{\geq0}$,
and $\nu_d>0$ to mean $\im\nu_d\belongs\RR_{>0}$.
}
\rmk{
We prefer the somewhat informal name ``decay weight'' because
while the regularity assumptions on $\nu$ are very convenient in \cite{wang2021_HLH_vs_RMT} (see e.g.~\cite{wang2021_HLH_vs_RMT}*{Remark~8.4, and the \Holder argument in \S9}),
they are not morally essential.
}

\defn{
Define $J_{\bm{c},X}(n)$ as in \cite{wang2021_HLH_vs_RMT}*{Definition~3.43};
in our setting, $J_{\bm{c},X}(n) = I_{\bm{c}}(n)$, but it will be helpful to keep the $X$-dependence explicit.
}

Given $\bm{c}\in\RR^m\setminus\set{\bm{0}}$,
let $\tilde{\bm{c}}\defeq\bm{c}/\norm{\bm{c}}$,
so that $\abs{F^\vee(\tilde{\bm{c}})}^{-1}\in[\Theta(1),\infty]$ measures the ``degeneracy'' of $(V_\RR)_{\bm{c}}$.
In the following statement, we only need $V$ to be smooth, not diagonal; in fact, we only need $V$ to be smooth along $V(\RR)$, not $V(\CC)$.
\lem{
[\cite{wang2021_HLH_vs_RMT}*{Lemma~4.5}]
\label{LEM:uniform-[J1]--[J4]}
Suppose $(F,w)$ is \emph{clean} in the sense of Definition~\ref{DEFN:support-smooth-clean}.
Then for some decay weight $\nu_m>0$,
\emph{fixed} in terms of $F,w$,
the following statements hold,
uniformly over $(X,\bm{c},n)\in\RR_{>0}\times\RR^m\times\RR_{>0}$:
\begin{enumerate}[{label=[$J$\arabic*]}]
    \item \emph{Modulus cutoff}:
    $J_{\bm{c},X}(n)=0$
    % holds uniformly over a certain range of the form $n\gg_{F,w} Y\defeq X^{3/2}$
    % (with implied constant depending only on $F,w$).
    holds unless $n\ll_{F,w}Y$.
    
    \item \emph{Integral bound}:
    If we interpret $\bm{c}/\abs{F^\vee(\tilde{\bm{c}})}^{-1}$ as $\bm{0}$ if $\bm{c}=\bm{0}$,
    then we \emph{always} have
    \mathd{
    \abs{J_{\bm{c},X}(n)}
    \ll_{F,w}
    \min\left[1,\left(\frac{X\norm{\bm{c}}}{n}\right)^{1-m/2}\right]
    \cdot \nu_m\left(\frac{\bm{c}}{X^{1/2}}\right)^3
    \cdot \nu_m\left(\frac{X\bm{c}}{n\abs{F^\vee(\tilde{\bm{c}})}^{-1}}\right).
    }
    
    \item \emph{Homogeneous} dimensional analysis:
    For each integer $j\geq0$,
    the same bound holds for $\abs{\partial_{\log{n}}^j J_{\bm{c},X}(n)}$,
    up to an $O_{F,w,j}(1)$ factor loss.
    (Here $\partial_{\log{n}}\defeq n\cdot\partial_n$.)
    
    \item \emph{Vertical} variation:
    In [$J$3],
    we can replace $J_{\bm{c},X}(n)$ with $\partial_{\log\bm{c}}^{\bm{\alpha}}J_{\bm{c},X}(n)$
    (where we define $\partial_{\log\bm{c}}^{\bm{\alpha}}\defeq\prod_{i\in[m]}\partial_{\log c_i}^{\alpha_i}$ for $\bm{\alpha}\in\ZZ_{\geq0}^m$),
    up to an \emph{additional} multiplicative loss of
    \mathd{
    O_{F,w,\bm{\alpha}}(1)
    \cdot \prod_{i\in[m]}\left(1+\frac{X\abs{c_i}}{n}\right)^{\alpha_i}.
    }
\end{enumerate}
}

\rmk{
\cites{heath1996new,heath1998circle} proved most of Lemma~\ref{LEM:uniform-[J1]--[J4]} \emph{except for}
[$J$2]--[$J$4]'s ``decay over $n\ll\norm{X\bm{c}/\abs{F^\vee(\tilde{\bm{c}})}^{-1}}$''.
(Hooley's ``pre-delta method'' works \cites{hooley1986HasseWeil,hooley_greaves_harman_huxley_1997} have a similar lacuna.)
}

\rmk{
\label{RMK:integral-estimates-without-cleanliness-assumption}
Without the cleanliness assumption,
similar but messier bounds in [$J$2]--[$J$4] might still hold,
at least for diagonal $F$
(cf.~Lemma~\ref{LEM:n-aspect-I_c(n)-estimates}),
and plausibly---at least in a weaker sense---even for general $F$
(using the the o-minimal geometric framework of \cite{basu2021stationary}).
It would be interesting, and likely useful or enlightening, to find such bounds for smooth pairs $(F,w)$, say, even for diagonal $F$.
However,
in order to focus on the most essential issues in this thesis,
we have decided to keep the
(qualitatively harmless)
cleanliness assumption on $(F,w)$.
}

% Before proceeding,
% we discuss a technical point:
The requirement $\nu_m\in\mcal{S}(\RR^m)$ in Lemma~\ref{LEM:uniform-[J1]--[J4]} (via Definition~\ref{DEFN:decay-weight-nu})
may seem unnaturally restrictive at first glance.
But in fact,
the following (surely folklore-type) result shows that
when \emph{proving} Lemma~\ref{LEM:uniform-[J1]--[J4]},
% https://www.merriam-webster.com/dictionary/without%20regard%20to or https://www.merriam-webster.com/dictionary/without%20regard%20for
it suffices (for [$J$2]--[$J$4]) to obtain ``decay bounds'' \emph{without regard to regularity}.
%Alternatively, if one is willing to restrict to ``textbook'' weights w (as Hooley often does), then one could always work with \exp(-c*r^e) type functions (which are clearly Schwartz...).
\prop{
\label{PROP:regularizing-decay-bounds}
Given an \emph{arbitrary} function $f\maps\RR^d\to\RR_{\geq0}$ such that
$f(\bm{a})\ll_b (1+\norm{\bm{a}})^{-b}$ (for all $b\in\ZZ_{\geq0}$),
there must exist a decay weight $\nu_d>0$ with $f\leq\nu_d$.
}

\pf{
See \cite{wang2021_HLH_vs_RMT}*{Proposition~4.7}.
}

Given Proposition~\ref{PROP:regularizing-decay-bounds}, let us end by summarizing some of the main ideas behind Lemma~\ref{LEM:uniform-[J1]--[J4]};
see \cite{wang2021_HLH_vs_RMT}*{Appendix~B} for details.

\pf{
[Partial proof sketch for Lemma~\ref{LEM:uniform-[J1]--[J4]}]
[$J$1]:
See Proposition~\ref{PROP:rigorous-modulus-cutoff-Y}.

[$J$3]:
This is based on a recursive structure proven by an integration by parts argument of \cite{heath1996new} (using the homogeneity of $F=0$).

[$J$2]:
This involves a lot of stationary phase;
the assumption that $(F,w)$ is clean makes the analysis much cleaner (allowing conceptual arguments based on gradient descent, for instance).
% To uniformly handle the necessary derivatives,
Let $r\defeq n/Y$ and $\bm{v}\defeq X\bm{c}/n$; then it suffices to bound, suitably uniformly over Schwartz functions $q,\phi$, the \emph{$(q\circ r,\phi)_{r,\bm{v}}$-integral}
\mathd{
\int_{u\in\RR}du\;q(ru)\int_{\bm{x}\in\RR^m}d\bm{x}\;\phi(\bm{x})\ub{e(uF(\bm{x})-\bm{v}\cdot\bm{x})}_{e(\psi(\bm{x}))},
}
% for arbitrary Schwartz $q,\phi$, uniformly in terms of certain norms involving $q,\phi$,
where $\Supp\phi\belongs\Supp{w}$.
% Up to norms involving $q,\phi$, we want a bound of
% % the $(q\circ r,\phi)$ integral by
% \mathd{
% O_{k,q,\phi,F}(1)
% \cdot \min(1,\norm{\bm{v}}^{1-m/2})
% \cdot (1+\norm{r\bm{v}})^{-k}(1+\norm{F^\vee(\tilde{\bm{v}})\bm{v}})^{-k}.
% }
% Let $\psi(\bm{x})\defeq uF(\bm{x})-\bm{v}\cdot\bm{x}$.
% Often either the trivial bound or the principle of \emph{non-stationary phase} (i.e. integration by parts if $\grad{\psi}\gg \max(\abs{u},\norm{\bm{v}})\gg 1$) suffices.
If $\grad{\psi}(\bm{x})=u\grad{F}(\bm{x})-\bm{v}\gg\abs{u}+\norm{\bm{v}}\gg 1$, then the principle of \emph{non-stationary phase} suffices.
In many other ranges, the trivial bound suffices.
The critical range occurs when $\abs{u}\asymp_{F,w}\norm{\bm{v}}\gg 1$, with $\bm{x}\in\Supp{w}$ restricted to $\norm{\grad{\psi}(\bm{x})}\leq\delta\abs{u}$.
In this setting, fix $u,\bm{v}$.
Locally over $\bm{x}$,
% by gradient descent over $\bm{x}$,
there exists a unique $\bm{s}\approx\bm{x}$ with $\grad{\psi}(\bm{s})=\bm{0}$.
In fact $\bm{s}$ is \emph{Morse}, i.e.~if $\bm{y}=\bm{x}-\bm{s}$ then $\psi(\bm{x})-\psi(\bm{s})\approx uQ_{\bm{s}}(\bm{y})\in\map{Sym}^2(\RR^m)^\vee$ looks non-degenerate.
% Fix $u$.
% , and write $e(\psi(\bm{x}))\approx e(\psi(\bm{s}))\cdot e(uQ_{\bm{s}}(\bm{y}))$.
% and let $\lambda=2\pi u$.
Then, by stationary phase expansion,
\mathds{
\int_{\bm{x}\in\RR^m}d\bm{x}\;\phi(\bm{x})e(\psi(\bm{x}))
% \mscr{I}(\varphi,\lambda,Q)
% &\defeq
% \int_{\bm{y}\in\RR^m}d\bm{y}\;\varphi(\bm{y})e^{i\lambda Q(\bm{y})}
&\approx e(\psi(\bm{s}))
\cdot\int_{\bm{y}\in\RR^m}d\bm{y}\;\phi(\bm{s}+\bm{y})e(uQ(\bm{y})) \\
% &= i^{\sgn(\lambda Q)/2}\abs{\pi/\lambda}^{m/2}\abs{\det{Q}}^{-1/2}\int_{\RR^m}d\bm{\xi}\;\hat{\varphi}(\bm{\xi})e^{-i\pi^2Q^\vee(\bm{\xi})/\lambda} \\
% &\approx i^{\sgn(\lambda Q)/2}\abs{\pi/\lambda}^{m/2}\abs{\det{Q}}^{-1/2}\int_{\RR^m}d\bm{\xi}\;\hat{\varphi}(\bm{\xi}) \\
&\approx e(\psi(\bm{s}))
\cdot\frac{i^{\sgn(uQ)/2}\phi(\bm{s})}{\abs{2u}^{m/2}\abs{\det{Q}}^{1/2}},
}
roughly speaking.
(One heuristic for the growth rate $\abs{u}^{-m/2}$ as $\abs{u}\to \infty$ is to compare the integral over $\bm{y}$ with the discrete Gauss-like sum $\abs{u}^{-m}\sum_{\ZZ^m\ni u\bm{y}\ll u}e(Q(u\bm{y})/u)$, which one might expect to exhibit square-root cancellation over $\ZZ^m\ni u\bm{y}\ll u$.)
Hooley and Heath-Brown integrated $\abs{u}^{-m/2}$ absolutely over $\abs{u}\asymp \norm{\bm{v}}$.
But there is really a factor of
$e(\psi(\bm{s}))$ in front, so by a calculation using non-stationary phase over $\abs{u}\asymp \norm{\bm{v}}$ (as $\bm{s}$ varies with $u$), decay occurs if $\abs{F^\vee(\tilde{\bm{v}})\bm{v}}\gg 1$ is large.
}

% [insert some stuff from Simplifying the oscillatory integral factor in the delta method]

% The critical range (even for non-clean, non-separable weights $w$) seems to be $\abs{u}\asymp \norm{\bm{v}}\gg 1$.
% (Rough goal: we  want decay as ``$\norm{\bm{v}}\to \infty$''.)
% In this range, we might hope to localize to $\abs{x_i-s_i}\ll \delta \abs{s_i}$ (if $s_i\in \RR$), outside an integral with decay $O_{\delta,N}(\max(1, \abs{v_i}/\abs{u}^{1/3})^{-N})$ in the $i$th factor.
% Idea (?): The localization is helpful unless $\abs{v_i}$ is small (the cutoff seems to be $\abs{v_i}\asymp 1$).
% Furthermore, if $\abs{v_i}\ll 1$, then one might be able to localize $x_i$ to $3ux_i^2\ll 1$.
% So we should get a good bound by
% using the localization over some set $S\belongs [m]$ of indices,
% doing all the stationary phase expansion over these indices,
% then integrating by parts over $u$ (this includes all indices, even those not in $S$; so we'll be considering things like $e(u\cdot (\sum_{i\in S} s_i^3 + \sum_{i\notin S} x_i^3))$?),
% then using upper bounds over the $x_i$-integrals for $i\notin S$,
% then finally integrating over $u$,
% and optimizing the final bound over $S$.

% [if we can work this out integral bound, then maybe a similar bound holds for $S(n)$; also it may provide inspiration for non-diagonal case]

\section{Pointwise sum estimates}

Assume $F$ is diagonal and $m\geq 4$.
Write $F = F_1x_1^3+\dots+F_mx_m^3$, with $F_1,\dots,F_m\in \ZZ\setminus \set{0}$.
For technical convenience and discussion, we make the following definition:
\defn{
For each nonempty set $U\belongs[m]$,
choose $F_U^\vee\in\QQ^\times\cdot\map{rad}(F^\vee\vert_{c_j=0,\;j\notin U})$ with $F_U^\vee\in \ZZ[c_i]_{i\in U}$.
}

\ex{
If $F=x_1^3+\dots+x_m^3$,
then $F_{\set{1,2}}^\vee\propto\prod(c_1^{3/2}\pm c_2^{3/2})=c_1^3-c_2^3$,
while $F_{\set{1}}^\vee\propto c_1$.
(Cf.~the factorization \eqref{EXPR:diagonal-discriminant-factorization} in \S\ref{SUBSEC:diagonal-dual-example}.)
}

Recall the general pointwise bound Proposition~\ref{PROP:pointwise-bound} on $\wt{S}_{\bm{c}}(n)$.
By taking $F^\vee(\bm{c})$ into account, we can sometimes do better.
For convenience, let $\bm{F}\defeq (F_1,\dots,F_m)$ and $\lcm(\bm{F})\defeq \lcm(F_1,\dots,F_m)$.
The following is essentially \cite{wang2021_HLH_vs_RMT}*{Lemma~4.11}:
\lem{
\label{LEM:bad-sum-vanishing-and-boundedness-criteria}
Assume $\lcm(\bm{F})$ is cube-free.\footnote{This assumption can surely be removed with more work, but it is extremely mild anyways.}
Up to scaling $F^\vee,F_U^\vee$ once and for all,
the following \emph{vanishing and boundedness criteria} hold
uniformly over $p$:
\begin{enumerate}[(1)]
    \item If $\bm{c}\in\ZZ_p^m$ and $v_p(F^\vee(\bm{c}))\leq1$,
    then $\abs{\wt{S}_{\bm{c}}(p)}\ll_{F}1$.
    
    \item If $\bm{c}\in\ZZ_p^m$ and $v_p(F^\vee(\bm{c}))\leq1$,
    then $\abs{\wt{S}_{\bm{c}}(p^2)}\ll_{F}1$.
    
    \item If $\bm{c}\in\ZZ_p^m$,
    then $S_{\bm{c}}(p^l)=0$ for all integers $l\geq 2+v_p(F^\vee(\bm{c}))$.
    
    \item If $U\belongs[m]$ and $\card{U}\geq 2$,
    then (2)--(3) hold for $\bm{c}\in\ZZ_p^U\times\set{0}^{[m]\setminus U}$,
    \emph{even if} we ``replace'' $F^\vee$
    with $F_U^\vee\in\ZZ[\bm{c}\vert_U]$.
    % The analogous ``modified (1)'' may or may not be true.
    
    \item If $i\in[m]$ and $U\defeq\set{i}$,
    then (2)--(3) hold for $\bm{c}\in\ZZ_p^U\times\set{0}^{[m]\setminus U}$,
    \emph{even if} we ``replace'' $v_p(F^\vee(\bm{c}))$
    with $\frac32v_p(F_U^\vee(c_i))-\frac12\cdot\bm{1}_{l=3}$.
    (Here,
    when ``modifying'' (2),
    we let $l\defeq2$,
    so that $\bm{1}_{l=3}=0$.
    The $\bm{1}_{l=3}$ only plays a role when ``modifying'' (3).)
\end{enumerate}
}

\pf{
[Proof sketch]
For (1), use Theorem~\ref{THM:codimension-2-criteria-for-boundedness-of-E_c}(1) and \cite{poonen2020valuation}*{Theorem~1.1}.

The proofs of (2)--(5), especially (3)--(5), are much more technical; the main ingredient is Theorem~\ref{THM:polar-map-fiber-criteria-for-vanishing-and-boundedness-of-S_c(p^l)} below, proven in \cite{wang2021_HLH_vs_RMT}*{Appendix~D}.
Essentially one does a lot of Hensel lifting, and needs to design a suitable ``step-by-step ladder'' into increasingly deep ``approximate singularities'' of $(V_{\QQ_p})_{\bm{c}}$.
Unfortunately,
despite the simplicity of the idea,
the full argument involves many clumsy preparations
(designed to align ourselves in an ``efficient direction'' at each ``ladder step'').
}

\rmk{
\label{RMK:failure-of-prime-boundedness-criteria-for-quadratic-F}
If $F$ were quadratic (rather than cubic),
with $m\geq4$ \emph{even},
then Lemma~\ref{LEM:bad-sum-vanishing-and-boundedness-criteria}(1) would be false.
This is because in general,
if $p\neq2$,
then every rank-$r$ quadratic form $Q/\FF_p$ in $s\geq3$ variables has
$\#Q(\FF_p)=\#\PP^{s-2}(\FF_p)\pm p^{(s-r)/2}\bm{1}_{2\mid r}\cdot p^{(s-2)/2}$;
cf.~Proposition~\ref{PROP:dichotomy-for-quadrics}.
}

On a first reading of the following statement, it may help to focus on the special but ``general'' case $v_p(\lcm(6\bm{F}))=v_p(\bm{c})=0$.
\thm{
[\cite{wang2021_HLH_vs_RMT}*{Theorem~D.1}]
\label{THM:polar-map-fiber-criteria-for-vanishing-and-boundedness-of-S_c(p^l)}
Fix $o\in\set{0,1}$.
Fix $\bm{c}\in\ZZ_p^m$ and $l\in\ZZ$.
Assume the following:
\begin{enumerate}[(1)]
    \item $v_p(\lcm(\bm{F}))\leq 2$;
    
    \item $l\geq 2+10v_p(\lcm(3\bm{F}))+8v_p(4)+\frac32v_p(\bm{c})$;
    and
    
    \item $l-2-v_p(48)+o\geq v_\wp(F(\bm{z}))$
    for all tuples $(\lambda,K,\bm{z},v_\wp)$ consisting of
    a unit $\lambda\in\ZZ_p^\times$,
    a finite extension $K/\QQ_p$,
    a solution $\bm{z}\in K^m$ to $\grad{F}(\bm{z})=\lambda\bm{c}$,
    and a valuation $v_\wp\maps K^\times\to\QQ$
    % extending $v_p\maps\QQ_p^\times\to\ZZ$
    normalized by $v_\wp(p)=v_p(p)=1$.
\end{enumerate}
Then $\bm{1}_{o=0}\cdot S_{\bm{c}}(p^l)=0$.
Also, $\bm{1}_{(o,l)=(1,2)}\cdot \abs{S_{\bm{c}}(p^l)}\leq 2^{m-1}p^{l(1+m)/2}$.
}

\rmk{
Theorem~\ref{THM:polar-map-fiber-criteria-for-vanishing-and-boundedness-of-S_c(p^l)} simplifies greatly
when $F$ is $\PP^{m-1}_{\FF_p}$-smooth,
i.e.~$p\nmid3\lcm(\bm{F})$.
\emph{If} the simplified statement extends directly (without change) to general $\PP^{m-1}_{\FF_p}$-smooth cubic forms $F/\ZZ_p$,
a different proof would be needed
(maybe less explicit,
and more geometric or model-theoretic).
It might then take even more work to find a reasonable statement
(and proof)
for general $\PP^{m-1}_{\QQ_p}$-smooth cubic forms $F/\ZZ_p$.
}

\rmk{
% [Weaker results with easier proofs]
If one strengthened the assumption ``$l\geq v_\wp(F(\bm{z}))+O(1)$'' to ``$l\geq 2v_\wp(F(\bm{z}))+O(1)$'',
then Theorem~\ref{THM:polar-map-fiber-criteria-for-vanishing-and-boundedness-of-S_c(p^l)} would likely be
(i) much easier to prove:
it might suffice to use \cite{wang2021_HLH_vs_RMT}*{Proposition~D.21}, with $d=\frac12l-O(1)$;
and (ii) still satisfactory (for our purposes),
when complemented with suitable results for $l\leq O(1)$;
but (iii) messier, and less enlightening.
}

\rmk{
% [Algorithm to compute $S_{\bm{c}}$]
% Why bother proving the stronger results?
% They give us more information.
One could likely adapt the proof of Theorem~\ref{THM:polar-map-fiber-criteria-for-vanishing-and-boundedness-of-S_c(p^l)} (see \cite{wang2021_HLH_vs_RMT}*{Appendix~D}) to obtain a fairly efficient algorithm to compute $(S_{\bm{c}}(p^l))_{l\geq2}$,
at least when $p\nmid6\lcm(\bm{F})$.\footnote{It would be nice to numerically verify
Theorem~\ref{THM:polar-map-fiber-criteria-for-vanishing-and-boundedness-of-S_c(p^l)} (or at least some of its proof ingredients).}
Such work might also eventually help to improve, or at least clarify,
existing bounds on $S_{\bm{c}}(p^l)$ (e.g.~those from \cites{hooley1986HasseWeil,hooley_greaves_harman_huxley_1997,heath1998circle} that ``resort to diagonality''),
either for diagonal $F$ or in general.
}

\rmk{
A similar result should hold without the assumption $v_p(\lcm(\bm{F}))\leq 2$;
see \cite{wang2021_HLH_vs_RMT}*{Remark~D.12} for a possible first step towards such an extension
(under the present approach).
In the same vein,
there might be a variant of Theorem~\ref{THM:polar-map-fiber-criteria-for-vanishing-and-boundedness-of-S_c(p^l)} suited for weak-approximation questions.
But for practical reasons,
we have restricted ourselves to Theorem~\ref{THM:polar-map-fiber-criteria-for-vanishing-and-boundedness-of-S_c(p^l)}.
}

The assumption $v_p(\lcm(\bm{F}))\leq 2$ essentially lets us ``cleanly anchor ourselves'' to certain ``minimally degenerate'' indices.
The relevant ``clean initial combinatorics'' is captured by Observation~\ref{OBS:implications-of-cube-free-F_i-assumption} below.
\newcommand{\rootset}{R}
\newcommand{\splitset}{U}
% https://tex.stackexchange.com/questions/546004/is-it-alright-to-use-newcommand-inside-of-the-body-of-a-document
\defn{
\label{DEFN:minimally-degenerate-indices-S_min}
% If $v_p(c_i^3/F_i)\geq v_p(c_m^3/F_m)$ for all $i\in[m]$,
% then let $\splitset_{\min}\defeq\set{i\in[m-1]
% : v_p(c_i^3/F_i)=v_p(c_m^3/F_m)}$.
Given $\bm{c}\in\ZZ_p^m$,
let $\rootset_{\min}\defeq\map*{arg\,min}_{i\in[m]}v_p(c_i^3/F_i)$,
% https://tex.stackexchange.com/questions/5223/command-for-argmin-or-argmax and https://tex.stackexchange.com/questions/389038/declaremathoperator-vs-operatorname suggest using \map* to get the \lim like subscript behavior (though for inline math like here, it doesn't really matter)
and if $m\in R_{\min}$,
then let $\splitset_{\min}\defeq \rootset_{\min}\setminus\set{m}$.
% Note: R - rooted set (i.e. including "root" m); S - split set (see later); U - unrooted/unramified set. Avoid assigning \rootset,\splitset the same letter, since later below \splitset always denotes a subset of $[m-1]$.
}

\obs{
% [Implications of cube-free assumption]
\label{OBS:implications-of-cube-free-F_i-assumption}
Fix $\bm{c}\in\ZZ_p^m$.
After permuting $[m]$ if necessary,
\emph{assume} $v_p(c_1^3/F_1)\geq\dots\geq v_p(c_m^3/F_m)$.
Now fix $i,j\in[m]$ with $i\geq j$.
Then $v_p(\lcm(\bm{F}))\leq2$ implies
\begin{enumerate}[(1)]
    \item $v_p(c_i)\geq v_p(c_j)$;
    
    \item $\splitset_{\min}=\set{k\in[m-1] : (v_p(c_k),v_p(F_k))=(v_p(c_m),v_p(F_m))}$;
    and
    
    \item if $v_p(c_i)-v_p(c_j)\in\set{0}\cup[2,\infty]$ or $2\mid v_p(c_i/F_i)-v_p(c_j/F_j)$,
    then $v_p(c_i/F_i)\geq v_p(c_j/F_j)$.
\end{enumerate}
}

\pf{
Suppose we write the integers $2+v_p(c_i^3/F_i)=3v_p(c_i)+[2-v_p(F_i)]$ in base $3$,
noting that $2-v_p(F_i)\in\set{0,1,2}$.
Then we find that ``$v_p(c_i^3/F_i)\geq v_p(c_j^3/F_j)$'' is equivalent to
``$(v_p(c_i),-v_p(F_i))\geq (v_p(c_j),-v_p(F_j))$ under the lexicographic ordering of $\ZZ^2$''.

Now fix $i,j$ with $i\geq j$.
The lexicographic ordering immediately implies (1)--(2),
as well as the case ``$v_p(c_i)=v_p(c_j)$'' of (3).
Now assume $v_p(c_i)\neq v_p(c_j)$.
Then $v_p(c_i/F_i)-v_p(c_j/F_j)
\geq v_p(c_i)-v_p(c_j)-2
\geq -1+\bm{1}_{v_p(c_i)\geq v_p(c_j)+2}$,
which yields the remaining cases of (3).
% whether $v_p(c_i)-v_p(c_j)\geq 2$ or $2\mid v_p(c_i/F_i)-v_p(c_j/F_j)$.
}

\section{Conditional average sum estimates}
\label{SEC:SFSC-and-Sarnak-Xue-type-average-sum-bound}

Let us state some quantitative forms of the Square-free Sieve Conjecture (cf.~\cite{miller2004one}*{p.~956} and \cites{granville1998abc,poonen2003squarefree,bhargava2014geometric})---restricted to a certain range (specified in terms of a real exponent $b\geq 0$)---regarding ``unlikely divisors'' of certain polynomial outputs.
Function field analogs of these statements (for arbitrary $b\geq 0$) are, at least up to literature gaps, known unconditionally;
see \cite{poonen2003squarefree}*{Lemma~7.1} for a general qualitative statement for multivariate polynomials,
and \cite{lando2015square}*{Propositions~3.2 and~3.3} for a quantification for univariate polynomials (which can surely be extended to multivariate polynomials by adapting \cite{bhargava2014geometric}*{Theorem~3.3} to the function field setting).

\cnj{
[SFSC$_{p,b}$]
There exists an absolute $\delta>0$ such that
\mathd{
\#\set{\bm{c}\in\ZZ^m\cap[-Z,Z]^m
:\exists\;\textnormal{a prime $p\in\ZZ\cap[P,2P]$ with $p^2\mid F^\vee(\bm{c})$}}
\ll_{F^\vee}Z^mP^{-\delta}
}
holds uniformly over $Z\geq 1$ and $P\leq Z^{b/2}$.
% Here $\bm{c}$ ranges over \emph{all tuples} in $\ZZ^m$ with $\norm{\bm{c}}\leq Z$.
}

\cnj{
[SFSC$_{q,b}$]
There exists an absolute $\delta>0$ such that
\mathd{
\#\set{\bm{c}\in\ZZ^m\cap[-Z,Z]^m
:\exists\;\textnormal{a square-full $q\in\ZZ\cap[Q,2Q]$ with $q\mid F^\vee(\bm{c})$}}
\ll_{F^\vee}Z^mQ^{-\delta}
}
holds uniformly over $Z\geq 1$ and $Q\leq Z^b$.
% Here $\bm{c}$ ranges over \emph{all tuples} in $\ZZ^m$ with $\norm{\bm{c}}\leq Z$.
}

\rmk{
Note that (SFSC$_{p,b}$) is stable under scaling $F^\vee$.
Also,
if (SFSC$_{p,b}$) holds for $P\leq Z^{b/2}$,
then it extends (up to modified implied constant) to any \emph{fixed} range of the form $P\ll Z^{b/2}$.
Similar remarks hold for (SFSC$_{q,b}$).
}

\rmk{
It can (probably) be shown that (SFSC$_{p,b}$) implies (SFSC$_{q,b}$), up to changing $\delta$.
The point is that (in the difficult case $b>1$) a square-full integer $q\in [Z, Z^b]$ is divisible by a square $q'\in [Z^{2/3}, Z^b]$, which in turn either has a prime-squared factor $p^2\in [Z^{1/3}, Z^b]$, or a square factor $d^2\in [Z^{1/3}, Z^{2/3}]$ (with $d$ composite).
}

In view of Lemma~\ref{LEM:bad-sum-vanishing-and-boundedness-criteria}(4)--(5), it is also natural to consider the following augmented versions of (SFSC$_{p,b}$) and (SFSC$_{q,b}$).

\cnj{
[SFSC$_{p,b}$+]
If $U\belongs[m]$ is nonempty,
then the analog of (SFSC$_{p,b}$) holds for $F_U^\vee$ over $(c_i)_{i\in U}\in\ZZ^U\cap[-Z,Z]^U$.
In particular, (SFSC$_{p,b}$) holds.
}

\cnj{
[SFSC$_{q,b}$+]
If $U\belongs[m]$ is nonempty,
then the analog of (SFSC$_{q,b}$) holds for $F_U^\vee$ over $(c_i)_{i\in U}\in\ZZ^U\cap[-Z,Z]^U$.
In particular, (SFSC$_{q,b}$) holds.
}

Proposition~\ref{PROP:pointwise-bound},
Lemma~\ref{LEM:bad-sum-vanishing-and-boundedness-criteria},
and (SFSC$_{q,6}$+)
together imply the following statement, at least when $F$ is diagonal and $\lcm(\bm{F})$ is cube-free:
\cnj{
[B3]
\label{CNJ:bad-sum-average}
Let $F/\ZZ$ be a $\PP^{m-1}_\QQ$-smooth cubic form in $m\geq 4$ variables.
Restrict to $F^\vee(\bm{c})\neq 0$.
Then there exists an absolute $\delta>0$ such that
uniformly over $Z>0$ and $N\leq Z^3$,
we have
\mathd{
\psum_{\bm{c}\in\ZZ^m}
\bm{1}_{\norm{\bm{c}}\leq Z}
\left(
\sum_{n_{\bm{c}}\in [N,2N]}
\frac{\abs{\wt{S}_{\bm{c}}(n_{\bm{c}})}}{N^{1/2}}
\cdot\bm{1}_{n_{\bm{c}}\mid F^\vee(\bm{c})^\infty}
\right)^2
% \sum_{n_{\bm{c}}\in [N,2N]}(n_{\bm{c}}^{-1}\abs{\wt{S}_{\bm{c}}(n_{\bm{c}})}^2\cdot \bm{1}_{n_{\bm{c}}\mid F^\vee(\bm{c})^\infty})
% \cdot \max_{n_{\bm{c}}\in [N,2N]}(n_{\bm{c}}^{-1}\abs{\wt{S}_{\bm{c}}(n_{\bm{c}})}^2\cdot \bm{1}_{n_{\bm{c}}\mid F^\vee(\bm{c})^\infty})
\ll_{F}Z^mN^{-\delta}.
}
}

For the (conditional) proof, see \cite{wang2021_HLH_vs_RMT}*{Lemma~7.43};
in certain key ranges, the $N$-power saving in (B3) comes from Lemma~\ref{LEM:bad-sum-vanishing-and-boundedness-criteria} and (SFSC$_{q,6}$+).
For a discussion of our reliance on (SFSC$_{q,6}$+), see e.g.~Remark~\ref{RMK:Ekedahl-geometric-sieve-reduce-range?} below.

\rmk{
``B'' refers to the ``badness'' of $n_{\bm{c}}$ and $S_{\bm{c}}(n_{\bm{c}})$; cf.~\S\ref{SEC:using-L-function-hypotheses-on-average}'s [B2'], as discussed in Remark~\ref{RMK:[B2']-intuition}.
For each fixed $\bm{c}$, the sum over $n_{\bm{c}}$ is quite sparse, so there is not much difference between $\sum$ and $\max$, but for small $N$ the statement with $\sum$ rather than $\max$ is slightly stronger.
}

\rmk{
\label{RMK:failure-of-(B3)-for-quadratic-F}
If $F$ were \emph{quadratic},
then (B3) would be false in general.
One way to see this is to
take $N$ to be a small power of $Z$,
restrict $n_{\bm{c}}$ to be prime,
and apply Remark~\ref{RMK:failure-of-prime-boundedness-criteria-for-quadratic-F}.
}

\rmk{
For us,
the second moment---as stated above---is most convenient.\footnote{Note that once we have a power saving
for the second moment,
we also have---due to the trivial bound on $S_{\bm{c}}$---a power saving
for all moments with exponent in $[2,2+\Omega(\delta)]$.}
However,
the first moment---or anything higher---could still be useful.
Also note that while a bound of $Z^{m+\eps}N^\eps$
(in place of $Z^mN^{-\delta}$ in (B3))
follows from Proposition~\ref{PROP:pointwise-bound} for diagonal $F$
(cf.~[B2'] in \S\ref{SEC:using-L-function-hypotheses-on-average}),
no comparable result is known for general $F$
(a significant source of difficulty in work such as \cite{hooley2014octonary});
see the discussion in \cite{wang2021_large_sieve_diagonal_cubic_forms}*{Appendix~A} for more details.
}

\rmk{
It would be good to understand the restriction $N\leq Z^3$ at a deeper level;
is it really needed?
Our conditional \emph{proof} of (B3)
morally only needs the restriction over the locus $c_1\cdots c_m=0$.
(However, in the proof,
the restriction $N\leq Z^3$
does help at least at a technical level, by letting us get away with the limited range of (SFSC$_{q,6}$+).)
The full \emph{truth} seems unclear.
}

\rmk{
In principle,
in our \emph{applications} of (B3),
it would probably be OK to ``drop'' the absolute value from $\abs{\wt{S}_{\bm{c}}(n_{\bm{c}})}$---provided we assume the corresponding statement holds for all sub-intervals of $[N,2N]$,
and not just $[N,2N]$ itself.
}

Now we discuss some possible approaches to weakening the hypothesis (SFSC$_{q,6}$+).

\rmk{
\label{RMK:Ekedahl-geometric-sieve-reduce-range?}
I expect that if instead of Lemma~\ref{LEM:bad-sum-vanishing-and-boundedness-criteria}(1), one uses Theorem~\ref{THM:codimension-2-criteria-for-boundedness-of-E_c}(2), then a quantitative forms of Ekedahl's geometric sieve (cf.~\cite{bhargava2014geometric}*{Theorem~3.3}) would allow one to replace the ingredient (SFSC$_{q,6}$+) with (SFSC$_{q,3}$+) (and thus with (SFSC$_{p,3}$+)).
Furthermore, if one restricts (B3) to $F^\vee(\bm{c})\cdot c_1\cdots c_m\neq 0$ and not just $F^\vee(\bm{c})\neq 0$, then I believe (SFSC$_{q,6}$) (and thus (SFSC$_{q,3}$) or (SFSC$_{p,3}$)) would suffice;
in fact, this further-restricted version of (B3), say (B3G), would suffice for the main goals of Chapter~\ref{CHAP:using-mean-value-L-function-predictions}.
}

\rmk{
Going beyond Remark~\ref{RMK:Ekedahl-geometric-sieve-reduce-range?},
say one uses Theorem~\ref{THM:polar-map-fiber-criteria-for-vanishing-and-boundedness-of-S_c(p^l)} (or any improvement thereof)
in place of Lemma~\ref{LEM:bad-sum-vanishing-and-boundedness-criteria}(2)--(3).
Then ultimately, one might be left with problems
such as Problem~\ref{PROB:toy-SFSC-replacement-problem} below (to give a toy example).
}

\prob{
\label{PROB:toy-SFSC-replacement-problem}
Show that uniformly over $Z\geq1$ and $Q\leq Z^3$,
the count
\mathd{
\#\set{\bm{c}\ll Z
: \exists\;p\asymp Q^{1/2}
\;\textnormal{and}\;
(\bm{x},\lambda)\in\ZZ_p^m\times\ZZ_p^\times
\;\textnormal{with}\;
\grad{F}(\bm{x})=\lambda\bm{c}
\;\textnormal{and}\;
p^2\mid F(\bm{x})}
}
is $\ll_{F} Z^mQ^{-\delta}$.
(Here $\bm{c}$ denotes a tuple in $\ZZ^m$,
and $p$ denotes a prime.)
}

\rmk{
While a direct use of
Theorem~\ref{THM:polar-map-fiber-criteria-for-vanishing-and-boundedness-of-S_c(p^l)} (with $(o,l)=(1,2)$)
to control large values of $\wt{S}_{\bm{c}}(p^2)$
only ``guarantees'' $\bm{x}\in\ol{\QQ}_p^m$,
a closer inspection of the proof of Theorem~\ref{THM:polar-map-fiber-criteria-for-vanishing-and-boundedness-of-S_c(p^l)}
shows, under mild hypotheses,
that $S_{\bm{c}}(p^2)=0$ unless
there exists $(\bm{x},\lambda)\in\ZZ_p^m\times\ZZ_p^\times$ with $\grad{F}(\bm{x})\equiv\lambda\bm{c}\bmod{p}$---a condition
not so far from the toy problem's condition $\grad{F}(\bm{x})=\lambda\bm{c}$.
(With more care when $p\mid c_1\cdots c_m$,
we could state a more precise toy problem---but
at least when $Q\gg Z^2$ is sufficiently large,
this would not make a difference.)
}

\rmk{
The toy problem above vaguely resembles \cite{kowalski2020proportion},
but with an additional ``multi-quadratic cover'' aspect due to $\grad{F}$.
Optimistically,
one might try to combine \cite{kowalski2020proportion} with techniques for ``Weyl sums for square roots'' developed by \cite{duke2012weyl} and others.
}

\rmk{
% Mention Jutila (possibly restrict to square-free moduli? or moduli coprime to a given integer - maybe useful for inhomogeneous quadratic forms for instance)
As another approach to simplifying the work or input required,
one might try combining Jutila's ``approximate circle method'' with \cites{duke1993bounds,heath1996new},
in the hope of ``reducing'' the delta method to
``simple'' moduli $n$ of our choice
(e.g.~square-free $n$ coprime to $6F_1\cdots F_m$).
This may or may not be possible;
the inherent ``error bound'' in Jutila's circle method
may or may not be too large for our purposes in \cites{wang2021_isolating_special_solutions,wang2021_HLH_vs_RMT}.
}

\chapter{Using the Ratios Conjectures}
% Using mean-value \texpdf{$L$}{L}-function predictions
\label{CHAP:using-mean-value-L-function-predictions}

\section{Introduction}

Let $F,V,w,\dots$ be as in \S\ref{SEC:delta-method-setup}.
Assume $m=6$ and $F=x_1^3+\dots+x_6^3$, as in \S\ref{SEC:using-L-function-hypotheses-on-average}.
Under certain standard number-theoretic hypotheses,
the paper \cite{wang2021_HLH_vs_RMT}---to be summarized in the present chapter---``removes the $\eps$ from \S\ref{SEC:using-L-function-hypotheses-on-average}'' (and goes beyond it, proving HLH when $(F,w)$ is clean);
this has statistical implications for sums of three cubes, by Observation~\ref{OBS:3Z^odot3-lower-bound-by-Cauchy} and Theorem~\ref{THM:enough-HLH-implies-100pct-Hasse-principle-for-3Z^odot3}.
The following hypotheses suffice:
\begin{enumerate}[(1)]
    \item some Langlands-type conjectures
    % some parts of the (modern?) Langlands program
    % Langlands reciprocity (mostly) and functoriality (maybe a little bit, at least if we want to take the general Sarnak-Shin-Templier route (to avoid assuming/using cases of Tate's global semi-simplicity conjecture))
    (applied to certain Galois representations ``coming from geometry''),
    plus GRH
    (for certain automorphic $L$-functions);
    
    \item some RMT-type predictions
    (based on \cite{conrey2008autocorrelation}*{\S5.1, (5.6)} and \cite{sarnak2016families})---namely
    % certain predictions of Random Matrix Theory (RMT) type---namely / in essence,
    the $L$-function Ratios Conjectures over the \emph{multi-parameter} geometric family $\bm{c}\mapsto L(s,V_{\bm{c}})$ (with $L(s,V_{\bm{c}})$ defined as in Example~\ref{EX:one-sided-first-order-approximations-of-Phi} and \S\ref{SEC:using-L-function-hypotheses-on-average});
    %allowing adelic perturbation for the $100\%$ result.
    
    \item a quantitative form of the Square-free Sieve Conjecture (see \S\ref{SEC:SFSC-and-Sarnak-Xue-type-average-sum-bound}); and
    
    \item the constancy of the local Hasse--Weil Euler factors $L_p(s,V_{\bm{c}})$,
    when $\bm{c}$ is restricted to a $p$-adic residue class
    % Note: a $p$-adic residue class can also be thought of as a p$-adic box (not a p$-adic $\ell^2$-ball, though).
    of modulus $p\cdot \gcd(F^\vee(\bm{c})^{O(1)}, p^\infty)$---in the spirit of Krasner's lemma (cf.~the general but possibly ``ineffective'' results of \cite{kisin1999local}).
\end{enumerate}

\rmk{
Under the current state of knowledge,
we expect (1)--(2) to be the most serious hypotheses;
(4) to be provable, and thus removable, by a suitable expert;
and (3) to be somewhere in between.
}

In \S\ref{SEC:precise-standard-conjectural-hypotheses},
we will state more precise hypotheses,
especially regarding (1)--(2).
For now,
we make six
(inessential) remarks on hypotheses~(1)--(4).

\rmk{
Very roughly,
to improve on \S\ref{SEC:using-L-function-hypotheses-on-average},
% ---which lie fairly close to the ``boundary of producing points''---
the paper \cite{wang2021_HLH_vs_RMT} ``factors''
certain key sums into pieces addressed by
(2),
from the world of ``RMT-based heuristics'',
% originating from RMT,
or by (3),
from the world of ``unlikely divisors'' (see especially Conjecture~\ref{CNJ:bad-sum-average}~(B3) and the surrounding discussion).
}

\rmk{
It might be worthwhile to find a suitable ``elementary'' replacement for the Ratios Conjectures
whose formulation would not require analytic continuation or GRH.
But \cite{conrey2008autocorrelation}*{\S5.1, (5.6)} itself is stated directly in terms of $L$ and $1/L$.
}

\rmk{
In (1)---unlike in \cites{hooley1986HasseWeil,hooley_greaves_harman_huxley_1997,heath1998circle}'s ``Hypothesis~HW''---we need to know
not just about $L(s,V_{\bm{c}})$ as $\bm{c}$ varies,
but also about
% their exterior square $L$-functions, and $L(s,V)$
$L(s,V)$,
and about $L(s,V_{\bm{c}},\bigwedge^2)$
and its poles as $\bm{c}$ varies.
For us,
such polar information
% % (at least ``almost always'' with a power saving exceptional set)
% can either come
% % from functoriality, Ramanujan, and GRH
% from Langlands and GRH,
% % or alternatively,
% % from Poincar\'{e} duality, Tate's global semi-simplicity conjecture, and reciprocity
% or from a ``slightly more geometric'' set of conjectures.
comes from Langlands,
via standard ideas of a representation-theoretic flavor.
}

\rmk{
In (2),
we take the family of $V_{\bm{c}}$'s over $\set{\bm{c}\in\ZZ^6: F^\vee(\bm{c})\neq0}$,
indexed by very clean level sets $\norm{\bm{c}} = \ast$ for the ``$\eps$-removal'' result,
but later indexed with an additional ``adelic perturbation'' for the HLH result.
}

\rmk{
% Note re: Miller's (SFSC)-type assumption - not strictly weaker; just different. (He doesn't need power saving, but he also needs very large primes.) But if we didn't restrict to medium-sized moduli, our SFSC assumption would in fact be stronger than Miller's...
In work towards ``geometric'' RMT conjectures,
\cite{miller2004one} used a similar---but more qualitative---version of (3),
for sieve-theoretic purposes
\emph{different from ours}.
That said,
it might be interesting to interpret \cites{duke1993bounds,heath1996new}'s delta method in a sieve-theoretic framework
(cf.~the fact that $0$ is the only integer divisible by arbitrarily large primes;
also cf.~the ``philosophy of sieves and assemblers'' suggested by \cite{manin2021homotopy}).
}

\rmk{
(4) is a technical statement
that we use first to prove the existence of
certain local averages (such as those required for the ``recipe for the Ratios Conjectures''),
% Comment 2/17/21 (resolved): maybe better to say: "certain local averages (such as those in the recipe of [CFZ08])"... since we also use it for other technical averages towards (RA1'E') maybe?
and second to pacify certain ``small bad error moduli'' for HLH.
}

\section{Precise statements of our main hypotheses}
% Precise statements of our ``standard conjectural'' hypotheses
\label{SEC:precise-standard-conjectural-hypotheses}

(Recall that we are assuming $F=x_1^3+\dots+x_6^3$.
But in general, for $F$ with nonzero discriminant, almost all of what we say in \S\ref{SEC:precise-standard-conjectural-hypotheses} will apply if $m=6$; most of it will apply if $m\geq 3$ and $2\mid m$; and some of it will apply if $m\geq 3$ and $2\nmid m$.)

Having already stated the necessary forms of the Square-free Sieve Conjecture (SFSC) in \S\ref{SEC:SFSC-and-Sarnak-Xue-type-average-sum-bound},
it remains to discuss our other hypotheses, which involve certain ``geometric'' $L$-functions.
This is technical,
but the following three comments may help.
\begin{enumerate}[(1)]
    \item Geometric $L$-functions capture
    % ``the variation of local behavior in $p$'',
    ``the horizontal variation of local behavior''
    (i.e.~variation in $p$),
    in a way susceptible to
    the familiar formalism of representation theory.
    
    \item At least at ``good'' primes $p$,
    one can compute all local data in question
    \emph{cleanly and concretely}
    in terms of certain Frobenius eigenvalues.
    (See e.g.~Definition~\ref{DEFN:good-frienly-Hasse-Weil-L-function-factors}
    for a friendly definition of
    the specific $L$-functions $L(s,V_{\bm{c}})$
    at ``good'' primes.
    For the connection to Definition~\ref{DEFN:geometric-L-functions} below,
    see Observation~\ref{OBS:smooth-proper-base-change-for-various-representations-M} near p.~\pageref{OBS:smooth-proper-base-change-for-various-representations-M}.)
    
    \item At least morally,
    the $L$-functions relevant to us
    are no more complicated than
    the $L$-functions $L(s,A)$ of \emph{abelian varieties} $A$,
    and various tensor-square $L$-functions thereof.
    Therefore,
    familiarity with fairly well-known Galois representations
    (e.g.~$\ell$-adic Tate modules $V_\ell A\defeq T_\ell A\otimes\QQ$,
    and various tensor squares thereof)
    basically suffices.
\end{enumerate}

\defn{
[Cf.~\cite{taylor2004galois}]
\label{DEFN:geometric-L-functions}
Fix, once and for all, a prime $\ell_0$ and an inclusion $\iota\maps\QQ_{\ell_0}\inject\CC$.
Let $\rho\maps G_\QQ\to\GL(M)$ denote an \emph{$\ell_0$-adic} representation of $G_\QQ$.

Suppose $\rho$
\emph{arises from geometry},
in the ``arbitrary subquotient'' sense of
\cite{taylor2004galois}*{pp.~79--80, the two paragraphs before Conjecture~1.1~(Fontaine-Mazur)}.
Then $\rho$ is \emph{de Rham} at $\ell_0$,
and \emph{pure} of some weight $w\in\ZZ$,
so \cite{taylor2004galois}*{\S2, pp.~85--86} defines
(among other invariants of $\rho$)
\begin{enumerate}[(1)]
    \item a global \emph{conductor} $q(\rho)$
    and \emph{root number} $\eps(\rho)\defeq\eps(\iota\rho)$,
    and
    
    \item an \emph{algebraically normalized} local factor
    $L_{v,\textnormal{Taylor}}(s,\iota\rho)$
    at each place $v\leq\infty$.
\end{enumerate}
Now define the \emph{analytically normalized} local factors
$L_v(s,\rho)\defeq L_{v,\textnormal{Taylor}}(s+w/2,\iota\rho)$.
Globally, set
$L(s,\rho)\defeq\prod_{p<\infty}L_p(s,\rho)$
and $\Lambda\defeq L_\infty L$.
And for each prime $p$, let $\set{\tilde{\alpha}_{\rho,j}(p)}_j\defeq\set{p^{-w/2}\alpha_{\rho,j}(p)}_j$ denote
the multiset of \emph{normalized}
eigenvalues of geometric Frobenius on the
($\leq\dim(M)$-dimensional)
representation of $W_{\QQ_p}/I_{\QQ_p}$ defining $L_p(s,\rho)$.
Finally,
% let $\tilde{\lambda}_\rho(n)\defeq n^{-w/2}\lambda_\rho(n)$ denote the $n$th coefficient of $L(s,\rho)$
let $\tilde{\lambda}_\rho(n)\defeq [n^s]L(s,\rho)$
denote the $n$th coefficient of $L(s,\rho)$
for $n\geq1$.

For a smooth projective variety $Y$ over a field $k$, let $H^d(Y)\defeq H^d(Y_{\ol{k}}, \QQ_{\ell_0})$;
for an \emph{embedded} smooth projective variety $Z\belongs \PP^N_k$, let $H^d_{\map{diff}}(Z)\defeq H^d(Z)/H^d(\PP^N_k)$.
When $M\defeq H^d_{\map{diff}}(Z)$ for a \emph{smooth projective complete intersection} $Z\belongs \PP^{r+d}_\QQ$ of dimension $d\geq 0$ and codimension $r\geq 1$,
we \emph{abbreviate} $q(M),L(s,M),\dots$ as $q(Z),L(s,Z),\dots$.
(Here if $2\nmid d$,
then in fact $H^d_{\map{diff}}(Z)=H^d(Z)$;
but when $2\mid d$,
the ``$\map{diff}$'' is important.)
}

\rmk{
Here
each $L_p(s,\rho) = \prod_{j}(1-\tilde{\alpha}_{\rho,j}(p)p^{-s})^{-1}$
has some ``degree'' $\#\set{j}\leq\dim{M}$.
\emph{Purity} means
``$\#\set{j}=\dim{M}$
and $\abs{\tilde{\alpha}_{\rho,j}(p)}=1\;\forall{j}$''
holds for all $p$ outside a \emph{finite} set.
}

% \rmk{
% We could alternatively follow
% Serre 1970,
% where instead of arbitrarily fixing $\ell$,
% one assumes ``independence of $\ell$''
% (sometimes known, as in \cite{hooley1986HasseWeil} and \cite{heath1998circle}).
% % (which is likely known for the $\ell$-adic representations we are interested in).
% }

\rmk{
Say $Z/\QQ$ is a smooth projective complete intersection of dimension $d\geq0$.
Then only the \emph{middle cohomology} $H^d(Z)$---or
rather,
only the ``primitive quotient'' $H^d_{\map{diff}}(Z)$ thereof---is interesting.
This justifies the definition
$L(s,Z)\defeq L(s,M)$ above.

Now say $d\geq1$.
Then $H^d(Z)=H^d(\PP^{1+d})\oplus\ker(L\maps H^d(Z)\to H^{d+2}(Z))$,
% https://ncatlab.org/nlab/show/Lefschetz+decomposition
where $L$ denotes the Lefschetz operator.
So $H^d_{\map{diff}}(Z)\cong\ker(L)$.
But this is almost always false for $d=0$.
So even though $d\geq1$ always for us,
we prefer to define $H^d_{\map{diff}}(Z)$ as in Definition~\ref{DEFN:geometric-L-functions}.
}

\defn{
\label{DEFN:L(s,V_c)-and-coefficients-and-eigenvalues}
Fix $\bm{c}\in\ZZ^m$ with $F^\vee(\bm{c})\neq0$.
Then $V_{\bm{c}}$ is a smooth projective complete intersection in $\PP^{m-1}_\QQ$ of dimension $m_\ast\defeq m-3$.
So in particular,
Definition~\ref{DEFN:geometric-L-functions} \emph{fully defines $L(s,V_{\bm{c}})$}.
For $n\geq1$,
now let $\tilde{\lambda}_{\bm{c}}(n)\defeq [n^s]L(s,V_{\bm{c}})$
and $\mu_{\bm{c}}(n)\defeq [n^s]L(s,V_{\bm{c}})^{-1}$.
And for all $p,j$, let $\tilde{\alpha}_{\bm{c},j}(p)\defeq \tilde{\alpha}_{M_{\bm{c}},j}(p)$, where $M_{\bm{c}}\defeq H^{m_\ast}_{\map{diff}}(V_{\bm{c}})$.
}

\rmk{
A close reading of \cite{taylor2004galois}*{\S\S1--2}
(keeping in mind that certain geometric facts are \emph{truly of local origin},
% https://math.stackexchange.com/questions/1364560/continuity-of-galois-representations-from-cohomology
% https://math.stackexchange.com/questions/3310312/has-anyone-got-a-reference-as-to-why-%c3%a9tale-galois-representations-are-de-rham
even though \cite{taylor2004galois} often ``starts'' globally)
shows that given $n$,
one can define $\tilde{\lambda}_{\bm{c}}(n),\mu_{\bm{c}}(n)$
on all of $\set{\bm{c}\in\prod_{p\mid n}\ZZ_p^m:F^\vee(\bm{c})\neq0}$.
We could probably avoid using extended definitions like these,
but they are convenient in some local calculations.
}

After (SFSC),
our second technical hypothesis,
Conjecture~\ref{CNJ:(EKL)},
is an ``effective Krasner-type lemma''
(an ``ineffective'' version likely already being known by \cite{kisin1999local}).

\cnj{[EKL]
\label{CNJ:(EKL)}
Let $S\defeq\set{\bm{c}\in\wh{\ZZ}^m:F^\vee(\bm{c})\neq0}$.
Then there exists a \emph{nonzero} homogeneous polynomial $H\in\ZZ[\bm{c}]$ such that
% given $n\geq1$, the coefficient $\tilde{\lambda}_{\bm{c}}(n)$ depends only on $(n^\infty,H(\bm{c}))$ and $\bm{c}\bmod{(n^\infty,H(\bm{c}))}$.
% over $(\bm{c},n)\in S\times\ZZ_{\geq1}$, the value of $\tilde{\lambda}_{\bm{c}}(n)$ is determined by $n$, $(n^\infty,H(\bm{c}))$, and $\bm{c}\bmod{(n^\infty,H(\bm{c}))}$.
$\tilde{\lambda}_{\bm{c}}(n)$ is
\emph{constant} on the fibers of
the map $\ZZ_{\geq1}\times S \to \ZZ_{\geq1}\times \ZZ_{\geq0}\times 2^{\wh{\ZZ}}$ given by
\mathd{
(n,\bm{c}) \mapsto (n,r,\bm{c}+\map{rad}(n)r\wh{\ZZ}),
\;\textnormal{where $r\defeq (n^\infty, H(\bm{c}))\in \ZZ_{\geq0}$}.
}
}

\rmk{
The point is that $L_p(s,V_{\bm{c}})$ should morally be computable from something like a ``local minimal model'' of $(V_{\bm{c}})_{\QQ_p}$.
% If $V_{\bm{c}}$ is smooth over $\FF_p$, then $\bm{c}\bmod{p}$ is enough data.
% In general we may have to ``change coordinates'' $p$-adically, but hopefully something like $\bm{c}\bmod{p^{v_p(F^\vee(\bm{c}))}}$ is still enough data.
At least for elliptic curves in Weierstrass form,
$v_p(\Delta)$ controls minimality,
and $L_p$ is determined by the coefficients modulo $p$ of a minimal model.
% At least for elliptic curves in Weierstrass form, $v_p(\Delta)$ controls minimality (see \href{https://www.lmfdb.org/knowledge/show/ec.q.minimal_weierstrass_equation}{LMFDB} or \cite{iwaniec2004analytic}*{p. 364, paragraph surrounding (14.29)}), and given a minimal model, the local factor depends only on the coefficients modulo $p$ (see \cite{iwaniec2004analytic}*{p. 365}, noting that split multiplicative reduction is an arithmetic condition over $\FF_p$).
% %https://www.lmfdb.org/knowledge/show/ec.split_multiplicative_reduction depends only on minimal model modulo $p$
}

\ques{
Does (EKL) hold with $H=\Theta_F(1)\cdot F^\vee$
(for a suitable constant $\Theta_F(1)$)?
}

Now we move on to our seemingly more fundamental hypotheses---those involving
(for the most part)
\emph{global aspects of $L$-functions}.
As suggested in Remark~\ref{RMK:moral-argument-for-assuming-Langlands-directly-rather-than-HW-type-hypotheses},
we will phrase these hypotheses in terms of
automorphic representations $\pi$ of $\GL_\bullet$ over $\QQ$.
Doing so is technical,
but the following remark
(partly based on the survey \cite{farmer2019analytic})
may help.

\rmk{
\label{RMK:expository-remarks-on-isobaric-category}
We will only work with
\emph{cuspidal} $\pi$'s
(on $\GL_\bullet$ over $\QQ$),
or more generally,
\emph{isobaric} $\pi$'s.
These $\pi$'s have
well-defined $L$-functions $L(s,\pi)$,
and good formal properties
(due to Rankin, Selberg,
Langlands,
Godement, Jacquet, Shalika,
and others):
\begin{enumerate}[(1)]
    \item If $\pi$ is cuspidal,
    then $L(s,\pi)$ is \emph{primitive} in the sense of
    \cite{farmer2019analytic}*{Lemma~2.4},
    and has certain standard analytic properties \cite{farmer2019analytic}*{Theorem~3.1}.
    Furthermore,
    the following statements hold:
    \begin{enumerate}
        \item Given a cuspidal $\pi$,
        let $\omega\maps\wh{\ZZ}^\times\times\RR_{>0}\to\CC^\times$
        denote the (continuous) central character of $\pi$.
        Then the critical line of $L(s,\pi)$ is $\Re(s)=1/2$
        if and only if
        $\omega$ is \emph{unitary}
        (in which case
        one might informally say
        ``$\pi$ is of weight $0$'').
        % for weight 0 terminology, see http://math.caltech.edu/papers/drama-kor2.pdf
        
        \item In the setting of (a),
        suppose $\omega$ is unitary.
        Say $\pi$ is on $\GL_d$,
        let $q(\pi)$ be the conductor of $\pi$,
        and write $L_p(s,\pi)=\prod_{1\leq j\leq d}(1-\alpha_{\pi,j}(p)p^{-s})^{-1}$ for each $p\nmid q(\pi)$.
        Then the number $\prod_{1\leq j\leq d}\alpha_{\pi,j}(p)\in\CC$ is \emph{algebraic for all $p\nmid q(\pi)$}
        if and only if
        $\omega$ is a Dirichlet character
        (i.e.~$\omega$ is of \emph{finite order};
        i.e.~$\omega\vert_{\RR_{>0}}=1$;
        i.e.~$\pi$ is \emph{balanced} in the sense of \cite{farmer2019analytic}*{\S2.2}).
    \end{enumerate}
    
    \item For each isobaric $\pi$,
    there is a \emph{unique}
    multiset $\set{\pi_1,\dots,\pi_r}$,
    consisting of \emph{cuspidals},
    such that $L(s,\pi)=L(s,\pi_1)\cdots L(s,\pi_r)$.
    We call the $\pi_i$'s \emph{cuspidal constituents} of $\pi$.
    
    \item \emph{Strong multiplicity one}
    % for isobaric $\pi$'s
    (a strengthening of (2)'s ``uniqueness statement''):
    If $\pi,\pi'$ are isobaric,
    and $L_p(s,\pi)=L_p(s,\pi')$ for all but finitely many primes $p$,
    then $L(s,\pi)=L(s,\pi')$---or
    equivalently,
    $\set{\pi_1,\dots,\pi_r}=\set{\pi'_1,\dots,\pi'_{r'}}$,
    in the notation of (2).
    
    \item Fix an isobaric $\pi$,
    % with cuspidal constituents $\pi_1,\dots,\pi_r$.
    say corresponding to $\set{\pi_1,\dots,\pi_r}$
    in the notation of (2).
    Suppose each $\pi_i$ has \emph{unitary} central character.
    Then $L(s,\pi)$ has \emph{real coefficients}
    % If L(s,pi) has real coefficients, then at least in a unitary context, it is standard to call L(s,pi) "self-dual": see e.g. https://www.lmfdb.org/knowledge/show/lfunction.self-dual
    if and only if
    $\pi$ is \emph{self-dual}
    (i.e.~$L(s,\pi)=L(s,\pi^\vee)$;
    i.e.~$\set{\pi_1,\dots,\pi_r}=\set{\pi_1^\vee,\dots,\pi_r^\vee}$).
    % The key is that $L(s,\sigma^\vee)=\conj{L}(s,\sigma)$ for unitary cuspidal $\sigma$ (see \cite{farmer2019analytic}*{proof of Theorem~3.1} for a reference stating this fact explicitly; but it can probably be found elsewhere as well, with proof?)
\end{enumerate}
(Confusingly,
isobaric $\pi$'s are always irreducible as automorphic representations.
% see e.g. Theorem 1.2.10 in https://arxiv.org/pdf/math/0609460.pdf
So in (1) it is indeed better to use the adjective ``primitive'' than ``irreducible''.)
}

For our automorphic purposes,
the following convenient definition
(though not as precise as the ``algebraicity'' and ``arithmeticity'' definitions of Clozel and Buzzard--Gee)
will suffice:
\defn{
\label{DEFN:nice-isobaric-pi's}
% For convenience,
% call an isobaric $\pi$ \emph{unitary}
% if each cuspidal constituent of $\pi$ has unitary central character.
% % \footnote{so that $\pi$ is ``pure of weight $0$'' (i.e. \emph{each cuspidal constituent} of $\pi$ is of weight $0$)
% (The term ``unitary isobaric'' seems to have varying definitions throughout the literature;
% % https://arxiv.org/pdf/0710.0676.pdf (Dinakar Ramakrishnan, Remarks on the symmetric powers of cusp forms on GL(2))
% we are following the conventions of e.g.~\cite{booker2016converse}.)
Call an isobaric $\pi$ \emph{nice}
if each cuspidal constituent of $\pi$ has \emph{unitary, finite-order} central character.
}

Using Definition~\ref{DEFN:nice-isobaric-pi's},
we can now state Conjecture~\ref{CNJ:(HW2)},
which cleanly extends
Hypothesis HW (or rather (1) in Remark~\ref{RMK:first-mention-of-standard-consequences-of-Langlands-and-GRH})
from $L(s,V_{\bm{c}})$
to a host of related $L$-functions.
\cnj{[HW2]
\label{CNJ:(HW2)}
% (HW$\bigotimes$2)
Fix $\bm{c}\in\ZZ^m$ with $F^\vee(\bm{c})\neq0$.
Let $M_{\bm{c}}\defeq H^{m_\ast}_{\map{diff}}(V_{\bm{c}})$
and $M_V\defeq H^{1+m_\ast}_{\map{diff}}(V)$.
Now let $M$ denote one of the representations
\mathd{
M_{\bm{c}},M_V,M_{\bm{c}}\wedge M_{\bm{c}},\map{Sym}^2{M_{\bm{c}}},M_{\bm{c}}\otimes M_{\bm{c}},M_{\bm{c}}\otimes M_{\bm{c}}^\vee.
}
Let $S$ denote $\disc(F)$ or $F^\vee(\bm{c})$,
according as $M=M_V$ or $M\neq M_V$.
Then
\begin{enumerate}[(a)]
    \item $M$ itself \emph{arises from geometry}
    (pure of weight $m_\ast$, $1+m_\ast$, or $2m_\ast$
    according as $M$ is $M_{\bm{c}}$, $M_V$, or neither),
    with $\dim{M}=O_m(1)$
    and $q(M)\mid\map{rad}(S)^{O_m(1)}\mid S^{O_m(1)}$;
    
    \item the gamma factor $L_\infty(s,M)$
    lies in a \emph{finite set} depending only on $m$;
    %This should follow at least formally from Hodge theory (cf.~Conjecture 1.2(part 3) in Taylor).
    
    \item there exists a \emph{nice isobaric} $\pi_M$ on $\GL_{\dim{M}}$ over $\QQ$
    %http://math.caltech.edu/papers/drama-kor2.pdf Theorem~1.2.10 notes that isobaric is still technically irreducible automorphic rep (just not irreducible in the ``Langlands sum'' sense?)
    such that
    (i) $L_v(s,M) = L_v(s,\pi_M)$ at all places $v\leq\infty$,
    and (ii) $(q(M),\eps(M))=(q(\pi_M),\eps(\pi_M))$;
    
    % \item finite order,
    % functional equation,
    % holomorphic except possibly for poles at $s=1$ corresponding to trivial constituents of $\pi_M$;
    
    \item each cuspidal constituent of $\pi_M$
    satisfies the \emph{Generalized Ramanujan Conjectures (GRC)}---whence
    $\abs{\tilde{\alpha}_{M,j}(p)}\leq 1$ for all $p,j$,
    and $L_\infty(s,M)$ is holomorphic on $\set{\Re(s)>0}$;
    and
    % Here $\omega$ should be of finite order (i.e.~trivial on $\RR_{>0}$), cf.~fact that ``tempered'' in GRC means something more precise than what we have above (but there's no reason to include the extra precision unless we do it at all places?)... see ``Analytic L-functions: Definitions, theorems, and connections'' by Farmer et al.
    %$\omega$ being trivial on $\RR_{>0}$ should be related to Hodge-Tate weights, cf.~Conjecture 1.3 and Conjecture 1.2(part 3) in Taylor 2004.
    % $\RR_{>0}$-trivial,\footnote{cf.~the assumptions on $\pi$ near the beginning of \cite{sarnak2016families}*{\S1}}
    
    \item $L(s,\pi_M)$ satisfies the \emph{Grand Riemann Hypothesis (GRH)}.
\end{enumerate}
}

\rmk{
Here ``HW'' stands for ``Hasse--Weil''
(cf.~``Hypothesis HW'').
The ``2'' refers to the ``second-order'' nature in which $L(s,V),L(s,V_{\bm{c}},\bigwedge^2)$ will appear later.
Here and later,
we let $L(s,V_{\bm{c}},\bigwedge^2)\defeq L(s,M_{\bm{c}}\wedge M_{\bm{c}})$,
and similarly define $L(s,V_{\bm{c}},\map{Sym}^2),L(s,V_{\bm{c}},\bigotimes^2),\dots$.
}

\rmk{
% It is conjectured that $M$ should be semi-simple (as an $\ell_0$-adic representation of $G_\QQ$),
% in which case \cite{taylor2004galois}*{p.~100, prior to Conjectures 3.4--3.5} ``specifies'' $\pi_M$ precisely.
% This,
% and some of the other
Some of the ``conjectures'' above (excluding GRH) may be known in some generality or specificity;
see \S\ref{SUBSEC:discuss-individual-L-function-aspects-(HW2)-and-(HWSp)} (which begins near p.~\pageref{SUBSEC:discuss-individual-L-function-aspects-(HW2)-and-(HWSp)}) for some discussion.
}

\obs{
\label{OBS:smooth-proper-base-change-for-various-representations-M}
In the notation of (HW2),
fix a prime $p\nmid\ell_0 S$.
% if $M\neq M_V$ and $p\nmid F^\vee(\bm{c})$, \emph{and} $p\neq2$ if $M\neq M_{\bm{c}}$,
Then the multiset $\set{\alpha_{M,j}(\rho)}_j$
\emph{coincides} with the multiset of eigenvalues of
geometric Frobenius on
% $M_{\FF_p}\in\set{H^{m_\ast}((\mcal{V}_{\bm{c}})_{\ol{\FF}_p},\QQ_{\ell'}),\dots}$, for any prime $\ell'\nmid 2p$;
``the obvious $G_{\FF_p}$-representation $\mcal{M}_{\FF_p}\in\set{H^{m_\ast}_{\map{diff}}((\mcal{V}_{\bm{c}})_{\FF_p}),\dots}$ corresponding to $M$''.
(For proof,
use smooth proper base change---and some linear algebra on $\bigwedge^2,\map{Sym}^2,-\otimes-$.)
%This should follow from smooth proper base change for the ``pure'' tensors, and then since $p\neq 2$ we should be able to use $S_2$-equivariance to extract $\bigwedge^2,\dots$?
%https://ayoucis.wordpress.com/2015/07/24/a-different-viewpoint-on-etale-cohomology/
%https://ayoucis.wordpress.com/2015/01/26/some-examples-of-geometric-galois-representations/

% EDIT 3/13/2021 (moved "item (c)" to the present remark): the factors of 2 in S are probably unnecessary, and (c) should probably not be stated as part of this conjecture; instead (perhaps in a remark here or in an appendix) we should say that if p\nmid\ell_0 S, then M is unramified at p, and furthermore the corresponding W_{Q_p}- and G_{F_p}- representations are compatible by smooth proper base change.
% (For a concrete example with symmetric powers at *all primes*, see Sarnak's letter to Mazur; or see http://www.lmfdb.org/L/degree3/EllipticCurve/SymmetricSquare/.)
% See also https://math.stackexchange.com/questions/23899/formulas-for-the-top-coefficients-of-the-characteristic-polynomial-of-a-matrix for e.g. trace of an operator (e.g. Frobenius) on exterior powers.

In particular,
this explains the connection between
the (full, sophisticated) Definition~\ref{DEFN:L(s,V_c)-and-coefficients-and-eigenvalues}
and the (partial, simpler) Definition~\ref{DEFN:good-frienly-Hasse-Weil-L-function-factors}
for $L(s,V_{\bm{c}})$.
}

\obs{
\label{OBS:Rankin-Selberg-L-functions-under-(HW2)}
Assume (HW2),
and fix $\bm{c}$.
Then $L(s,V_{\bm{c}},\bigotimes^2)$ is (isobaric) automorphic,
so by strong multiplicity one (Remark~\ref{RMK:expository-remarks-on-isobaric-category}(3)),
\emph{the Rankin--Selberg $L$-function $L(s,\pi_{M_{\bm{c}}},\bigotimes^2)$
% as defined by local Langlands perhaps?
is automorphic,
and equal to $L(s,V_{\bm{c}},\bigotimes^2)$}.
Similar comments apply to $L(s,\pi_{M_{\bm{c}}},\bigwedge^2),\dots$.

In particular,
% (HW2) implies RH for $\zeta(s)$,
$L(s,\pi_{M_{\bm{c}}}\times\pi_{M_{\bm{c}}}^\vee)$---which
has a pole at $s=1$ for general reasons---must
be (isobaric) automorphic,
and must satisfy GRH.
So \emph{RH for $\zeta(s)$ must hold}, under (HW2).
}

For convenience,
we now refine (HW2) to
a conjecture more specific to our family of $M_{\bm{c}}$'s.
From here to the end of \S\ref{SEC:precise-standard-conjectural-hypotheses}, it will be important that $2\mid m$.
\cnj{[HW$\map{Sp}$]
\label{CNJ:(HWSp)}
Conjecture \emph{(HW2) holds,
and} each $\pi_M$ in (HW2) is \emph{self-dual}.
\emph{Furthermore},
for each $\bm{c}$ in (HW2),
there exists a \emph{nice isobaric} $\phi_{\bm{c},2}$ over $\QQ$ such that
\begin{enumerate}[(a)]
    \item $L_v(s,V_{\bm{c}},\bigwedge^2) = \zeta_v(s)L_v(s,\phi_{\bm{c},2})$ holds at all places $v\leq\infty$,
    and
    
    \item the conductors and $\eps$-factors match accordingly.
\end{enumerate}
}

\rmk{
Fix $\bm{c}$,
and say $M_{\bm{c}}$ is \emph{irreducible}.
(Under Langlands-type conjectures and GRH,
one can prove that $M_{\bm{c}}$ is \emph{typically} irreducible.
See also Proposition~\ref{PROP:absolute/strong/potential-irreducibility-for-V_c-when-c=(1,2,3,4,5,6)} for some heuristic evidence for a stronger statement.)
Then Schur's lemma and Poincar\'{e} duality
(and the fact that $m_\ast=m-3$ is \emph{odd})
suggest (via Langlands-type conjectures)
that the putative $\pi_{M_{\bm{c}}}$ in (HW2)
should be \emph{cuspidal self-dual symplectic} as defined on \cite{sarnak2016families}*{p.~533},
and in particular that $L(1,\pi_{M_{\bm{c}}},\bigwedge^2)=\infty$.
Hence the ``$\map{Sp}$'' in ``HW$\map{Sp}$''.

It is then no surprise\footnote{since we are working over a number field (rather than a global function field)}
that under (HW2), say,
% though a weak large sieve would probably suffice (in place of GRH) for the SST recipe...
the \emph{homogeneity} type of the family $\bm{c}\mapsto L(s,V_{\bm{c}})$ is ``symplectic''
(in the sense of \cite{sarnak2016families});
see \S\ref{SUBSEC:discuss-expected-RMT-statistical-nature-of-our-families-of-L-functions}
(which begins near p.~\pageref{SUBSEC:discuss-expected-RMT-statistical-nature-of-our-families-of-L-functions})
for details.
Under \cite{sarnak2016families}*{Universality Conjecture},
one then expects the \emph{symmetry} type to be ``orthogonal''---a
point that we will clarify in due time
(even though the RMT-based conjectures we use
will come mostly from \cite{conrey2008autocorrelation}
rather than \cite{sarnak2016families}).
}

See \S\ref{SUBSEC:discuss-individual-L-function-aspects-(HW2)-and-(HWSp)}
for a more thorough discussion of (HW2) and (HW$\map{Sp}$),
and where they come from.
But most directly relevant to us
% in (HW2) and (HWSp)
are the \emph{analytic estimates} (in Remark~\ref{RMK:standard-consequences-of-(HW2)})
implied by (HW2) and (HWSp),
leading via \cite{iwaniec2004analytic}*{Perron's formula (5.111)} to
\emph{pointwise} ``square-root cancellation up to $\eps$'' bounds (such as (3) in Remark~\ref{RMK:first-mention-of-standard-consequences-of-Langlands-and-GRH}).
Such bounds ``barely'' fail us.
Fortunately,
there exist
more precise \emph{mean-value} predictions of RMT type.
For some (but not all) of our purposes here and elsewhere,
we can work with \emph{smooth weights};
to this end, recall Definition~\ref{DEFN:decay-weight-nu}.

Let $\pi_{\bm{c}}\defeq\pi_{M_{\bm{c}}}$.
In \S\ref{SUBSEC:applying-CFZ-ratios-conjecture-recipe} (which begins near p.~\pageref{SUBSEC:applying-CFZ-ratios-conjecture-recipe}),
we will explain---assuming (HW2) and (EKL)---how
the ``recipe'' (or ``heuristic'') of \cite{conrey2008autocorrelation},
% carried over to / interpreted (made sense of; see https://www.vocabulary.com/dictionary/interpret "When you interpret something, you make sense of it. You could interpret a graph, a foreign language, or even Mona Lisa's odd smile.") in ... the general framework / natural/general setting ... of SST
when interpreted in the general framework of \cite{sarnak2016families}*{pp.~534--535, Geometric Families and Remark~(i)},
leads to the following clean two-part ``Ratios Conjecture''
(and to the slightly messier ``Ratios Conjecture'' (RA1) stated later below as Conjecture~\ref{CNJ:(RA1)}):
\cnj{
[Cf.~\cite{conrey2008autocorrelation}*{\S5.1, (5.6)}]
\label{CNJ:(R1)--(R2)}
% [(R1)--(R2)]
% By the Ratios Conjecture,\footnote{stated with a smoothly weighted sum over $\bm{c}$, rather than with a sharp cutoff}
Restrict to $F^\vee(\bm{c})\neq0$,
let $s\defeq\sigma+it$,
and \emph{assume (HW2)}.
Fix a \emph{decay weight $\nu_m\geq0$}.
Then the following hold:
\begin{enumerate}[label=(R\arabic*)]
    \item Fix $\sigma>1/2$.
    Then uniformly over $Z\geq 1$ and $\abs{t}\leq Z^\hbar$, we have
    \mathds{
    &\psum_{\bm{c}\in\ZZ^m}
    \nu_m\left(\frac{\bm{c}}{Z}\right)
    \cdot \frac{1}{L(s,\pi_{\bm{c}})} \\
    &= O_{\nu_m,\sigma}((Z^m)^{1-\delta})
    + \psum_{\bm{c}\in\ZZ^m}
    \nu_m\left(\frac{\bm{c}}{Z}\right)
    \cdot \zeta(2s)L(s+1/2,V)
    A_F(s),
    }
    for some absolute $\hbar,\delta>0$ independent of $\sigma$.
    
    \item Fix $\sigma_1,\sigma_2>1/2$.
    Then uniformly over $Z\geq 1$ and $\norm{\bm{t}}\leq Z^\hbar$, we have
    \mathds{
    &\psum_{\bm{c}\in\ZZ^m}
    \nu_m\left(\frac{\bm{c}}{Z}\right)
    \cdot\frac{1}{L(s_1,\pi_{\bm{c}})L(s_2,\pi_{\bm{c}})} \\
    &= O_{\nu_m,\bm{\sigma}}((Z^m)^{1-\delta})
    + \psum_{\bm{c}\in\ZZ^m}
    \nu_m\left(\frac{\bm{c}}{Z}\right)
    \cdot A_{F,2}(\bm{s})
    \zeta(s_1+s_2)
    \prod_{j\in[2]}\zeta(2s_j)L(s_j+1/2,V),
    }
    for some absolute $\hbar,\delta>0$ independent of $\bm{\sigma}$.
\end{enumerate}
Here $A_F(s),A_{F,2}(\bm{s})$ are certain Euler products
(defined in \S\ref{SUBSUBSEC:deriving-(R1)--(R2)},
in terms of $F$),
\emph{absolutely convergent} on the regions $\Re(s),\Re(\bm{s})\geq 1/2-\delta'$, respectively,
for some absolute $\delta'>0$.
}

Before proceeding,
we make four remarks on Conjecture~\ref{CNJ:(R1)--(R2)}.

\rmk{
The letter ``R'' signifies
``Random Matrix Theory''
or ``Ratios Conjecture'' (or ``recipe'' thereof).
% When citing~\cite{conrey2008autocorrelation}, we may occasionally write ``R\cite{conrey2008autocorrelation}'' for emphasis.
}

\rmk{
Following \cite{conrey2007applications}*{(2.11a)--(2.11b)},
one could require $\sigma,\bm{\sigma}\leq 1/2+\delta'$,
and restrict the ``absolute convergence'' statement to $\Re(s),\Re(\bm{s})\in[1/2-\delta',1/2+\delta']$,
to be safe (see \cite{conrey2007applications}*{Remark~2.3}).
Our applications would certainly permit that.
But since our ``ratios'' only involve $1/L$ and not $L$,
the present formulation of Conjecture~\ref{CNJ:(R1)--(R2)} should be OK.
}
\rmk{
We restrict $t$ to $[-Z^\hbar,Z^\hbar]$
to (comfortably) respect the constraint \cite{conrey2007applications}*{(2.11c)} on ``vertical shifts''.
We could alternatively allow \emph{arbitrary} $t\in\RR$,
but only after including a sufficiently large factor of the form $(1+\abs{t})^{O(1)}$ in the error terms of (R1)--(R2).

(Here we are being careful,
but it might actually be possible to relax \cite{conrey2007applications}*{(2.11c)} to allow $t\in[-Z^{O(1)},Z^{O(1)}]$;
cf.~\cite{bettin2020averages}*{p.~4, the sentence before Conjecture~2}.)
}

\rmk{
\cite{conrey2008autocorrelation}*{\S5.1, (5.6)} has
a ``square-root error term''---too much to ask for
in general.\footnote{See \cite{diaconu2021third}*{Theorem~1.2}
for an unconditional example in the setting of \cite{conrey2005integral}.
See also \cite{cho2022omega}*{first two sentences of the paragraph before Theorem~1.4,
% and final two sentences of Remark~1.3,
as $\sigma\to0^+$}
for a presumably unconditional example in the setting of one-level densities.}
However,
all theoretical and numerical data so far suggest
\emph{power-saving error terms}---and
an \emph{arbitrary} power saving is all we need (and all we assume).
% all we need or assume.
}

Actually, to prove $N_F(X)\ll X^3$,
we do not need (R2) itself,
but only the weaker statement (R2') below.
However,
we need to keep uniformity in mind when integrating against certain \emph{varying weights} $f(s)$.
So we first make a definition:
\defn{
\label{DEFN:(Z,nu,hbar)-good-contour-weight-f}
Let $f\maps\set{\Re(s)\in(1/5,6/5)}\to\CC$ be Schwartz on vertical strips.
Given $Z,\nu_1,\hbar$,
suppose $\abs{f(s)}\leq\nu_1(t)$ for all $\sigma\in (1/5,6/5)$ and $\abs{t}\geq Z^\hbar$.
Also let $M_I(f)\defeq \sup_{\sigma\in I}\norm{f(\sigma+it)}_{L^1(t\in\RR)}$ and suppose $M_{(1/5,6/5)}(f)<\infty$.
Then we say $f$ is \emph{$(Z,\nu_1,\hbar)$-good}.
}

\cnj{
[R2']
\label{CNJ:(R2')}
\emph{Assume} (HW2).
Then there exist absolute constants $\hbar,\delta>0$ such that
if $\nu_1,\nu_m\geq0$ are decay weights
and $f(s)$ is an \emph{$(Z,\nu_1,\hbar)$-good holomorphic} function,
then uniformly over any \emph{given} range of the form $1\ll N\ll Z^3$,
we have
\mathd{
\psum_{\bm{c}\in\ZZ^m}
\nu_m\left(\frac{\bm{c}}{Z}\right)
\left\lvert\int_{(\sigma)}ds\,\frac{\zeta(2s)^{-1}L(s+1/2,V)^{-1}}{L(s,\pi_{\bm{c}})}
\cdot f(s)N^s\right\rvert^2
\ll_{\nu_1,\nu_m} M^{(2)}(f)\cdot Z^mN
}
for all $\sigma>1/2$,
where the implied constant on the right-hand side depends only on $\nu_1,\nu_m$.
Here $M^{(2)}(f)\defeq [1 + M_{\set{1/2}}(f^2) + M_{[1/2-\delta,1/2+\delta]}(s^3 f(s))]^2$.
}

\rmk{
% Here ``(R2')'' stands for ``applied (R2)'':
In \cite{wang2021_HLH_vs_RMT}*{\S7.4},
we prove (R2')
\emph{assuming} (HW2) and (R2).
}

\rmk{
Sometimes,
``sharp-up-to-constant upper bounds'' like (R2')
can be ``easier'' than
true ``leading-order asymptotics'' like (R1)--(R2).
See e.g.~\cite{harper2013sharp}.
}

On the other hand,
(R1) itself \emph{does not suffice}
to fully establish HLH (in our setting).
Rather,
we need a slight ``adelic perturbation'' (R$\bd{A}_\ZZ$1) of (R1).
% Discuss archimedean perturbation as well\dots
% at least seems reasonable that if we don't deform boxes/regions too much then the non-archimedean local averages should suffice to predict RMT-type averages.
To state the Ratios Conjecture (R$\bd{A}_\ZZ$1),
we first make a convenient definition:
\defn{
Let $r_{-}\defeq\min(r,0)$ and $r_{+}\defeq\max(r,0)$ for every $r\in\RR$.
Then given $\bm{Z}\in\RR^m$,
we let $\mcal{B}(\bm{Z})
\defeq [(Z_1)_{-},(Z_1)_{+}]\times\cdots\times[(Z_m)_{-},(Z_m)_{+}]$.
}

\cnj{[R$\bd{A}_\ZZ$1,
or RA1]
\label{CNJ:(RA1)}
Fix $\sigma>1/2$.
Then for some absolute $\hbar,\delta>0$ independent of $\sigma$,
we have
\mathd{
\psum_{\bm{c}\equiv\bm{a}\bmod{n_0}}
\bm{1}_{\bm{c}\in\mcal{B}(\bm{Z})}
\cdot \frac{1}{L(s,\pi_{\bm{c}})}
= O_\sigma(\card{\mcal{F}}^{1-\delta})
+ \psum_{\bm{c}\equiv\bm{a}\bmod{n_0}}
\bm{1}_{\bm{c}\in\mcal{B}(\bm{Z})}
\cdot \zeta(2s)L(s+1/2,V)
A_F^{\bm{a},n_0}(s),
}
uniformly over $\abs{t},n_0\leq Z^\hbar$ and $(\bm{a},\bm{Z})\in\ZZ^m\times\RR^m$
with $\abs{Z_1},\dots,\abs{Z_m}\in [Z^{1-\hbar},Z]$.
Here $A_F^{\bm{a},n_0}(s)$ is a certain Euler product
(defined in \S\ref{SUBSUBSEC:deriving-(RA1)},
in terms of $F,\bm{a},n_0$),
absolutely convergent on $\set{\Re(s)\geq 1/2-\delta'}$ for some absolute $\delta'>0$.
}

% \rmk{
% For convenience,
% we may write (RA1) instead of (R$\bd{A}_\ZZ$1).
% }

\rmk{
Note that the ``prediction'' (i.e.~the ``right-hand side'') is sensitive to both
the archimedean data ``$\bm{Z}$''
and the non-archimedean data ``$\bm{a}\bmod{n_0}$''.
Furthermore,
in (RA1),
we forbid $\bm{Z}$ from being too lopsided,
and we also forbid $n_0$ from being too large.
For more details on the philosophy behind (RA1),
see Remark~\ref{RMK:multi-parameter-family-features} (near p.~\pageref{RMK:multi-parameter-family-features}).
}

\section{Some critical calculations}

\subsection{A second-order approximation}
\label{SUBSEC:approx-Phi(c,s)-series-past-1/2-to-second-order}

Recall, from Chapter~\ref{CHAP:delta-method-review} and \S\ref{SEC:using-L-function-hypotheses-on-average},
the Dirichlet series $\Phi(\bm{c},s),L(s,V_{\bm{c}})$,
and the ``first-order approximation''
$\Phi = \Psi_1\Psi_2$,
with $\Psi_1\defeq 1/L$
and $\Psi_2\defeq \Phi L$.
As suggested in Remark~\ref{RMK:[B2']-intuition},
the ``first-order error'' $\Psi_2$
is a ``source of $\eps$'' in \S\ref{SEC:using-L-function-hypotheses-on-average},
coming from both \emph{good} primes $p\nmid F^\vee(\bm{c})$ and \emph{bad} primes $p\mid F^\vee(\bm{c})$.

From the perspective of the good primes---which define ``standard'' data---we would like a more precise ``standard'' approximation of $\Phi$.
We thus
% multiply $\Psi_2\defeq\Phi\cdot L(s,V_{\bm{c}})$ by $L(1/2+s,V)L(2s,V_{\bm{c}},\bigwedge^2)$ to adjust for
further approximate $\Phi/L(s,V_{\bm{c}})^{-1}$ by $L(1/2+s,V)^{-1}L(2s,V_{\bm{c}},\bigwedge^2)^{-1}$,
in view of the ``second-order phenomena''
\mathd{
\wt{S}_{\bm{c}}(p)-\mu_{\bm{c}}(p)
= \wt{S}_{\bm{c}}(p)-\wt{E}_{\bm{c}}(p)
= -p^{-1/2}\wt{E}_F(p)
= -p^{-1/2}\tilde{\lambda}_V(p)
\quad\textnormal{for $p\nmid F^\vee(\bm{c})$}
}
(valid for \emph{even} integers $m\geq4$)
and
\mathd{
\wt{S}_{\bm{c}}(p^2)-\mu_{\bm{c}}(p^2)
= 0-\sum_{i<j}\tilde{\alpha}_{\bm{c},i}(p)\tilde{\alpha}_{\bm{c},j}(p)
= -\tilde{\lambda}_{V_{\bm{c}},\bigwedge^2}(p)
\quad\textnormal{for $p\nmid F^\vee(\bm{c})$}.
}


\rmk{
If $m$ were \emph{odd},
we would instead have
$\wt{S}_{\bm{c}}(p)-\tilde{\lambda}_{\bm{c}}(p)
= p^{-1/2}\tilde{\lambda}_V(p)$
for $p\nmid F^\vee(\bm{c})$,
and also instead care about the identity
$\wt{S}_{\bm{c}}(p^2)-\tilde{\lambda}_{\bm{c}}(p^2)
= -\tilde{\lambda}_{V_{\bm{c}},\map{Sym}^2}(p)$
for $p\nmid F^\vee(\bm{c})$.
}

For technical reasons,
we separately study the ``standard Hasse--Weil part'' of $\Phi$
and the ``bad part'' of $\Phi$,
based on the following definition:
\defn{
\label{DEFN:factor-Phi^HW-into-Phi_1-Phi_2-Phi_3}
Given $\bm{c}\in\ZZ^m$ with $F^\vee(\bm{c})\neq0$,
let $\Phi^{\map{HW}}(\bm{c},s)\defeq\prod_{p\nmid F^\vee(\bm{c})}\Phi_p(\bm{c},s)$
and $\Phi^{\map{B}}(\bm{c},s)\defeq\prod_{p\mid F^\vee(\bm{c})}\Phi_p(\bm{c},s)$.
% Then let $\Phi_3^{\map{HW}}\defeq\Phi^{\map{HW}}\Phi_1^{-1}\Phi_2^{-1}$.
If $m\geq 3$ is even, define
\mathds{
\Phi_1(\bm{c},s)
&\defeq L(s,V_{\bm{c}})^{-1}
L(1/2+s,V)^{-1}
\zeta(2s)^{-1} \\
\Phi_2(\bm{c},s)
&\defeq \zeta(2s)/L(2s,V_{\bm{c}},\textstyle{\bigwedge^2}) \\
\Phi_3
= \Phi_3^{\map{HW}}(\bm{c},s)
&\defeq \Phi^{\map{HW}}\Phi_1^{-1}\Phi_2^{-1}
= \Phi^{\map{HW}}(\bm{c},s)
L(s,V_{\bm{c}})
L(1/2+s,V)
L(2s,V_{\bm{c}},\textstyle{\bigwedge^2}).
}
For each $j\in[3]$,
let $a_{\bm{c},j}(n)$ be the $n^s$ coefficient of $\Phi_j(\bm{c},s)$.
}

\ex{
$a_{\bm{c},2}(n)=0$ if $n$ is not a square.
}

\rmk{
Here $\Phi_1(\bm{c},s)$ is precisely the ``mollified'' version of $1/L(s,V_{\bm{c}})$ considered in Conjecture~\ref{CNJ:(R2')}~(R2')!
And $\Phi_2(\bm{c},s)=L(2s,\phi_{\bm{c},2})^{-1}$ under Conjecture~\ref{CNJ:(HWSp)}~(HWSp).
}

\rmk{
For each $j\in[3]$,
the definition of $a_{\bm{c},j}(n)$ can be extended to
tuples $\bm{c}\in\prod_{p\mid n}\ZZ_p^m$ with $F^\vee(\bm{c})\neq0$.
In local calculations,
we sometimes abuse notation accordingly.
}

\prop{
% [A2]
%$\frac12\approx$
\label{PROP:approx-Phi-past-1/2}
Fix $\bm{c}\in\ZZ^m$ with $F^\vee(\bm{c})\neq0$,
and suppose $m\in\set{4,6}$.
\emph{Assume} Conjecture~\ref{CNJ:(HW2)}~(HW2).
% If $\Re(s)>1/3$, then $\Phi_1,\Phi_2$ are \emph{meromorphic} and $\Phi_3$ is \emph{holomorphic}.
Then $\Phi_3$ \emph{converges absolutely} over $\Re(s)>1/3$;
\emph{locally}, we have, uniformly over $\bm{c},p,l$, that
\mathd{
a_{\bm{c},3}(p)\cdot\bm{1}_{p\nmid F^\vee(\bm{c})}=0
\quad\textnormal{and}\quad
a_{\bm{c},3}(p^2)
\cdot\bm{1}_{p\nmid F^\vee(\bm{c})}\ll p^{-1/2}
\quad\textnormal{and}\quad
a_{\bm{c},3}(p^l)\ll_\eps p^{l\eps}.
}
% having local factors
% \mathd{
% \Phi_{3,p}(\bm{c},s) = 1 + O(p^{-1/2-2\Re(s)}) + O(p^{-3\Re(s)})
% }
% at \emph{good primes} $p\nmid F^\vee(\bm{c})$.
Also, \emph{under} (HWSp),
$\Phi_2$ is \emph{holomorphic} on the region $\Re(s)>1/4$.
% Here we freely assume Ramanujan for all $L$-functions in sight.\footnote{This is to ensure clean, uniform error terms when multiplying or dividing by various local $L$-factors.}
}

\pf{
Our definition of $\Phi^{\map{HW}}$,
along with GRC,
ensures that each local factor $\Phi_{j,p}(\bm{c},s)$ is ``well-behaved over $\Re(s)>0$''
in a standard sense---even at ``bad'' primes $p\mid F^\vee(\bm{c})$.
Specifically,
we have $\abs{a_{\bm{c},j}(p^l)}\ll_\eps p^{l\eps}$
uniformly over tuples $\bm{c}$ (with $F^\vee(\bm{c})\neq0$),
primes $p$,
and integers $l\geq 1$.
% (To be clear,
% the only restriction on $\bm{c},p,l$ in the previous sentence is that $F^\vee(\bm{c})\neq0$.)

In particular,
``bad local factors'' do \emph{not} affect holomorphy,
or absolute convergence,
over any subset of the half-plane $\Re(s)>0$.
Now fix $\bm{c}$
and a prime $p\nmid F^\vee(\bm{c})$,
so that
$\Phi_p(\bm{c},s) = 1 + \wt{S}_{\bm{c}}(p)p^{-s}$,
where
\mathd{
\wt{S}_{\bm{c}}(p)
= \wt{E}_{\bm{c}}(p)-p^{-1/2}\wt{E}_F(p)
= -\tilde{\lambda}_{\bm{c}}(p)-p^{-1/2}\tilde{\lambda}_V(p).
}
% Multiplying $\Phi_p(\bm{c},s)$ by $L_p(s,V_{\bm{c}})$ gives
The product $\Phi_p(\bm{c},s) L_p(s,V_{\bm{c}})$ simplifies to
\mathds{
&(1 - \tilde{\lambda}_{\bm{c}}(p)p^{-s} - \tilde{\lambda}_V(p)p^{-1/2-s})
(1 + \tilde{\lambda}_{\bm{c}}(p)p^{-s} + \tilde{\lambda}_{\bm{c}}(p^2)p^{-2s} + O(p^{-3s})) \\
&= 1 - \tilde{\lambda}_V(p)p^{-1/2-s} + [\tilde{\lambda}_{\bm{c}}(p^2)-\tilde{\lambda}_{\bm{c}}(p)^2]p^{-2s} + O(p^{-1/2-2s}) + O(p^{-3s}).
}
To get further cancellation,
we multiply by
\mathd{
L_p(1/2+s,V)
= 1+\tilde{\lambda}_V(p)p^{-1/2-s}+O(p^{-1-2s}),
}
and also---motivated by the identities
``$(\tilde{\lambda}_{\bm{c}}(p^2),
\tilde{\lambda}_{\bm{c}}(p)^2)
=(\tilde{\lambda}_{V_{\bm{c}},\map{Sym}^2}(p),
\tilde{\lambda}_{V_{\bm{c}},\bigotimes^2}(p))$
if $p\nmid F^\vee(\bm{c})$''
and ``$\bigotimes^2 = \map{Sym}^2\oplus\bigwedge^2$ in general''---by
\mathd{
L_p(2s,V_{\bm{c}},\textstyle{\bigwedge^2})
= 1+\tilde{\lambda}_{V_{\bm{c}},\bigwedge^2}(p)p^{-2s}+O(p^{-4s}),
}
to get
\mathd{
\Phi_p(\bm{c},s)L_p(s,V_{\bm{c}})
L_p(1/2+s,V)
L_p(2s,V_{\bm{c}},\textstyle{\bigwedge^2})
= 1
+ O(p^{-1/2-2s})
+ O(p^{-3s}).
}
By definition, the left-hand side is precisely the ``second-order error'' $\Phi_{3,p}$, as desired.
}

\rmk{
From the perspective of $\Phi_1$,
we consider $\Phi_2,\Phi_3$ ``error factors''
and need to separately consider large and small ``error moduli''
(with only the small moduli participating in
the RMT-type predictions we use).
}

\rmk{
If $m$ were \emph{odd},
then $\Phi^{\map{HW}}(\bm{c},s)$ would be
\mathd{
\approx L(s,V_{\bm{c}})
L(1/2+s,V)
/L(2s,V_{\bm{c}},\map{Sym}^2).
}
}

\cor{[$\Phi$3E]
\label{COR:average-bound-on-Phi_3}
Assume the setting and hypotheses of Proposition~\ref{PROP:approx-Phi-past-1/2}.
Let $\mcal{S}\defeq\set{\bm{c}\in\ZZ^m:F^\vee(\bm{c})\neq0}$.
Fix $A\in\RR_{>0}$.
Then uniformly over $Z,N>0$,
we have
\mathd{
\sum_{\bm{c}\in\mcal{S}}
\bm{1}_{\norm{\bm{c}}\leq Z}
\left(\sum_{n\in[N,2N]}\abs{a_{\bm{c},3}(n)}\right)^A
\ll_{A,\eps}
Z^mN^{(1/3+\eps)A}.
}
}

\pf{
If we define $0^0\defeq1$,
then the result trivially holds when $A=0$.
By \Holder over $\bm{c}\in\mcal{S}$,
it thus suffices to prove the (extended) result for $A\in\ZZ_{\geq0}$.
The case $A=0$ is trivial
(as noted already),
so from now on,
assume $A\in\ZZ_{\geq1}$.

Given $\bm{c},n$,
let $n_{\bm{c}}\defeq (n,F^\vee(\bm{c})^\infty)$ and $n^{\bm{c}}\defeq n/n_{\bm{c}}$.
Proposition~\ref{PROP:approx-Phi-past-1/2} implies---uniformly over $(\bm{c},n)\in\mcal{S}\times\ZZ_{\geq1}$---that
$\abs{a_{\bm{c},3}(n)}
\ll_\eps n^\eps
\cdot \bm{1}_{n^{\bm{c}}=\map{sq}(n^{\bm{c}})}
\cdot (n^{\bm{c}}/\map{cub}(n^{\bm{c}}))^{-1/2}$.

If we fix $\bm{c}\in\mcal{S}$,
then it follows that
\mathd{
\sum_{n\in[N,2N]}\abs{a_{\bm{c},3}(n)}
\ll_\eps N^\eps
\psum_{n_{\bm{c}}\leq 2N}
\psum_{n^{\bm{c}}\asymp N/n_{\bm{c}}}
\bm{1}_{n^{\bm{c}}=\map{sq}(n^{\bm{c}})}
\cdot (\map{cub}(n^{\bm{c}})/n^{\bm{c}})^{1/2}.
}
(On the right-hand side,
we think of $n_{\bm{c}},n^{\bm{c}}$ as separate variables,
subject to the constraints $n_{\bm{c}}\mid F^\vee(\bm{c})^\infty$ and $n^{\bm{c}}\perp F^\vee(\bm{c})$.)
But in general,
the sum of $(\map{cub}(a)/a)^{1/2}$ over \emph{square-full} $a\asymp A$
is at most the sum over \emph{cube-full} $d\ll A$ of the quantity
\mathd{
(d/A)^{1/2}\cdot\#\set{\textnormal{square-full}\;a\asymp A
: d\mid a\;\textnormal{and}\;\sqrt{a/d}\in\ZZ}
\ll (d/A)^{1/2}\cdot (A/d)^{1/2}
= 1.
}
Furthermore,
$\#\set{\textnormal{cube-full}\;d\ll A}\asymp A^{1/3}$,
so ultimately we conclude that
\mathd{
\sum_{n\in[N,2N]}\abs{a_{\bm{c},3}(n)}
\ll_\eps N^\eps
\psum_{n_{\bm{c}}\leq 2N}
(N/n_{\bm{c}})^{1/3}
\leq N^{1/3+\eps}
\psum_{n_{\bm{c}}\leq 2N}1.
}

In general,
if $G\in\ZZ_{\neq0}$,
then $\#\set{u\leq 2N:u\mid G^\infty}\ll_{\eps}(2N\cdot G)^\eps$ \cite{heath1998circle}*{antepenultimate display of p.~683}.
If we take
$G\defeq F^\vee(\bm{c})\ll_{F}\norm{\bm{c}}^{O_m(1)}$,
then Corollary~\ref{COR:average-bound-on-Phi_3} immediately follows,
\emph{provided} $2N>Z^\eta$ holds with $\eta\defeq 1/(2A)$,
say.

Now suppose $2N\leq Z^\eta$,
expand $(\smallpsum_{n_{\bm{c}}\leq 2N}1)^A$ as a sum over $u_1,\dots,u_A\leq 2N$,
and switch the order of $\bm{c},\bm{u}$.
Then for each $\bm{u}$,
we have $\map{rad}(u_1\cdots u_A)
\leq u_1\cdots u_A
\leq Z^{A\eta}\leq Z$.
By Lang--Weil
and the Chinese remainder theorem
(and the well-known bound
$O(1)^{\omega(\map{rad}(-))}
\ll_{\varepsilon}\map{rad}(-)^\varepsilon$),
we may therefore
reduce Corollary~\ref{COR:average-bound-on-Phi_3}
to the statement that for all $\varepsilon>0$,
we have
\mathd{
\sum_{u_1,\dots,u_A\leq 2N}
\map{rad}(u_1\cdots u_A)^{\varepsilon-1}
\ll_{A,\varepsilon}
(2N)^{O(A^2\varepsilon)}.
}
To prove this last statement,
observe that
if $u_1,\dots,u_A\leq 2N$ and $\map{rad}(u_1\cdots u_A)=r$,
then $u_1,\dots,u_A\mid r^\infty$
and $r\leq (2N)^A$;
but $\card{\set{u\leq 2N:u\mid r^\infty}^A}\ll_{A,\varepsilon}(2N\cdot r)^{A\varepsilon}$
and $\sum_{r\leq (2N)^A}
(2N\cdot r)^{A\varepsilon}
r^{\varepsilon-1}
\ll_{A,\varepsilon}(2N)^{O(A^2\varepsilon)}$,
since $A\geq1$.
}

\rmk{
Such care to keep Corollary~\ref{COR:average-bound-on-Phi_3}~($\Phi$3E) ``$Z^\eps$-free'' is only important in the proof of Theorem~\ref{THM:generic-HLH-error-bounds}(a),
not in the proof of Theorem~\ref{THM:generic-HLH-error-bounds}(b).
}


\subsection{Computing local averages}
\label{SUBSEC:computing-local-averages}

The local behavior of $\bm{c}\mapsto\tilde{\lambda}_{\bm{c}}(n)$ or $\bm{c}\mapsto\mu_{\bm{c}}(n)$ on average
plays a basic role in our understanding of families of $L$-functions.
% To get a feel for the underlying structure, we compute the first few natural moments by hand.
For reference,
write $L_p(s) = \prod_{i}(1-\tilde{\alpha}_i(p)p^{-s})^{-1}$
and $1/L_p(s) = \prod_{i}(1-\tilde{\alpha}_i(p)p^{-s})$,
so e.g.~
\mathd{
\mu_{\bm{c}}(p)
=-\sum_{i}\tilde{\alpha}_{\bm{c},i}(p)
=-\tilde{\lambda}_{\bm{c}}(p)
}
and
\mathd{
% \tilde{\lambda}_{\bm{c}}(p^2)
% &= \sum_{i\leq j}\tilde{\alpha}_{\bm{c},i}(p)\tilde{\alpha}_{\bm{c},j}(p)
% = \frac{\tilde{\lambda}_{\bm{c}}(p)^2+\sum_{i}\tilde{\alpha}_{\bm{c},i}(p)^2}{2} \\
\mu_{\bm{c}}(p^2)
= \sum_{i<j}\tilde{\alpha}_{\bm{c},i}(p)\tilde{\alpha}_{\bm{c},j}(p)
= \frac{\tilde{\lambda}_{\bm{c}}(p)^2-\sum_{i}\tilde{\alpha}_{\bm{c},i}(p)^2}{2}.
}

To prove that certain averages \emph{exist},
we will assume
Conjectures~\ref{CNJ:(EKL)}~(EKL)
and~\ref{CNJ:(HW2)}~(HW2).
But to satisfactorily \emph{estimate} said averages,
we will take a concrete point-counting approach
(though one could use monodromy groups instead;
see Remark~\ref{RMK:vertical-monodromy} below).
The result is Proposition~\ref{PROP:(LASp)} below.
To concisely state the averages relevant to us,
we first make a convenient archimedean definition:
\defn{
Say that a tuple $\bm{Z}\in\RR^m$ is \emph{$\beta$-lopsided} if $\abs{Z_i}\leq\abs{Z_j}^\beta$ for all $(i,j)\in[m]^2$.
Then view
$\set{\bm{Z}\in\RR^m
:\textnormal{$\bm{Z}$ is $\beta$-lopsided}}$
as a topological subspace of $\RR^m\cup\set{\bm{\infty}}\cong S^m$.
}

\prop{
[LocAv,
or LA]
\label{PROP:(LASp)}
Fix $F$ with $m\geq4$ \emph{even},
and \emph{assume} (EKL) and (HW2).
Now fix $(\bm{a},n_0,\beta,n)\in\ZZ^m\times\ZZ_{\geq1}\times\RR_{\geq1}\times\ZZ_{\geq1}$.
If we restrict $\bm{c}$ to the locus $F^\vee(\bm{c})\neq 0$,
then the following two limits
\emph{exist},
and are \emph{independent of $\beta$}:
\matha{
\bar{\mu}_F^{\bm{a},n_0}(n)
&\defeq \lim_{\bm{Z}\to\bm{\infty}}
\left(\frac{1}{\card{\mcal{B}(\bm{Z})\cap\set{\bm{a}\bmod{n_0}}}}
\cdot \psum_{\bm{c}\equiv\bm{a}\bmod{n_0}}
\bm{1}_{\bm{c}\in\mcal{B}(\bm{Z})}
\cdot \mu_{\bm{c}}(n)\right) \\
\bar{\mu}_{F,2}^{\bm{a},n_0}(n_1,n_2)
&\defeq \lim_{\bm{Z}\to\bm{\infty}}
\left(\frac{1}{\card{\mcal{B}(\bm{Z})\cap\set{\bm{a}\bmod{n_0}}}}
\cdot \psum_{\bm{c}\equiv\bm{a}\bmod{n_0}}
\bm{1}_{\bm{c}\in\mcal{B}(\bm{Z})}
\cdot \mu_{\bm{c}}(n_1)\mu_{\bm{c}}(n_2)\right).
}
Furthermore,
for some absolute constants $\delta,\delta'>0$,
the following statements hold.
\begin{enumerate}[{label=(LA\arabic*)}]
    \item The function
    $n\mapsto \bar{\mu}_F^{\bm{a},n_0}(n)$
    is \emph{multiplicative}.
    For $\Re(s)\geq 1/2-\delta$,
    we have
    \mathd{
    \sum_{l\geq0}p^{-ls}\bar{\mu}_F^{\bm{a},n_0}(p^l)
    = 1+(\tilde{\lambda}_V(p)p^{-s-1/2}+p^{-2s})+O(p^{O(v_p(n_0)-1-\delta'}),
    }
    uniformly over $p,s,\bm{a},n_0,\beta$.
    
    \item The function
    $(n_1,n_2)\mapsto
    \bar{\mu}_{F,2}^{\bm{a},n_0}(n_1,n_2)$
    is \emph{multiplicative}
    (i.e.~if $\gcd(n_1n_2,n'_1n'_2)=1$,
    then $\bar{\mu}_{F,2}^{\bm{a},n_0}(\bm{n})\bar{\mu}_{F,2}^{\bm{a},n_0}(\bm{n}')
    = \bar{\mu}_{F,2}^{\bm{a},n_0}(\bm{n}\bm{n}')$).
    For $\Re(s)\geq 1/2-\delta$,
    we have
    \mathd{
    \sum_{\bm{l}\geq0}p^{-\bm{l}\cdot\bm{s}}\bar{\mu}_{F,2}^{\bm{a},n_0}(p^{\bm{l}})
    = 1+p^{-s_1-s_2}
    +\sum_{j\in[2]}(\tilde{\lambda}_V(p)p^{-s_j-1/2}+p^{-2s_j})
    + O(p^{O(v_p(n_0))-1-\delta'}),
    }
    uniformly over $p,s,\bm{a},n_0,\beta$.
\end{enumerate}
}

\defn{
% [Default $\bm{a},n_0$]
Let $\bar{\mu}_F(n)\defeq \bar{\mu}_F^{\bm{0},1}(n)$,
and $\bar{\mu}_{F,2}(n_1,n_2)\defeq \bar{\mu}_{F,2}^{\bm{0},1}(n_1,n_2)$.
}

\defn{
\label{DEFN:singular-c's-local-coefficient-0-convention}
For $p$-adic calculations only,
we use the convenient convention
``$\mu_{\bm{c}},a_{\bm{c},1},a_{\bm{c},2},a_{\bm{c},3}\defeq0$''
for all $\bm{c}\in\ZZ_p^m$ with $F^\vee(\bm{c})=0$.
(In particular,
for such $\bm{c}$'s,
we set $\mu_{\bm{c}}(1)\defeq0$,
etc.,
so $\mu_{\bm{c}},a_{\bm{c},1},a_{\bm{c},2},a_{\bm{c},3}$ are \emph{not} multiplicative in the standard sense.)
}

\pf{
[Proof of existence and multiplicativity of limits]
By (EKL),
\cite{wang2021_HLH_vs_RMT}*{Corollary~6.3},
and the bound $\abs{\tilde{\alpha}(p)}\leq1$ (from GRC),
it is routine to show
that the $p$-adic averages
\mathd{
\EE_{\bm{c}\in\ZZ_p^m\cap\set{\bm{a}\bmod{n_0}}}
[\mu_{\bm{c}}(p^l)]
\quad\textnormal{and}\quad
\EE_{\bm{c}\in\ZZ_p^m\cap\set{\bm{a}\bmod{n_0}}}
[\mu_{\bm{c}}(p^{l_1})\mu_{\bm{c}}(p^{l_2})]
}
are well-defined,
and furthermore
(by the Chinese remainder theorem)
that these $p$-adic averages determine $\bar{\mu}_F^{\bm{a},n_0}(n),\bar{\mu}_{F,2}^{\bm{a},n_0}(n_1,n_2)$ in the obvious way.
(Note that in $\ZZ^m$,
the locus $F^\vee(\bm{c})=0$ has ``density $0$'' in the boxes defining $\bar{\mu}_F^{\bm{a},n_0}(n),\bar{\mu}_{F,2}^{\bm{a},n_0}(n_1,n_2)$.)
}

\pf{
[Beginning of proof of required estimates]
If $\delta'\leq1$, say,
then by GRC,
we may ``absorb'' the case $p\mid n_0$ into the error term $O(p^{O(v_p(n_0))}p^{-1-\delta'})$.
Now assume $p\nmid n_0$.
If $3\cdot (1/2-\delta)\geq 1+\delta'$,
% If $\delta,\delta'$ are sufficiently small, then in the local Dirichlet series over $l,\bm{l}$,
% the terms with $l\geq 3$ or $\abs{\bm{l}}\geq 3$
% all fit in the error term $O(p^{-1-\delta'})$.
then we are left with analyzing the contributions from
$l\in\set{1,2}$ in (LA1),
and $\abs{\bm{l}}\in\set{1,2}$ in (LA2).
In other words,
up to re-defining $\delta,\delta'$,
we must prove the following.
\begin{enumerate}[(1)]
    \item $\EE_{\bm{c}\in\ZZ_p^m}[\mu_{\bm{c}}(p)]
    = \tilde{\lambda}_V(p)p^{-1/2}
    + O(p^{-1/2-\delta'})$,
    for $l=1$.
    
    \item $\EE_{\bm{c}\in\ZZ_p^m}[\mu_{\bm{c}}(p^2)]
    = 1 + O(p^{-\delta'})$,
    for $l=2$.
    (Cf.~``essentially symplectic'' in \S\ref{SUBSEC:discuss-expected-RMT-statistical-nature-of-our-families-of-L-functions},
    which begins near p.~\pageref{SUBSEC:discuss-expected-RMT-statistical-nature-of-our-families-of-L-functions}.)
    
    \item $\EE_{\bm{c}\in\ZZ_p^m}[\mu_{\bm{c}}(p)^2]
    = 1 + O(p^{-\delta'})$,
    for $\bm{l}=(1,1)$.
    (Cf.~``essentially cuspidal and self-dual''.)
\end{enumerate}
(Note that (1)--(2) ``cover''
not just the cases ``$l\in\set{1,2}$'' in (LA1),
but also the cases ``$\abs{\bm{l}}\in\set{1,2}$ with $l_1l_2=0$'' in (LA2).)

But in each of (1)--(3),
the locus $p\mid F^\vee(\bm{c})$ fits in an $O(p^{-1})$ error term,
by Lang--Weil and GRC.
We may then restrict to $p\nmid F^\vee(\bm{c})$,
in which case
$\mu_{\bm{c}}(p)
= -\tilde{\lambda}_{\bm{c}}(p)
\defeq -\tilde{\tr}(\map{Frob}\vert_{H^{m_\ast}})
= \wt{E}_{\bm{c}}(p)$
and $\mu_{\bm{c}}(p)^2
= \wt{E}_{\bm{c}}(p)^2$,
while $\mu_{\bm{c}}(p^2)
= \frac12(\tilde{\lambda}_{\bm{c}}(p)^2-\sum_{i}\tilde{\alpha}_{\bm{c},i}(p)^2)
= \frac12(\wt{E}_{\bm{c}}(p)^2+\wt{E}_{\bm{c}}(p^2))$.\footnote{If $p\nmid F^\vee(\bm{c})$,
then $\mu_{\bm{c}}(p^2)=(-1)^2\tilde{\lambda}_{V_{\bm{c}},\bigwedge^2}(p)$,
but we only need this $\bigwedge^2$ interpretation elsewhere.}
(Note that these formulas for $\mu_{\bm{c}}(p^\bullet)$
are only fully correct when $m_\ast$ is \emph{odd}.)

It thus remains to prove the following;
but these follow from Corollary~\ref{COR:smooth-locus-moment-calculations}.
\begin{enumerate}[{label=(\alph*)}]
    \item $\EE_{\bm{c}\in\FF_p^m}[\wt{E}_{\bm{c}}(p)
    \bm{1}_{p\nmid F^\vee(\bm{c})}]
    = \tilde{\lambda}_V(p)p^{-1/2}
    + O(p^{-1/2-\delta'})$.
    
    \item $\EE_{\bm{c}\in\FF_p^m}[\wt{E}_{\bm{c}}(p^2)
    \bm{1}_{p\nmid F^\vee(\bm{c})}]
    = 1 + O(p^{-\delta'})$.
    
    \item $\EE_{\bm{c}\in\FF_p^m}[\wt{E}_{\bm{c}}(p)^2
    \bm{1}_{p\nmid F^\vee(\bm{c})}]
    = 1 + O(p^{-\delta'})$.
\end{enumerate}
(Note here that (a) implies (1),
and (c) implies (3),
while (b)--(c) imply (2).)
}

\subsection{``Deriving'' the Ratios Conjectures}
\label{SUBSEC:applying-CFZ-ratios-conjecture-recipe}

Assume the hypotheses,
(EKL) and (HW2),
of Proposition~\ref{PROP:(LASp)}~(LocAv).
Then for the ``ratio'' $1/L$
(or for ``pure products'' thereof),
the recipe \cite{conrey2008autocorrelation}*{\S5.1}---carried over
% Google dictionary for "carry over" - "extend beyond the normal or original area of application" or "retain something and apply or deal with it in a new context."
% https://www.macmillandictionary.com/us/dictionary/british/carry-over_1 - "if something carries over or is carried over from one situation into another, it has the same effect in the new situation as it had in the old one"
to the general setting of \cite{sarnak2016families}*{pp.~534--535, Geometric Families and Remark~(i)}---now
makes sense for the families $\bm{c}\mapsto\pi_{\bm{c}}$ underlying Conjectures~\ref{CNJ:(R1)--(R2)} and~\ref{CNJ:(RA1)}.
% https://books.google.com/ngrams/graph?content=help+them+accordingly%2C+accordingly+help+them&year_start=1800&year_end=2019&corpus=26&smoothing=3&direct_url=t1%3B%2Chelp%20them%20accordingly%3B%2Cc0#t1%3B%2Chelp%20them%20accordingly%3B%2Cc0
We will soon ``derive'' these two conjectures accordingly.

\rmk{
If one were willing to directly apply a similar ``recipe'' to one of $\Phi_1\Phi_2,\Phi^{\map{HW}},\Phi$
(and not just $1/L$ or $\Phi_1$),
then our work would simplify correspondingly.
But it is unclear how generally a naive recipe like that could hold,
without a clear supporting model like a classical random matrix ensemble.
}

\rmk{
\label{RMK:technical-points-regarding-ratios-heuristic-in-SST-type-geometric-setting}
In our specific ``geometric'' Ratios Conjectures,
(i) we are assuming (EKL), not just (HW2);
(ii) the $L$-functions $L(s,V_{\bm{c}})$ are not all primitive,
as \cite{conrey2005integral} explicitly requires,
and \cite{conrey2008autocorrelation} thus implicitly requires;
and (iii) we do not order our families by conductor
(or by discriminant, for that matter).
This is all OK:
% for the following reasons:
\begin{enumerate}[(1)]
    \item Regarding (i),
    we use (EKL) here (in \S\ref{SUBSEC:applying-CFZ-ratios-conjecture-recipe})
    merely to prove that certain local averages \emph{exist}.
    This is technically important,
    but far from the heart of RMT-type matters.
    % So point~(i) seems to point to
    % a literature gap,
    % more so than a fundamental obstacle.
    In fact,
    it would be harmless to explicitly include (EKL) as a hypothesis in Conjectures~\ref{CNJ:(R1)--(R2)} and~\ref{CNJ:(RA1)},
    if one wanted to do so.
    
    \item Point~(ii) is minor.
    Using (HW2),
    one can show that
    $L(s,V_{\bm{c}})$ is primitive
    for all $\bm{c}$'s outside a fairly sparse set.
    (See ``essentially cuspidal'' in \S\ref{SUBSEC:discuss-expected-RMT-statistical-nature-of-our-families-of-L-functions},
    which begins near p.~\pageref{SUBSEC:discuss-expected-RMT-statistical-nature-of-our-families-of-L-functions}.)
    So for our purposes
    (locally and globally),
    the question of whether to
    include all $L(s,V_{\bm{c}})$'s,
    or only primitive ones,
    is essentially cosmetic.
    
    Alternatively,
    it may well be that primitivity is not essential, even on average;
    see the sentence ``Non-primitive families can also be handled\dots'' on \cite{conrey2005integral}*{p.~34}.
    
    \item Regarding (iii),
    we are still following the RMT-based philosophy underlying \cite{conrey2008autocorrelation}.
    We are just indexing by different level sets,
    as is natural for multi-parameter families like ours;
    cf.~\cite{sarnak2016families}*{p.~535, Remark~(i); and p.~560, second paragraph after (25)}.
    
    For a more thorough discussion of the expected RMT-type models for our families,
    see \S\ref{SUBSEC:discuss-expected-RMT-statistical-nature-of-our-families-of-L-functions},
    which begins near p.~\pageref{SUBSEC:discuss-expected-RMT-statistical-nature-of-our-families-of-L-functions}.
\end{enumerate}
}

\subsubsection{Deriving (R1)--(R2)}
\label{SUBSUBSEC:deriving-(R1)--(R2)}

To ``derive'' Conjecture~\ref{CNJ:(R1)--(R2)}(R1),
replace each term
\mathd{
L(s,\pi_{\bm{c}})^{-1}
= \sum_{n\geq 1}\mu_{\bm{c}}(n)n^{-s}
}
on the ``left-hand side of (R1)'' with
its ``naive expected value over $\bm{c}$'', i.e.~
\mathd{
\sum_{n\geq 1}\bar{\mu}_F(n)n^{-s}
= \prod_p\sum_{l\geq0}p^{-ls}\bar{\mu}_F(p^l)
= \prod_p\left[1+(\tilde{\lambda}_V(p)p^{-s-1/2}+p^{-2s})+O(p^{-1-\delta'})\right]
}
for $\Re(s)\geq 1/2-\delta$.
This ``naive average'' factors as $\zeta(2s)L(s+1/2,V)A_F(s)$,
for a certain Euler product $A_F(s)$
that converges absolutely
on the half-plane $\Re(s)\geq 1/2-\delta$.

To ``derive'' Conjecture~\ref{CNJ:(R1)--(R2)}(R2),
similarly replace
\mathd{
L(s_1,\pi_{\bm{c}})^{-1}L(s_2,\pi_{\bm{c}})^{-1}
= \sum_{n_1,n_2\geq 1}\mu_{\bm{c}}(n_1)\mu_{\bm{c}}(n_2)\bm{n}^{-\bm{s}},
}
for $\Re(\bm{s})\geq 1/2-\delta$, with
\mathd{
% \sum_{\bm{n}\geq\bm{1}}\bar{\mu}_{F,2}(n_1,n_2)\bm{n}^{-\bm{s}}
\prod_p\sum_{\bm{l}\geq0}p^{-\bm{l}\cdot\bm{s}}\bar{\mu}_{F,2}(p^{\bm{l}})
= \prod_p\left[1+p^{-s_1-s_2}
+\sum_{j\in[2]}(\tilde{\lambda}_V(p)p^{-s_j-1/2}+p^{-2s_j})
+ O(p^{-1-\delta'})\right],
}
which factors as $A_{F,2}(\bm{s})\zeta(s_1+s_2)\prod_{j\in[2]}\zeta(2s_j)L(s_j+1/2,V)$.

\rmk{
Here $\zeta(2s),L(s+1/2,V),\zeta(s_1+s_2),\dots$ are called \emph{polar factors}.
}

\rmk{
When there are $L$'s in the numerator,
the recipe must be modified using the approximate functional equation for $L$.
(In particular,
the root number, gamma factor, and conductor
play a visible role on average,
whereas in (R1)--(R2) they seem to be be hidden in the $\sigma$-dependence,
i.e.~how close $s$ is allowed to be to the critical line.)
}

\rmk{
Under a quasi-GRH for $\zeta(s),L(s,V)$,
the errors in (R1)--(R2) remain essentially the same
(up to a factor of $O_{\sigma,\bm{\sigma},\eps}(Z^{\hbar\eps})$)
even after dividing both sides of (R1) by $\zeta(2s)L(s+1/2,V)$,
or both sides of (R2) by $\prod_{j\in[2]}\zeta(2s_j)L(s_j+1/2,V)$.
}


\subsubsection{Deriving (RA1)}
\label{SUBSUBSEC:deriving-(RA1)}

To ``derive'' Conjecture~\ref{CNJ:(RA1)}~(RA1),
replace each term
\mathd{
L(s,\pi_{\bm{c}})^{-1}
= \sum_{n\geq 1}\mu_{\bm{c}}(n)n^{-s}
}
on the ``left-hand side of (RA1)'', for $\Re(s)\geq 1/2-\delta$, with
its ``naive expected value over $\bm{c}\equiv\bm{a}\bmod{n_0}$'', i.e.~
\mathds{
\sum_{n\geq 1}\frac{\bar{\mu}_F^{\bm{a},n_0}(n)}{n^s}
&= \prod_p\sum_{l\geq0}\frac{\bar{\mu}_F^{\bm{a},n_0}(p^l)}{p^{ls}} \\
&= \prod_p\left[1+(\tilde{\lambda}_V(p)p^{-s-1/2}+p^{-2s})+O(p^{O(v_p(n_0))-1-\delta'})\right].
}
This ``naive average'' factors as $\zeta(2s)L(s+1/2,V)A_F^{\bm{a},n_0}(s)$,
for a certain Euler product $A_F^{\bm{a},n_0}(s)$
that converges absolutely
on the half-plane $\Re(s)\geq 1/2-\delta$.

\section{Main conditional results}

Let $\Sigma_\textnormal{gen}$ denote the contribution to the right-hand side of eq.~\eqref{EQN:normalized-delta-method} from the locus $F^\vee(\bm{c})\neq 0$.
(Explicitly, $\Sigma_\textnormal{gen}\defeq X^{m-3}\sum_{n\geq1}
\sum_{\bm{c}\in\ZZ^m} \bm{1}_{F^\vee(\bm{c})\neq 0}\cdot n^{-(m-1)/2}\wt{S}_{\bm{c}}(n)\wt{I}_{\bm{c}}(n)$.)

\thm{
[\cite{wang2021_HLH_vs_RMT}*{Theorem~3.49}]
\label{THM:generic-HLH-error-bounds}
Fix $F$ diagonal with $m\in\set{4,6}$,
and fix $w\in C^\infty_c(\RR^m)$ with $(F,w)$ clean.
Assume $\lcm(\bm{F})$ is cube-free.
Then
\begin{enumerate}[(a)]
    \item $\abs{\Sigma_\textnormal{gen}}\ll X^{3(m-2)/4}$ holds
    \emph{under (HWSp), (SFSC$_{q,6}$+), and (R2')};
    and
    
    \item $\abs{\Sigma_\textnormal{gen}}\ll X^{3(m-2)/4-\Omega(1)}$ holds
    \emph{under (HWSp), (SFSC$_{q,6}$+), (RA1), and (EKL)}.
\end{enumerate}
}

\pf{
[Proof sketch]
We use (SFSC$_{q,6}$+) precisely to ensure that Conjecture~\ref{CNJ:bad-sum-average}~(B3) holds (see \S\ref{SEC:SFSC-and-Sarnak-Xue-type-average-sum-bound}).
Now factor $\Phi$ using Definition~\ref{DEFN:factor-Phi^HW-into-Phi_1-Phi_2-Phi_3}.
For (a), begin with a framework for ``restriction and separation'' going beyond Remark~\ref{RMK:using-partial-summation-to-factor-out-integral} (see \cite{wang2021_HLH_vs_RMT}*{\S5} for details), and then use \Holder appropriately between ``good'' and ``bad'' factors;
some important ingredients are (B3), Lemma~\ref{LEM:uniform-[J1]--[J4]}, Conjecture~\ref{CNJ:(R2')}~(R2'), \cite{wang2021_HLH_vs_RMT}*{Proposition~7.27}, and Corollary~\ref{COR:average-bound-on-Phi_3}~($\Phi$3E).
For (b), we handle some ranges (namely those with large ``error moduli'') as in (a).
Over the remaining ranges, we then decompose $\Sigma_\textnormal{gen}$ ``adelically'' into pieces---based on the polynomial $H$ from Conjecture~\ref{CNJ:(EKL)}~(EKL)---up to a small exceptional set constructed in \cite{wang2021_HLH_vs_RMT}*{\S7.8} by algorithmic tree-like means.
We then estimate these pieces via local calculations and Poisson summation.
}

\rmk{
In Theorem~\ref{THM:generic-HLH-error-bounds}(b),
the power saving $\Omega(1)$ deserves some clarification,
because the situation here
is not as clear-cut as
that in Corollary~\ref{COR:Disc-vanishing-sum-with-explicit-linear-density}.
\begin{enumerate}[(1)]
    \item The saving $\Omega(1)$ can be safely taken to be independent of $w$:
    no \emph{exponent} anywhere in \cite{wang2021_HLH_vs_RMT}
    truly depends on the weight $w$.
    (In particular,
    (SFSC), (RA1), and (EKL)
    depend only on $F$, not on $w$.)
    
    \item \emph{If} (SFSC), (RA1), and (EKL)
    are assumed to be sufficiently ``exponent-uniform'' over $F$,
    then one can take $\Omega(1)$ to be independent of $F$
    (but still dependent on $m$).
\end{enumerate}
Regarding the implied constant in Theorem~\ref{THM:generic-HLH-error-bounds}(b),
one could probably work out an explicit $w$-dependence
of the form
$(1+\diam(\Supp{w})+\norm{w}_{O_F(1),\infty})^{O_F(1)}$ (where $\norm{w}_{k,\infty}$ denotes a Sobolev norm),
given enough patience.
An explicit $F$-dependence
would probably take even more patience.
}

\thm{
[\cite{wang2021_HLH_vs_RMT}*{Theorem~3.39}]
\label{THM:HLH-and-3-cubes-application}
Fix $F$ diagonal with $m=6$.
Assume $\lcm(\bm{F})$ is cube-free.
\begin{enumerate}[(a)]
    \item Say $F=x_1^3+\dots+x_6^3$,
    and \emph{assume (HWSp), (SFSC$_{q,6}$+), and (R2')}.
    Then $N_F(X)\ll X^3$ holds as $X\to\infty$.
    Therefore,
    a positive fraction of integers
    lie in $\set{x^3+y^3+z^3: (x,y,z)\in \ZZ_{\geq 0}^3}$.
    % can be represented as a sum of three nonnegative integer cubes.
    
    \item Alternatively,
    % Alternatively / Separately / Independently (i.e. dropping the assumptions of (a), and starting afresh)
    \emph{assume (HWSp), (SFSC$_{q,6}$+), (RA1), and (EKL)}.
    Then for any \emph{given} $w\in C^\infty_c(\RR^m)$ with $(F,w)$ clean,
    the pair $(F,w)$ is HLH
    (with a power saving),
    in the sense of Definition~\ref{DEFN:HLH-asymptotic}.
    Therefore,
    \begin{enumerate}[(i)]
        \item the Hasse principle holds for $V/\QQ$;
        and also
        
        \item asymptotically $100\%$ of integers $a\not\equiv\pm4\bmod{9}$
        lie in $\set{x^3+y^3+z^3: (x,y,z)\in \ZZ^3}$,
        % can be represented as a sum of three integer cubes.
        if $F=x_1^3+\dots+x_6^3$.
    \end{enumerate}
\end{enumerate}
}

\pf{
[Proof of Theorem~\ref{THM:HLH-and-3-cubes-application}(a) assuming Theorem~\ref{THM:generic-HLH-error-bounds}(a)]

By a \Holder argument (see \cite{wang2021_HLH_vs_RMT} for details), $N_F(X)\ll X^3$.
Consequently,
% if we take $F=x_1^3+\dots+x_6^3$, then
Observation~\ref{OBS:3Z^odot3-lower-bound-by-Cauchy} implies \emph{positive lower density} of $\set{x^3+y^3+z^3: (x,y,z)\in \ZZ_{\geq 0}^3}$.
% \footnote{The same holds for positive-density subsets of $\ZZ_{\geq 0}^3$.}
}

\pf{[Proof of Theorem~\ref{THM:HLH-and-3-cubes-application}(b) assuming Theorem~\ref{THM:generic-HLH-error-bounds}(b)]

Using Corollary~\ref{COR:Disc-vanishing-sum-with-explicit-linear-density}---and the cleanliness assumption in Theorem~\ref{THM:HLH-and-3-cubes-application}(b)---we find that
Theorem~\ref{THM:generic-HLH-error-bounds}(b) directly implies HLH for $(F,w)$,
in fact with (an unnecessary) power saving.
Theorem~\ref{THM:HLH-and-3-cubes-application}(b)(i) then immediately follows,
upon choosing a weight $w\in C^\infty_c(\RR^6)$ with $(F,w)$ clean and $c_{\textnormal{HLH},F,w}>0$ (doable by hand, or via Proposition~\ref{PROP:localize-away-from-hessian}).
On the other hand,
Theorem~\ref{THM:HLH-and-3-cubes-application}(b)(ii) follows from Theorem~\ref{THM:enough-HLH-implies-100pct-Hasse-principle-for-3Z^odot3}
(essentially due to \cite{diaconu2019admissible}).
}

\rmk{
See the introductions to \cites{heath1999solubility,heath2007cubic} for
some history,
and (what is likely)
the state of affairs,
on the Hasse principle for diagonal cubic forms
(although \cite{heath2007cubic} also discusses non-diagonal cubic forms)---with
the unconditional record being $m=7$,
due to \cite{baker1989diagonal}.
% We refer the reader to the introduction of \cite{heath1999solubility} for
% (what is likely)
% the state of affairs regarding the Hasse principle for diagonal cubic forms---with the record $m=7$ due to \cite{baker1989diagonal}---save for \cite{swinnerton2001solubility}'s work
% conditional on the finiteness of Tate--Shafarevich groups
% (over quadratic extensions of $\QQ$)
% for certain families of elliptic curves.
% % Note Dec 12, 2020: Bombieri's RH official statement says 5 variables is known under GRH, but this might be a typo? Mention it.
(Over global function fields of characteristic $\geq 7$,
the record is $m\geq 6$,
achieved by geometric techniques \cite{tian2017hasse}.)
}

\rmk{
In Theorem~\ref{THM:HLH-and-3-cubes-application} (and in Theorem~\ref{THM:generic-HLH-error-bounds}),
one could relax the assumption (HWSp) to (HW2),
% by slightly tedious means
% (briefly sketched in Appendix~\ref{SUBSEC:replacing-(HWSp)-input-with-(HW2)}).
% at the cost of cleanliness
but that would muddy the proof,
% more than we would like
with little benefit.
(For a brief sketch of the necessary modifications,
see \cite{wang2021_HLH_vs_RMT}*{Appendix~A.4}.)
}

\rmk{
In Theorem~\ref{THM:HLH-and-3-cubes-application} (and in Theorem~\ref{THM:generic-HLH-error-bounds}),
I expect that the assumption (SFSC$_{q,6}$+) can be relaxed to (SFSC$_{p,3}$); see Remark~\ref{RMK:Ekedahl-geometric-sieve-reduce-range?}.
Furthermore, I believe (the proof of) Theorem~\ref{THM:generic-HLH-error-bounds} would directly generalize to $\PP^{m-1}_\QQ$-smooth $F$ if one replaced the assumption (SFSC$_{q,6}$+) with Conjecture~\ref{CNJ:bad-sum-average}~(B3);
however, to generalize (the proof of) Theorem~\ref{THM:HLH-and-3-cubes-application} accordingly, one would also need to generalize Corollary~\ref{COR:Disc-vanishing-sum-with-explicit-linear-density}.
}

\rmk{
I expect that with a lot of additional technical work (cf.~Remark~\ref{RMK:integral-estimates-without-cleanliness-assumption}), one could replace the cleanliness condition in Theorem~\ref{THM:HLH-and-3-cubes-application}(b) with the condition that $(F,w)$ be smooth.
(If successful, this would, in particular, conditionally imply the original HLH conjecture of Hooley from Example~\ref{EX:6-cubes-zero-locus}.)
I suspect that if $(F,w)$ is smooth, then in the key ``generic range'' where $\abs{c_1},\dots,\abs{c_m}$ are all roughly of size $X^{1/2}$ (up to a factor of $X^{\pm\delta}$), a slightly deformed version of Lemma~\ref{LEM:uniform-[J1]--[J4]} should hold (because in the critical range $\abs{u}\asymp \norm{\bm{v}}\gg 1$, the zeros of $u\grad{F}(\bm{x})-\bm{v}$ would then lie relatively far from $(\map{hess}{F})_\RR(\RR)$), and thus the proof of Theorem~\ref{THM:generic-HLH-error-bounds} should remain adequate.
(The complementary ``non-generic range'' of $\bm{c}$'s can be handled under GRH, following e.g.~\S\ref{SEC:using-L-function-hypotheses-on-average}.)
}

\section{Supplementary material on \texpdf{$L$}{L}-functions}
% Complementary commentary on conjectural aspects
% Complementary commentary on conjectural aspects of the paper
\label{SEC:complementary-commentary-on-conjectural-aspects-and-friends}
% A detailed commentary on... --> Complementary commentary on... (seems more realistic / less burdensome)
%comments/commentary/reaction/notes on conjectural features/aspects/facets/angles/sides (of or related to our paper/analysis)
%Diving deeper into our standard conjectures (and related conjectures?)
%A deeper dive into our standard conjectures
% Deeper discussion of our Langlands-based hypotheses

\subsection{A discussion of (HW2) and (HWSp)}
\label{SUBSEC:discuss-individual-L-function-aspects-(HW2)-and-(HWSp)}

\subsubsection{Generalities}

Fix $F$ with $m\geq3$.
(Here we \emph{allow arbitrary $m\geq3$ and $\PP^{m-1}_\QQ$-smooth $F$},
until further notice.)
Fix a tuple $\bm{c}\in \ZZ^m$ with $F^\vee(\bm{c})\neq 0$.
In the notation of Definition~\ref{DEFN:L(s,V_c)-and-coefficients-and-eigenvalues} and Conjecture~\ref{CNJ:(HW2)}~(HW2),
then define $m_\ast,M_{\bm{c}},M_V$ and fix $(M,S)$.

\rmk{
In Definition~\ref{DEFN:geometric-L-functions},
we fixed an auxiliary pair $(\ell_0,\iota)$.
But we could have instead used Serre's definition of Hasse--Weil $L$-functions
(and its extension to motives),
% https://en.wikipedia.org/wiki/Motivic_L-function
where one \emph{assumes} ``$\ell$-independence of the local factors $L_p$''
(known for us by
\cite{laskar2017local}*{Corollary~1.2},\footnote{which is
based in part on Scholze's results on the weight-monodromy conjecture}
since our $M$'s are ``tensor-generated'' by smooth projective hypersurfaces).
% (known at least for diagonal $F$ with $m\in\set{4,6}$,
% due to a connection with algebraic Hecke characters if $M=M_V$,
% and with abelian varieties if $M\neq M_V$).
% (which is likely known for the $\ell$-adic representations we are interested in, at least for $F$ diagonal with $m\in\set{4,6}$).
Note that \cites{hooley1986HasseWeil,heath1998circle} both defined $L(s,V_{\bm{c}})$ following Serre;
for $m\in\set{4,6}$,
the $\ell$-independence of the $L_p(s,V_{\bm{c}})$'s was already known at the time,
due to connections with abelian varieties;
cf.~\cite{taylor2004galois}*{paragraph after Conjecture~1.2}.
}

\rmk{
It is conjectured that $M$ should be semi-simple (as an $\ell_0$-adic representation of $G_\QQ$),
in which case \cite{taylor2004galois}*{p.~100, prior to Conjectures~3.4--3.5} precisely ``specifies'' (HW2)'s putative $\pi_M$.
Semi-simplicity is known at least
for $M=M_{\bm{c}}$ when $m\in\set{4,6}$,
and thus
(by a general representation-theoretic argument\footnote{using
(i) a representation-theoretic passage
from a \emph{topological} subgroup $H\belongs\GL_{\dim M_{\bm{c}}}(\QQ_{\ell_0})$
to its \emph{Zariski} closure in $\GL_{\dim M_{\bm{c}}}/\QQ_{\ell_0}$,
based on the relative \emph{coarseness} of the Zariski topology;
and (ii) \cite{milne2017algebraic}*{Corollary~22.44},
a general result on \emph{algebraic} groups over fields of characteristic zero})
for all $M$'s ``tensor-generated'' by such $M_{\bm{c}}$'s.
}

We now briefly elaborate on the (conjectural) rationale for each part of (HW2).
\begin{enumerate}[(a)]
    \item By definition,
    $M_{\bm{c}},M_V$ arise from geometry,
    pure of weight $m_\ast,1+m_\ast$,
    respectively.
    And if $M\neq M_{\bm{c}},M_V$,
    then by the K\"{u}nneth $G_\QQ$-isomorphism,
    $M$ is a subquotient of $H^{2m_\ast}((V_{\bm{c}})_{\ol{\QQ}}\times (V_{\bm{c}})_{\ol{\QQ}}, \QQ_{\ell_0})$---whence $M$ arises from geometry,
    pure of weight $2m_\ast$.
    
    It is known that $\dim M_{\bm{c}},\dim M_V\ll_{m}1$
    (uniformly over $\bm{c}$),
    so $\dim M\ll_{m}1$ in general as well.
    
    Also,
    $M$ is unramified away from $S$
    (by smooth proper base change),
    so we should have $q(M)\mid\map{rad}(S)^{O_m(1)}$---trivially if $M=M_V$,
    and by the ``bounded depth in families'' observation of \cite{sarnak2016families}*{\S2.11} if $M\neq M_V$.
    
    \item The \emph{finiteness} of the \emph{set} of gamma factors $L_\infty(s,M)$
    (for a \emph{given} value of $m$,
    as $\bm{c}$ varies)
    should follow at least formally from Hodge theory;
    see \cite{taylor2004galois}*{p.~79, definition of $\map{HT}(-)$;
    p.~86, definition of $\Gamma(-,s)$; and
    p.~80, Conjecture~1.2}.
    
    \item This is a precise instance of the \emph{Langlands reciprocity conjecture};
    cf.~\cite{sarnak2016families}*{Conjecture~4}.
    See also \cite{sarnak2016families}*{pp.~534--535, Geometric Families} for some context.
    
    \item We say that a cuspidal $\pi$ \emph{satisfies GRC} if
    it is tempered at all places.
    If each cuspidal constituent of the putative isobaric $\pi_M$
    is tempered,
    then
    $\abs{\tilde{\alpha}_{M,j}(p)}\leq 1$ for all $p,j$,
    and also $L_\infty(s,M)$ is holomorphic on $\set{\sigma>0}$;
    but the implication,
    even in the cuspidal case,
    may not literally be an equivalence
    (cf.~\cite{farmer2019analytic}*{Axiom~4 and Lemma~3.2}).
    
    \item We say that $L(s,\pi_M)$ \emph{satisfies GRH} if all of its zeros in the region $\Re(s)>0$ lie on the line $\Re(s)=1/2$.
    (Under our ``assumptions'' on $\pi_M$ from (HW2)(c)--(d),
    it would be \emph{equivalent} to require GRH to hold for all ``cuspidal constituent'' $L$-functions.
    This is because all such ``constituents'' are known to be
    % https://mathoverflow.net/questions/168884/standard-zero-free-region-of-automorphic-l-function-on-gln
    zero-free for $\Re(s)\geq 1$,
    by \cite{iwaniec2004analytic}*{Theorem~5.42}.)
\end{enumerate}

% \todo[inline]{Be more careful in (HW2), (HWSp) about possible zeros at $1$.
% Maybe need to write out isobaric sum / state stuff like GRH for each cuspidal constituent more carefully.
% EDIT 3/14/2021: Should be resolved now...}

\rmk{
\label{RMK:standard-consequences-of-(HW2)}
(HW2) and Godement--Jacquet imply that $L(s,M)$
has finite order
and a standard functional equation
(with $\abs{\eps(M)}=1$, etc.),
and is holomorphic except possibly for poles at $s=1$ corresponding to trivial constituents of $\pi_M$.
(Here the \emph{unitarity} in Definition~\ref{DEFN:nice-isobaric-pi's} restricts poles to $\Re(s)=1$,
and the \emph{finiteness} restricts poles to $s=1$.)
% \footnote{\cites{hooley1986HasseWeil,hooley_greaves_harman_huxley_1997,heath1998circle} \emph{directly} assumed similar analytic properties.
% We could have done the same when stating (HW2).
% But it is arguably more natural to assume automorphy,
% % https://en.wikipedia.org/wiki/Converse_theorem
% in view of ``analytic-to-automorphic'' converse theorems in certain natural families of Dirichlet series.}
% 
Together with (HW2)(b),
GRC, and GRH,
these analytic properties
imply the uniform estimate
$1/L(s,M)\ll_{m,\eps}q(M)^\eps(1+\abs{s})^\eps$ over $\bm{c},M,s$ with $\Re(s)\geq1/2+\eps$;
see e.g.~\cite{iwaniec2004analytic}*{Theorem~5.19 and the ensuing paragraph}.
}

We now explain the (conjectural) rationale for (HWSp).
\begin{enumerate}[(1)]
    \item The first part of (HWSp) is (HW2),
    which we have already explained.
    
    \item For simplicity, say $m\geq 4$, i.e.~$m_\ast\geq 1$.
    Given $X\in\set{V_{\bm{c}},V}$ of dimension $d\in\set{m_\ast,1+m_\ast}$,
    consider the cup-product pairing
    \mathd{
    \psi\maps
    H^d(X_{\ol{\QQ}},\QQ_{\ell_0})
    \times H^d(X_{\ol{\QQ}},\QQ_{\ell_0})
    \to H^{2d}(X_{\ol{\QQ}},\QQ_{\ell_0})
    \cong \QQ_{\ell_0}(-d),
    }
    % (up to ``orientation'', i.e.~appropriate Tate twists)
    where $\QQ_{\ell_0}(-d)$ denotes the Tate motive of weight $2d$.
    % for skew-symmetric vs. symmetric, look at Milne's LEC discussion of cup-product pairing, skew-symmetric, etc.; alternatively, note the phrase "Graded-commutative ring" used in Peter Haine's notes on etale cohomology.
    % 
    By Poincar\'{e} duality, $\psi$ is non-degenerate.
    In fact, $\psi$ induces a non-degenerate pairing on $H^d_{\map{diff}}(X)=H^d(X)/H^d(\PP^{m-1})$, as we now recall.
    
    \emph{Case~1: $d$ is odd.}
    Then $H^d(\PP^{m-1})=0$, so $H^d_{\map{diff}}(X)=H^d(X)$.
    Thus $\psi$ can (trivially) be viewed as a non-degenerate pairing on $H^d_{\map{diff}}(X)$.
    
    \emph{Case~2: $d$ is even.}
    Then $\psi$ is symmetric (and non-degenerate), and $H^d(\PP^{m-1})$ is one-dimensional.
    It is known that the restriction $\psi\vert_{H^d(\PP^{m-1})}$ is non-degenerate; this follows, for instance, from Poincar\'{e} duality (for a copy of $\PP^d$ sitting in $\PP^{m-1}$) and the functoriality of cup products.
    % https://kconrad.math.uconn.edu/blurbs/linmultialg/bilinearform.pdf (Theorem 3.12)
    Therefore, we have an orthogonal direct sum decomposition $H^d(X) = H^d(\PP^{m-1})\oplus H^d(\PP^{m-1})^\perp$, and $\psi\vert_{H^d(\PP^{m-1})^\perp}$ is non-degenerate.
    Via the decomposition, $H^d_{\map{diff}}(X)\cong H^d(\PP^{m-1})^\perp$, so $\psi\vert_{H^d(\PP^{m-1})^\perp}$ can be viewed as a non-degenerate pairing on $H^d_{\map{diff}}(X)$.
    
    It follows that there are non-degenerate pairings
    $M_{\bm{c}}\times M_{\bm{c}}
    \to \QQ_{\ell_0}(-m_\ast)$
    and $M_V\times M_V
    \to \QQ_{\ell_0}(-1-m_\ast)$,
    whence $M\cong M^\vee(-w)$ if $M\in\set{M_{\bm{c}},M_V}$.
    Hence $M\cong M^\vee(-w)$ even if $M\notin\set{M_{\bm{c}},M_V}$.
    In every case,
    $M$ is self-dual up to a Tate twist of weight $2w$.
    So $\pi_M$ should be self-dual on the nose.
    
    \item Say $2\mid m$,
    i.e.~$2\nmid m_\ast$.
    Then the aforementioned pairing
    $M_{\bm{c}}\times M_{\bm{c}}
    \to \QQ_{\ell_0}(-m_\ast)$
    is \emph{skew-symmetric}.
    So under Tate's global semi-simplicity conjecture,
    the representation $M_{\bm{c}}\wedge M_{\bm{c}}$ should decompose as $\QQ_{\ell_0}(-m_\ast)\oplus M'_{\bm{c},2}$,
    for some representation $M'_{\bm{c},2}$ of weight $2m_\ast$.
    Then $M'_{\bm{c},2}$ should correspond to some \emph{nice isobaric} $\phi_{\bm{c},2}$,
    with the expected compatibilities (a)--(b) of (HWSp).
\end{enumerate}

\rmk{
There is an intuitive reason for $M_{\bm{c}},M_V$ to be self-dual:
at least at good primes,
$M_{\bm{c}},M_V$ arise from \emph{point counts},
which are obviously always \emph{integral}
(and therefore \emph{real}).
% (Assuming (HW2), this could perhaps be converted into a proof of self-duality, by considering conjugate representations on the automorphic side, and invoking strong multiplicity one.)
}

\rmk{
Suppose $2\mid m$,
and assume (HWSp).
Then most of the standard analytic properties of $L(s,V_{\bm{c}},\bigwedge^2)$---based on (HW2)---carry over to $L(s,\phi_{\bm{c},2})
= L(s,V_{\bm{c}},\bigwedge^2)/\zeta(s)$.
Furthermore,
$L(s,\phi_{\bm{c},2})$ is zero-free for $\Re(s)\geq 1$.
By GRH for $L(s,V_{\bm{c}},\bigwedge^2)$,
it follows that 
$1/L(s,\phi_{\bm{c},2})\ll_{m,\eps}q(V_{\bm{c}})^\eps(1+\abs{s})^\eps$
holds uniformly over $\bm{c},s$ with $\Re(s)\geq1/2+\eps$.
}

Let $\pi_{\bm{c}}\defeq\pi_{M_{\bm{c}}}$.
Before proceeding,
note that if $2\mid m$ and $M_{\bm{c}}$ is \emph{irreducible},
then we expect the putative $\pi_{\bm{c}}$ to be \emph{cuspidal self-dual symplectic} as defined on \cite{sarnak2016families}*{p.~533}.
Out of idle curiosity,
we raise the following refined question,
which could be too optimistic.
\ques{
\label{QUES:nature-of-cuspidal-constituents}
Say $2\mid m$,
and assume (HW2).
Then is it true that each cuspidal constituent $\pi_{\bm{c},i}$ of $\pi_{\bm{c}}$ is \emph{self-dual} and \emph{symplectic},
with $L(s,\pi_{\bm{c},i},\bigwedge^2)$ analytic except for a simple pole at $s=1$?
(If this is really true,
then all $\pi_{\bm{c},i}$ must be nontrivial,
i.e.~$L(s,\pi_{\bm{c}})$ must be entire.)
}

\rmk{
The fact that $\pi_{\bm{c}}$ may not be cuspidal muddies the waters.
Perhaps \cite{gross2016langlands} can clarify matters (with the notion of a symplectic motive).
% , though I am confused by some remarks on cuspidality criteria.\footnote{For example, unless there is a typo or I missed a condition, Gross seems to be saying that $A = E_1\times E_2$ (with $E_1,E_2$ non-isogeneous elliptic curves, so that $\Aut(A)$ is finite) should correspond to a cuspidal, and not just an isobaric, representation.
% But \url{https://arxiv.org/abs/1705.03054}, which attributes a precise automorphy conjecture to Gross--Langlands, assumes $A$ is simple over $\QQ$, so maybe there is just a typo.}
% Some natural questions to answer here are:
% \begin{enumerate}[(1)]
%     \item Is a sub-motive of a self-dual motive self-dual?
%     (Presumably no, since you can add a motive to its dual.)
    
%     \item Is a sub-motive of a symplectic motive symplectic?
%     (Maybe sometimes, since the symplectic pairing restricts---but not always, because there could be some issue with degeneracy of the restricted pairing?)
    
%     \item Is a symplectic motive necessarily self-dual?
    
%     \item For the above questions, what if we \emph{restrict attention to sub-motives $M$ of $H_1(A)$ of abelian varieties $A$} (i.e. must $M$ be self-dual and symplectic)?
    
%     \item How should one think about self-dual representations coming from geometry?
%     (Presumably point counts are not the whole answer.)
% \end{enumerate}
}

\ex{
Say $m=4$.
% i.e.~$m_\ast = 1$.
Then $\pi_{\bm{c}}$ is the representation generated by the weight $2$ modular cusp form of level $N_{J(V_{\bm{c}})}$ associated to the elliptic curve $J(V_{\bm{c}})/\QQ$.
% \footnote{Here $V_{\bm{c}}$ is isomorphic to a plane cubic $C\belongs\PP^2$.
% It is possible for a plane cubic $C$ to have no $\QQ_p$-points for some $p$, in which case we would not have $C\times\QQ_p\cong J(C)\times\QQ_p$.
% However, because $H^1$ \href{https://mathoverflow.net/questions/135648/is-first-etale-cohomology-of-a-variety-always-dual-to-a-tate-module}{depends only} on $J(C)$, we are OK.}
%https://arxiv.org/abs/1311.5578
%When V is a non-singular algebraic curve of genus g, H1 is a free Zℓ-module of rank 2g, dual to the Tate module of the Jacobian variety of V.
Here $L(s,\pi_{\bm{c}},\bigwedge^2) = \zeta(s)$.
}

\ex{
Fix $F$ diagonal,
and suppose $\bm{c}=(0,\dots,0,1)$.
Then $V_{\bm{c}}$ is a diagonal cubic hypersurface of dimension $m_\ast$,
so $L(s,V_{\bm{c}})$ is a product of \emph{normalized} $L$-functions attached to
algebraic Hecke characters of weight $m_\ast$ on $\QQ(\zeta_3)$.
% Hecke chars live on the automorphic side; on the geometric side, see notion of Jacobi motive in http://pmb.univ-fcomte.fr/2018/PMB_Watkins.pdf (which Watkins attributes to Anderson)... we can associate Jacobi motives to said Hecke characters (coming from Jacobi sums) on the automorphic side
% https://mathoverflow.net/questions/111851/explicit-examples-of-algebraic-hecke-characters-with-infinite-image
Furthermore, if $2\mid m$, then no factors of $\zeta(s)$ can appear, so $L(s,V_{\bm{c}})$ must be entire.
(These facts about $L(s,V_{\bm{c}})$ are classical; see Lemma~\ref{LEM:L(s,V)-poles} below for details.)
}

% \lem{
% [Classical]
% \label{LEM:L(s,V)-poles}
% If $F$ is diagonal, then $r_F\defeq -\ord_{s=1} L(s,V)$
% equals the number of pairs $\set{\bm{a},-\bm{a}}\belongs \set{1,-1}^m$ such that $F_1^{-a_1}\cdots F_m^{-a_m}\in (\QQ^\times)^3$ and $\sum_{i\in [m]} a_i = 0$.
% }

% \rmk{
% For $m$ odd,
% or for typical $F$,
% we have $r_F=0$.
% For even $m$ with $F$ Fermat,
% we have $r_F = \frac12\binom{m}{m/2} = \binom{m-1}{m/2-1}$
% (so that e.g.~$r_F=3$ if $m=4$,
% and $r_F=10$ if $m=6$).
% }

\subsubsection{Alternative motivic descriptions}

Fix $m,F,\bm{c}$ as before.
Then $V_{\bm{c}}$ is isomorphic to a smooth projective cubic hypersurface of dimension $m_\ast\geq0$.

Often the representation $M_{\bm{c}}$ can also be realized from other perspectives,
at least \emph{up to semi-simplification}.
Recall that by Chebotarev and Brauer--Nesbitt,
% https://en.wikipedia.org/wiki/Brauer%E2%80%93Nesbitt_theorem
two ``finitely ramified'' $\ell$-adic representations of $G_\QQ$ agree up to semi-simplification if and only if their local $L$-factors agree at all but finitely many primes.
% Note that at an unramified prime $p$, the Frobenius traces (encoded in the local Euler factor) should completely determine the original local $W_{\QQ_p}$-representation, at least up to Frobenius-semi-simplification?
In particular,
the $L$-function $L(s,-)$ from Definition~\ref{DEFN:geometric-L-functions},
at least ``up to finitely many factors'',
is a ``complete and well-defined'' invariant for ``such representations up to semi-simplification''.

\rmk{
Semi-simplification and ``finite factor-fudging'' \emph{should} be unnecessary
(in view of Tate's global semi-simplicity conjecture),
but to be safe (unconditionally speaking),
we allow them.
In any case,
they are convenient.
}

Returning to $V_{\bm{c}}$,
we now give some alternative descriptions of $M_{\bm{c}}$ for $m\in\set{4,5,6}$.
Here we let ``$\approx$'' denote ``equality up to finitely many Euler factors''.
(Experts may well know more precise information,
either in general or in our specific situations;
e.g.~for item~(1) below, see the discussion in Appendix~\ref{CHAP:natural-modularity-questions} surrounding Proposition~\ref{PROP:first-etale-cohomology-passes-to-Albanese}.)
\begin{enumerate}[(1)]
    \item If $m\in\set{4,6}$,
    then there exists an abelian variety $A_{\bm{c}}$ of dimension $(\dim M_{\bm{c}})/2\in\set{1,5}$
    % associated to $V_{\bm{c}}$
    such that $L(s,V_{\bm{c}})\approx L(s,A_{\bm{c}})$.
    % \footnote{The gamma factors can then be double-checked against \cite{iwaniec2004analytic}*{pp.~146--147}.}
    For $m=4$,
    we can take $A_{\bm{c}}$ to be the Jacobian $J(V_{\bm{c}})$ of the genus one curve $V_{\bm{c}}$,
    as noted on \cite{heath1998circle}*{p.~680}.
    For $m=6$,
    we can take the Albanese $A(F(V_{\bm{c}}))$ of the Fano surface $F(V_{\bm{c}})$ of lines on $V_{\bm{c}}$.\footnote{See \cite{debarre2021lines}*{Theorem~4.1} for a computational perspective,
    explaining at least the Fano surface connection.
    For a more complete discussion,
    % see \cite{bombieri1967local},
    % or \url{https://arxiv.org/abs/1304.4076},
    see \cite{reid1972complete}*{Appendix~4.3} or \cite{murre1974some}.}
    
    \item If $m=5$,
    then there exists a $6$-dimensional Artin representation $\rho_{\bm{c}}$ associated to $V_{\bm{c}}$ such that $L(s,V_{\bm{c}})\approx L(s,\rho_{\bm{c}})$.
    % \footnote{We can then in principle compute the gamma factor more concretely by following \cite{iwaniec2004analytic}*{p.~142}.}
    At least at all but finitely many primes,
    this is explained in \cite{manin1986cubic}.
    % Manin's book on cubic forms.
    To include all primes, one could apply \cite{poonen2017rational}*{Proposition~9.2.6} to $(V_{\bm{c}})_{\ol{\QQ}}$.
    (The point is that the $\ell_0$-adic cycle class homomorphism \cite{poonen2017rational}*{(7.6.2)} for $(V_{\bm{c}})_{\ol{\QQ}}$ is an isomorphism, since the cohomological Brauer group of $(V_{\bm{c}})_{\ol{\QQ}}$ vanishes \cite{poonen2017rational}*{Corollary~6.9.11}.
    See also \cite{jahnel2014brauer}*{Remark~III.4.10, Theorem~III.7.9(ii), Lemma~III.8.4, and Example~III.8.7(ii)}.)
\end{enumerate}
Strictly speaking,
such observations are unnecessary.
But the question of reciprocity for
abelian varieties
(see e.g.~\cites{gross2016langlands,boxer2021abelian})
and Artin motives
is currently better studied than
that for hypersurfaces---hence worth mentioning
(especially for $m\geq5$,
where automorphy seems to remains open for $M_{\bm{c}}$ in general;
e.g.~for $m=6$, see Appendix~\ref{CHAP:natural-modularity-questions} for some discussion).
% (See Appendix~\ref{CHAP:natural-modularity-questions} for an irreducibility computation justified using abelian varieties.)

\rmk{
Recall the non-degenerate pairing $\psi\maps H^{m_\ast}(V_{\bm{c}})\times H^{m_\ast}(V_{\bm{c}})\to H^{2m_\ast}(V_{\bm{c}})$.
\begin{enumerate}[(1)]
    \item If $m_\ast=1$,
    then $\psi$ is essentially the Weil pairing on the elliptic curve $J(V_{\bm{c}})/\QQ$.
    Here $\psi$ is \emph{symplectic} and $\bigwedge^2 H^1(V_{\bm{c}})\cong\QQ_{\ell_0}(-1)$ is trivial up to Tate twist
    (cf.~the fact that $\tilde{\alpha}_p\tilde{\beta}_p = 1$ if $p\nmid N$).
    
    \item If $m_\ast=2$,
    then $\psi$ is the \emph{symmetric} intersection pairing on the cubic surface $V_{\bm{c}}$.
    Since $2$ is even,
    $H^2(V_{\bm{c}}, \QQ_{\ell_0}(1))\cong \Pic(V_{\bm{c}}\times \ol{\QQ})\otimes \QQ_{\ell_0}$ splits as $[K_{V_{\bm{c}}}]\ZZ\oplus K_{V_{\bm{c}}}^\perp$.
    It is the \emph{primitive} part $K_{V_{\bm{c}}}^\perp = [-H]^\perp$ (which is isomorphic to $M_{\bm{c}}$)
    % (a $6$-dimensional Artin representation $G_\QQ\to W(E_6)$)
    that defines $L(s, V_{\bm{c}})$ (a degree $6$ Artin $L$-function).
    % Here $\psi_{\RR}\vert_{H^\perp}$ is negative definite by the Hodge index theorem.
    
    \item If $m_\ast=3$,
    then $\psi$ is again \emph{symplectic}.
    It is likely closely related to the Weil pairing associated to $(A_{\bm{c}},\lambda_{\bm{c}})$,
    where $\lambda_{\bm{c}}\maps A_{\bm{c}}\to A^\vee_{\bm{c}}$ denotes a certain principal polarization defined in terms of $V_{\bm{c}}$.
    (For one possible construction of $\lambda_{\bm{c}}$, see \cite{debarre2021lines}*{Remark~4.2, par.~2, involving a ``difference morphism'' on $F(-)$}.)
    % (The point is that (at least if $V_{\bm{c}}$ contains a rational line) $A_{\bm{c}}$ is the \emph{Prym variety} associated to a certain double cover of curves related to $V_{\bm{c}}$;
    % see \cite{debarre2021lines}*{\S4.3}, \cite{reid1972complete}*{Appendix~4.3}, or \cite{murre1974some}.)
    % https://en.wikipedia.org/wiki/Weil_pairing#Generalisation_to_abelian_varieties
    % https://en.wikipedia.org/wiki/Prym_variety
\end{enumerate}
% The geometry of cubic hypersurfaces is quite rich; we will not get into it here.
}

% \subfile{conditional}

\subsection{A statistical discussion of our families of \texpdf{$L$}{L}-functions}
\label{SUBSEC:discuss-expected-RMT-statistical-nature-of-our-families-of-L-functions}
% The statistical nature of our families...
% Statistical remarks/background/context on/for our families...
% A discussion of statistical expectations/predictions for our families...
% The main/standard/noteworthy expected/predicted statistical aspects/features/properties/remarks/questions/statistics of

We now discuss the expected statistical nature of our families of $L$-functions.
Fix $F$ with $m\in\set{4,6,8,\dots}$,
and for convenience,
assume Conjecture~\ref{CNJ:(HW2)}~(HW2) (but not (HWSp)).

\subsubsection{The Sarnak--Shin--Templier framework}

Let $\pi_{\bm{c}}\defeq\pi_{M_{\bm{c}}}$.
Then $\pi_{\bm{c}}$ should be cuspidal for almost all $\bm{c}$.
Indeed,
the proof of Proposition~\ref{PROP:(LASp)} shows (unconditionally) that
$p^{-m}\sum_{\bm{c}\in \FF_p^m} \bm{1}_{p\nmid F^\vee(\bm{c})}
\cdot \abs{\tilde{\lambda}_{\bm{c}}(p)}^2
= 1+O(p^{-\delta})$ as $p\to\infty$.
% At least under
% ($\dim M\ll_m 1$,) GRC, GRH, and the expected properties of Rankin--Selberg $L$-functions
% (cf.~the idea of computing $\dim V^G$ in group representation theory by analyzing $V\otimes V^\ast$),
It follows that
as $Z\to\infty$,
we have
\mathds{
1+O(Z^{-\delta'})
% Use $\sum (\log{p}) \sim \int (1/\log{x})\cdot\log{x} \sim Z^\delta$ to get the main term.
% Use GRC, and $\sum (\log{p})/p \sim \int (1/\log{x})\cdot(\log{x})/x \sim Z^\delta$, to bound the error from $\bm{1}_{p\mid F^\vee}-\bm{1}_{F^\vee=0}$.
&= \frac{1}{Z^{\delta}\cdot(2Z)^m}
\sum_{p\leq Z^{\delta}}(\log{p})
\cdot \sum_{\bm{c}\in[-Z,Z]^m}\bm{1}_{F^\vee(\bm{c})\neq0}
\cdot \abs{\tilde{\lambda}_{\bm{c}}(p)}^2 \\
% Use $\dim M\ll_m 1$, GRC, GRH, and Rankin--Selberg theory (noting that $L(s,M_{\bm{c}}\otimes M_{\bm{c}}^\vee)$ must coincide with its automorphic analog, by strong multiplicity one for isobarics).
&\geq 1+O_\eps(Z^{-\delta/2+\eps})
+ \frac{1}{(2Z)^m}
\sum_{\bm{c}\in[-Z,Z]^m}\bm{1}_{F^\vee(\bm{c})\neq0}
\cdot \bm{1}_\textnormal{$\pi_{\bm{c}}$ is not cuspidal},
}
by a ``representation-theoretic'' analysis of (the poles at $s=1$ of) the $L$-functions $L(s,M_{\bm{c}}\otimes M_{\bm{c}}^\vee)$
via (HW2), Observation~\ref{OBS:Rankin-Selberg-L-functions-under-(HW2)}, and \cite{iwaniec2004analytic}*{\S5.6's Exercise~6 and \S5.7's Theorem~5.15}.
So $\#\set{\bm{c}\in[-Z,Z]^m
: \textnormal{$F^\vee(\bm{c})\neq0$, and $\pi_{\bm{c}}$ is not cuspidal}}
\ll Z^{m-\delta''}$.

Similarly (still under (HW2)),
via other (unconditional) local moments from the proof of Proposition~\ref{PROP:(LASp)},
% (proven either by explicit point counting, or by monodromy),
one can statistically analyze the poles of $L(s,M_{\bm{c}}\otimes M_{\bm{c}}),L(s,M_{\bm{c}}\wedge M_{\bm{c}})$ at $s=1$.
Such an analysis
% The resulting information
(when combined with the previous paragraph)
reveals the family $\bm{c}\mapsto\pi_{\bm{c}}$ to be \emph{essentially}
cuspidal, self-dual, and symplectic
(in the sense of \cite{sarnak2016families}*{p.~538, (i)--(iii)}),
so that $\ord_{s=1}L(s,\pi_{\bm{c}},\bigwedge^2)=-1$ almost always.

\rmk{
\label{RMK:vertical-monodromy}
The computations above are closely related to ``vertical'' or ``local'' monodromy.
Since $V$ is smooth of even dimension $m-2\in\set{2,4,\dots}$,
with hyperplane sections $V_{\bm{c}}$ of odd dimension $m_\ast\in\set{1,3,\dots}$,
a general result of Deligne
(see \cite{katz2004larsen}*{Introduction, pp.~1--2},
around the line ``For $n$ odd, the monodromy group $G_d$ is\dots'')
shows that the ``Zariski closure of the monodromy of the local system''
\mathd{
\bm{c}\mapsto
H^{m_\ast}(V_{\bm{c}})/H^{m_\ast}(V)
= H^{m_\ast}(V_{\bm{c}})
}
of rank $N_1
= \frac{2^{m_\ast+2}+2(-1)^{m_\ast}}{3}
\in \set{2,10,\dots}$
on the space of ``smooth, degree $d=1$, hypersurface sections''
(in \cite{katz2004larsen}'s setup\footnote{we take the ``universal family'' of smooth hyperplane sections,
but a ``sufficiently general Lefschetz pencil'' would also suffice according to \cite{katz2004larsen}})
is the ``full symplectic group'' $\map{Sp}(N_1)$.
(Cf.~\cite{katz2014sato}*{\S8, ``hypersurface examples''}, which are of the same spirit.)

In particular,
by Deligne--Katz equidistribution (\cite{katz2014sato}*{Theorem~5.1}, as applied in \cite{sarnak2016families}*{\S2.11}),
our family has Sato--Tate group $\map{Sp}(N_1,\CC)$ in the sense of \cite{sarnak2016families}.
}

In any case, by \cite{sarnak2016families}*{Conjecture~2},
the low-lying $L$-function zeros associated to our family
% our family's / its associated low-lying $L$-function zeros
should have symmetry type
consisting of
$\SO_{\map{even}}$ and $\SO_{\map{odd}}$;
cf.~the discussion on \cite{sarnak2016families}*{p.~549}
for Dwork families of odd degree,
i.e.~with ``$n$ even''.
% 
Thus in \emph{all} RMT-type predictions,
% (e.g.~for low-lying zeros of $L(s,\pi_{\bm{c}})$),
the family $\bm{c}\mapsto L(s,\pi_{\bm{c}})$ should have ``symmetry type'' composed of
$\SO_{\textnormal{even}}$ and $\SO_{\textnormal{odd}}$;
cf.~\cite{sarnak2016families}*{pp.~540--541, paragraph discussing ``moments of $L$-values''}.
(The root numbers are probably evenly distributed, but their distribution is actually irrelevant to our main results; see Remark~\ref{RMK:remarkable-universality-and-role-of-root-numbers} below.)

\rmk{
\label{RMK:multi-parameter-family-features}
RMT-type predictions should apply equally well to all ``natural'' parameterizations of a given family,
in the spirit of Remark~\ref{RMK:technical-points-regarding-ratios-heuristic-in-SST-type-geometric-setting}(3) and \cite{sarnak2016families}*{p.~535, Remark~(i); and p.~560, second paragraph after (25)}.
(See e.g.~Conjecture~\ref{CNJ:(RA1)}~(RA1), a mean-value prediction over certain ``nearly homogeneous'' regions $\mcal{B}(\bm{Z})\cap\set{\bm{a}\bmod{n_0}}$,
where one can think of $\mcal{B}(\bm{Z})$ as being ``truly homogeneous, but \emph{with small coefficients}''.)
In this regard,
\emph{multi-parameter} families raise some interesting questions
(involving ``lopsidedness'' and ``singularities'' of weights)
% (e.g.~the optimal error terms might be even subtler than
% those for single-parameter families).
that are subtler than those for \emph{single-parameter} families.
}

\subsubsection{Remarks on (R1)--(R2) and (RA1)}

In \S\ref{SUBSEC:applying-CFZ-ratios-conjecture-recipe},
we applied the RMT-based heuristic behind \cite{conrey2008autocorrelation}*{\S5.1, (5.6)}
to state certain Ratios Conjectures (R1)--(R2) and (RA1).

\rmk{
For a discussion of uniformity in $\sigma,t$,
see \cite{conrey2007applications}*{(2.11b)--(2.11c)}.
In particular,
(2.11b) suggests that in (R1)--(R2) and (RA1),
we could take $\sigma-1/2$ as small as $\Omega(1/\log{Z})$.
But it is cleaner for us to fix $\sigma>1/2$---which is good enough for us anyways,
since we assume \emph{power-saving} error terms in (R1)--(R2) and (RA1).
}

% \rmk{
% Using the techniques of \cite{conrey2007applications}*{\S7},
% one should be able to interpret Conjectures~\ref{CNJ:applied-RMT-1/L-integrated}~(RA1') and~\ref{CNJ:(R2')}~(R2') in terms of certain residues of $1/L(s,\pi_{\bm{c}})$.
% % (at the zeros of $L(s,\pi_{\bm{c}})$).
% }

\rmk{
\label{RMK:remarkable-universality-and-role-of-root-numbers}
The Ratios Conjectures involve not just the low-lying zeros of $L(s,\pi_{\bm{c}})$,
but rather ``all zeros up to height $\approx 1$'' (morally).
Nonetheless,
one still expects a remarkable degree of universality:
see e.g.~\cite{conrey2008autocorrelation}*{Conjectures~5.3--5.4} for orthogonal examples of the Ratios Conjectures,
where the presence of ``$L$'s in the denominator'' on the left-hand side
leads to ``polar factors of $\zeta(2s)$'' on the right-hand side,
just as in (R1)--(R2) and (RA1).

Interestingly,
root numbers and functional equations play no role in the ``recipe'' for $1/L$.
This somehow reflects the naive intuition that $1/L$ is ``more random'' than $L$.
In other aspects, though,
$1/L$-moments over orthogonal families seem loosely analogous to
$L$-moments over symplectic families
(and vice versa).
}


\section{Miscellaneous writeups}

\subsection{Poles given by the original variety}

Now assume $F$ is diagonal.
% For $m=4$ and $F$ Fermat,
% $L(s,V) = \zeta_{\QQ(\zeta_3)}(s)^3 = \zeta(s)^3 L(s,\chi_3)^3$.
% % What happens in general?
% In general,
% see Lemma~\ref{LEM:L(s,V)-poles} below.
Then one can ``compute'' $L(s,V)$; see Lemma~\ref{LEM:L(s,V)-poles} below.

\defn{
Given a nontrivial multiplicative character $\chi\maps\FF_q^\times\to\CC^\times$,
define the standard Gauss sum $g(\chi)\defeq \sum_{x\in\FF_q}\chi(x)e_p(\Tr_{\FF_q/\FF_p}(x))$,
and let $\tilde{g}(\chi)\defeq g(\chi)/q^{1/2}$.
%other additive characters are gotten by replacing x with ax for a\in\FF_q fixed (https://math.stackexchange.com/questions/1384203/additive-character-of-a-finite-field-trace-map-to-middle-field), so not much different (e.g. for a cubic character, g(\chi)^3 is independent of the choice of additive character)
}

\prop{[See e.g.~\cite{ireland1990classical}*{Chapter~10, Theorem~2}]
Let $q$ be a power of a prime $p\nmid 3F_1\cdots F_m$.
If $q\equiv2\bmod{3}$,
then $E(q) = 0$.
If $q\equiv1\bmod{3}$,
and $\chi_3 = \chi_{3,q}\maps\FF_q^\times\to\mu_3\belongs\CC^\times$ denotes \emph{either} of the two multiplicative characters of order $3$,
then
\mathd{
E(q) = q^{-1+m/2}\psum_{\bm{a}}\chi_3(F_1^{-a_1}\cdots F_m^{-a_m})\tilde{g}(\chi_3^{a_1})\cdots\tilde{g}(\chi_3^{a_m}),
}
where $a_i\in\set{1,-1}$ and $3\mid a_1+\dots+a_m$.
}

Although over each prime $p$ there are two possible choices of ``compatible'' $\chi_{3,q}$
(such that $\chi_{3,q^r} = \chi_{3,q}\circ N_{\FF_{q^r}/\FF_q}$ whenever $q\equiv1\bmod{3}$),
the cyclotomic field $K\defeq\QQ(\zeta_3)$ itself
(in which the $\chi$ are valued)
only has finitely many automorphisms---and thus
provides a way to glue different $p$ together,
via the cubic residue symbol $\chi_{3,\wp}$.

\defn{
For a prime $\wp\nmid 3$ of $K$ with residue field $k\defeq\mcal{O}_K/\wp$,
and a unit residue $x\in k^\times$,
let $\chi_{3,\wp}(x)\in \mu_3$ with $\chi_{3,\wp}(x)\equiv x^{(N\wp - 1)/3}\bmod{\wp}$.
Then,
for each integer $r\geq 1$ and given field extension $\ell/k$ of degree $r$,
let $\chi_{3,\wp^r}\defeq \chi_{3,\wp}\circ N_{\ell/k}$ be the unique character $\ell^\times\to\mu_3$ extending $\chi_{3,\wp}\maps k^\times\to\mu_3$.
}

Since $g(\chi)$ is well-defined
(i.e.~independent of the realization of $\FF_q$),
we can make sense of $g(\chi_{3,\wp^r})$ by identifying $\ell$ with $\FF_{N\wp^r}$,
i.e.~$g(\chi_{3,\wp^r})\defeq \sum_{x\in \ell}\chi_{3,\wp^r}(x)e_p(\Tr_{\ell/\FF_p}(x))$.
Now by the previous proposition
(though there might be a more conceptual approach phrased in terms of the $K$-automorphisms of $V_K$),
the indices $\bm{a}$ decompose into pairs $\set{\bm{a},-\bm{a}}$,
each defining a Hecke $L$-function over $K$.
This is the content of the following result:
\cor{
For good primes $p\nmid 3F_1\cdots F_m$,
we have
\mathd{
L_p(s,V)
\defeq \exp\left((-1)^{m-2}\sum_{r\geq1}\wt{E}(p^r)(p^{-s})^r/r\right)
= \pprod_{\set{\bm{a},-\bm{a}}}\prod_{\wp\mid p}(1-\psi_{\bm{a}}(\wp)(N\wp)^{-s})^{-1},
}
where $\wp$ denotes a prime in $\mcal{O}_K$ and $\psi_{\bm{a}}$ is the unique \emph{primitive} Hecke character on $K$ satisfying $\psi_{\bm{a}}(\wp) = \chi_{3,\wp}(\bm{F}^{-\bm{a}})(-\tilde{g}(\chi_{3,\wp}^{a_1}))\cdots (-\tilde{g}(\chi_{3,\wp}^{a_m}))$ for all $\wp\nmid 3F_1\cdots F_m$.
}

\rmk{
For the construction of the (classical) Hecke characters $\psi_{\bm{a}}$,
see \cite{weil1952jacobi}.
(Actually, \cite{weil1952jacobi} might only yield a possibly imprimitive character of conductor dividing $(3F_1\cdots F_m)^\infty$,
but this suffices.)
Strictly speaking,
\cite{weil1952jacobi}*{p.~489, Theorem} only addresses the product of $g$'s,
but the Artin symbol $\wp\mapsto \chi_{3,\wp}(\bm{F}^{-\bm{a}})$ is harmless by class field theory over $K$
(cf.~\cite{weil1952jacobi}*{final paragraph of p.~494,
regarding the $L$-function of a diagonal curve}).
}

\rmk{
Each $\psi_{\bm{a}}$ is $\ol{\QQ}$-valued,
i.e.~``algebraic''
(also called ``type $A_0$'').
% Whether we work idelically or classically does not matter.
In fact,
one can show that each ``un-normalized'' Hecke character $N_{K/\QQ}^{m/2}\cdot\psi_{\bm{a}}$ maps into $K$.
}

\pf{
Let $T = p^{-s}$.
Using the Hasse--Davenport relation \cite{iwaniec2004analytic}*{p.~275, Theorem~11.4} and $\chi_{3,\wp^r} = \chi_{3,\wp}\circ N_{\ell/k}$ as in \cite{weil1949numbers}*{p.~506, (8)} gives
\mathd{
(-1)^m\sum_{r\geq1}\wt{E}(p^r)T^r/r
% &= \sum_{r\geq 1}\psum_{\bm{a}}
% \chi_{3,\wp}(\bm{F}^{-\bm{a}})^r
% (-\tilde{g}(\chi_{3,\wp}^{a_1}))^r\cdots(-\tilde{g}(\chi_{3,\wp}^{a_m}))^r
% T^r/r \\
= -\psum_{\bm{a}}\log(1-\chi_{3,\wp}(\bm{F}^{-\bm{a}})(-\tilde{g}(\chi_{3,\wp}^{a_1}))\cdots(-\tilde{g}(\chi_{3,\wp}^{a_m}))T),
}
for either choice of $\wp\mid p$,
if $p\equiv1\bmod{3}$.
Similarly,
if $p\equiv2\bmod{3}$ (and $\wp = (p)$),
then
\mathd{
(-1)^m\sum_{r\geq1}\wt{E}(p^r)T^r/r
= -\frac12\psum_{\bm{a}}\log(1-\chi_{3,\wp}(\bm{F}^{-\bm{a}})(-\tilde{g}(\chi_{3,\wp}^{a_1}))\cdots(-\tilde{g}(\chi_{3,\wp}^{a_m}))T^2),
}
in which case the automorphism $x\mapsto x^p$ of $k$ equates the contributions from $\bm{a}$ and $p\bm{a}\equiv -\bm{a}\bmod{3}$
(because $\bm{F}^{-\bm{a}}\equiv \bm{F}^{-p\bm{a}}\bmod{p}$ and $g(\chi_{3,\wp}) = g(\chi_{3,\wp}^p)$),
thus removing the factor of $1/2$.
% Since $(-1)^m = (-1)^{m-2}$, we get the desired formula for $L_p(s,V)$.
The desired formula for $L_p(s,V)$ follows.
}


\lem{
\label{LEM:L(s,V)-poles}
With the above notation,
% $L(s,V)$ is a product of Hecke $L$-functions over $K$,
$L(s,V)=\smallpprod_{\set{\bm{a},-\bm{a}}}L(s,\psi_{\bm{a}})$.
Furthermore,
$L(s,V)$ has a pole at $s=1$ of total order $r_F\geq 0$
equal to the number of pairs $\set{\bm{a},-\bm{a}}$ such that $F_1^{-a_1}\cdots F_m^{-a_m}\in (\QQ^\times)^3$ and $\sum_{i\in[m]} a_i = 0$.
}

\rmk{
For $m$ odd,
or for typical $F$,
we have $r_F=0$.
For $F$ Fermat with $m$ even,
$r_F = \frac12\binom{m}{m/2} = \binom{m-1}{m/2-1}$ (so that e.g.~$r_F=3$ if $m=4$, and $r_F=10$ if $m=6$).
}

\pf{
First,
to fully prove the factorization on the nose,
combine \cite{anderson1986cyclotomy}*{Theorem~8(II), Theorem~6, and Corollary~5.7.2};
here Theorem~8(II) identifies $M_V(-1)$
(and hence $M_V$)
% since $\QQ_{\ell_0}(-1)=H^2(E)$ for all elliptic curves over $\QQ$ (and we can choose $E$ to have CM)
as a motive ``potentially of complex multiplication type''
(which can therefore be analyzed by Theorem~6 and Corollary~5.7.2).

It remains to compute $r_F
\defeq -\ord_{s=1}L(s,V)$.
But work of Hecke
(see e.g.~\cite{iwaniec2004analytic}*{Theorem~3.8} for the imaginary quadratic case we are in)
immediately implies that
\mathd{
r_F = \#\set{\textnormal{pairs}\;\set{\bm{a},-\bm{a}}
: \psi_{\bm{a}}\;\textnormal{is trivial}}.
}
Now fix $\bm{a}$.
Since $\tilde{g}(\chi_3)\tilde{g}(\chi_3^{-1}) = 1$,
the ``defining formula'' for $\psi_{\bm{a}}$
(from the previous corollary)
simplifies,
telling us that
$\psi_{\bm{a}}(\wp)
= \chi_{3,\wp}(\bm{F}^{-\bm{a}})
(-\tilde{g}(\chi_{3,\wp}))^{\sum a_i}$
for all but finitely many primes $\wp$.
If $\bm{F}^{-\bm{a}}\in (\QQ^\times)^3$ and $\sum a_i = 0$,
then certainly $\psi_{\bm{a}}$ must be trivial.

Conversely,
suppose $\psi_{\bm{a}}$ is trivial;
then $\chi_{3,\wp}(\bm{F}^{-\bm{a}})(-\tilde{g}(\chi_{3,\wp}))^{\sum a_i} = \psi_{\bm{a}}(\wp) = 1$ for almost all $\wp$.
For such $\wp$,
cubing yields $(-\tilde{g}(\chi_{3,\wp}))^{3\sum a_i} = 1$.
If $\sum a_i\neq 0$,
then $-\tilde{g}(\chi_{3,\wp})$ would be restricted to lie in a finite set,
contradicting the known equidistribution of Kummer sum angles.
% proven by Heath-Brown and Patterson (building on work of Kubota?).
% https://en.wikipedia.org/wiki/Kummer_sum
(Alternatively,
it should suffice to use Stickelberger's factorization \cite{lang2012cyclotomic}*{Chapter~1, Theorem~2.2} of $g(\chi_3)$.)
Thus $\sum a_i = 0$,
so 
$\chi_{3,\wp}(\bm{F}^{-\bm{a}}) = 1$ for almost all $\wp$,
from which Chebotarev implies $\bm{F}^{-\bm{a}}\in (\QQ^\times)^3$,
completing the proof.
}

\rmk{
For $m=4$,
a simpler treatment is possible via Artin representation theory \cite{jahnel2014brauer}*{pp.~213--216},
since for each $\bm{a}$,
the $g$'s cancel out:
$\sum a_i$ is $0\bmod{3}$ and $m\bmod{2}$,
and thus $0$.
(For example, if $F = x_1^3+\dots+x_4^3$, then $L(s,V) = \zeta_{\QQ(\zeta_3)}(s)^3$.)
For $m=6$, though,
we need to address the ``transcendental motive''
(in the language of \cite{swinnerton2014diagonal})
associated to $\bm{a}=\pm\bm{1}$,
even though it does not ultimately contribute poles.
}

\subsection{Typical analytic ranks, versus \texpdf{$r_F$}{rF}}

For diagonal $F$,
Lemma~\ref{LEM:L(s,V)-poles} expresses $L(s,V)$ as a product of Hecke $L$-functions and determines the order $r_F$ of the pole at $s=1$.
(For arbitrary $\PP^{m-1}_\QQ$-smooth $F$,
the discussion below should still hold conditionally on automorphy for $L(s,V)$.)
Naively,
$r_F$ should have something to do with
% the ``special'' or ``diagonal-type'' contribution from coming from maximal rational linear subvarieties of $V$
special subvarieties of $V$
(cf.~Chapter~\ref{CHAP:isolating-special-solutions}).
This is partly true,
but the full story,
via the global Tate conjecture for codimension-$(1+m_\ast)/2$ cycles on $V$ modulo suitable equivalence
(e.g.~$\Pic(V)$ for $m_\ast=1$,
when $V$ is a cubic surface),
seems to involve other cycles on $V$ as well.
(One should perhaps work with homological or numerical equivalence.
We will not be too precise about such technical questions on algebraic cycles,
including base change or rationality issues.)

When $F^\vee(\bm{c})\neq 0$,
let $r_{\bm{c}}$ be the \emph{analytic rank},
i.e.~central order of vanishing,
of $L(s,\pi_{\bm{c}})$.
The following may not be strictly necessary,
but seems good to discuss:
\cnj{
[Nagao-type conjecture]
\label{CNJ:r_c-vs-r_F-family-conjecture}
If $m\in\set{4,6}$,
then for almost all $\bm{c}\in\ZZ^m$ with $F^\vee(\bm{c})\neq 0$,
we have $r_{\bm{c}}\in\set{r_F,r_F+1}$.
}

\rmk{
For typical $F$,
Lemma~\ref{LEM:L(s,V)-poles} says $r_F=0$,
and we certainly typically expect that $r_{\bm{c}}\in\set{0,1}$.
For specific $F$,
comparing typical $r_{\bm{c}}$ with $r_F$ might require BSD-on-average or similar.
As some algebraic evidence for $m=4$ (i.e.~$m_\ast=1$),
\cite{MO233170restrict_picard} has shown
(for any \emph{fixed} smooth cubic surface $V$)
that the restriction map $\Pic(V)\to \Pic(V_{\bm{c}})$ is \emph{typically injective} as $\bm{c}$ varies,
so that typically $r_{\bm{c}}\geq r_F$.

For $m=6$ (i.e. $m_\ast=3$),
is it true that the relevant restriction map is typically injective?
(If not,
then the conjecture above must be modified according to the dimension of the kernel.)
}

We now analyze the maximal linear $\QQ$-subvarieties for $m\in\set{4,6}$ when $F$ is Fermat,
and explain (with the $m=4$ example) why robust numerical testing requires at least a bit of care.

\ex{
For $m=4$ and $F$ Fermat,
the genus one curve $V_{\bm{c}}$ has three rational parametric points coming from the three rational lines of the cubic surface $V$,
so (taking one of the three to be the origin)
we certainly expect $J(V_{\bm{c}})$ to typically have (algebraic) rank at least $3-1=2$.
Naive random sampling of $V_{\bm{c}}$
(with $\norm{\bm{c}}_\infty\leq 6$)
leads to a rank distribution of
$[5, 23, 154, 166, 35, 3, 0, 0, \dots]$,
at first suggesting typical (analytic) ranks $2,3$.

However, $r_F=3$.
A more careful computation,
% inspired by Sarnak's mentioning to us of Tate's and Nagao's conjectures (see August 5, 2020 email about Miller and Rosen--Silverman)
sampling with $\norm{\bm{c}}_\infty\leq 10$ and avoiding the locus $\prod_{i<j}(c_i-c_j)=0$ (which arose from guesswork---but it would be good to check how $\Pic(V)\to \Pic(V_{\bm{c}})$ behaves for such $\bm{c}$),
gives a rank distribution of
$[0, 0, 10, 70, 42, 1, 0, 0, \dots]$,
suggesting instead typical (algebraic) ranks $3,4$,
in line with Conjecture~\ref{CNJ:r_c-vs-r_F-family-conjecture}.
Probably we missed a typical third point in the image of $\Pic(V)\to \Pic(V_{\bm{c}})$,
which could be computed in principle from \cite{browning2009quantitative}*{p.~139, \S8.3.1}
(which explicitly describes $\Pic(V)$,
or equivalently---by \cite{jahnel2014brauer}*{top of p.~211}---$\Pic(V_{\overline{\QQ}})^{G_\QQ}$,
since $V(\bd{A}_{\QQ})\neq\emptyset$).

(Disclaimer:
Nearly all of the ranks computed as part of the data should be correct,
but a small portion may be incorrect due to compromises made by typical algorithms,
or due to the fact that verifying large analytic ranks remains an open problem.)

% [Code:
% % Make github repository?
% to be added.]
}


\ex{  
For $m=6$ and $F$ Fermat,
$V_{\bm{c}}$ has $15$ rational parametric lines coming from the $5\cdot 3\cdot 1 = 15$ rational $2$-planes
of the cubic fourfold $V$
(one for each partition $\mcal{J}$ of $[6]$ into $2+2+2$).
These lines on $V_{\bm{c}}$
(i.e.~points on $F(V_{\bm{c}})$)
correspond to points on $A(F(V_{\bm{c}}))$
(taking one of the points to be the origin of $A_{\bm{c}}$);
\emph{what are the relations} between these points?

By \cite{clemens1972intermediate}*{p.~286},
the analysis might involve configurations like $V_{\bm{c}}\cap \set{x_1+x_2=0}\cap\set{x_3+x_4+x_5+x_6=0}$
(a union of three coplanar lines).
There are $\binom{6}{2} = 15$ such \emph{triangles}.
Each line lies in $3$ triangles.
Thus we can sequentially ``remove a line from an existing triangle'' $\geq 5$ times
(so that at the end,
the remaining $\leq 10$ lines ``span'' or ``triangulate'' the rest).
This may be suboptimal;
it is a question of linear algebra to find a ``triangle-reduced basis''.
% In fact, we can visually construct $5$ lines, no two in the same triangle: label a regular pentagon with $1,2,3,4,5$ and put $6$ in the center; to construct the partition $\mcal{J}_i$ defining the $i$th line, connect $6$ to $i$ as one part, and draw two perpendicular segments to form the remaining two parts of $\mcal{J}_i$.}

What other ``typical cycles'' on $V_{\bm{c}}$
(coming from $V$)
are there?
}


\rmk{
In both cases,
at least some of
the rational parametric cycles on $V_{\bm{c}}$ come from linear spaces on $V$ of the largest possible dimension (in the sense of \S\ref{SEC:maximal-linear-under-duality}).
% and Starr's \href{http://www.math.stonybrook.edu/~jstarr/papers/appendix3.pdf}{notes}, \emph{A fact about linear spaces on hypersurfaces}).
}

\rmk{
For $m=6$,
it may be worth testing analytic ranks numerically,
perhaps through the logarithmic derivative $L'/L$.
(Completely computing local $L$-factors of cubic threefolds seems to quickly get expensive.
But to test ranks, one should only need to work with small moduli.)
Notably,
\cite{swinnerton1967application}*{bottom of p.~290, with $r=2$} provides a relevant variant of BSD
``supported by unpublished results of Bombieri and Swinnerton-Dyer for the cubic threefold'' \cite{swinnerton1967application}*{p.~291};
I cannot tell if these ``unpublished results'' are more numerical or theoretical, however.
}

\chapter{Variations}
% Some other problems
\label{CHAP:variations}

\S\ref{SEC:other-open-problems-of-a-similar-nature} discusses problems inspired by Hypothesis HW, GRH on average, and large sieves, in the spirit of \S\ref{SEC:using-L-function-hypotheses-on-average}.

\S\ref{SEC:possibly-averaging-deforming-enlarging-or-smoothing-the-delta-method}
discusses questions related to
``perturbing'' the delta method---which seems to be
an interesting direction of research (cf.~\cite{marmon2019hasse}).

In \S\ref{SEC:speculation-on-other-Diophantine-problems},
we speculate on some Diophantine problems
that may or may not be within reach under
standard hypotheses similar to those in Chapter~\ref{CHAP:using-mean-value-L-function-predictions}.

% Research statement draft (Potential future directions)

\section{Problems inspired by Hypothesis HW}
% Other open problems of a similar nature
\label{SEC:other-open-problems-of-a-similar-nature}

% Consider adding an appendix with related/equivalent problems where Hypothesis HW / similar GRH on average would likely suffice.

% Good chance to link/explain an important part of Brudern's thesis and also mention Vaughan 2015 survey "Squares ... partitions".

\subsection{Problems with a cubic flavor}

\subsubsection{The cubic convexity barrier}

Beating the Hua bound
(or ``breaking convexity'' as \cite{wooley1995breaking} might say)
lies in the following family of (probably equivalent) problems:
\begin{enumerate}[(1)]
    \item To show that the (smoothly weighted) count of solutions $\bm{x}\in\ZZ^6$ to $x_1^3+\dots+x_6^3=0$,
    in the region $\norm{\bm{x}}\ll X$,
    is $O(X^{7/2-\delta_3})$ for some $\delta_3>0$.
    % or sharp cutoff
    %known proofs interpolate between $L^4$ and $L^8$/$L^\infty$.
    
    \item To show that the analogous smooth count for $x_1^3+\dots+x_8^3=0$ satisfies a Hardy--Littlewood asymptotic $cX^5 + O(X^{5-\delta_4})$ with a power saving $\delta_4>0$.
    
    \item In general,
    for $s\geq 4$,
    to show that the analogous count for $x_1^3+\dots+x_{2s}^3=0$ satisfies a Hardy--Littlewood asymptotic of the form $cX^{2s-3} + O(X^{3s/2-1-\delta_s})$ for some $\delta_s>0$.
\end{enumerate}
These ``even Fermat bounds'' would all follow from an improvement of the current generic cubic Weyl sum bound of $O_\eps(X^{3/4+\eps})$,
which appears quite difficult to beat;
see e.g.~\cite{heath2009bounds} assuming $abc$.
In any case,
the beauty of \cites{hooley1986HasseWeil, hooley_greaves_harman_huxley_1997,heath1998circle} lies in the averaging over arcs
(and the further averaging over moduli going beyond Kloosterman),
which is morally independent of the issue of pointwise Weyl bounds.
% in fact can be viewed as going beyond Weyl bounds.

The equivalent family above would also imply similar bounds for odd numbers of variables,
e.g.~an upper bound for ``$2s=5$''
(a problem raised by \cite{bombieri2009problems}).
% On the other hand,
At first glance,
there does not appear to be an \emph{equivalent} Diophantine problem involving $4$ variables (where ``$s=2$''),
or any odd number ``$2s$'',
although the general ``restricted arcs'' form of the $4$ variable problem (see problem~(5) below) could in principle pare away
most, if not all, but the most extreme minor arcs for $s\geq 3$.

\subsubsection{Other problems with a cubic flavor}

Besides the ``familiar'' or ``classical'' Fermat cubic problems above,
other natural problems in the same vein include the following:
\begin{enumerate}[(1)]
\setcounter{enumi}{3}
    \item To show that each smooth projective cubic surface $V(F)$ over $\QQ$ has $O_{F,\eps}(X^{3/2+\eps})$ solutions $\bm{x}\ll X$ away from its rational lines---or at least
    $O_F(X^{12/7-\delta})$,
    to beat \cite{salberger_2015}'s bound.
    (Note however that one of the nice features of \cite{salberger_2015} is uniformity over $F$,
    which we do not consider here.)
    
    \item To show that $x_1^3+\dots+x_4^3=0$ has $O_\eps(X^{2+\eps})$ solutions,
    in a way that generalizes
    to show that uniformly over $M\leq X^{3/2}$,
    we have
    \mathd{
    \int_{\theta\in\mf{M}(M)}
    d\theta\, \abs{T(\theta)}^4
    \ll_\eps X^\eps
    \cdot (X+M^2/X),
    }
    where $T(\theta)\defeq
    \sum_{\abs{x}\leq X}
    e(\theta x^3)$
    and $\mathfrak{M}(M)\defeq
    \bigcup_{q\leq M}
    \set{\theta\in\RR/\ZZ
    : \abs{q\theta-a}\leq M/X^3}$.
    
    (Unconditionally,
    \cite{brudern1991ternary} obtained $O_\eps(X^\eps)\cdot (X+M^2/X+M^{7/2}/X^3)$.)
    
    \item For $k\in\set{3,4}$,
    to show that the equation
    $x_1^2+x_2^3+x_3^k = y_1^2+y_2^3+y_3^k$,
    with each \emph{monomial} restricted to be of size at most $N$,
    satisfies a
    (smoothly weighted)
    Hardy--Littlewood asymptotic of the form
    $cN^{1+2\delta_k} + O_\eps(N^{1+\delta_k+\eps})$,
    where $\delta_k\defeq
    (1/2+1/3+1/k)-1$.
    
    \item For $k\in\set{3,4}$,
    to show that
    $x_1^2+x_2^3+x_3^k$
    positively represents all $n\leq N$ but an exceptional set of size
    $O_\eps(N^{1-\delta_k+\eps})$---which \cite{brudern1991ternary}
    proved unconditionally for $k=5$,
    and obtained weaker bounds towards for $k=3,4$.
\end{enumerate}

\rmk{
Progress on (4) would already be interesting for ``somewhat general'' (e.g.~diagonal) $F$'s.
But for cubic surfaces $V(F)$ with $\QQ$-lines,
unconditional bounds beyond \cite{salberger_2015} are already known---e.g.~$O_\eps(X^{4/3+\eps})$ for the Fermat cubic.

(Note that general diagonal $V(F)$'s have no $\QQ$-lines.)
}

\rmk{
The bound in (5),
if true,
may or may not be optimal
(up to $\eps$).
It is optimal for $M=X^{3/2}$,
at least.
}

\rmk{
The asymptotic error term $O_\eps(N^{1+\delta_k+\eps})$ in (6),
% This
if valid,
would likely be optimal up to $\eps$,
due to ``trivial loci'' such as $\bm{x}=\bm{y}$.
It could be interesting to try to obtain
a ``secondary term'' asymptotic that precisely captures the influence of such loci;
cf.~\cite{vaughan2015squares}*{Theorem~1.4} regarding the mixed ternary forms $x_1^2+x_2^2+x_3^k$ for $k\geq3$.
}

\rmk{
By mimicking \cite{brudern1991ternary},
one can show that the bound in (5) implies the ``exceptional set bound'' in (7);
see \cite{restricted_cubic_moments}*{Theorem~1.12} for details.
% The asymptotic in (6)---or at least ideas from a proof thereof---should imply the bound in (7).
It is also reasonable to expect
``(5) to imply (6)''
and ``(6) to imply (7)''---at least morally---but
% such implications would need to be carefully worked out.
such expectations would need to be carefully checked.
% (and then---if indeed correct---realized).
}

\subsubsection{Conditional approaches}

The bounds in (1)--(3),
and the bound in (4) for diagonal $F$,
can be established conditionally under
Hypothesis HW for smooth projective cubic hypersurfaces of dimension
$3,5,2s-3$,
and $1$,
respectively,
in certain families.
See \cite{wang2021_large_sieve_diagonal_cubic_forms}*{Theorem~1.28} (whose philosophy we sketched in \S\ref{SEC:using-L-function-hypotheses-on-average}) for a unified treatment,
which also clarifies how one may ``relax'' Hypothesis HW.

Similarly,
under Hypothesis HW for smooth hyperplane sections of $V(x_1^3+\dots+x_4^3)$,
one can mimic \cite{wang2021_large_sieve_diagonal_cubic_forms}*{Theorem~1.28} to prove the bound in (5),
and consequently the bound in (7).
Again,
some ``relaxation'' of Hypothesis HW is possible;
see \cite{restricted_cubic_moments}*{Theorem~1.6} for details.
(The situation for (6) may well be similar,
but would need to be carefully analyzed and written up.)
So at least in (5) and (7),
a tiny improvement over \cite{brudern1991ternary}'s bounds should already be possible using existing general automorphic large sieve inequalities;
see \cite{restricted_cubic_moments}*{Remark~1.9} for more details.
But truly satisfying progress on any of (1)--(7) will probably (and hopefully) require significant new ideas.

\rmk{
In the absence of progress on GRH itself,
the ``GRH-on-average philosophy'' provides one of the most alluring approaches towards (1)--(7).

However,
other ideas could also plausibly lead to interesting progress on (1)--(7).
For example,
(4) might be susceptible to clever slicing methods
(e.g.~those developed by Salberger or Heath-Brown),
especially when combined with advances in the theory of irrational varieties
(including upper-bound sieves
and elliptic-curve statistics).

From another direction,
Bourgain and Demeter (among others) have applied
decoupling theory on curves (for instance)
to prove interesting point-counting bounds,
often even in the absence of translation invariance.
Although
the most striking applications of decoupling
to \emph{classical} Diophantine problems
so far seem to have been restricted to degree $\geq5$ or codimension $\geq2$,
the full power of decoupling
remains to be understood,
and one cannot yet rule out the possibility of applying decoupling
(or one of its non-archimedean allies, as developed by Wooley and others)
% https://english.stackexchange.com/questions/224781/is-allied-topics-a-term-reserved-for-the-academia
towards (1)--(7).
}

% % general cubic forms
% % (Sarnak asked what is known re: Hua-type bound.
% % Deligne suggests $X^{3.75}$ should certainly be known.
% % What can Salberger prove.)

% \section{Competing approaches}

% unlikely intersections

% slicing methods (Oberwolfach, cubic surfaces)

% has the $x_1+\dots+x_6=t$ slice been tried?
% i.e. $h_1+h_2+h_3=t$.

% decoupling (best Hua for large degree, idea of completing systems)

% Bourgain's inequality (close to special subvariety)

% \section{Function field analogs}

% may help guide to the truth, and suggest natural families to look at when proving bounds
% Finally, especially for the non-homogeneous problems, 
% Finally, it might be worth thinking a little about function field analogs of 

\subsection{On families of quadrics}

Aside from cubic problems,
\emph{quadratic} problems---especially those in \emph{families}---provide an immediate opportunity for deeper exploration via the delta method.
Such problems might also play a role in, or at least relate to, Diophantine analysis on varieties that are \emph{not} complete intersections.

\ex{
Consider Manin's conjecture for
% non-c.i.'s such as
the conic bundle $a_1x_1^2+a_2x_2^2+a_3x_3^2=0$ in $\PP^2\times\PP^2$.
(\cites{le2015density,blomer2018manin} have expressed the opinion that this seems ``out of reach'' of existing techniques;
the analog in $\PP^3\times\PP^3$ is known \cite{browning2020density}.
On the other hand, Heath-Brown may have a forthcoming proof; see \cite{heath2022distribution}*{final paragraph of \S1}.)

The whole variety $E$ itself
(i.e.~the total space)
is probably not a complete intersection.
% I think $E$ is a smooth projective $3$-fold with $\dim H^2(E)=\dim H^2(\PP^2\times\PP^2)=2\neq1=\dim H^2(\PP^3)$; one should be able to prove this (in any Weil cohomology theory https://en.wikipedia.org/wiki/Weil_cohomology_theory - use the weak Lefschetz axiom) by expressing $E$ as a hyperplane section in the image of the composition $\PP^2\times \PP^2\inject \PP^2\times \PP^5\inject \PP^{17}$ (a quadratic Veronese embedding followed by a Segre embedding).
But the ``left-hand side'' above certainly defines a \emph{family} of ternary quadratic forms $F_0$.

Now recall that point counting on a \emph{ternary} affine quadric $F_0=b$
with $F_0$ fixed and definite---or with $(F_0,b)$ fixed and $F_0$ indefinite---is
basically understood,
% but varying more than that
but varying $F_0$ or $b$ in general
can get tricky \cite{friedlander2013small}.
Nonetheless,
in the \emph{homogeneous} case $b=0$ above,
the situation might be better---with
a certain interesting family of classical (Dirichlet or Artin) $L$-functions
plausibly relating to Manin's conjecture for $E$.
}

\subsection{On intersections of quadrics}
% Counting on other varieties

% Also perhaps state a large sieve "Problem" that would be hypothetically useful for pairs of quadrics in 5 or 7 variables... it might be nice to write down with expected ranges of variables, etc. even though the 2-D Kloosterman method is not yet fully developed (but link to Vishe's work).

% However,
% if one had such a satisfactory Kloosterman method,
% then preliminary calculations suggest that it would be possible,
% under automorphy and GRH for smooth projective varieties in certain families,
% to prove ... (quantitatively)
% to prove the Hasse principle for pairs of quadrics in $7$ variables.

For simplicity,
we discuss only individual smooth projective complete intersections $W\defeq V(F_1,\dots,F_R)$ in $\PP^{m-1}_\QQ$ with $m\geq5$ and $R\defeq2$ and $\deg{F_1}=\dots=\deg{F_R}=d\defeq2$,
though there are certainly many other interesting (affine or projective) varieties one could consider (individually or on average).
% \item It would be interesting to study $x_1^3+x_2^3+x_3^3+kx_4^3 = 0$ on average over $k$, just as one studies $x_1^3+x_2^3+x_3^3 = n$ on average---but it is unclear how to get started.
% %families of rational forms seems interesting but unclear how to get a handle on.
% \item Non-homogeneous problems, e.g. the mixed ternary \cite{brudern1991ternary}, may also be interesting.

Given $\bm{F}\defeq (F_1,\dots,F_R)/\ZZ$,
fix $w\in C^\infty_c(\RR^m)$ supported away from $\bm{0}$.
For $X\geq1$,
let $N_{\bm{F},w}(X)\defeq
\sum_{\bm{x}\in\ZZ^m}w(\bm{x}/X)
\cdot\bm{1}_{\bm{F}(\bm{x})=\bm{0}}$,
and if $m=5$,
let $N'_{\bm{F},w}(X)$ denote the corresponding weighted count restricted $\bm{x}\in\ZZ^m$ not lying on any $\QQ$-line of $W$.

For references on what is known about the Hasse principle for $W$,
we refer the reader to the introduction of \cite{vishe2019rational}.
When $m=5$,
the ``Hasse principle up to Brauer--Manin'' is unknown but conjectured to hold,
and from a quantitative point of view the best we know in general is Salberger's (possibly unpublished) bound
% https://citeseerx.ist.psu.edu/viewdoc/download?doi=10.1.1.368.4592&rep=rep1&type=pdf (Browning, Resent progress on the quantitative arithmetic of del Pezzo surfaces)
$N'_{\bm{F},w}(X)\ll_{\bm{F},\eps} X^{3/2+\eps}$.

Let $D\in\ZZ[\bm{c}]$ denote a suitable discriminant polynomial,
so that a given hyperplane section $W_{\bm{c}}$ of $W$ is singular if and only if $D(\bm{c})=0$.
Let
\mathd{
S_{\bm{F},\bm{c}}(q)
\defeq \psum_{\bm{a}\in(\ZZ/q)^R}\sum_{\bm{x}\in(\ZZ/q)^m}
e_q(\bm{a}\cdot\bm{F}(\bm{x})+\bm{c}\cdot\bm{x})
}
(cf.~the complete exponential sums defined in \cite{heath2017simultaneous}),
where we restrict $\bm{a}$ to be \emph{primitive} modulo $q$.
Then let $\wt{S}_{\bm{F},\bm{c}}(q)\defeq q^{-(m+R)/2}S_{\bm{F},\bm{c}}(q)$,
let $\Phi(\bm{F},\bm{c},s)\defeq
\sum_{q\geq1}\wt{S}_{\bm{F},\bm{c}}(q)q^{-s}$,
and let $\Psi_1(\bm{F},\bm{c},s)=\sum_{q\geq1}b_{\bm{F},\bm{c}}(q)q^{-s}$ denote a first-order approximation of $\Phi(\bm{F},\bm{c},s)$ in the sense suggested by Definition~\ref{DEFN:one-sided-first-order-Euler-product-approximation-of-Phi}.

Optimistic \emph{preliminary calculations}
(subject to errors or unforeseen difficulties)
% e.g.~even assuming a fully satisfactory 2-D Kloosterman method, one would have check carefully if \ell^2 GRH is enough (depending on whether the ``bad part'' of the problem can be controlled satisfactorily in \ell^2).
suggest the following conjecture:
\cnj{
Fix $\bm{F},w,\Psi_1$.
Let $Y\defeq X^{d/(1+1/R)}
= X^{4/3}$.
\emph{Assume that} for all
$X\geq1$ and $Z\geq Y/X=X^{1/3}$,
and for all
$N\ll Y$ and intervals $I\belongs[N/2,2N]$,
we have a uniform bound of the form
\mathd{
\psum_{\bm{c}\in[-Z,Z]^m}
\left\lvert
\sum_{q\in I}b_{\bm{F},\bm{c}}(q)
\right\rvert^2
\ll_\eps Z^\eps
\max(Z^m,Y)\cdot N,
}
where we restrict to $\bm{c}$ with $D(\bm{c})\neq0$.
\emph{Also assume} the existence of a ``fully satisfactory'' two-dimensional Kloosterman method.
Now let $\theta\defeq
\frac{dR(m-1-R)}{2(R+1)}
= \frac{2(m-3)}{3}$.
Then
\begin{enumerate}[(1)]
    \item $N'_{\bm{F},w}(X)
    \ll_\eps X^{\theta+\eps}
    = X^{4/3+\eps}$ if $m=5$;
    
    \item $N_{\bm{F},w}(X)
    \ll_\eps X^{\theta+\eps}
    = X^{2+\eps}$ if $m=6$;
    and
    
    \item $N_{\bm{F},w}(X)
    = cX^{m-dR}
    + O_\eps(X^{\theta+\eps})
    = cX^{m-4}
    + O_\eps(X^{2(m-3)/3+\eps})$
    if $m\geq7$
    (with $c=c_{\bm{F},w}$ being
    the usual Hardy--Littlewood prediction if $m\geq8$,
    and a more complicated constant that also incorporates $\QQ$-planes on $W$ if $m=7$).
    % Note that $\floor{(m-1)/2}\leq 2(m-3)/3$ for $m\geq8$; see https://www.wolframalpha.com/input/?i=solve+floor%28%28m-1%29%2F2%29+%3C%3D+2%28m-3%29%2F3
\end{enumerate}
}

\rmk{
The ``elementary GRH on average'' assumption above,
and that in \cite{wang2021_large_sieve_diagonal_cubic_forms}*{Theorem~1.28},
% should belong to a wider family of natural ``average hypotheses''.
% on any of which significant progress would be interesting.
should belong to a wider framework of ``average hypotheses'' over ``natural but thin'' geometric families.
% "thin" (informally, or in the sense of Serre) or "non-universal" families (e.g.~sub-families of a universal family of hypersurfaces of given dimension and degree)
Hopefully at least one such family is sufficiently rich yet tractable to inspire significant new ideas.
}

\rmk{
% \section{Finding low-degree automorphic families in the circle method}
% Anyways, aside from the aforementioned cubic $L$-functions, what other (simple) families can we find in the circle method?
% And how simple are the resulting families?
Over $D(\bm{c})\neq0$,
the Hasse--Weil $L$-functions $L(s,W_{\bm{c}})$ are quite rich.
Roughly speaking,
each $W_{\bm{c}}$
(after passing to certain standard $\ell$-adic Galois representations)
% http://pbelmans.ncag.info/cohomology-tables/
corresponds (up to Tate twist)
to an abelian variety of dimension $(m-3)/2$ if $2\nmid m$,
and to an Artin representation of dimension $m$ if $2\mid m$.
(See e.g.~\cite{bernardara2016chow}*{Theorem~2.1} or \cite{reid1972complete}.)
}

\rmk{
% Only ``doubled systems'' $\bm{Q}(\bm{x}) = \bm{Q}(\bm{y})$ have been rigorously analyzed at all by averaging over $\bm{a}\bmod{q}$ \cite{heath2017simultaneous}, but unlike for a single equation, the singular locus of $\bm{Q}(\bm{x}) = \bm{Q}(\bm{y})$ has positive dimension, so that the hyperplane sections are singular (a complicated desingularization may be possible?), and no averaging over $q$ has been done.
The $\bm{F}$'s considered in \cite{heath2017simultaneous} define \emph{singular} $W$'s
(with singular loci of positive dimension, in fact),
so are not included in the conjecture above.
But it would be interesting to see how much further one can push \cite{heath2017simultaneous} under automorphy and GRH for varieties ``related to'' hyperplane sections of $W$.
(The hyperplane sections $W_{\bm{c}}$ for \cite{heath2017simultaneous} are all singular,
but some desingularization might be possible.)
}

\rmk{
The ``two-dimensional Kloosterman method'' in \cite{heath2017simultaneous} is based on ``positivity''
(cf.~the works \cites{hooley1986HasseWeil,hooley_greaves_harman_huxley_1997} avoiding \cites{duke1993bounds,heath1996new}),
and hence is not ``fully satisfactory''
(in that it does not readily generalize to arbitrary $\bm{F}$'s).
Finding a suitable ``fully satisfactory'' generalization,
even in the rational function field case
(where the one-dimensional Kloosterman method is ``trivial to set up''),
seems to be an interesting but challenging problem;
see \cite{vishe2019rational} for some fascinating progress
(with fairly good cancellation---but one might optimistically hope for even more).
% % From Vishe's paper: "Ideally, one would also like to obtain some extra cancellations from the sum over different lines dc of a fixed height in (1.7). However, so far, we have been unable to do so."
% in this direction.
}

\rmk{
% In general,
% it would be very interesting to adapt the Kloosterman method to the setting of universal torsors.
% (think about this more carefully.
% e.g. for a cubic surface would Kloosterman method be possible on some ``lifted circle method'' for the universal torsor?
% or would Kloosterman method maybe be useful for torsors of non-c.i.'s?)
% % From Derenthal's thesis: "Batyrev and Popov [BP04] have determined the Cox ring of smooth
% % Del Pezzo surfaces of degree ≥ 3. In the case of smooth cubic surfaces, this
% % realizes the universal torsor as an open subset of a 9-dimensional variety,
% % defined by 81 equations in 27-dimensional affine space (see Section 3.3).
% % However, estimating the number of points on the universal torsor seems to
% % be very hard in this case."
At least for complete intersections,
it might be interesting to try ``lifting'' the Kloosterman method to the setting of universal torsors,
but this could be difficult
(and maybe only directly relevant to $W$'s of small dimension,
e.g.~projective surfaces).
}

\section{Enlarging or deforming the delta method}
% Enlarging or deforming delta (Possibilities for enlarging or deforming the delta method)
\label{SEC:possibly-averaging-deforming-enlarging-or-smoothing-the-delta-method}
% averaging, deforming, enlarging/expanding/extending/elongating, smoothing
% Deforming or enlarging (the search space in) the delta method

For a nice historical introduction to \cites{duke1993bounds,heath1996new}'s delta method,
we refer the reader to \cite{iwaniec2004analytic}*{Chapter~20}.
In the rest of \S\ref{SEC:possibly-averaging-deforming-enlarging-or-smoothing-the-delta-method},
we discuss some possibilities---mostly far from being fully understood so far---for modifying or perturbing the delta method.

\subsection{Deforming or enlarging the search space}

Let $(F,w)$ be a smooth pair for some $\PP^{m-1}_\QQ$-smooth homogeneous $F\in \ZZ[x_1,\dots,x_m]$ of degree $\geq 1$.
Then we have
\mathd{
(1+O_A(Y^{-A}))
\cdot \frac{N_{F,w}(X)}{X^{m-\deg{F}}}
= X^{-m+\deg{F}}Y^{-2}
\sum_{n\geq1}
\sum_{\bm{c}\in\ZZ^m}
n^{-m}S_{\bm{c}}(n)
I_{\bm{c},X,Y}(n)
}
for all $(X,Y)\in\RR_{>0}^2$,
where for all $(\bm{c},n)\in\RR^m\times\RR_{>0}$,
we let
\mathd{
I_{\bm{c},X,Y}(n)
\defeq \int_{\bm{x}\in\RR^m}
d\bm{x}\,w(\bm{x}/X)
h(n/Y, F(\bm{x})/Y^2)
e(-\bm{c}\cdot\bm{x}/n)
\quad
\textnormal{(following \cite{heath1996new})}.
}

For simplicity,
we have restricted to \emph{homogeneously} expanding weights $\bm{x}\mapsto w(\bm{x}/X)$,
but it could be useful to work more generally.
Note the following:
\begin{enumerate}[(1)]
    \item given $F,w,X$,
    the weighted count $N_{F,w}(X)$ is independent of $Y$;
    and
    
    \item to explore the solutions to $F=0$,
    we have \emph{both} parameters $w,X$ at our disposal.
\end{enumerate}

\subsubsection{Changing the region}

At a basic level,
we have already seen the possible benefits of (2) in Chapter~\ref{CHAP:approx-variances} (which is based on \cite{diaconu2019admissible}).
But in Chapters~\ref{CHAP:isolating-special-solutions}--\ref{CHAP:using-mean-value-L-function-predictions},
we fixed $Y\defeq X^{3/2}$, even as we let $w$ vary;
perhaps one could get sharper results by letting $Y$ vary with $w$.

% Enlarging: Diaconu, Ghosh, Sarnak; Heath-Brown's conjecture, Booker--Sutherland data

% Deforming in another problem: Marmon, Vishe

\subsubsection{Changing the modulus cutoff parameter}

% From Overleaf "Deformed integral estimates and a larger delta method"...
Typically one chooses $Y\asymp_{F,w}X^{(\deg{F})/2}$;
what if one perturbs $Y$
by $X^{\pm\delta}$,
or $(\log{X})^{\pm\delta}$, or\dots?
\cites{munshi2015pairs,marmon2019hasse} have made a first step in this direction.
% (for quadric pencils, or quartics, in many variables).
These works are based on \emph{modifying} $I_{\bm{c},X,Y}(n)$.

But we do not seem to have a deep understanding of what choices should work.
Conceptually,
there are at least two natural problems one could look at.
(For concreteness,
we fix $(\deg{F},m)=(3,6)$,
but the discussion below certainly generalizes easily.)
\begin{enumerate}[(1)]
    \item Given $X'\ll X$,
    count points $\bm{x}\ll X$ with $F(\bm{x})\ll (X')^3$ such that $F(\bm{x}) = 0$.
    (The basic question here is whether the condition $F(\bm{x})\ll (X')^3$ can be used to give better integral estimates,
    or not.
    A tantalizing open question is to recover the Hua bound $N_{x_1^3+\dots+x_6^3}(X)\ll_\eps X^{7/2+\eps}$ unconditionally using the delta method.)
    
    \item Given a separable form $F = F_1(\bm{x}_1) - F_2(\bm{x}_2)$,
    and $X_1,X_2\geq X$,
    count points with $\norm{\bm{x}_1}\asymp X_1$ and $\norm{\bm{x}_2}\asymp X_2$ with $F_1(\bm{x}_1)\ll X^3$ and $F_2(\bm{x}_2)\ll X^3$,
    such that $F_1(\bm{x}_1) = F_2(\bm{x}_2)$.
    (For variance computations when $F_1=F_2$,
    we need to let both $X_1,X_2$ grow the same amount;
    for the Hasse principle for $F$,
    it suffices to simply let one of $X_1,X_2$ grow and keep the other fixed, e.g.~$X_2=X$.
    It could also be worth splitting $F$ in different ways,
    e.g.~not just $6=3+3$ or $6=3+1+1+1$ but also in between,
    like $6=4+1+1$.)
\end{enumerate}

\ques{
In both cases,
there is freedom in how we choose the modulus cutoff $Y$ in the delta method;
what are the best choices?
}

\rmk{
Even after one has chosen $Y$,
it is unclear what the ``best form'' of the delta method is.
For (1),
after choosing $Y\defeq(X')^{(\deg{F})/2}$ for a certain $X'\ll X$,
\cite{marmon2019hasse}---inspired by \cite{munshi2015pairs}---take the Fourier transform of $U(X^3\xi/Y^2)h(r,X^3\xi/Y^2)$
(and then plug in $\xi = F(\tilde{\bm{x}})$)
to get something closely resembling a ``heuristic circle method with denominators $n\ll Y$''.
Can we do better by explicitly ``remembering'' the condition $F(\tilde{\bm{x}})\ll Y^2/X^3$
(given that the Fourier transform ``forgets'' it)?

And how does the technique for (1) generalize to (2)?
Should one apply the two-variable Fourier transform to $U(X^3\xi_1/Y^2)U(X^3\xi_2/Y^2)h(r,X^3[\xi_1+\xi_2]/Y^2)$
(and then plug in $\xi_i = F_i(\tilde{\bm{x}}_i)$),
moving away from the classical ``heuristic'' circle method
(but still likely having decay in $u_1,u_2,u_1-u_2$ ``localizing'' towards the classical $u_1=u_2=u$),
or should one directly reduce to the ``heuristic'' (smoothed) circle method and deal with the weight $\prod U(X^3F_i(\tilde{\bm{x}}_i)/Y^2)$ later in the integral estimates?
}

\subsection{Smoothing or averaging}

% Deforming and smoothing: Sarnak
Given an expression from the delta method,
one could try to include it into some kind of weighted average---whether it be over archimedean or adelic parameters,
or over number fields or varieties,
or perhaps (maybe by positivity) even over expressions without immediate Diophantine interpretation.
Ideally,
one would like to preserve as much information as possible
about the original varieties of interest.

\rmk{
% Sarnak has suggested the rough idea of ``deforming and smoothing''
One might hope to somehow ``deform and smooth''
(in the Fourier-analytic spirit of e.g.~\cite{qu2007selberg}*{Lemma~3.2} or \cite{selberg1989collected}*{p.~169 (p.~10 in Paper~12), use of Parseval})
to reduce the RMT-based input in
Theorem~\ref{THM:HLH-and-3-cubes-application}
(or at least in the more \emph{qualitative} parts thereof)
to statements about \emph{low-lying} zeros---though
I currently do not see how to do so.
% implement such a strategy effectively.
}

Note that on the ``right-hand side'' of the delta method,
the $(X,Y,w)$-dependence essentially lies in the factors $(X^{\deg{F}}/Y^2)
\cdot X^{-m}I_{\bm{c},X,Y}(n)$.

\ques{
Are there interesting examples where one can average over some or all of $X,Y,w$ in the delta method to simplify \emph{qualitative} Diophantine analysis?
}

% The toy phenomenon we 
As an oversimplified toy example,
consider the following \emph{$X$-averaging} of a Mellin transform:
\mathds{
\int_{X}d^\times{X}\,g(X)
\int_{n}d^\times{n}\,f(n/X)n^s
&= \int_{n,v}d^\times{n}\,d^\times{v}\,
g(n/v)f(v)n^s \\
&= \int_{v}d^\times{v}\,f(v)
\int_{n}d^\times{n}\,g(n/v)n^s,
}
where $v\defeq n/X$.
Even if $f$ is mysterious,
the $X$-averaging moves the $n$-dependence away from $f$,
leaving the Mellin decay in $\abs{s}$ to be dictated by the smoothness of $g$.

\rmk{
In fact,
such ``convolution creation'' has appeared before in work on delta-like methods.

In order to prove $N_{F,w}(X)\ll_\eps X^{3+\eps}$ under Hypothesis HW,
\cite{hooley_greaves_harman_huxley_1997}*{p.~180, (17)}
uses dyadic averaging in $X$
to avoid the interesting but harder derivative estimates of \cite{heath1998circle} (who does not smooth at all).
% I did not appreciate this until recently, but
The point is that without smoothing,
the \emph{trivial} $n$-derivative bounds on $I_{\bm{c}}(n)$ would have a fatal loss as $n\to 0$
(i.e.~at the end of the argument,
the final $n$-exponent would become negative,
whereas it was zero before differentiating).\footnote{This is why \cite{hooley1986HasseWeil} needed to separately consider small $n$,
i.e.~``junior arcs'' \cite{hooley1986HasseWeil}*{p.~81, \S9},
getting a final exponent of $60/19+\eps=3.15\ldots+\eps$ instead of $3+\eps$.
See \cite{hooley_greaves_harman_huxley_1997}*{p.~176} for Hooley's commentary.}
% in which he also happens to suggest that his methods ``may well'' have applications to
% ternary representation problems
% (cf.~Corollary~\ref{COR:HLH-and-3-cubes-application},
% as well as the conjectures in \cite{hooley1986some}).
% EDIT 3/16/2021: misread Hooley; he meant general (e.g. non-diagonal) ternary cubic forms, not general ternary representation problems}
However,
with averaging,
the derivative on $n$ can be moved to a harmless ``smoothing'' factor,\footnote{Strictly speaking,
Hooley does not smooth,
but rather uses a positivity-based design,
together with the ``convolution-like sub-structure'' created by averaging,
to ``eliminate'' the $n$-dependence.}
without Heath-Brown's \emph{improved} $n$-derivative bounds.

Similarly,
one could likely simplify parts of \cite{wang2021_large_sieve_diagonal_cubic_forms} (especially in archimedean aspects, at least in the $m$-even Fermat case) by dyadic (or slightly larger) smoothing.
% (I was unaware of the benefits of Hooley's smoothing at the time, so followed Heath-Brown's more difficult approach.)
% Of course,
% there is a tradeoff between the amount of averaging and the precision of the final results.
}

\rmk{
We call the toy example above ``oversimplified''
because while the factor $f(n/X)$ only depends on $n/X$,
the actual factor $(X^{\deg{F}}/Y^2)
\cdot X^{-m}I_{\bm{c},X,Y}(n)$ of interest depends on
$n/Y,X\bm{c}/n,X^{\deg{F}}/Y^2$
(and $F,w$)---though at least
if $\deg{F}=2$ and we restrict to $Y=X$,
then $(X^{\deg{F}}/Y^2)
\cdot X^{-m}I_{\bm{c},X,Y}(n)$ depends only on $n/X,\bm{c}$ (and $F,w$).
}

\rmk{
The limitations of averaging---and the extent of resulting internal cancellation (if any)---are unclear,
due to the (currently) mysterious nature of
the multi-parameter families $I_{\bm{c},X,Y}(n)$ and $I^\vee_{\bm{c},X,Y}(s)$
(with $F,w$ dependence suppressed).
Experimentation,
whether theoretical or computational,
could be enlightening.
}

\section{Problems inspired by the Ratios Conjectures}
% Speculation on other problems
\label{SEC:speculation-on-other-Diophantine-problems}

% Recall Theorem~\ref{THM:generic-HLH-error-bounds}.
\subsection{What is the optimal power saving?}

We have not attempted to optimize
the power saving $X^{\Omega(1)}$ in Theorem~\ref{THM:generic-HLH-error-bounds}(b),
since there could be a more efficient approach waiting to be understood.
However,
a sufficiently large power saving might bring the Hasse principle for $m=5$ within reach
(perhaps under GLH and \cite{conrey2005integral}'s standard moment conjectures,
applied to certain Artin $L$-functions).

From the opposite angle,
it would also be very interesting to determine any hard limits on what power savings can be expected,
e.g.~coming from Brauer--Manin obstructions for $m=4$
(but there the $F^\vee(\bm{c})=0$ contribution could at least sometimes muddy the analysis).

\rmk{
When $F$ is non-diagonal, $m\leq 5$, \emph{or} $\deg{F}\geq 4$,
even the $F^\vee(\bm{c})=0$ contribution itself
deserves to be better understood
(and it may be the most tractable starting point);
see the discussion in
% \cite{wang2021_isolating_special_solutions}*{\S1.4, Remark~1.24;
% \S1.3, Remark~1.7;
% \S1.3.1, Remark~1.19; \S5.1, Remark~5.1;
% and \S1.5}.
Remark~\ref{RMK:non-diagonal-F},
Example~\ref{EX:II.1.7},
\cite{wang2021_isolating_special_solutions}*{Remarks~1.19 and~5.1},
and Question~\ref{QUES:II.1.26}.
And when $m=4$,
even the $\bm{c}=\bm{0}$ contribution deserves to be better understood;
% see e.g.~\cite{wang2021_isolating_special_solutions}*{Remark~B.4},
% and cf.~\cite{browning2009quantitative}*{\S8.3.3's heuristic analysis}.
see e.g.~Remark~\ref{RMK:II.B.4}.
}

\ques{
% [Cf.~\cite{wang2021_isolating_special_solutions}*{\S1.3, Remark~1.15}]
[Cf.~Remark~\ref{RMK:II.1.15}]
\label{QUES:where-lies-the-manin-peyre-vs-major-arc-based-heuristic-difference?}
When the Manin--Peyre prediction for $m=4$
\emph{differs} from
the ``$\QQ$-lines plus major arcs'' heuristic of \cite{browning2009quantitative}*{\S8.3.3}---as
is the case when $F=x_1^3+\dots+x_4^3$ \cite{browning2009quantitative}*{p.~149},
for instance---where does the \emph{difference} lie, precisely, in the delta method?
}

Besides $V(x_1^3+\dots+x_4^3)$,
another example for Question~\ref{QUES:where-lies-the-manin-peyre-vs-major-arc-based-heuristic-difference?} is $V(5x_1^3+12x_2^3+9x_3^3+10x_4^3)$
(due to \cite{cassels1966hasse}),
% which in fact \emph{fails the Hasse principle} over $\QQ$.
which in fact has a \emph{Brauer--Manin obstruction to rational points} over $\QQ$.
The latter example is arguably cleaner,
% https://www.merriam-webster.com/dictionary/in%20that
in that it has $V(F^\vee)(\QQ)=\emptyset$.

\pf{
[Proof that $V(F^\vee)(\QQ)=\emptyset$]
Let $m\defeq4$,
let $\bm{F}\defeq(5,12,9,10)$,
let $F\defeq\sum_{i\in[m]}F_ix_i^3$,
and let $V\defeq V(F)/\QQ$.
Fix $\bm{c}\in\QQ^m\setminus\set{\bm{0}}$ with $F^\vee(\bm{c})=0$,
i.e.~$F^\vee(\bm{c})=0$.
Then by the classical ``factorization'' of $F^\vee(\bm{c})$ (see \eqref{EXPR:diagonal-discriminant-factorization} in \S\ref{SUBSEC:diagonal-dual-example}),
we have $\sum_{i\in[m]}\pm(c_i^3/F_i)^{1/2}=0$ for some choice of signs.
But $\bm{c}\neq\bm{0}$,
so in particular,
there must exist
a nonempty set $\mcal{I}\belongs[m]$,
and rationals $d,x_i\in\QQ^\times$ for $i\in\mcal{I}$,
such that
(i) $c_i^3/F_i = F_i^2x_i^6d^3\in d\cdot(\QQ^\times)^2$ for all $i\in\mcal{I}$,
and (ii) $\sum_{i\in\mcal{I}}F_ix_i^3=0$.
But $V(\QQ)=\emptyset$,
% and $(x_i)_{i\in\mcal{I}}\in(\QQ^\times)^\mcal{I}\belongs\QQ^\mcal{I}\setminus\set{\bm{0}}$,
and $(x_i)_{i\in\mcal{I}}\in\QQ^\mcal{I}\setminus\set{\bm{0}}$,
so (ii) is impossible.
Thus $\bm{c}$ cannot exist.
}

In view of Remark~\ref{RMK:II.1.15},
such an example naturally raises the following question:
\ques{
Given a smooth cubic surface $V=V(F)/\QQ$ with
$V(\bd{A}_\QQ)\neq\emptyset$ and $V(\bd{A}_\QQ)^{\Br}=\emptyset$
(in the notation of \cite{jahnel2014brauer}*{Definition~IV.1.1 and Notation~IV.2.5}),
is it necessarily true that
$V(F^\vee)(\QQ)=\emptyset$,
i.e.~$V^\vee(\QQ)=\emptyset$
(in the notation of \S\ref{SEC:alg-geom-background})?
}

\subsection{Hooley's critical ternary triumvirate}

% \rmk{
% See $a^2+b^4+c^4$ discussion at the end of ``Cubic seminar talk notes v2 part 1''.
% }

Note that
\mathd{
1 = \frac13+\frac13+\frac13 = \frac12+\frac14+\frac14 = \frac12+\frac13+\frac16.
}
Let $R_4(a),R_6(a)$ denote
% the number of ways to write
% $a$ as a sum $y^2+x_1^4+x_2^4$ of a square and two biquadrates (fourth powers),
% $a=y^2+x_1^4+x_2^4$ with $y,\bm{x}\geq 0$
$\#\set{y,\bm{x}\geq0: y^2+x_1^4+x_2^4=a}$
% and $R_6(a)$ the number of ways
% as a sum $x^2+y^3+z^6$ of a square, a cube, and a sixth power, with $x,y,z\geq 0$.
% (resp.~$a=x^2+y^3+z^6$ with $x,y,z\geq 0$).
and $\#\set{x,y,z\geq0: x^2+y^3+z^6=a}$,
respectively.
\cite{hooley1986some} made \emph{precise asymptotic $\ell^2$ conjectures} for $R_4,R_6$
alongside the analogous \cite{hooley1986some}*{Conjecture~2} for $r_3$.
% $\eps$-free version of \emph{Hypothesis $K^\ast$} for $r_3$.
Hooley's ternary problems belong to the framework of
a ``$cba$ conjecture''
(with a goal---opposite the $abc$ conjecture---of \emph{producing points});
cf.~\cite{harpaz2017geometry}*{Conjecture~1.11}.

We focus on $R_4$
(for technical reasons to arise later).
% , though some of the initial analysis also applies to $R_6$.
Just like for $r_3$,
the first moment of $R_4$ grows linearly:
$\sum_{a\leq B}R_4(a)
% \asymp \#\set{(y,x_1,x_2)\in[0,B^{1/2}]\times[0,B^{1/4}]^2}
\propto (B^{1/2})(B^{1/4})^2(1+o(1)) = B(1+o(1))$
as $B\to\infty$.
%, and likewise for $R_6$
But now,
$\ell^2$ is in some sense easier than for $r_3$:
for $R_4$,
the divisor bound easily implies the following result.
\prop{
% [Divisor bound]
\emph{Unconditionally}, $\sum_{a\leq B}R_4(a)^2\ll_\eps B^{1+\eps}$.
% (and likewise for $R_6$).
}

% \pf{
% Routine by the divisor bound.
% }

\rmk{
\cite{hooley1979new} proved $\sum_{a\leq B}R_4(a)^2\ll_\eps B(\log{B})^{4/\pi-1+\eps}$
% by analyzing $\Delta(n)\defeq \max_u\sum_{u<d\leq eu}\bm{1}_{d\mid n}$ on average over $n$.
by introducing and analyzing
``$\Delta$-functions''
(i.e.~certain divisor sums).
% ---an idea which later played an important role in \cites{vaughan1985probleme,vaughan1986waring}'s asymptotic proof of $G(3)\leq 8$.
\cite{robert2011nombre} improved \cite{hooley1979new}'s bound to
(something a bit stronger than)
$O_\eps(B(\log{B})^\eps)$;
% building on works such as \cite{hall1986average};
but $O(B)$ seems to remain open.
}

\ques{
\label{QUES:remove-eps-from-divisor-bound-for-R_4-second-moment}
Can the bound $O_\eps(B^{1+\eps})$ be improved to $O(B)$
for $R_4$ (or for $R_6$)?
}

\obs{
%Let $Y\defeq X^2$.
If $\#\set{(\bm{x},\bm{y})
\in [0,X]^4\times[X^2/2,X^2]^2
: y_1^2+x_1^4+x_2^4 = y_2^2+x_3^4+x_4^4}\ll X^4$,
then $\set{a\in\ZZ_{\geq0}: R_4(a)\neq 0}$ has \emph{positive lower density}.
}

\pf{
Use double counting and Cauchy
as for $r_3$,
but with the large variable ``$y$'' restricted to $[X^2/2,X^2]$.
}

% Although most of the ideas above should also hold for $R_6$, we will eventually use the homogeneity of $F_4(\bm{x})\defeq x_1^4+x_2^4-x_3^4-x_4^4$ below.

For the question above (restricted to $\norm{\bm{y}}\asymp X^2$ for simplicity),
we now describe a possible conditional approach whose roots lie close in spirit to work of Hooley (e.g.~\cites{hooley1981waring,hooley1983some}).
Let $F_4(\bm{x})=F_4(x_1,\dots,x_4)\defeq x_1^4+x_2^4-x_3^4-x_4^4$.
Let $n\defeq y_1+y_2$ and $N\defeq X^{\deg{F_4}/2} = X^2$.
Then
% the point count above over $(\bm{x},\bm{y})\in [0,X]^4\times[N,2N]^2$ is \emph{at most}
\mathd{
\#\set{(\bm{x},\bm{y})
\in [0,X]^4\times[N/2,N]^2
: (y_2-y_1)(y_2+y_1) = F_4(\bm{x})}
\leq \sum_{n\in[N,2N]}
\sum_{\bm{x}\in[0,X]^4}\bm{1}_{n\mid F_4(\bm{x})}.
}
By positivity,
it is harmless to over-extend the right-hand side to a smooth sum
\mathd{
\mcal{S}_4
\defeq \sum_{n\geq 1}D(n/N)
\sum_{\bm{x}\in\ZZ^4}\bm{1}_{n\mid F_4(\bm{x})}\cdot w(\bm{x}/X)
}
with $D\in C^\infty_c(\RR_{>0})$ and $w\in C^\infty_c(\RR^4)$.
Given $n$,
we then expand $\bm{1}_{n\mid F_4(\bm{x})}$
% as $n^{-1}\sum_{a\bmod{n}}e_n(aF_4(\bm{x}))$
over ``vertices'' $a\bmod{n}$ on the circle,
using additive characters $t\mapsto e_n(at)$,
to get
\mathd{
\mcal{S}_4
= \sum_{n\geq1} D(n/N)
n^{-1}\sum_{a\in \ZZ/n}
\sum_{\bm{x}\in \ZZ^4} e_n(aF_4(\bm{x}))
\cdot w(\bm{x}/X).
}
(One might call this
%more than a single n here!
the ``polygon(s) method'' for divisor-type problems;
it is very close in spirit to the circle method.)
Now let
$S_{4,\bm{c}}(n)
\defeq \sum_{a\in \ZZ/n}\sum_{\bm{z}\in (\ZZ/n)^4}e_n(aF_4(\bm{z}) + \bm{c}\cdot\bm{z})$.
Then by Poisson summation,
one can show that
\mathd{
\mcal{S}_4
= \sum_{n\geq1}D(n/N)
n^{-1}\sum_{\bm{c}\in\ZZ^4}
S_{4,\bm{c}}(n)\cdot (X/n)^4\hat{w}(X\bm{c}/n).
}

\cnj{
Automorphy and GRH for smooth hyperplane sections of the form $V_{\PP^3}(F_4, \bm{c}\cdot\bm{x})_\QQ$ (with $\bm{c}\in \ZZ^4\setminus \set{\bm{0}}$)
should recover a ``$B^{1+\eps}$-like'' bound,
$\mcal{S}_4\ll_\eps X^{4+\eps}$.
}

\rmk{
These hyperplane sections are essentially \emph{smooth plane quartic curves}.
In general, a smooth projective curve over $\QQ$ has the same Hasse--Weil $L$-function as its Jacobian (an abelian variety of dimension equal to the genus of the curve).
So the hypotheses above concern the $L$-functions of certain abelian varieties of dimension $3$.
}

Assuming the question of proving $\mcal{S}_4\ll X^4$ is indeed
open---and not susceptible to hypersurface slicing methods---the
% (such as the \emph{determinant method})---the
above approach is interesting in that one might be able to
\emph{remove the $\eps$} conditionally (in the fashion of Chapter~\ref{CHAP:using-mean-value-L-function-predictions}).

\subsection{On quartics}

A ``standard'' application of the delta method, without cancellation over $\bm{c}$, is (at least morally) limited in scope to quadratic and cubic problems---at least as far as hypersurfaces are concerned.
It is amusing, but probably incorrect, to compare this situation to the following classical fact:
every Diophantine equation over $\ZZ$ (or $\QQ$) is equivalent to a system of quadratic equations, and thus to a single quartic equation;
an analogous statement holds for homogeneous Diophantine equations over $\ZZ$ (or $\QQ$), by \cite{mumford1970varieties}*{Theorem~1} (a result based on Veronese embeddings).
% See https://arxiv.org/pdf/2001.06687.pdf for more details?

In ``natural'' quartic and higher-degree equations, there should be more structure that remains to be uncovered.
For instance,
\cite{marmon2019hasse} uses ``averaging over $a\bmod{q}$'' to get the best known results on general quartic projective hypersurfaces,
although it uses a ``non-standard modulus cutoff'' in the delta method.
Furthermore,
Question~\ref{QUES:remove-eps-from-divisor-bound-for-R_4-second-moment} presents
a ``borderline quartic'' example where
the ``standard delta method'' should provide a reasonable conditional avenue forward.
There is also some hope of using RMT-type predictions to analyze quartics beyond \cite{marmon2019hasse},
but any attempt at a fully rigorous conditional analysis might face serious difficulties at the moment.

At the same time,
it is very natural to wonder if there might be a better (more
efficient) way to ``complete exponential sums'' for quartics like $F_4$.
So it cannot hurt to list a few half-baked thoughts,
in the hope of inspiring further discussion or creativity:
\begin{enumerate}[(1)]
    \item The tantalizing Question~\ref{QUES:remove-eps-from-divisor-bound-for-R_4-second-moment} may provide a good ``critical'' testing ground for old and new quartic techniques alike.
    
    \item At least for smooth projective cubic hypersurfaces,
    one expects all ``special subvarieties'' to be \emph{linear}.
    But at least \emph{some} smooth projective quartic hypersurfaces
    contain special \emph{quadrics}:
    see e.g.~Example~\ref{EX:Wooley-special-quadratic-subvarieties}.
    
    The contrast between cubics and quartics
    seems vaguely parallel to the
    difference between Szemer\'{e}di's theorem for $3$-term and $4$-term arithmetic progressions---the former being of a ``linear'' nature,
    and the latter of a ``quadratic'' nature.
    (But maybe ``quadratic'' means something else here,
    in that the set of $4$-term arithmetic progressions
    is cut out by a system of $2$ linear equations.)
    
    \item For homogeneous quartics,
    maybe $L$-functions associated to
    more complicated slices (e.g.~degree $2$ hypersurfaces)
    should be relevant,
    not just (linear) hyperplane slices as in the standard Kloosterman method for homogeneous equations?
    
    \item The standard Kloosterman method is based on \emph{interpolation}---i.e.~\emph{completing} incomplete sums---via
    % (\emph{linear})
    Nyquist--Shannon sampling.
    (Given a weight $w\in C^\infty_c(\RR)$,
    a modulus $n\gg X$,
    and a residue $z\bmod{n}$,
    one replaces
    $\sum_{x\equiv z\bmod{n}}w(x/X)$
    with a ``dual'' sum of length $\lessapprox n/X$---a certain linear combination of \emph{linear} phases $e_n(c\cdot z)$.)
    
    Is there a useful ``nonlinear'' version of Nyquist--Shannon sampling?
    % - Googling "nonlinear nyquist shannon sampling" leads to
    % https://en.wikipedia.org/wiki/Nonuniform_sampling
    
    \item One might try to replace ``exact'' sums with ``approximate'' sums.
    For example,
    in the context of $\mcal{S}_4$
    (as defined after Question~\ref{QUES:remove-eps-from-divisor-bound-for-R_4-second-moment}),
    one could write
    \mathd{
    \sum_{x^2\equiv z^2\bmod{n}}w(x/X)
    = \sum_{y\equiv z^2\bmod{n}}
    \bm{1}_{y=\square}\cdot w(y^{1/2}/X)
    \quad\textnormal{for any given $z\bmod{n}$},
    }
    and then try to detect $\bm{1}_{y=\square}$ ``statistically''
    % using Legendre symbols
    (in the spirit of the ``square sieve'').
    
    \item Perhaps one could seek inspiration
    from quadratic Fourier analysis,
    or from formulas in sphere packing that contain $f(\sqrt{n})$'s,
    or from
    \cite{kumaraswamy2018zeros}*{pp.~25--27, \S2.2 Other versions of the $\delta$-method}'s
    discussion of Jutila's and Munshi's ideas,
    or from\dots.
\end{enumerate}

\appendix

% \chapter{Discriminants}
% % Background on discriminants
% \subfile{correctly_defining_V_c_discriminant}
% copy from Paper~I, v1

\chapter{Modularity questions}
\label{CHAP:natural-modularity-questions}

% \section{Natural modularity questions}
% \label{SEC:natural-modularity-questions-for-V_c-when-m=6}
% % \section{Endomorphisms of our abelian fivefolds}

In this appendix,
we fix a $\PP^{m-1}_\QQ$-smooth cubic form $F/\ZZ$ in $m\in\set{4,6}$ variables,
and consider the Hasse--Weil $L$-functions $L(s,V_{\bm{c}})$ appearing in Example~\ref{EX:one-sided-first-order-approximations-of-Phi}.
Fix $\bm{c}$.
(Here we restrict attention to $\bm{c}\in \ZZ^m$ with $V_{\bm{c}}$ smooth of dimension $m_\ast$.)

\section{A general discussion}

Let us first recall some useful definitions and facts.
Let $\ell$ be a prime,
and $K$ a field of characteristic $\neq \ell$.
Let $X$ be a smooth, projective, geometrically integral variety over $K$.
Let $P_{X/K}$ denote the \emph{Picard variety} of $X$ (following \cite{poonen2017rational}*{\S5.7.3}, say),
and $A_{X/K}$ the \emph{Albanese variety} of $X$ (following \cite{poonen2017rational}*{Example~5.12.11}, say).
It is known that $P_{X/K},A_{X/K}$ are \emph{dual} abelian varieties over $K$ \cite{poonen2017rational}*{Theorem~5.7.20}.

\prop{
[Standard]
\label{PROP:first-etale-cohomology-passes-to-Albanese}
In this setting,
we have a canonical isomorphism $H^1(X_{\ol{K}}, \QQ_\ell)\cong V_\ell(P_{X/K})(-1)\cong V_\ell(A_{X/K})^\vee$ of $\ell$-adic $G_K$-representations.
}

\pf{
To get the statement, one can combine, for instance, \cite{PetersenMO56378semisimplicity}*{Remark} and \cite{BellaicheMO132271Weil_pairing}*{par.~1}.
(Alternatively, see \cite{ghorpade2008etale}*{Proposition~9.6}, which however assumes ``for simplicity'' that $X$ has a $K$-rational point.)
The standard proof follows \cite{TomMO135648first_etale_cohomology}*{Piotr Achinger's comment and Adel Betina's answer}.
By $\ell$-adic Kummer theory (cf.~\cite{poonen2017rational}*{\S7.6.3}) and \cite{poonen2017rational}*{Proposition~6.6.1},
one obtains isomorphisms $H^1(X_{\ol{K}}, (\ZZ/\ell^n)(1))\cong \Pic(X_{\ol{K}})[\ell^n]$ for $n\geq 1$, compatible with multiplication by $\ell$ (as $n$ varies).
The result follows.
(See \cite{poonen2017rational}*{\S\S7.5.3--7.5.4} for some useful background on $\ell$-adic cohomology, Tate modules, and Tate twists.)
}

We now return to the original setting of this appendix.
If $m=4$,
then $V_{\bm{c}}$ is a genus one curve,
so $L(s,V_{\bm{c}})$ is known to be the $L$-function of a weight $2$ modular cusp form
(since $L(s,V_{\bm{c}})=L(s,J(V_{\bm{c}}))$ by Proposition~\ref{PROP:first-etale-cohomology-passes-to-Albanese} and \cite{poonen2017rational}*{Example~5.12.12},
and elliptic curves over $\QQ$ are known to be modular).

Now suppose $m=6$.
Let $A_{\bm{c}}$ denote the Albanese variety of $F(V_{\bm{c}})/\QQ$,
the Fano surface of lines on $V_{\bm{c}}$.
If we consider the $\ell$-adic $G_{\QQ}$-representations
$M_{\bm{c}}\defeq H^3(V_{\bm{c}}\times \ol{\QQ}, \QQ_\ell)$ and $V_\ell(A_{\bm{c}})^\vee = H^1(A_{\bm{c}}\times \ol{\QQ}, \QQ_\ell)$
defining $L(s,V_{\bm{c}}),L(s,A_{\bm{c}})$, respectively,
then $M_{\bm{c}}(1)\cong H^1(F(V_{\bm{c}})\times \ol{\QQ}, \QQ_\ell)\cong V_\ell(A_{\bm{c}})^\vee$, by \cite{reid1972complete}*{Appendix~4.3, Corollaries~4.29} (or \cite{charles2015conjecture}*{paragraph containing diagram~(5)}; cf.~\cite{debarre2021lines}*{first paragraph of the proof of Theorem~4.1}) and Proposition~\ref{PROP:first-etale-cohomology-passes-to-Albanese}.
In particular, $L(s,V_{\bm{c}}) = L(s,A_{\bm{c}})$.

Here $\dim A_{\bm{c}}=5$.
% so in general, using only that $A_{\bm{c}}$ is an abelian variety over $\QQ$ of dimension $5$,
% one does not yet know how to prove that $A_{\bm{c}}$ is automorphic.
But at least in general,
abelian varieties $A/\QQ$ of dimension $5$ are not yet known to be automorphic.
(In this connection,
the current state of the art is \cite{boxer2021abelian},
which ``potentially addresses'' abelian \emph{surfaces} over certain fields like $\QQ$.)

However,
one might ask if (at least) for certain $F$'s,
like $x_1^3+\dots+x_6^3$,
the $A_{\bm{c}}$'s might have some special structure that could ``reduce'' the complexity of $L(s,A_{\bm{c}})$.
In this vein,
% https://english.stackexchange.com/questions/266275/question-about-the-use-of-in-this-spirit-expression
recall that the Hasse--Weil $L$-function of a diagonal projective hypersurface over $\QQ$ of degree $d\geq2$ always factors into $L$-functions of degree $\phi(d)$ associated to certain Hecke characters on $\QQ(\zeta_d)$;
at least up to bad factors,
this observation is due to Weil \cite{weil1952jacobi}.

Now let $F\defeq x_1^3+\dots+x_6^3$.
% and $V\defeq V(F)/\QQ$.
% Sarnak (2019) asked if an analog of Weil's observation holds for smooth hyperplane sections $V_{\bm{c}}$ of $V$.
Does some loose analog of Weil's observation hold for the ``codimension-$2$ diagonal systems'' $V_{\bm{c}}$?
In the most direct sense,
the answer to this basic question is probably negative in general
(see Proposition~\ref{PROP:absolute/strong/potential-irreducibility-for-V_c-when-c=(1,2,3,4,5,6)} below).
So even when $F=x_1^3+\dots+x_6^3$,
the conjectured automorphy of $V_{\bm{c}}$ (or $A_{\bm{c}}$) does not seem to easily follow from existing progress on the Langlands program;
% there is a good chance that interesting new ideas are needed in order to (understand/probe... satisfactorily)
one may well need interesting new ideas in order to
probe the family $\bm{c}\mapsto L(s,V_{\bm{c}})$ from an automorphic point of view.

\section{Notes on an explicit diagonal system}
% Notes on an explicit diagonal system, and a conjecture
% Preliminary notes on an explicit diagonal system
% ("example" is perhaps too vague for section title; should clarify: "example" of *what*? certainly not an example where modularity/automorphy is being proved or disproved... but also need to clarify if it's an example of a "natural modularity question" or of...)
% An explicit example and some ensuing comments
% An explicit negative example and some resulting questions

% Let $F = x_1^3+\dots+x_6^3$ and $\bm{c} = (1,2,\dots,6)$.
% Then $F^\vee(\bm{c})=?$.
% Assume that $L_p(s,V_{\bm{c}}) = L_p(s,A_{\bm{c}})$ at the following primes: $p=5,11,?$.

% [Code: to be added.]

% \prop{
% $\End_{\ol{\QQ}}(A_{\bm{c},\ol{\QQ}}) = \ZZ$.
% }

% \pf{
% To be added.
% }

Armed with the theory of abelian varieties,\footnote{Thanks to Will Sawin for initial help in this direction.}
we can prove the following statement:
\prop{
\label{PROP:absolute/strong/potential-irreducibility-for-V_c-when-c=(1,2,3,4,5,6)}
Let $F\defeq x_1^3+\dots+x_6^3$ and $\bm{c}\defeq(1,2,3,4,5,6)$.
% Then for each number field $K/\QQ$,
% each $\ell$-adic $G_K$-representation associated to the $L$-function $L(s,V_{\bm{c}}\times K)$ over $K$ is irreducible.
Then for each prime $\ell$ and number field $K/\QQ$,
the $\ell$-adic $G_{\QQ}$-representations $M_{\bm{c}},V_\ell(A_{\bm{c}})$ are irreducible as $\ell$-adic $G_K$-representations.
}

\rmk{
Proposition~\ref{PROP:absolute/strong/potential-irreducibility-for-V_c-when-c=(1,2,3,4,5,6)} does not rule out the possibility that $L(s,V_{\bm{c}})$ could be
``deconstructed'' in a subtler way,
e.g.~as the $9$th symmetric power $L$-function of a $\GL_2$ representation.
}

\rmk{
It may or may not be natural to contrast Proposition~\ref{PROP:absolute/strong/potential-irreducibility-for-V_c-when-c=(1,2,3,4,5,6)} with \cite{patterson1997asymptotic}*{Theorem~3.1},
a result showing that
the ``one-dimensional constituents of $S_{\bm{c}}(q)$''
(before taking mixed $6$th moments over $a\bmod{q}$)
are ``nice'' over $\QQ(\zeta_3)$
(and in fact,
amenable to statistical analysis via cubic metaplectic forms \cite{louvel2011distribution}).
}

The conclusion of Proposition~\ref{PROP:absolute/strong/potential-irreducibility-for-V_c-when-c=(1,2,3,4,5,6)} is equivalent to the statement that $\End(A_{\bm{c}}\times\ol{\QQ})=\ZZ$.
Thus Proposition~\ref{PROP:absolute/strong/potential-irreducibility-for-V_c-when-c=(1,2,3,4,5,6)} naturally suggests the following conjecture:
\cnj{
\label{CNJ:typical-or-generic-irreducibility}
For \emph{almost all} $\bm{c}\in\ZZ^m$ with $F^\vee(\bm{c})\neq0$,
we have $\End(A_{\bm{c}}\times\ol{\QQ})=\ZZ$.
% (Is this a standard conjecture for nice geometric families of abelian varieties?
% See Masser 1996, Specialization of endomorphism rings of abelian varieties.)
}

\rmk{
By \cite{masser1996specialization},
the conjecture above should be equivalent to a question about the generic fiber of the family $A_{\bm{c}}$.
Furthermore,
a ``typical'' special fiber should control the possible behavior at the generic fiber.
But Proposition~\ref{PROP:absolute/strong/potential-irreducibility-for-V_c-when-c=(1,2,3,4,5,6)} does not seem to imply the conjecture ``for free'' (without further calculation).
}

\pf{
[Proof of Proposition~\ref{PROP:absolute/strong/potential-irreducibility-for-V_c-when-c=(1,2,3,4,5,6)}]
For convenience,
let $A\defeq A_{\bm{c}}$.
Now fix $\ell$.
We know that $M_{\bm{c}}(1)\cong V_\ell(A)^\vee$ as $\ell$-adic $G_\QQ$-representations.
So by Faltings' isogeny and semi-simplicity theorems
% \footnote{Thanks to Will Sawin for suggesting
% % (at IHP in 2019)
% the use of Faltings' isogeny theorem.}
(and the fact that endomorphism rings of abelian varieties are torsion-free as $\ZZ$-modules),
our desired irreducibility statement is equivalent to the statement that $\End(A_K)=\ZZ$ holds for all $K/\QQ$.

We now prove the latter statement---which is independent of $\ell$.
To begin,
we re-define $\ell\defeq2$
for convenience.
By Appendix~\ref{SUBSEC:local-zeta-code-and-data},
$V_{\bm{c}}$ has good reduction at all primes $p\in[5,100]$,
so by smooth proper base change,
$M_{\bm{c}}$ is unramified at all $p\in[5,100]$.
Since $M_{\bm{c}}(1)\cong V_\ell(A)^\vee$,
it follows that $V_\ell(A)$ is unramified at all $p\in[5,100]$.
So by the N\'{e}ron--Ogg--Shafarevich criterion,
% https://www.lmfdb.org/knowledge/show/ag.conductor
% https://en.wikipedia.org/wiki/N%C3%A9ron%E2%80%93Ogg%E2%80%93Shafarevich_criterion
% see Milne, Theorem 3.5 https://www.jmilne.org/math/CourseNotes/AV.pdf for a proof of Neron--Ogg--Shafarevich
$A$ has good reduction at all $p\in[5,100]$.

For an abelian variety $B$ over a field, let $\End^0(B)\defeq \End(B)\otimes_{\ZZ} \QQ$;
then $\End(B)$ is always a subring of $\End^0(B)$, finitely generated as a $\ZZ$-module.
In particular,
to achieve our original goal,
it suffices to prove that $\End^0(A_K)=\QQ$ holds for all $K/\QQ$.

Now fix $K/\QQ$,
and fix a N\'{e}ron model $\mcal{A}_K/\mcal{O}_K$ of $A_K$.
Then for each prime $\wp$ of $\mcal{O}_K$,
let $k_\wp\defeq\mcal{O}_K/\wp$ and $f_\wp\defeq[k_\wp:\FF_p]$.
Finally,
for each \emph{$A_K$-good} prime $\wp$,
let $A_\wp\defeq\mcal{A}_K\times k_\wp$ denote the reduction of $A_K$ modulo $\wp$;
let $\pi_\wp\in\End(A_\wp)$ denote the ``geometric Frobenius'' \emph{morphism} $A_\wp\to A_\wp$ over $k_\wp$ (as in \cite{stacks-project}*{\href{https://stacks.math.columbia.edu/tag/03SQ}{Tag 03SQ}});
and \emph{if} $\wp\nmid\ell$,
let $P_\wp\in1+t\ZZ[t]$ denote the ``\emph{reverse} characteristic polynomial'' of $\pi_\wp$.
Then given $\wp\mid p\in[5,100]$,
we have $P_\wp(t)
= \prod(1-\beta_i^{f_\wp}t)$,
where $P_{p\ZZ}(t)
= \prod(1-\beta_it)$,
where for $p\in\set{5,7,11,13}$ we know by Appendix~\ref{SUBSEC:local-zeta-code-and-data} that
\begin{enumerate}[(1)]
    \item $P_{5\ZZ}(t)
    = (5t^2+1)\cdot
    (625t^8+100t^6+4t^2+1)$
    at $p=5$,
    
    \item $P_{7\ZZ}(t)
    = (7t^2+t+1)\cdot
    (7t^2+4t+1)^4$
    at $p=7$,
    
    \item $P_{11\ZZ}(t)
    = 161051t^{10} + 14641t^{9} + 3993t^{8} + 2420t^{7} + 539t^{6}
    + 14t^{5} + 49t^{4} + 20t^{3} + 3t^{2} + t + 1$
    at $p=11$,
    % ZZ.<t> = PolynomialRing(ZZ); latex(...) or latex(factor(...)) in SageMath
    and
    
    \item $P_{13\ZZ}(t)
    = (13t^{2} + 1) \cdot
    (13t^{2} + 7t + 1) \cdot
    (13t^{2} + 4t + 1)^{3}$
    at $p=13$.
\end{enumerate}

In order to put the data above to use,
note that
for each $A_K$-good prime $\wp$,
the theory of N\'{e}ron models furnishes
a canonical \emph{reduction map}
$\End(A_K)\to\End(A_\wp)$,
% https://math.stackexchange.com/questions/1834474/endomorphism-ring-of-an-abelian-variety-and-its-reduction-mod-mathfrakp
which is known to be \emph{injective} (see e.g.~\cite{HuismanMSE1834526End_reduction_map}).
On the other hand,
for each $A_K$-good $\wp$,
the results
\cite{tate1966endomorphisms}*{Theorems~1(a) and~2(a)}
(combined with the fact,
noted in \cite{tate1966endomorphisms}*{p.~141, par.~2},
that over $k_\wp$,
``isogeny-class factorizations'' correspond to ``simple $E$-algebra factorizations'')
imply that
\mathd{
\End^0(A_\wp)\cong
E_{\wp,1}\times\cdots,
}
where $E_{\wp,1},\dots$ are certain \emph{central simple algebras} (CSA's) over the \emph{number field factors} $F_{\wp,1},\dots$ of the commutative subalgebra $\QQ[\pi_\wp]\belongs\End^0(A_\wp)$,
and where
\mathd{
F_{\wp,j}
\cong \QQ[T]/(T^{10}P_{\wp,j}(1/T))
\cong \QQ[t]/(P_{\wp,j}(t))
\quad\textnormal{and}\quad
[E_{\wp,j}:F_{\wp,j}]=m_{\wp,j}^2
}
correspond to an arbitrary factorization $P_\wp(t)=P_{\wp,1}(t)^{m_{\wp,1}}\cdots$ of $P_\wp(t)$ into irreducibles over $\QQ$ of the form $1+t\ZZ[t]$.

Finally,
fix $\wp\mid11$ and $\wp'\mid5$.
Then by Lemma~\ref{LEM:P_5^f,P_11^f-algebraic-factorization-properties}(1),
$P_\wp$ is irreducible over $\QQ$.
So $\End^0(A_\wp)=\QQ[\pi_\wp]\cong F_{\wp,1}\cong L_{11}$.
The algebra $\End^0(A_K)$ is thus isomorphic to a number field $F_0\belongs L_{11}$.
Now on the one hand,
$F_0\belongs L_{11}$ automatically implies $[F_0:\QQ]\in\set{1,5,10}$,
by Lemma~\ref{LEM:P_5^f,P_11^f-algebraic-factorization-properties}(1).
On the other hand,
by Lemma~\ref{LEM:P_5^f,P_11^f-algebraic-factorization-properties}(2)--(3)
(applying (2) if $2\nmid f_{\wp'}$,
and (3) if $2\mid f_{\wp'}$),
we must have $\End^0(A_{\wp'})\cong E_{\wp',1}\times E_{\wp',2}$,
where each $E_{\wp',j}$ is a CSA of dimension $m_{\wp',j}^2\in\set{1,4}$ over a number field $F_{\wp',j}$ of degree $[F_{\wp',j}:\QQ]\in\set{1,2,4,8}$.
Now fix $j\in[2]$,
and let $F'_j$ denote the image of the composition $F_0\cong\End^0(A_K)\to\End^0(A_{\wp'})\surject E_{\wp',j}$.
Let $F_{\wp',j}\vbra{F'_j}$ denote the $F_{\wp',j}$-subalgebra of $E_{\wp',j}$ generated by $F'_j$.
Then certainly $[F_{\wp',j}\vbra{F'_j}: F_{\wp',j}]\leq [E_{\wp',j}: F_{\wp',j}] = m_{\wp',j}^2\leq 4$.
But $F_{\wp',j}\vbra{F'_j}$ is an $F'_j$-module, and hence a \emph{free} $F'_j$-module (since $F'_j$ is a field).
(In fact, $F_{\wp',j}\vbra{F'_j}$ is a \emph{commutative} $F'_j$-algebra, since $F_{\wp',j}$ is \emph{central} in $E_{\wp',j}$.)
Therefore, $[F'_j: \QQ]\mid [F_{\wp',j}\vbra{F'_j}: \QQ]\perp 5$.
But the surjection $F_0\surject F'_j$ is \emph{nonzero},
% i.e.~$F'_j$ is isomorphic to a \emph{nonzero} quotient of $F_0$,
because $E_{\wp',j}\neq0$
(and all of our ring maps,
including $F_0\to E_{\wp',j}$,
are \emph{unital}).
Also,
$F_0$ is a \emph{field}.
Therefore,
$F'_j\cong F_0$.
It follows that
$[F'_j:\QQ]=[F_0:\QQ]$,
whence $[F_0:\QQ]\in \set{1,5,10}\cap \ZZ_5^\times = \set{1}$.
So $[\End^0(A_K):\QQ]=[F_0:\QQ] = 1$,
% i.e.~$\End^0(A_K)=\QQ$,
as desired.
}

\rmk{
% maybe need to extend $K$ to quadratic extension such that $5$ attains an even residue degree extension?
% is this always possible?
% Note that this reduction (i.e.~passing to field extensions) is OK because $\End(A_K)\inject\End(A_{K'})$ for field extensions $K'/K$.
% % https://mathoverflow.net/questions/349443/endomorphisms-of-weil-restriction-of-cm-elliptic-curves
One could slightly shorten the proof,
by replacing $K$ with a suitable quadratic extension $K'$
(if necessary)
to reduce to the case where there exists $\wp\mid5$ with $2\mid f_\wp$.
}

Let $Q_5(t)\defeq P_{5\ZZ}(t)$ and $Q_{11}(t)\defeq P_{11\ZZ}(t)$ be
the specific polynomials at $p\in \set{5, 11}$ written above.
We need the following lemma:
\lem{
\label{LEM:P_5^f,P_11^f-algebraic-factorization-properties}
Given $p\in\set{5,11}$ and $f\geq1$,
write $Q_p(t)=\prod(1-\beta_i t)$,
let
\mathd{
Q_{p^f}(t)\defeq\prod(1-\beta_i^ft),
}
let $Q_{p^f,1}(t)^{m(p^f,1)}\cdots$ denote a factorization of $Q_{p^f}(t)$ into irreducibles over $\QQ$ of the form $1+t\ZZ[t]$,
and let $\sqrt{Q_{p^f}}\defeq
Q_{p^f,1}\cdots$ denote a ``radical'' of $Q_{p^f}$.
\begin{enumerate}[(1)]
    \item The polynomial $Q_{11}(t)$ is irreducible,
    and the field $L_{11}\defeq\QQ[t]/Q_{11}$
    contains exactly $1$ nontrivial subfield $L'_{11}$.
    Here $[L_{11}:\QQ]=10$ and $[L'_{11}:\QQ]=5$.
    Furthermore,
    for each $f\geq1$,
    the polynomial $Q_{11^f}(t)$ is irreducible,
    and $\QQ[t]/Q_{11^f}\cong L_{11}$.
    
    \item We can write
    $Q_5=Q_{5,1}Q_{5,2}$
    with $\deg Q_{5,1}=2$ and $\deg Q_{5,2}=8$.
    Under these conventions,
    the fields $F_{5,j}\defeq\QQ[t]/Q_{5,j}$ have degree $\deg Q_{5,j}$.
    Furthermore,
    for each \emph{odd} $f\geq1$,
    we have $Q_{5^f}=\sqrt{Q_{5^f}}$
    and $\QQ[t]/Q_{5^f}\cong F_{5,1}\times F_{5,2}$.

    \item We can write
    $\sqrt{Q_{5^2}}=Q_{5^2,1}Q_{5^2,2}$
    with $\deg Q_{5^2,1}=1$ and $\deg Q_{5^2,2}=4$.
    Under these conventions,
    the fields $F_{5^2,j}\defeq\QQ[t]/Q_{5^2,j}$ have degree $\deg Q_{5^2,j}$.
    Furthermore,
    for each \emph{even} $f\geq2$,
    we have $Q_{5^f}=(\sqrt{Q_{5^f}})^2$
    and $\QQ[t]/\sqrt{Q_{5^f}}\cong F_{5^2,1}\times F_{5^2,2}$.
\end{enumerate}
}

\rmk{
It is important to view $L_{11},F_{5,2}$ as ``abstract'' (rather than ``embedded'') number fields,
since they are not Galois
(even though $F_{5,1},F_{5^2,1},F_{5^2,2}$ happen to be Galois).
}

Before proceeding,
note in general that $Q_q(t)\mid Q_{q^f}(t^f)$,
so the algebra map $\QQ[t]\to\QQ[t]/Q_q$
given by $t\mapsto t^f$
factors through a ``canonical'' map
\mathd{
\phi_{q,f}
\maps \QQ[t]/Q_{q^f}
\to \QQ[t]/Q_q
}
between two $10$-dimensional algebras.
In general,
$\phi_{q,f}$ need not be bijective
(or equivalently,
injective or surjective),
but it is always nonzero
(because it sends $1$ to $1$).
But we can still analyze $Q_{q^f}(t)$ reasonably well,
using the Galois-theoretic correspondence
between irreducible polynomials over $\QQ$ on the one hand,
and $G_\QQ$-orbits in $\ol{\QQ}$ on the other.

\pf{[Proof of (1)]
Fix $f\geq1$.
Let $\phi\defeq\phi_{11,f}$.
By Appendix~\ref{SUBSEC:local-zeta-code-and-data},
we know that $Q_{11}$ is irreducible,
i.e.~$L_{11}$ is a field.
So $G_\QQ$ acts \emph{transitively} on the roots of $Q_{11}$ in $\ol{\QQ}$,
and hence also on the roots of $Q_{11^f}$.
Thus $\sqrt{Q_{11^f}}$ is irreducible---i.e.~$\QQ[t]/Q_{11^f}$ is a \emph{local} Artinian ring.
Since $\phi\neq0$,
% it follows that $\QQ[t]/\sqrt{Q_{11^f}}$ is a subfield of $L_{11}$.
it follows that $\QQ[t]/Q_{11^f}\cong L_{11}$ holds if and only if $Q_{11^f}=\sqrt{Q_{11^f}}$.

So to prove (1),
it remains to
show that $Q_{11^f}(t)$ is square-free
(and thus irreducible),
and compute the subfields of $L_{11}$.
(We can do these tasks in either order.)

For the first task,
write $Q_{11}(t)=\prod(1-\beta_it)$ with $\beta_1,\dots,\beta_{10}\in\ol{\QQ}$;
then we need to show that $\beta_1^f,\dots,\beta_{10}^f$ are pairwise distinct.
Assume for contradiction
that $\beta_i^f=\beta_j^f$ for some distinct $i,j$.
Then $\beta_i/\beta_j$ is a \emph{primitive} $n$th root of unity for some $n\mid f$.
% ===
% Way 1 (seems slow):
% Find all $\zeta_n\in\bigcup_{i,j}\QQ(\beta_i,\beta_j)$.
% In principle can do this by primitive elements: use SageMath to find minimal polynomials of primitive elements of $\QQ(\beta_1,\beta_j)$ for $j=2,3,\dots,10$.
% (Or better: Factor $Q_{11}$ over $\QQ(\beta_1)$, and note that if $\beta_u,\beta_v$ belong to the same $G_{\QQ(\beta_1)}$-orbit, then $\QQ(\beta_1,\beta_u)\cong\QQ(\beta_1,\beta_v)$.
% So it's enough to look at one extension of $\QQ(\beta_1)$ per factor of $Q_{11}$.)
% Then use SageMath to find all roots of unity in each of the fields individually.
% ---
% Way 2 (possibly unreliable or luck-based):
% If $\beta_j\in\QQ(\beta_i)$,
% then $\zeta_n\in\QQ(\beta_i)$.
% If $\beta_j\notin\QQ(\beta_i)$,
% then $\beta_j\in\QQ(\beta_i,\zeta_n)/\QQ(\beta_i)$,
% so some $\QQ(\beta_i)$-irreducible factor of $Q_{11}$ factors in $\QQ(\beta_i,\zeta_n)$.
% ---
% Way 3 (perhaps most direct):
% Note that $\zeta_n\in\QQ(\beta_i,\beta_j)$ would have to have degree $\phi(n)\mid10d$ for some $d\in\set{1,2,\dots,9}$.
% Then compute $[\QQ(\beta_1^n):\QQ]$ for all possible $n$'s (a simple for loop should suffice).
% Alternatively,
% compute $\Disc Q_{11^n}$ for all possible $n$'s (this may be annoying though).
% ===
In particular,
$\zeta_n\in\QQ(\beta_i,\beta_j)$.
Since $[L_{11}:\QQ]=10$,
it follows that $\phi(n)
= [\QQ(\zeta_n):\QQ]\mid 10d$
for some $d\in\set{1,2,\dots,9}$.
But in general,
% https://math.stackexchange.com/questions/527946/prove-that-phin-geq-sqrtn-2
% https://math.stackexchange.com/questions/301837/is-the-euler-phi-function-bounded-below
$\phi(n)\geq\sqrt{n/2}$,
so $90\geq\sqrt{n/2}$,
whence $n<2\cdot100^2$.
By Appendix~\ref{SUBSEC:number-field-subfield-code-and-data},
we conclude that $n\leq 330$.
Hence by Appendix~\ref{SUBSEC:number-field-subfield-code-and-data},
$[\QQ(\beta_i^n):\QQ]=10$,
so the Galois conjugates $\beta_1^n,\dots,\beta_{10}^n$ must be pairwise distinct---contradicting the fact that $(\beta_i/\beta_j)^n=1$.
% This contradiction completes the first task.

For the second task,
see Appendix~\ref{SUBSEC:number-field-subfield-code-and-data}.
}

\pf{[Proof of (2)]
By Appendix~\ref{SUBSEC:local-zeta-code-and-data},
we can (and must) take
$Q_{5,1}\defeq 5t^2 + 1$
and $Q_{5,2}\defeq 625t^8 + 100t^6 + 4t^2 + 1$.
(These are irreducible,
i.e.~$F_{5,1}$ and $F_{5,2}$ are fields.)

Now fix $f\geq1$ with $2\nmid f$.
Let $\phi\defeq\phi_{5,f}$.
For convenience,
write $Q_{5,1}=\prod_{i\leq2}(1-\beta_it)$ and $Q_{5,2}=\prod_{i\geq3}(1-\beta_it)$ with $\beta_1,\dots,\beta_{10}\in\ol{\QQ}$.
Then let $Q_{5,1,f}\defeq\prod_{i\leq2}(1-\beta_i^ft)$ and $Q_{5,2,f}\defeq\prod_{i\geq3}(1-\beta_i^ft)$.
For each $j$,
Galois theory implies that $\sqrt{Q_{5,j,f}}$ is irreducible;
cf.~the proof of (1).
Furthermore,
$Q_{5,1,f}$ is clearly square-free,
since $\beta_1^f=-\beta_2^f\neq\beta_2^f$.

In fact,
$Q_{5,2,f}$ is also square-free.
Assume not;
then arguing as in (1),
we obtain distinct $i,j\geq3$,
and an integer $n\leq330$,
such that $n\mid f$ and $(\beta_i/\beta_j)^n=1$.
But here $2\nmid f$,
so $2\nmid n$.
So by Appendix~\ref{SUBSEC:number-field-subfield-code-and-data},
$[\QQ(\beta_i^n):\QQ]=8$,
so the Galois conjugates $\beta_3^n,\dots,\beta_{10}^n$ must be pairwise distinct---a contradiction.

From the above information,
we conclude that $Q_{5,j,f}$ is irreducible for each $j$.
Since $\deg Q_{5,1,f}\neq\deg Q_{5,2,f}$,
it then follows that the polynomial $Q_{5^f}=Q_{5,1,f}Q_{5,2,f}$ is square-free.
Consequently,
by the Chinese remainder theorem modulo $Q_{5^f},Q_5$,
we can identify $\phi$ with the product of the canonical maps $\psi_j\maps\QQ[t]/Q_{5,j,f}\to\QQ[t]/Q_{5,j}$.
But $\psi_1,\psi_2$ are nonzero maps between \emph{fields}---and thus isomorphisms.
So $\phi$ is an isomorphism itself,
whence $\QQ[t]/Q_{5^f}\cong\QQ[t]/Q_5\cong F_{5,1}\times F_{5,2}$,
as desired.
}

\pf{[Proof of (3)]
Following (2) and its proof,
write $Q_{5,1}=\prod_{i\leq2}(1-\beta_it)$ and $Q_{5,2}=\prod_{i\geq3}(1-\beta_it)$ with $\beta_1,\dots,\beta_{10}\in\ol{\QQ}$.
The polynomials $Q_{5,1},Q_{5,2}$ are \emph{even},
% so $Q_5(t)^2=Q_5(t)Q_5(-t)=Q_{5^2}(t^2)$.
so we may assume that $\beta_1+\beta_2=\beta_3+\beta_4=\cdots=0$.
Then in particular,
$Q_{5^2}(t)
= Q_{5^2,1}(t)^2Q_{5^2,2}(t)^2$,
where $Q_{5^2,1}\defeq\prod_{2\mid i\leq2}(1-\beta_i^2t)=Q_{5,1}(t^{1/2})=5t + 1$
and $Q_{5^2,2}\defeq\prod_{2\mid i\geq3}(1-\beta_i^2t)=Q_{5,2}(t^{1/2})=625t^4 + 100t^3 + 4t^1 + 1$.
(Note that $Q_{5^2,j}(t^2)=Q_{5,j}(t)$,
and each $Q_{5,j}$ is irreducible,
so each $Q_{5^2,j}$ is certainly irreducible.)

Now fix $f\geq1$ with $2\mid f$.
Let $\phi\defeq\phi_{5^2,f/2}$.
Let $Q_{5^2,1,f/2}\defeq\prod_{2\mid i\leq2}(1-\beta_i^ft)$ and $Q_{5^2,2,f/2}\defeq\prod_{2\mid i\geq3}(1-\beta_i^ft)$,
and let $Q_{5,1,f}\defeq\prod_{i\leq2}(1-\beta_i^ft)=Q_{5^2,1,f/2}^2$ and $Q_{5,2,f}\defeq\prod_{i\geq3}(1-\beta_i^ft)=Q_{5^2,2,f/2}^2$.
For each $j$,
Galois theory implies that $\sqrt{Q_{5^2,j,f/2}}$ is irreducible;
cf.~the proof of (1).
Furthermore,
$Q_{5^2,1,f/2}$ is clearly square-free,
since $\deg Q_{5^2,1,f/2}=1$.

In fact,
$Q_{5^2,2,f/2}$ is also square-free.
Assume not;
then arguing as in (1),
we obtain distinct \emph{even} indices $i,j\geq3$,
and an integer $n\leq330$,
such that $n\mid f/2$ and $(\beta_i^2/\beta_j^2)^n=1$.
So by Appendix~\ref{SUBSEC:number-field-subfield-code-and-data},
$[\QQ((\beta_i^2)^n):\QQ]=4$,
so the Galois conjugates $(\beta_4^2)^n,\dots,(\beta_{10}^2)^n$ must be pairwise distinct---a contradiction.

We conclude that $Q_{5^2,1,f/2}$ and $Q_{5^2,2,f/2}$ are irreducible.
Since $\deg Q_{5^2,1,f/2}\neq\deg Q_{5^2,2,f/2}$,
it then follows that $\sqrt{Q_{5^f}}=Q_{5^2,1,f/2}Q_{5^2,2,f/2}$ and $Q_{5^f}=(\sqrt{Q_{5^f}})^2$.
By studying the ``canonical'' maps $\phi'\maps\QQ[t]/\sqrt{Q_{5^f}}\to\QQ[t]/\sqrt{Q_{5^2}}$ and $\psi'_j\maps\QQ[t]/Q_{5^2,j,f/2}\to\QQ[t]/Q_{5^2,j}$,
we obtain an isomorphism $\QQ[t]/\sqrt{Q_{5^f}}\cong\QQ[t]/\sqrt{Q_{5^2}}\cong F_{5^2,1}\times F_{5^2,2}$,
as desired;
cf.~the proof of (2).
}

\section{Supporting code}

\subsection{Local zeta polynomials}
\label{SUBSEC:local-zeta-code-and-data}

Running (in SageMath)
\begin{verbatim}
def Delta(c_1,c_2,c_3,c_4,c_5,c_6):
    return Integer(expand(
        3*prod(sqrt(c_1)^3 + i[0]*sqrt(c_2)^3 + i[1]*sqrt(c_3)^3
        + i[2]*sqrt(c_4)^3 + i[3]*sqrt(c_5)^3 + i[4]*sqrt(c_6)^3
    for i in cartesian_product([[-1,1],[-1,1],
                                [-1,1],[-1,1],[-1,1]]))));
factor(Delta(1,2,3,4,5,6))
\end{verbatim}
% latex(factor(Delta(1,2,3,4,5,6)))
% yields $3^{13} \cdot 996001 \cdot 1898591 \cdot 107541241 \cdot 1722583559$.
yields \verb!3^13 * 996001 * 1898591 * 107541241 * 1722583559!.
% https://tex.stackexchange.com/questions/2790/when-should-one-use-verb-and-when-texttt
So in particular,
$V_{\bm{c}}$ in Proposition~\ref{PROP:absolute/strong/potential-irreducibility-for-V_c-when-c=(1,2,3,4,5,6)}
has good reduction at all primes $p\in[5,100]$.
Now in \cite{debarre2021lines}'s supplementary code \cite{github-alaface-CubLin-2016}*{{\tt algorithms.txt}},
re-define the function {\tt ZetaG} as follows:
% https://www.overleaf.com/learn/latex/code_listing#The_verbatim_environment
% http://magma.maths.usyd.edu.au/magma/handbook/text/8 to check not-equality
\begin{verbatim}
// Input: (c,q) with q an odd prime power,
//              c[6] ne 0 in Fq, and V_c/Fq smooth of dimension 3
// Output: the characteristic polynomial P_1(F(V_c))
ZetaG := function(c,q)
 Fq := FiniteField(q);
 P4<[x]> := ProjectiveSpace(Fq,4);
 X := Scheme(P4,c[6]^3*(&+[x[i]^3 : i in [1..5]])
                - (&+[c[i]*x[i] : i in [1..5]])^3);
 L := Scheme(P4,[x[2]+x[4], x[3]+x[5],
                            c[6]*x[1] - &+[c[i]*x[i] : i in [1..5]]]);
 R<t> := PolynomialRing(Rationals());
 n := [Mr(X,L,r) : r in [1..5]];
 u := -&+[n[r]/r*t^r : r in [1..5]];
 ll := Coefficients(R!&+[u^k/Factorial(k) : k in [0..5]])[1..6];
 cf := ll cat [q^i*ll[6-i] : i in [1..5]];
 return &+[cf[i]*t^(i-1) : i in [1..11]];
end function;
\end{verbatim}
Then from the Magma session
\begin{verbatim}
> load "algorithms.txt";
> ZetaG([1,2,3,4,5,6],5);
3125*t^10 + 1125*t^8 + 100*t^6 + 20*t^4 + 9*t^2 + 1
> ZetaG([1,2,3,4,5,6],7);
16807*t^10 + 40817*t^9 + 50421*t^8 + 40572*t^7 + 23478*t^6
+ 10182*t^5 + 3354*t^4 + 828*t^3 + 147*t^2 + 17*t + 1
> ZetaG([1,2,3,4,5,6],11);
161051*t^10 + 14641*t^9 + 3993*t^8 + 2420*t^7 + 539*t^6
+ 14*t^5 + 49*t^4 + 20*t^3 + 3*t^2 + t + 1
> ZetaG([1,2,3,4,5,6],13);
371293*t^10 + 542659*t^9 + 432809*t^8 + 234572*t^7 + 94718*t^6
+ 29666*t^5 + 7286*t^4 + 1388*t^3 + 197*t^2 + 19*t + 1
\end{verbatim}
% > ZetaG([1,2,3,4,5,6],9);
% Warning: the execution of this statement would exceed the memory limit for
% the student version (150 MB).  Magma will execute further commands provided
% they do not exceed the limit of 150 MB.
and the (SageMath) factorizations (over $\ZZ[t]$)
% ZZ.<t> = PolynomialRing(ZZ)
% factor(3125*t^10 + 1125*t^8 + 100*t^6 + 20*t^4 + 9*t^2 + 1)
\begin{enumerate}[(1)]
    \item \verb!(5*t^2 + 1) * (625*t^8 + 100*t^6 + 4*t^2 + 1)! at $p=5$,
    
    \item \verb!(7*t^2 + t + 1) * (7*t^2 + 4*t + 1)^4! at $p=7$,
    
    \item \verb!irreducible! at $p=11$,
    and
    
    \item \verb!(13*t^2 + 1) * (13*t^2 + 7*t + 1) * (13*t^2 + 4*t + 1)^3! at $p=13$,
\end{enumerate}
we obtain
% the local factors of $L(s-1/2,V_{\bm{c}})$
the polynomials $P_{5\ZZ},P_{7\ZZ},P_{11\ZZ},P_{13\ZZ}$
as quoted in the proof of Proposition~\ref{PROP:absolute/strong/potential-irreducibility-for-V_c-when-c=(1,2,3,4,5,6)}.

\subsection{Number field subfield computations}
\label{SUBSEC:number-field-subfield-code-and-data}

The SageMath code
\begin{verbatim}
max_tested = 0; max_discovered = 0
for n in range(1,2*100^2):
    max_tested = n
    for d in range(1,10):
        if (10*d) % euler_phi(n) == 0:
            max_discovered = n; break
print(max_tested,max_discovered)
\end{verbatim}
yields
\verb!19999 330!,
showing that
\mathd{
\max\set{n\in[1,2\cdot100^2-1]
: \exists\;d\in\set{1,2,\dots,9}
\;\textnormal{such that}\;
\phi(n)\mid 10d}
= 330.
}

We now study $Q_{11}$.
The SageMath code
% ===
% Attempt 1 (too slow):
% K.<a> = NumberField(161051*x^10 + 14641*x^9 + 3993*x^8 + 2420*x^7 + 539*x^6 + 14*x^5 + 49*x^4 + 20*x^3 + 3*x^2 + x + 1)
% R.<x> = K[]
% factor(161051*x^10 + 14641*x^9 + 3993*x^8 + 2420*x^7 + 539*x^6 + 14*x^5 + 49*x^4 + 20*x^3 + 3*x^2 + x + 1)
% L.<b> = K.extension(x^8 + (-14641*a^9 - 1331*a^8 - 363*a^7 - 220*a^6 - 49*a^5 - 14/11*a^4 - 49/11*a^3 - 20/11*a^2 + 8/11*a)*x^7 + (-1331*a^8 - 121*a^7 - 33*a^6 - 20*a^5 - 49/11*a^4 - 14/121*a^3 + 72/121*a^2 - 9/121*a + 1/11)*x^6 + (-1331*a^9 - 121*a^8 - 154*a^7 - 31*a^6 - 82/11*a^5 - 234/121*a^4 + 23/121*a^3 - 113/1331*a^2 + 72/1331*a)*x^5 + (-121*a^8 - 11*a^7 - 14*a^6 - 31/11*a^5 + 39/121*a^4 - 113/1331*a^3 + 56/1331*a^2 + 107/14641*a + 1/121)*x^4 + (-121*a^9 - 11*a^8 - 14*a^7 - 31/11*a^6 - 82/121*a^5 - 234/1331*a^4 + 23/1331*a^3 - 113/14641*a^2 + 72/14641*a)*x^3 + (-11*a^8 - a^7 - 3/11*a^6 - 20/121*a^5 - 49/1331*a^4 - 14/14641*a^3 + 72/14641*a^2 - 9/14641*a + 1/1331)*x^2 + (-11*a^9 - a^8 - 3/11*a^7 - 20/121*a^6 - 49/1331*a^5 - 14/14641*a^4 - 49/14641*a^3 - 20/14641*a^2 + 8/14641*a)*x + 1/14641)
% print(L.number_of_roots_of_unity())
% (seems slow?)
% EDIT: Note that K.composite_fields(K) (i.e. all possible compositums of K with itself) returns
% [Number Field in a with defining polynomial 161051*t^10 + 14641*t^9 + 3993*t^8 + 2420*t^7 + 539*t^6 + 14*t^5 + 49*t^4 + 20*t^3 + 3*t^2 + t + 1,
%  Number Field in a0 with defining polynomial 32989690295920386835890134305346271024600891770176817346352641662914097799127234258677851*t^80 - 23992502033396644971556461312979106199709739469219503524620103027573889308456170369947528*t^79 - 20720797210660738839071489315754682627022047723416843953080998069268358948212147137681956*t^78 - 8129690771646797056477809205224325241223961307752063177763836563062226955757875910478088*t^77 + 53451364879436064238848833139227118451306765560558326691494936515787734990157520052206542*t^76 - 8099784128513849254365186746446791692412766529710040369258658168126551559276807328465142*t^75 - 26994600144252838601559149400999836920785676547218672298190358171509115565180323782547202*t^74 - 20422214946654708291703003323202395627247018413874008516236645401523265979857568280897662*t^73 + 37917211352401107539974823888065884739342875683291288053142540629509286047785130291162329*t^72 + 5805204543061704202540458329590673340836166814279990822800319881950236000032949188525480*t^71 - 13223785106859286842310073012742628200703062448087100561294344050536061129619067379303174*t^70 - 16398808640480528229946773226751449605405079590848143558341406810338980375952924898736918*t^69 + 15306091721029833022115489303019830903469669684933516647231931681522332624605018825697676*t^68 + 3708468398073439420768235325749931062292012013137624997567533733394273630458375058783186*t^67 - 4872537212390862727530357848758814325114215554290397922099416579947473667750600050003730*t^66 - 3915540957618671511266540490198041268594467812965468956311337667873886744300954910060318*t^65 + 6187026024236675855650281694278751645576197671921946314354289186422876435866247693808957*t^64 - 2237246703501197349490421192165693788058297868233871903270024340243353583596096420335416*t^63 - 3925489398004663552874269866113537492955780981703259059075563562411526225285188319079390*t^62 + 2230998003676777036670114944695292975005560563428417547402206083134504557121734138482290*t^61 + 4438259877920894418075470736659634180605147948887553664295006644031480291480310636776774*t^60 - 2781971532540977276913386917089621666581676201994221770071106203932592358741207558778216*t^59 - 3334516116576979305028277476649903517465419770924776293969063216195501193226866876538580*t^58 + 1680842203520897366266026571917478897741957726956783646834739290976724160775546539278298*t^57 + 3246798441943039962965959853111262695340038561294531460826405426221940937827773883554816*t^56 - 1542833217203841297969050358485841384890265918040851841452354342811440711620301625334370*t^55 - 1930474786252313326143013507487175658055905028457130432838898654067685284671896586488456*t^54 + 344743391781431476061929075924762705615592978203094415123231414562820956623636348795724*t^53 + 1958949782472702811498003181352819640921900390473012747546811564324072173220912189114014*t^52 - 457222762761962361385803833410042450682130643180881626465687004383699656689162215121818*t^51 - 866357125601694994879336204562511230549592946570406145507644791973726653416633297101750*t^50 + 50037706042507264636934639913787241647480705939743927946760523646796242021413114573768*t^49 + 447898636105129739708538522742557660797104779805622298133565363641906645586242799865435*t^48 - 47061287408187646981720622870929286950509561649512838594154399545865741051759519868330*t^47 - 290852370042225468355913766862741551158740661130986662926025954943088424858903495540490*t^46 + 166866355828754914706410000504785058981878099172887435425595931380679191421728847701592*t^45 + 87308655870177006074248695808254088540083334379288386578234523131619390526454845975848*t^44 - 19986710956996818371543561421219857924602364227708531136569705181677232974042724543762*t^43 - 123282421452301269181567436788075347004460241597541110254452743753600300928559189141776*t^42 + 26682181990828211619387747109260165498737642132912214356716715508617119271392146190010*t^41 + 40815040247152524635753813566476215174132896246161740701702864240601393298671116956562*t^40 + 11797464793873577629191260156587612075383465392912541262351039029654156900598757058088*t^39 - 10640246970464881794080072931865952993834705037108114514517179166360740970228812773474*t^38 - 11734410318640078565172217915722901840931796560496795224349919262347211938412202987122*t^37 + 592982325923328262806127469920922272263308754873838695140336521915249433256405714320*t^36 + 494374768327638948817460281695480345069542414603791867809788188563047292315656586326*t^35 + 5235931986785936080514244055338431501589737202884687861121367777623754598886179439160*t^34 - 1495756031759487619768699079391982287581574388812962710070232244111945726333036131740*t^33 - 544436866925267921707482486839726923452075581996243278594357540774426648296924708475*t^32 - 348181539122965003982969863731347232943521437709969803093503908966635252515008594322*t^31 + 469609277746339346652956940028599634051840002089037029684579106699757447339048873768*t^30 - 150058393673826218493045869446030317468003798636072951204128261704860007749865525228*t^29 - 106321825110029506869380036656981557023680703362313893459144837189081525920440477764*t^28 + 99216621899015700638996052144190692288983037078564491171160676852281867191121033694*t^27 + 5412031074436719293069769811531709665121309849610490083551180737549118153581061040*t^26 - 25712726737314896290275538119638543512766895821144166921356781089363314387957465292*t^25 - 2171613463495396623054512978451530608697589321689376978474886257455719231086172616*t^24 + 14824781254149529595841465816761903038505651459049366060549942106055754501044040934*t^23 - 3564601631780060174280981844018096114380128556818125948333376967467862700654833688*t^22 - 2358935758960676451206984697224215516383937280841470035729476017980585105593584710*t^21 + 233541228024662787517213546855423204369450604584824275069126670149927178602380974*t^20 + 990847199841203231225649525847365351352424488976472451212028427428645863927481760*t^19 - 520906534234399062911741962542288734682707940743339395833490960944580402869393456*t^18 - 146949959813614494725849844109775929533555736523066892534270542681504895007910806*t^17 + 174487096444777608067581706695479377556681936411343401878514979476138359658716491*t^16 - 23121887705535250333830515879046296325236007037041576969936676160715872532063940*t^15 + 12983112802768344489777111317941192197319432810330894477895435168967652581979916*t^14 - 4682989976076729036165168474200561304471704345434700228466537453289926834707850*t^13 - 85597842515977730778534950876515040549837600760184678056252093610981705284772*t^12 - 732251045844299963827608770789343132887960507153331521125881827511003003869378*t^11 - 576551826546002591352108834964335042254370581286878989193371991308347891100890*t^10 + 347041242425584031822717287900967436593561636416654117658578882499695761589274*t^9 - 23372684901409149237160103147615154329686341330241500300021420584302782694561*t^8 + 23885784886228639795470796846542469937716135571809037817607173665721177099276*t^7 + 9723378998464615334927244217348861998716301387428250235441694410868849910192*t^6 - 2637717811352934519545049100980759827689317685911860737875622926645794139930*t^5 - 2626898939179641762809407175792366139997476146570943998857438730366878821732*t^4 + 315514497887795876110635756437872217410111075858313618223876985902830944902*t^3 - 42439587405788488652253167120396132966473476380457037584265239197858486400*t^2 - 88096275975621036774361522870339098436977074922239690181879763700908941708*t + 47766993464287459054101521892848369232828716452351125794824971966682018659]
% but this is too slow to work with if we do L = K.composite_fields(K)[1] and print(L.number_of_roots_of_unity())
% ---
% Attempt 2:
% Use Q_11 = 161051*t^10 + 14641*t^9 + 3993*t^8 + 2420*t^7 + 539*t^6 + 14*t^5 + 49*t^4 + 20*t^3 + 3*t^2 + t + 1
\begin{verbatim}
max_tested = 0; R.<t> = QQ[]
Q_11 = ZetaG([1,2,3,4,5,6],11); K.<a> = NumberField(Q_11)
for n in range(1,500):
    max_tested = n; L = K.subfield(a^n)[0]
    if L.absolute_degree() != 10:
        print(n)
print("Tested up to",max_tested)
\end{verbatim}
yields
\verb!Tested up to 499!,
showing that for each positive integer $n<500$,
the field $\QQ[t]/Q_{11}$ is generated by $\res{t}^n$
(i.e.~$\res{t}^n$ is a \emph{primitive element} for $\QQ[t]/Q_{11}$).
Meanwhile,
% the SageMath code
% Use Q_11 = 161051*t^10 + 14641*t^9 + 3993*t^8 + 2420*t^7 + 539*t^6 + 14*t^5 + 49*t^4 + 20*t^3 + 3*t^2 + t + 1
\begin{verbatim}
R.<t> = QQ[]; Q_11 = ZetaG([1,2,3,4,5,6],11)
K.<a> = NumberField(Q_11)
print(K.is_galois(),[Z[0].absolute_degree() for Z in K.subfields()])
\end{verbatim}
yields
\verb!False [1, 5, 10]!,
% \begin{verbatim}
% [Number Field in a0 with defining polynomial t,
%  Number Field in a1 with defining polynomial t^5 - 61*t^4 + 1436*t^3 - 16248*t^2 + 88107*t - 182732,
%  Number Field in a2 with defining polynomial t^10 - t^9 + 3*t^8 - 20*t^7 + 49*t^6 - 14*t^5 + 539*t^4 - 2420*t^3 + 3993*t^2 - 14641*t + 161051]
% \end{verbatim}
showing that
the field $L_{11}\defeq\QQ[t]/Q_{11}$ is non-Galois,
with exactly $1$ nontrivial subfield $L'_{11}$---and furthermore,
$[L'_{11}:\QQ]=5$.

We now study $Q_{5,2}$.
The SageMath code
\begin{verbatim}
max_tested = 0; R.<t> = QQ[]
Q_52 = 625*t^8 + 100*t^6 + 4*t^2 + 1; K.<a> = NumberField(Q_52)
for n in range(1,500,2):
    max_tested = n; L = K.subfield(a^n)[0]
    if L.absolute_degree() != 8:
        print(n)
print("Tested odds up to",max_tested)
\end{verbatim}
yields
\verb!Tested odds up to 499!,
showing that for each positive \emph{odd} integer $n<500$,
the field $\QQ[t]/Q_{5,2}$ is generated by $\res{t}^n$.

% For $Q_{5,2}$:
% \begin{verbatim}
% R.<t> = QQ[]; Q_52 = 625*t^8 + 100*t^6 + 4*t^2 + 1; K.<a> = NumberField(Q_52)
% print(K.is_galois(),[Z[0].absolute_degree() for Z in K.subfields()])
% \end{verbatim}
% gives \verb!False [1, 2, 2, 2, 4, 4, 4, 8]!

Finally,
we study $Q_{5^2,2}$.
The SageMath code
\begin{verbatim}
max_tested = 0; R.<t> = QQ[]
Q_522 = 625*t^4 + 100*t^3 + 4*t^1 + 1; K.<a> = NumberField(Q_522)
for n in range(1,500):
    max_tested = n; L = K.subfield(a^n)[0]
    if L.absolute_degree() != 4:
        print(n)
print("Tested up to",max_tested)
\end{verbatim}
yields
\verb!Tested up to 499!,
showing that for each positive integer $n<500$,
the field $\QQ[t]/Q_{5^2,2}$ is generated by $\res{t}^n$.

% For $Q_{5^2,2}$:
% \begin{verbatim}
% R.<t> = QQ[]; Q_522 = 625*t^4 + 100*t^3 + 4*t^1 + 1; K.<a> = NumberField(Q_522)
% print(K.is_galois(),[Z[0].absolute_degree() for Z in K.subfields()])
% \end{verbatim}
% gives \verb!True [1, 2, 2, 2, 4]!


% \chapter{Zero-density hypotheses}
% discussions with Mayank, link to axiomatizing blah and when does density blah

\chapter{Sharp cutoffs}
% Mostly useful for giving talks
\label{CHAP:sharp-cutoffs}

Let $F\in \ZZ[\bm{x}]=\ZZ[x_1,\dots,x_6]$ be a $\PP^5_\QQ$-smooth $6$-variable cubic form.
Recall, from Definition~\ref{DEFN:HLH-asymptotic},
the singular series $\mf{S}_F$,
the set $C(\map{SSV})$,
and the real density $\sigma_{\infty,F,w}\defeq \lim_{\eps\to0} \,(2\eps)^{-1}\int_{\abs{F(\bm{x})}\leq\eps}d\bm{x}\,w(\bm{x})$ for each $w\in C^\infty_c(\RR^6)$.
Let $C(T)\defeq \bigcup_{L\in C(\map{SSV})} L\belongs \QQ^6$.
For each $w\in C^\infty_c(\RR^6)$, let $N'_{F,w}(X)\defeq \sum_{\bm{x}\in \ZZ^6\setminus C(T)} w(\bm{x}/X)\cdot \bm{1}_{F(\bm{x})=0}$.

Let $K$ be a compact semi-algebraic subset of $\RR^6$.
In analogy with $N'_{F,w}(X)$, let
\mathd{
N'_{F,K}(X)\defeq \#\set{\bm{x}\in (\ZZ^6\cap XK)\setminus C(T): F(\bm{x})=0}.
}
For $\delta>0$ and $\bm{a}\in \RR^6$, let $B_\delta(\bm{a})$ denote the closed Euclidean ball of radius $\delta$ centered at $\bm{a}$.
% Now---and this is a subtle point if $K$ is not ``sufficiently transverse'' to $\set{\bm{x}\in \RR^6: F(\bm{x})=0} = C(V)(\RR)$---let $K'\defeq \set{\bm{x}\in K: F(\bm{x})=0} = K\cap C(V)(\RR)$,
% let $K'_\delta\defeq \bigcup_{\bm{a}\in K'} B_\delta(\bm{a})$,
% and define $\sigma_{\infty,F,K}$ to be the iterated limit
% \mathd{
% \lim_{\delta\to 0} \left(\lim_{\eps\to 0} \,(2\eps)^{-1}\vol\set{\bm{x}\in K'_\delta: \abs{F(\bm{x})}\leq \eps}\right).
% }
Now---and this is a subtle point if $K$ is not ``sufficiently transverse'' to $\set{\bm{x}\in \RR^6: F(\bm{x})=0} = C(V)(\RR)$---let $K_\delta\defeq \bigcup_{\bm{a}\in K} B_\delta(\bm{a})$,
and define $\sigma_{\infty,F,K}$ to be the iterated limit
\mathd{
\lim_{\delta\to 0} \left(\lim_{\eps\to 0} \,(2\eps)^{-1}\vol\set{\bm{x}\in K_\delta: \abs{F(\bm{x})}\leq \eps}\right).
}
I believe that $\sigma_{\infty,F,K}$ should exist and be finite, and that the following statement should hold, but have not checked carefully.
\cnj{
Fix $\eta>0$.
Suppose $\lim_{X\to\infty}(N'_{F,w}(X)/X^3) = \sigma_{\infty,F,w}\mf{S}_F$ holds for all $w\in C^\infty_c(\RR^6)$ with $\Supp{w}\belongs K_\eta$.
Then $\lim_{X\to\infty}(N'_{F,K}(X)/X^3) = \sigma_{\infty,F,K}\mf{S}_F$.
}

\pf{
[Plausible proof sketch]
For simplicity, assume $\bm{0}\notin K$.
(I believe the case $\bm{0}\in K$ should follow from the case $\bm{0}\notin K$ without too much trouble.)

To prove $\limsup_{X\to\infty}(N'_{F,K}(X)/X^3) \leq \sigma_{\infty,F,K}\mf{S}_F$, take decreasing opens $U_i\to K$ and use the hypothesis for suitable weights $w_i\in C^\infty_c(\RR^6)$ that are $1$ on $K$ and $0$ outside $U_i$;
I expect that $\lim_{i\to\infty} \sigma_{\infty,F,w_i} = \sigma_{\infty,F,K}$.

To finish, it should suffice to bound the ``error'' $(N'_{F,w_i}(X)-N'_{F,K}(X))/X^3$ by $o_{i\to\infty}(1)$ as $X\to\infty$.
To do so, take $\eps_i\to 0$ and use weights $w'_i$ that are $1$ on $\ol{U_i\cap (C(V)(\RR)\setminus K)}$ and $0$ outside an $\eps_i$-neighborhood thereof.
Then
\mathd{
0\leq N'_{F,w_i}(X)-N'_{F,K}(X)\leq N'_{F,w'_i}(X).
}
I expect that (since $C(V)$ is smooth away from $\bm{0}$, and $K$ is semi-algebraic) the boundary of $C(V)(\RR)\setminus K$ in $C(V)(\RR)$ is \emph{relatively} null (for any reasonable measure on $C(V)(\RR)$), and that consequently $\sigma_{\infty,F,w'_i} = o_{i\to\infty}(1)$.
}

\rmk{
When choosing a region $K$, we must avoid the situation $K = \QQ^6\cap [-1,1]^6$,
and possibly also (though I have not checked carefully) the situation where $\set{\bm{x}\in K: F(\bm{x})=0}$ is like a fat Cantor set, say.
So some restriction on $K$ (like the present ``compact semi-algebraic'' assumption) is necessary or at least convenient.
}

\rmk{
If correct, the ideas in this appendix should apply much more generally to most Manin-type settings.
But of course, the hard part is still counting with \emph{any} reasonable weight at all!
I just wanted to record some thoughts and subtleties that I have not seen thoroughly discussed elsewhere.
}

% \chapter{Chapter 1}  %insert chapter 1 title
% \input{chp1}    %insert chapter 1 file name

% \chapter{Chapter 2}  %continue adding chapter titles
% \input{chp2}    %and files

\cleardoublepage
\phantomsection
\addcontentsline{toc}{chapter}{Bibliography}
% https://tex.stackexchange.com/a/98995
% \renewcommand{\biblistfont}{\normalsize}
\bibliographystyle{amsxport}
% \nocite{*}
\bibliography{master.bib}

\end{document}